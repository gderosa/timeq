\documentclass[a4paper,12pt]{article}
\usepackage{hyperref}
\usepackage{amssymb}

\author{Guido De Rosa \\ \small\tt{gderosa@umail.ucc.ie}}
\title{Pauli's Theorem}
\begin{document}

\maketitle

\begin{abstract}
A proof is detailed of the (in)famous Pauli's ``theorem''~\cite[footnote~2]{PauliFootnote}.
\end{abstract}

\section{Introduction}

In a footnote in~\cite{PauliFootnote} Pauli says there's no time in QM. But he
doesn't give an explicit proof. We'll follow (and expand)~\cite{Galapon2002}.

\section{Proof of Pauli's formal argument}\label{proof}

Let's assume that there exists a self-adjoint time operator, $T$, canonically conjugate
to the Hamiltonian $H$, i.e.

$$
[T, H] = i\hbar
$$
Since T is self-adjoint, then for all
$\beta\in\mathbb{R}$, $U_{\beta} = \exp(- i \beta T / \hbar)$
is unitary. A formal
expansion of the exponential yields the commutator

$$
[U_{\beta}, H] = 
\left[
    \sum_{k=0}^{\infty} \frac{1}{k!} \left(- \frac{i\beta T}{\hbar} \right)^k, H
\right]
$$

\bibliographystyle{abbrv}
\bibliography{pauli}

\end{document}
\end{document}