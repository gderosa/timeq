\documentclass[a4paper]{article}
\usepackage{hyperref}
\usepackage{amsmath}
\usepackage{amssymb}
\usepackage{breqn}
%% http://www.maths.tcd.ie/~dwilkins/LaTeXPrimer/Theorems.html
%% https://www.sharelatex.com/learn/Theorems_and_proofs
%% http://tex.stackexchange.com/a/46262

\usepackage{amsthm}

\newtheorem{theorem}{Theorem}[section]
\newtheorem{lemma}[theorem]{Lemma}
\newtheorem{proposition}[theorem]{Proposition}
\newtheorem{corollary}{Corollary}[theorem]
\newtheorem{remark}{Remark}[subsection]

\newenvironment{definition}[1][Definition]{\begin{trivlist}
\item[\hskip \labelsep {\bfseries #1}]}{\end{trivlist}}
\newenvironment{example}[1][Example]{\begin{trivlist}
\item[\hskip \labelsep {\bfseries #1}]}{\end{trivlist}}

\newcommand{\remarkautorefname}{Remark}

\author{Guido De Rosa \\ \small\tt{gderosa@umail.ucc.ie}}
\title{Pauli's Theorem}

\begin{document}

\maketitle

\begin{abstract}
A proof is detailed of the (in)famous Pauli's ``theorem''~\cite{PauliFootnote}
on the impossibility of a time observable in quantum mechanics. Some possible
approaches towards overcoming Pauli's objections are reviewd and expanded as well.
\end{abstract}

\section{Introduction}

In a footnote in~\cite{PauliFootnote}, W. Pauli excluded the possibility
of a self-adjoint operator representing time as a quantum observable.
However, he did not provide an explicit proof.
Here a proof is given, based on~\cite{Galapon2002}, but expanded in more detail.

\section{Proof of Pauli's formal argument}\label{proof}

Let's assume that there exists a self-adjoint time operator, $T$, canonically conjugate
to the Hamiltonian $H$, i.e.

\begin{equation}
\label{THcommutator}
[T, H] = i\hbar
\end{equation}
Since T is self-adjoint, then for all
$\beta\in\mathbb{R}$, $U_{\beta} = \exp(- i \beta T / \hbar)$
is unitary. A formal
expansion of the exponential yields the commutator

\begin{equation}
[U_{\beta}, H]  = 
\left[
    \sum_{k=0}^{\infty} \frac{1}{k!} \left(- \frac{i\beta T}{\hbar} \right)^k, H
\right]         =
\sum_{k=0}^{\infty} \frac{1}{k!} \left(- \frac{i\beta}{\hbar} \right)^k [T^k, H]
\end{equation}.

As the commutator $[T, H]$ itself commutes with its operator $T$,
the following identity holds (See Lemma \ref{CommProp}):

$$
[T^k, H] = kT^{k-1}[T, H]
$$
hence:

\begin{multline}
[U_{\beta}, H]  = 
\sum_{k=0}^{\infty} \frac{1}{k!} \left(- \frac{i\beta}{\hbar} \right)^k kT^{k-1}[T, H] \\ =
\beta\sum_{k=1}^{\infty} \frac{1}{(k-1)!} \left(- \frac{i\beta}{\hbar} \right)^{k-1} T^{k-1} =
\beta\sum_{\kappa=0}^{\infty} \frac{1}{\kappa!} \left(- \frac{i\beta T}{\hbar} \right)^{\kappa}  =
\beta U_{\beta}
\end{multline}
where the term for $k=0$ in the first sum clearly vanishes, hence we can start the sum from 
$k=1$ then set $\kappa=k-1$.

Now, given an eigenvector $\varphi_{E}$ so that $H\varphi_{E}=E\varphi_{E}$, there has:

$$
HU_{\beta}\varphi_{E} = (U_{\beta}H - [U_{\beta}, H])\varphi_{E} =
EU_{\beta}\varphi_{E} - \beta U_{\beta}\varphi_{E} = (E-\beta)U_{\beta}\varphi_{E}
$$
showing that $U_{\beta}\varphi_{E}$ is another eigenvector of $H$ with eigenvalue
$E-\beta$. But $\beta$ is an arbitrary real number and $H$ a \emph{generic} Hamiltonian,
hence the spectrum of a generic Hamiltonian $H$ should
be the whole real line, which contradicts the discrete and semi-bounded energy spectrum
in fact found in most physical systems.

\appendix
\section{Commutator properties}
\begin{lemma}\label{CommProp}
If the commutator $[T, H]$ commutes with $T$ i.e.
$$[T, H]T~=~T[T, H]\,,$$ then the following holds:
\begin{equation}\label{eq:tkh}
[T^k, H] = kT^{k-1}[T, H]\,.
\end{equation}
\end{lemma}
This is particularly true when $[T, H]$ is a \emph{number} as in \eqref{THcommutator} where
$T$ and $H$ are the time and energy operator respectively.
\begin{proof}
First of all, the \eqref{eq:tkh} is trivially valid for $k = 1$.

For an arbitrary positive integer $k$ there has:
\begin{dmath}\label{tkhrecur}
[T^k, H] = T^{k-1}TH - HT^{k-1}T = T^{k-1}TH - T^{k-1}HT + T^{k-1}HT - HT^{k-1}T \\
    = T^{k-1}[T, H] + [T^{k-1}, H]T
\end{dmath}
Now, iterating the result in \eqref{tkhrecur},
\begin{dmath}\label{tkhrecurplus}
[T^k, H] = T^{k-1}[T, H] + [T^{k-1}, H]T
= T^{k-1}[T, H] + (T^{k-2}[T, H] + [T^{k-2}, H]T)T
= T^{k-1}[T, H] +  T^{k-1}[T, H] + [T^{k-2}, H]T^2
= 2T^{k-1}[T, H] + [T^{k-2}, H]T^2
= \hdots
= nT^{k-1}[T, H] + [T^{k-n}, H]T^n = \hdots
\end{dmath}
where the commutativity hypothesis $[T, H]T = T[T, H]$ has been used to obtain $T^{k-2}[T, H]T = T^{k-1}[T, H]$.

Now, \eqref{tkhrecurplus} can be continued until it reaches $n=k$ when the term
$[T^{k-n}, H]T^n$ vanishes and a result of $kT^{k-1}[T, H]$ follows.
\end{proof}

\section{Density operators, projectors and measurement}

\subsection{Projectors}

We shall assume, unless otherwise stated, that all vectors are
normalized to unity.

For a given observable represented by an operator $A$ with a
\emph{discrete, non degenerate spectrum}, the probability of the
outcome $a$ of a
measurement on a system in the pure state $\ket{\psi}$ is given by
$$
\pi_{a} = \norm{\braket{a}{\psi}}^2
        = \bra{\psi}\ket{a}\bra{a}\ket{\psi}
        = \mel{\psi}{P_{a}}{\psi}
$$
where $P_{a} = \ketbra{a}$ appears as the \term{projector} on the
one-dimensional eigenspace of  $a$.

If $a$ is degenerate, and $j = 1, \dots, J$ its degeneracy index,
such probability shall sum over it:
$$
\pi_{a} = \sum_{j=1}^{J}\norm{\braket{aj}{\psi}}^2
        = \sum_{j=1}^{J}\bra{\psi}\ket{aj}\bra{aj}\ket{\psi}
        = \mel{\psi}{P_{a}}{\psi}
$$
where $P_{a} = \sum_{j=1}^{J}\ketbra{aj}$
still is the projector on the
($J$-dimensional) eigenspace of $a$.

Generally, the probability that the outcome of a measuremen falls in
the set of eigenvalues $\sigma = \{a_{1}, \dots, a_{S}\}$ is
\begin{equation}\label{eq:pi_sigma}
\pi_{\sigma}  = \sum_{s=1}^{S}\sum_{j=1}^{J_{s}}\norm{\braket{sj}{\psi}}^2
              = \sum_{s=1}^{S}\sum_{j=1}^{J_{s}}\bra{\psi}\ket{sj}\bra{sj}\ket{\psi}
              = \mel{\psi}{P_{\sigma}}{\psi}
              ,
\end{equation}
where $P_{\sigma} = \sum_{s=1}^{S}\sum_{j=1}^{J_{s}}\ketbra{sj}$
is once again a projector --- on the ``generalized eigenspace'' spanned by all
eigenvectors $\{\ket{sj}\}$ above.

\subsection[Measure]{Measure\footnote{Not to be confused with \emph{measurement}.}}

\begin{remark}\label{measure_properties}
  Being $\pi_{\sigma}$ the \emph{probability} of the outcome of a measurement to
  fall in a given set $\sigma$, it has to be:
  \begin{enumerate}
    \item \label{measure_properties:first} $0$ on the empty set
    \item non negative
    \item \label{measure_properties:last} \term{additive} on disjoint sets
    \item equal to $1$ if $\sigma$ includes the whole spectrum of $A$.
  \end{enumerate}
\end{remark}
\begin{remark}
  Properties \ref{measure_properties:first}\dots\ref{measure_properties:last}
  are the defining properties of a \term{measure} \cite{EncMath_Measure}.
\end{remark}

\bibliographystyle{abbrv}
\bibliography{pauli}

\end{document}
\end{document}