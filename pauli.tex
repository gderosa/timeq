\documentclass[12pt]{article}
\usepackage{hyperref}

\author{Guido De Rosa \\ \small\tt{gderosa@umail.ucc.ie}}
\title{Pauli's Theorem}
\begin{document}
\maketitle
\begin{abstract}
This is the paper's abstract \ldots
\end{abstract}
\section{Introduction}
This is time for all good men to come to the aid of their party!

\paragraph{Outline}
The remainder of this article is organized as follows.
Section~\ref{previous work} gives account of previous work.
Our new and exciting results are described in Section~\ref{results}.
Finally, Section~\ref{conclusions} gives the conclusions.

\section{Previous work}\label{previous work}
A much longer \LaTeXe{} example was written by Gil~\cite{Gil:02}.

\section{Results}\label{results}
In this section we describe the results.

\section{Conclusions}\label{conclusions}
We worked hard, and achieved very little.

\bibliographystyle{abbrv}
\bibliography{pauli}

\end{document}
\end{document}