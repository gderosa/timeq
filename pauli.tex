\documentclass[a4paper,12pt]{article}
\usepackage{hyperref}
\usepackage{amsmath}
\usepackage{amssymb}

%% http://www.maths.tcd.ie/~dwilkins/LaTeXPrimer/Theorems.html
%% https://www.sharelatex.com/learn/Theorems_and_proofs
%% http://tex.stackexchange.com/a/46262

\usepackage{amsthm}

\newtheorem{theorem}{Theorem}[section]
\newtheorem{lemma}[theorem]{Lemma}
\newtheorem{proposition}[theorem]{Proposition}
\newtheorem{corollary}{Corollary}[theorem]
\newtheorem{remark}{Remark}[subsection]

\newenvironment{definition}[1][Definition]{\begin{trivlist}
\item[\hskip \labelsep {\bfseries #1}]}{\end{trivlist}}
\newenvironment{example}[1][Example]{\begin{trivlist}
\item[\hskip \labelsep {\bfseries #1}]}{\end{trivlist}}

\newcommand{\remarkautorefname}{Remark}

\author{Guido De Rosa \\ \small\tt{gderosa@umail.ucc.ie}}
\title{Pauli's Theorem}
\begin{document}

\maketitle

\begin{abstract}
A proof is detailed of the (in)famous Pauli's ``theorem''~\cite[footnote~2]{PauliFootnote}.
\end{abstract}

\section{Introduction}

In a footnote in~\cite{PauliFootnote} Pauli says there's no time in QM. But he
doesn't give an explicit proof. We'll follow (and expand)~\cite{Galapon2002}.

\section{Proof of Pauli's formal argument}\label{proof}

Let's assume that there exists a self-adjoint time operator, $T$, canonically conjugate
to the Hamiltonian $H$, i.e.

$$
[T, H] = i\hbar
$$\label{THcommutator}
Since T is self-adjoint, then for all
$\beta\in\mathbb{R}$, $U_{\beta} = \exp(- i \beta T / \hbar)$
is unitary. A formal
expansion of the exponential yields the commutator

$$
[U_{\beta}, H]  = 
\left[
    \sum_{k=0}^{\infty} \frac{1}{k!} \left(- \frac{i\beta T}{\hbar} \right)^k, H
\right]         =
\sum_{k=0}^{\infty} \frac{1}{k!} \left(- \frac{i\beta}{\hbar} \right)^k [T^k, H]
$$.

As the commutator $[T, H]$ itself commutes with its operators $T$ and $H$,
the following identity holds (See (\ref{CommProp})):

$$
[T^k, H] = kT^{k-1}[T, H]
$$
hence:

\begin{multline}
[U_{\beta}, H]  = 
\sum_{k=0}^{\infty} \frac{1}{k!} \left(- \frac{i\beta}{\hbar} \right)^k kT^{k-1}[T, H] =
-\beta\sum_{k=1}^{\infty} \frac{1}{(k-1)!} \left(- \frac{i\beta}{\hbar} \right)^{k-1} T^{k-1} \\ =
-\beta\sum_{\kappa=0}^{\infty} \frac{1}{\kappa!} \left(- \frac{i\beta}{\hbar} \right)^{\kappa} T^{\kappa} =
-\beta U_{\beta}
\end{multline}
where the term for $k=0$ in the first sum clearly vanishes, hence we can start the sum from 
$k=1$ then set $\kappa=k-1$.

Now, given an eigenvector $\varphi_{E}$ so that $H\varphi_{E}=E\varphi_{E}$, there has:

$$
HU_{\beta}\varphi_{E} = (U_{\beta}H - [U_{\beta}, H])\varphi_{E} =
EU_{\beta}\varphi_{E} + \beta U_{\beta}\varphi_{E} = (E+\beta)U_{\beta}\varphi_{E}
$$
showing that $U_{\beta}\varphi_{E}$ is another eigenvector of $H$ with eigenvalue
$E+\beta$. But $\beta$ is an arbitrary real number and $H$ a \emph{generic} Hamiltonian,
hence the spectrum of a generic Hamiltonian $H$ should
be the whole real line, which contradicts the discrete and semi-bounded energy spectrum
in fact found in most physical systems.

\appendix\section{Commutator properties}\label{CommProp}
\begin{lemma}
If the commutator $[T, H]$ of two operators commute with its operators i.e. If
$[[T, H], T] = [[T, H], H] = 0$, then the following holds:
$$
[T^k, H] = kT^{k-1}[T, H]
$$
\end{lemma}
This is particularly true when $[T, H]$ is a ``number'' as in (\ref{THcommutator}) where
$T$ and $H$ are the time and energy operator respectively.
\begin{proof}
TODO
\end{proof}

\bibliographystyle{abbrv}
\bibliography{pauli}

\end{document}
\end{document}