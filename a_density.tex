\section{Density operators, projectors and measurement}

\subsection{Projectors}

We shall assume, unless otherwise stated, that all vectors are
normalized to unity.

For a given observable represented by an operator $A$ with a
\emph{discrete, non degenerate spectrum}, the probability of the
outcome $a$ of a
measurement on a system in the pure state $\ket{\psi}$ is given by
$$
\pi_{a} = \norm{\braket{a}{\psi}}^2
        = \bra{\psi}\ket{a}\bra{a}\ket{\psi}
        = \mel{\psi}{P_{a}}{\psi}
$$
where $P_{a} = \ketbra{a}$ appears as the \term{projector} on the
one-dimensional eigenspace of  $a$.

If $a$ is degenerate, and $j = 1, \dots, J$ its degeneracy index,
such probability shall sum over it:
$$
\pi_{a} = \sum_{j=1}^{J}\norm{\braket{aj}{\psi}}^2
        = \sum_{j=1}^{J}\bra{\psi}\ket{aj}\bra{aj}\ket{\psi}
        = \mel{\psi}{P_{a}}{\psi}
$$
where $P_{a} = \sum_{j=1}^{J}\ketbra{aj}$
still is the projector on the
($J$-dimensional) eigenspace of $a$.

Generally, the probability that the outcome of a measuremen falls in
the set of eigenvalues $\sigma = \{a_{1}, \dots, a_{S}\}$ is
\begin{equation}\label{eq:pi_sigma}
\pi_{\sigma}  = \sum_{s=1}^{S}\sum_{j=1}^{J_{s}}\norm{\braket{sj}{\psi}}^2
              = \sum_{s=1}^{S}\sum_{j=1}^{J_{s}}\bra{\psi}\ket{sj}\bra{sj}\ket{\psi}
              = \mel{\psi}{P_{\sigma}}{\psi}
              ,
\end{equation}
where $P_{\sigma} = \sum_{s=1}^{S}\sum_{j=1}^{J_{s}}\ketbra{sj}$
is once again a projector --- on the ``generalized eigenspace'' spanned by all
eigenvectors $\{\ket{sj}\}$ above.

\subsection[Measure]{Measure\footnote{Not to be confused with \emph{measurement}.}}

\begin{remark}\label{measure_properties}
  Being $\pi_{\sigma}$ the \emph{probability} of the outcome of a measurement to
  fall in a given set $\sigma$, it has to be:
  \begin{enumerate}
    \item \label{measure_properties:first} $0$ on the empty set
    \item non negative
    \item \label{measure_properties:last} \term{additive} on disjoint sets
    \item equal to $1$ if $\sigma$ includes the whole spectrum of $A$.
  \end{enumerate}
\end{remark}

\begin{remark}
  Properties \ref{measure_properties:first}\dots\ref{measure_properties:last}
  are the defining properties of a \term{measure} \cite{EncMath_Measure}.
\end{remark}

The probability $\pi_{\sigma}$ in \eqref{eq:pi_sigma} is linear with respect to the projector $P_{\sigma}$ hence it's not difficult to derive that the same properties in \autoref{measure_properties} applies to $P_{\sigma}$ \emph{in the operator sense}.