\documentclass{book}

%%%%

%% Theorem-like environments

%% http://www.maths.tcd.ie/~dwilkins/LaTeXPrimer/Theorems.html
%% https://www.sharelatex.com/learn/Theorems_and_proofs
%% http://tex.stackexchange.com/a/46262
\newtheorem{theorem}{Theorem}[chapter]
\newtheorem{lemma}[theorem]{Lemma}
\newtheorem{proposition}[theorem]{Proposition}
\newtheorem{corollary}{Corollary}[theorem]
\newtheorem{remark}{Remark}[chapter]
\newcommand{\remarkautorefname}{Remark}

%% Adapted from https://tex.stackexchange.com/q/45817
\theoremstyle{definition}
\newtheorem{definition}{Definition}[section]
\newtheorem{conjecture}{Conjecture}[section]
\newtheorem{example}{Example}[section]

%%%%

%% Misc

\newcommand{\term}[1]{\emph{#1}}
\newcommand{\idop}{\mathbbm{1}}           % Identity operator
\newcommand{\hilb}[1]{\mathcal{#1}}       % Hilbert space
\newcommand{\setof}[1]{\left\{#1\right\}}
\newcommand{\ox}{\otimes}

%% Allows better formatting than \underset
%% https://tex.stackexchange.com/a/130553
\DeclareMathOperator*{\repr}{\equiv}      % represented in a basis, or "has components..."

\renewcommand{\op}{\hat}                  % overwriting physics \op = \ketbra
%\newcommand{\eqbydef}{\coloneqq}
\newcommand{\eqbydef}{\triangleq}
\newcommand{\superop}{\mathcal}

\newcommand{\dket}[1]{\left.\left| #1 \right\rangle\right\rangle}
\newcommand{\Dket}[1]{\left.\left| #1 \right\rangle\!\right\rangle}
\newcommand{\dbra}[1]{\left\langle\left\langle #1 \right|\right.}
\newcommand{\Dbra}[1]{\left\langle\!\left\langle #1 \right|\right.}
\newcommand{\dbraket}[2]{\left\langle\left\langle #1 \middle| #2 \right\rangle\right.\!}
\newcommand{\bradket}[2]{\!\left.\left\langle #1 \middle| #2 \right\rangle\right\rangle}
\newcommand{\braDket}[2]{\!\left.\left\langle #1 \middle| #2 \right\rangle\!\right\rangle}
\newcommand{\dbradket}[2]{\left\langle\left\langle #1 \middle| #2 \right\rangle\right\rangle}
\newcommand{\dketdbra}[2]{\dket{#1}\dbra{#2}}

\newcommand{\pwspace}{\hilb{H}_T \ox \hilb{H}_S}

\NewEnviron{eqsplit}{\begin{equation}\begin{split}\BODY\end{split}\end{equation}}
\NewEnviron{eqsplit*}{\begin{equation*}\begin{split}\BODY\end{split}\end{equation*}}

% Imaginary unit (not used much this way here though)
% https://tex.stackexchange.com/a/303698
\newcommand{\iu}{\mathrm{i}\mkern1mu}
\newcommand{\E}{\mathrm{e}}

\author{Guido De Rosa \\ \small\tt{gderosa@umail.ucc.ie}}
\title{Time as a quantum observable: from Pauli objection to Page and Wootters model --- improvements and applications}

\usepackage[safeinputenc]{biblatex}
\addbibresource{biblio.bib}

\begin{document}

\maketitle

\tableofcontents

\chapter*{Abstract}
A proof is detailed of the (in)famous Pauli's ``theorem''~\cite{PauliFootnote}
on the impossibility of a time observable in quantum mechanics. Some possible
approaches towards overcoming Pauli's objections are reviewd and expanded as well

\chapter{Motivation and experiments}
\begin{itemize}
  \item Time crystals?
  \item Time-resolved diffraction patters, temporal double slit experiment + Muga theory \url{https://arxiv.org/pdf/0812.3034.pdfß}
\end{itemize}

%\clearpage\includegraphics[width=\linewidth]{img/diffraction.jpg}


\chapter{Pauli's objection --- TODO: make this Historical Introduction?}
\section{Introduction}

In a footnote in~\parencite{PauliFootnote}, W. Pauli excluded the possibility
of a self-adjoint operator representing time as a quantum observable.
However, he did not provide an explicit proof.
Here a proof is given, based on~\parencite{Galapon2002}, but expanded in more detail.
\input{tex/pauli/proof}

\chapter{Approaches}
\section{Approaches}

They are

\begin{itemize}
\item early attempts reviewed in \cite{TQM1, TQM2}, Aharonov-Bohm, Kijowski etc.
\item detector model (Ruschhaupt, Muga) \cite{TQM1, TQM2}
\item
    time ``tensor'' position Hilbert space or ``second'' Schr\"odinger equation (Prvanovic)
\item time and entanglement (Page and Wootters model, Leggett-Garg inequality as \emph{time} version of Bell inequalities, experiments by Moreva, Genovese et al.)
\item approachs where not only spacetime but causality itself is not fundamental (indefinite causal order: Oreshkov, Brukner et al)
\item event-based approaches: 
    \begin{itemize}
        \item ``event'' wavefunction square integrable in 4D (how does it relate rigourously to detector model?)
        \item event-enhanced quantum theory (EEQT, Ruschhaupt et al.)
        \item 
    \end{itemize}
\end{itemize}

\chapter{Entanglement, decoherence, and measurement}
The environment is watching, Haroce-Raimond, from ``Exploring the quantum'', ch.4

Environment-Induced Decoherence and the Transition from Quantum to Classical

Partial trace, which is not a number, check wikipedia $\rightarrow$ consistent histories.
J. P. Paz W. H. Zurek

Von Neumann and Shanon entropy.

MIKIO NAKAHARA: QUANTUM COMPUTING: AN OVERVIEW

\section{Maximally mixed state of $\ket{e_1}$ and $\ket{e_1}$
Vs their maximal coherent superposition: density matrices}

See Nakahara notes, example II.2.

$$
\frac{1}{2}\begin{bmatrix}
  1 &0  \\
  0 &1  \\
\end{bmatrix}
\text{ Vs }
\frac{1}{2}\begin{bmatrix}
  1 &1  \\
  1 &1  \\
\end{bmatrix}
$$

The off-diagonal terms in the coherent superposition case indicate
``quantumness''. (explain)

\begin{remark}
  The density matrix of a pure state is either diag(000\dots1\dots000) or non-diagonal.

  It can be easily proven reasoning on eigenvalues\dots

  So, except the special/trivial case above, a pure state must have off-diagonal terms.
\end{remark}

Measurement destroys off-diagonal terms and turns a pure state into a mixed one.

Off-diagonal terms in the density matrix indicates ``quantum purity''.
(Or internal Entanglement as well, if the system being detected was bipartite itself?)

In some sense, measurement turned a q-statistical distribution of values for the observable,
into a classical probability distribution.

Recommended read: \cite{Zurek_Decoherence, Zurek_Decoherence2}.

\chapter{Critical review of Page and Wootters mechanism and proposed improvements}
\chaptermark{Critical review of Page and Wootters \dots}
\section{Blah blah}

Three papers: \cite{Lloyd:Time}, \cite{Marletto:Evolution}, \cite{Prvanovic}.

And an experiment: \cite{Moreva:synthetic,Moreva:illustration}.

The basic idea is an additional Hilbert space $\mathcal{H}_T$ where time is an observable
corresponding to
a self-adjoint operator whose mathematical properties are the same of position in the
ordinary Hilbert space of a quantum particle in one dimension.

In this language, the ordinary Hilbert space can be labeled $\mathcal{H}_S$;
and we consider the product space $\mathcal{H}_T \otimes \mathcal{H}_S$ as
the space in which both time and position are observables, and they act as
$\hat{t} \otimes \idop_S$ and $\idop_T \otimes \hat{x}$
respectively.

\section{Questions and observations on Lloyd's paper}

With regards to \cite{Lloyd:Time}, section B, \textit{Measurement}.

\begin{enumerate}
  \item What is a memory system in quantum mechanics? 
  \begin{itemize}
    \item Do they mean a non-Markovian system? Look at \cite{MeasurementMarkovian}\dots
    \item Memory in the sense of the Maxwell daemon?
    \item
      Any general theory of quantum memory systems? Fiducial states?
      Yes, see \ref{sec:qmemory} or most importantly \cite[Ch.~3]{PreskillMeasurement}.
    \item memory kernels \cite{CarmichaelOQS2017}
  \end{itemize}
  \item ``\emph{the case where a measurement is performed at time t1}'' suggests they are back to time as an external parameter\dots
\end{enumerate}

\section{Memory systems (and fiducial state?)}\label{sec:qmemory}
Besides, and perhaps better than the below, see \cite[Ch.~3]{PreskillMeasurement},
which cover and expands topics discussed in \cite{open_systems}. Chapter 2
of Preskill \cite{PreskillMeasurement} also is insightful, paeritcularly
subsection 2.3.1 The bipartite quantum system --- ancillary systems. Chapter 3
also defines Kraus operators. We believe that \cite{Lloyd:Time} should
have cited \cite{PreskillMeasurement} --- although they're just \emph{lecture notes}.

\clearpage\includegraphics[width=\linewidth]{img/pw/qmem/1.jpg}
\clearpage\includegraphics[width=\linewidth]{img/pw/qmem/2.jpg}
\clearpage\includegraphics[width=\linewidth]{img/pw/qmem/3.jpg}
\clearpage\includegraphics[width=\linewidth]{img/pw/qmem/4.jpg}

\clearpage\includegraphics[width=\linewidth]{img/pw/qmem/nonmarkov.jpg}

\section{Entanglement and decoherence}
See also \cite{EntanglementVsDecoherence}.

Decoherence is an irreversible process, it also happens in measurement.

According to Marletto and Vedral, arrow of time is increase in Entanglement
between the clock and the rest.

So, there seems to be a contradiction: is entanglement ``decreasing''
(i.e. destroyed by decoherence) with time
or increasing?

Perhaps we can avoid the contradiction saying that
entanglement between two finite systems is
destroyed while the entanglement of each of them with the universe
is increasing?

Or in other words, the entanglement of two clocks with the rest of universe
is increasing at the expenses of the entanglement between them?

And maybe we can conclude that two systems  of which we successfully
protect the entanglement from decoherence are not good clocks?

Or perhaps there's no relation\dots
\section{In relation to Leggett-Garg inequalities (TODO: move to another file)}
Ref \cite{LeggettGarg+PageWootters}.

But also Lloyd: \url{https://arxiv.org/abs/1608.05672},
\emph{Decoherent histories approach to the cosmological measure problem}.
Lindbladt, Markov, Open Systems.

Halliwell, \url{https://arxiv.org/abs/1604.01659}. Ancilla, decay (spontaneous emission?).

A phylosophical object to decoherent histories / Everett (Everett mentioned in Marletto/ref)
is at
\url{https://arxiv.org/abs/1603.04845}.



\chapter{Reconciling the Detector Model with the Page and Wootters mechanism?}
\section{Can a POVM on a system, if then we see it as part of a bipartite one,
equivalent to a PVM on the other, entangled, system?}
This will show an equivalence of models based on POVM with the Page and Wootters...?

Not exactly, but a POVM on a subsystems corresponds to a PVM on the tensor product space.

From \cite{PreskillNotes}, Ch.3 
\begin{quotation}
We have seen that
a pure state of the bipartite system AB may behave like a mixed state
when we observe subsystem A alone, and that an orthogonal measurement
of the bipartite system can realize a (nonorthogonal) POVM on A alone.
\end{quotation}

The same chapter talks about quantum operations, quantum channels and Kraus opertators.

We might want to look at exponential decay from \url{https://arxiv.org/abs/1704.07236},
then compare with exponential decay with P and W using Lloyd Giovannetti and Maccone (ref).

\subsection{4-partite universe?}
\begin{itemize}
  \item{The system being measured/detected}
  \item{The Ruschhaupt detector --- which does not measure time, but whose detection happens at a certain time}
  \item{The Page and Wootters clock, entangled with the system and/or the detector}
  \item{The rest of the Universe, aka the Environment, aka the Termal Bath or Reservoir}
\end{itemize}

Can any of the above be identified? If the lab is isolated enough,
the detector is the only macro object and can act as a Universe/bath/environment/reservoir\dots?

\chapter{When Time Crystals come into play}
\section{Intro}

References: \cite{crystal2,crystal3,crystal2012}.


\chapter{Thermal time hypothesis}
\section{Connes and Rovelli}

Reference \parencite{ConnesRovelliThermo}.

\section{More}

Relate with John Goold's works? The ancilla as a clock? --- Topical Review

Markovianity, histories.

Lloyd on arXiv: from clock to cloners; erasing; scrambling (as in Goold).

Lloyd on decoherent histories (Gellman, Hartle?).

Dechoerence / irreversibility / measurement.

Vedral / Lloyd. Discord.

Measuring entanglement: Quantification of Concurrence via Weak Measurement: 1611.00149.

Marletto/Vedral on Arrow of time. Arrow of time as increasing entanglement.

Arrow of time: 

\url{https://www.wired.com/2014/04/quantum-theory-flow-time/}

\url{https://en.wikipedia.org/wiki/Loschmidt%27s_paradox}

\url{https://www.quantamagazine.org/20160119-time-entanglement/}

\section{Misc}

\url{https://arxiv.org/pdf/1702.07706.pdf} \textit{The second law of thermodynamics at the microscopic scale}
Thibaut Josset,
Aix Marseille Univ. (David).

Maxwell's demon: https://arxiv.org/pdf/1702.05161.pdf

\chapter{Gravitation and High Energy}
\section{Feynmann path stuff}

Sokolovski 1703.01966, Feynmann paths...
\section{(dis)entanglement under gravity, decoherence, event formalism}

1703.08036 An experiment to test decoherence under gravity aka entangled photons undergoing different paths and how their entanglement is affected.

Are they getting entangled with the environment instead? (Merletto and Vedral).

The theoretical paper behind the space experiment: \url{https://arxiv.org/pdf/1406.3677.pdf}. Interestingly, it mentions 
\emph{event formalism}, and we thought about that: is an event something
representable as a proper vector in $\mathcal{L}^2(\mathbb{R}^4)$ --- where one of the dimensions is time?
TODO: deepen the event formalism if it's quantum.

Closed timelike curves are also the subject of a paper by Lloyd (cite!).

``Deutsch argued that
the usual paradoxes associated with such solutions of general
relativity can be resolved by quantum mechanics''



\chapter{Extras}
\section{An intro chap on motivation and experiments?}
\begin{itemize}
  \item Time crystals?
  \item Time-resolved diffraction patters, temporal double slit experiment + Muga theory \url{https://arxiv.org/pdf/0812.3034.pdfß}
\end{itemize}

%\clearpage\includegraphics[width=\linewidth]{img/diffraction.jpg}


\section{Time crystals}

References: \cite{crystal2,crystal3,crystal2012}.


\section{Irrev}
\subsection{Use open quantum systems theory in ``Decoherence and measurement'' chapter}
Schmidt decompositions, spatial and temporal states in
$\hilb{H}_T$ and $\hilb{H}_S$
are described as density operators
(mixed states). What if there isn't a ``perfect entanglment'' between space and time.

\section{Entanglement and decoherence (Arrow of time)}
See also \cite{EntanglementVsDecoherence}.

Decoherence is an irreversible process, it also happens in measurement.

According to Marletto and Vedral, arrow of time is increase in Entanglement
between the clock and the rest.

So, there seems to be a contradiction: is entanglement ``decreasing''
(i.e. destroyed by decoherence) with time
or increasing?

We can avoid the contradiction saying that
entanglement between two finite systems is
destroyed while the entanglement of each of them with the universe
is increasing.

\subsection{``Harmonic clocks''}

TODO: use the harmonic oscillator in \cite{HarmonicClocks}
as a PaW clock for the same packet that is measured in
Ruschhaupt's detector model.

Therein, fading wave function: is minus derivative an event?
L4 normalized?


\subsection{Misc}

Idea: use section B ``Measurement'' of \cite{Lloyd:Time}: detector as (binary) measument device.

``Philospher'': \url{https://arxiv.org/abs/1704.07236}.

Time of arrival and clocks: again, \cite{YearsleyHalliwell_Clocks}.
Which maybe suggests we should not wory too much of $H\ket{\Psi} = 0$. 

We don't. 

BUT please note \cite{YearsleyHalliwell_Clocks} uses a clock that is
\emph{coupled} with the system, while in PaW they are ``only'' entangled.
So their calculation may be unnecessarily complicated.
Maybe the weakjly coupling case can be used?

Other systems of interest: decays. Prvanovic new.

Reference \cite{ConnesRovelliThermo}.

Relate with John Goold's works? The ancilla as a clock? --- Topical Review

Markovianity, histories.

Lloyd on arXiv: from clock to cloners; erasing; scrambling (as in Goold).

Lloyd on decoherent histories (Gellman, Hartle?).

Dechoerence / irreversibility / measurement.

Vedral / Lloyd. Discord.

Measuring entanglement: Quantification of Concurrence via Weak Measurement: 1611.00149.

Marletto/Vedral on Arrow of time. Arrow of time as increasing entanglement.

Arrow of time: 

\url{https://www.wired.com/2014/04/quantum-theory-flow-time/}

\url{https://en.wikipedia.org/wiki/Loschmidt%27s_paradox}

\url{https://www.quantamagazine.org/20160119-time-entanglement/}

\url{https://arxiv.org/pdf/1702.07706.pdf} \textit{The second law of thermodynamics at the microscopic scale}
Thibaut Josset,
Aix Marseille Univ. (David).

Maxwell's demon: https://arxiv.org/pdf/1702.05161.pdf

\subsection{and paths}

Both \cite{YearsleyHalliwell_Clocks} and \cite{Gambini_PW}
reason in terms of paths and actions, maybe Feynmann stuff
in following chapter... and maybe conistent historiesapproach can help
towards linking PaW and ToA?

Also \url{http://quantum.phys.cmu.edu/CHS/CHS_transp.pdf}.

\subsection{Decays?}

We might want to look at exponential decay from \url{https://arxiv.org/abs/1704.07236},
then compare with exponential decay with P and W using Lloyd Giovannetti and Maccone (ref).

\subsubsection{Purification}

See https://arxiv.org/pdf/quant-ph/0512125.pdf, P-W time as a purifying ancilla
of the (Kijowski?) time.

\section{Misc/Multi/Extras/TODO/Outlook}

\url{https://arxiv.org/abs/1703.05876}
--- \emph{comment}: time measured and stored here
may be all classical information
so this paper may or may not be relevant for the topic.

But
``prototypes of clocks based on quantum principles,
such as entanglement and squeezing''
may make this interesting again, see reference therein.
They also cite Lloyd, Giovannetti and Maccone,
but a paper quite older than \cite{Lloyd:Time}.

\url{https://arxiv.org/abs/1603.02522}
\emph{Decoherence by spontaneous emission: a single-atom analog of superradiance}.
Decoherent histories, non-markovianity, open quantum systems.

\url{https://arxiv.org/abs/1007.2615} Time travel / Quantum CTC.

Carmichael et al. \cite{CarmichaelOQS2017} (Andreas's reading)
(non-markovianity).

Non-markovian, quantum-to-classical, open systems, David,
\url{https://arxiv.org/pdf/1703.09428.pdf}.

In his works, Zurek mentions:
DeWitt, Everett, gell_Mann, hartl, Many Worlds, consistent/decoherent histories:
idea: Lagrangian over a history? Principle of least action?

Zurek: ``Reduction of the Wavepacket: How Long Does it Take?'' (arxiv),
``quantum''' time? \cite{Zurek_Einselect} also mentions
``decoherence timescale''.

Von Neumann/Shannon entropy in measurement? Mention information problems
in quantum cosmology (where a quantum time is necessary)? Etc. etc.


\chapter{Pedagogical remarks (TODO: move to Intro? move to Conclusions?)}
A computation/information approach for a better understanding of quantum
physics.\footnote{
  Namely, Preskill notes \cite{PreskillNotes} as a better way of looking at QM,
  especially \textbf{measurement}.
}

\appendix
\chapter{Commutator properties}
\section{Commutator properties}
\subsection{Power}
\begin{lemma}\label{CommProp}
If the commutator $[T, H]$ commutes with $T$ i.e.
$$[T, H]T~=~T[T, H]\,,$$ then the following holds:
\begin{equation}\label{eq:tkh}
[T^k, H] = kT^{k-1}[T, H]\,.
\end{equation}
\end{lemma}
This is particularly true when $[T, H]$ is a \emph{number} as in \eqref{THcommutator} where
$T$ and $H$ are the time and energy operator respectively.
\begin{proof}
First of all, the \eqref{eq:tkh} is trivially valid for $k = 1$.

For an arbitrary positive integer $k$ there has:
\begin{dmath}\label{tkhrecur}
[T^k, H] = T^{k-1}TH - HT^{k-1}T = T^{k-1}TH - T^{k-1}HT + T^{k-1}HT - HT^{k-1}T \\
    = T^{k-1}[T, H] + [T^{k-1}, H]T
\end{dmath}
Now, iterating the result in \eqref{tkhrecur},
\begin{dmath}\label{tkhrecurplus}
[T^k, H] = T^{k-1}[T, H] + [T^{k-1}, H]T
= T^{k-1}[T, H] + (T^{k-2}[T, H] + [T^{k-2}, H]T)T
= T^{k-1}[T, H] +  T^{k-1}[T, H] + [T^{k-2}, H]T^2
= 2T^{k-1}[T, H] + [T^{k-2}, H]T^2
= \hdots
= nT^{k-1}[T, H] + [T^{k-n}, H]T^n = \hdots
\end{dmath}
where the commutativity hypothesis $[T, H]T = T[T, H]$ has been used to obtain $T^{k-2}[T, H]T = T^{k-1}[T, H]$.

Now, \eqref{tkhrecurplus} can be continued until it reaches $n=k$ when the term
$[T^{k-n}, H]T^n$ vanishes and a result of $kT^{k-1}[T, H]$ follows.
\end{proof}

\chapter{Density operators, projectors and measurement}
\section{Projectors}

We shall assume, unless otherwise stated, that all vectors are
normalized to unity.

For a given observable represented by an operator $A$ with a
\emph{discrete, non degenerate spectrum}, the probability of the
outcome $a$ of a
measurement on a system in the pure state $\ket{\psi}$ is given by
$$
\pi_{a} = \norm{\braket{a}{\psi}}^2
        = \bra{\psi}\ket{a}\bra{a}\ket{\psi}
        = \mel{\psi}{P_{a}}{\psi}
$$
where $P_{a} = \ketbra{a}$ appears as the \term{projector} on the
one-dimensional eigenspace of  $a$.

If $a$ is degenerate, and $j = 1, \dots, J$ its degeneracy index,
such probability shall sum over it:
$$
\pi_{a} = \sum_{j=1}^{J}\norm{\braket{aj}{\psi}}^2
        = \sum_{j=1}^{J}\bra{\psi}\ket{aj}\bra{aj}\ket{\psi}
        = \mel{\psi}{P_{a}}{\psi}
$$
where $P_{a} = \sum_{j=1}^{J}\ketbra{aj}$
still is the projector on the
($J$-dimensional) eigenspace of $a$.

Generally, the probability that the outcome of a measuremen falls in
the set of eigenvalues $\sigma = \{a_{1}, \dots, a_{S}\}$ is
\begin{equation}\label{eq:pi_sigma}
\pi_{\sigma}  = \sum_{s=1}^{S}\sum_{j=1}^{J_{s}}\norm{\braket{sj}{\psi}}^2
              = \sum_{s=1}^{S}\sum_{j=1}^{J_{s}}\bra{\psi}\ket{sj}\bra{sj}\ket{\psi}
              = \mel{\psi}{P_{\sigma}}{\psi}
              ,
\end{equation}
where $P_{\sigma} = \sum_{s=1}^{S}\sum_{j=1}^{J_{s}}\ketbra{sj}$
is once again a projector --- on the ``generalized eigenspace'' spanned by all
eigenvectors $\{\ket{sj}\}$ above.

\section[Measure]{Measure\footnote{Not to be confused with \emph{measurement}.}}

\begin{remark}\label{measure_properties}
  Being $\pi_{\sigma}$ the \emph{probability} of the outcome of a measurement to
  fall in a given set $\sigma$, it has to be:
  \begin{enumerate}
    \item \label{measure_properties:first} $0$ on the empty set
    \item non negative
    \item \label{measure_properties:last} \term{additive} on disjoint sets
    \item equal to $1$ if $\sigma$ includes the whole spectrum of $A$.
  \end{enumerate}
\end{remark}

\begin{remark}
  Properties \ref{measure_properties:first}\dots\ref{measure_properties:last}
  are the defining properties of a \term{measure} \parencite{EncMath_Measure}.
\end{remark}

The probability $\pi_{\sigma}$ in \eqref{eq:pi_sigma} is linear with respect to the projector
$P_{\sigma}$ hence it's not difficult to derive that the same properties in
\autoref{measure_properties} applies to $P_{\sigma}$, \emph{in the operator sense}.
In fact, the map $\sigma \subseteq \mathbb{R} \rightarrow P_{\sigma}$
implicitly defined in~\eqref{eq:pi_sigma} is a \term{projector-valued measure}.

The result is generalized,
in such a way to include both discrete and continous spectra,
by the following \cite{VonNeumann, Ballentine}
\begin{theorem}[spectral resolution]
  If $A$ is a self-adjoint operator,
  there is a unique projector-valued measure $E$
  defined on the Borel sets of $\mathbb{R}$
  such that
  \footnote{
    In \eqref{eq:spectral}, $a$ is a real number (not a set),
    but it's intended $E$ to be evaluated
    on the~\emph{interval}~from $-\infty$ to $a$.
  }
  \begin{equation}\label{eq:spectral}
    A=\int_{-\infty}^{\infty}a\, dE(a)
  \end{equation}
  and satisfying:
  \begin{align*}
    E(\mathbb{R})       & =\mathbf{1},\\
    E(B_{1}\cap B_{2}) & =E(B_{1})E(B_{2})\,.
  \end{align*}
\end{theorem}

In terms of this theorem, the projector in \eqref{eq:pi_sigma} is
\begin{equation}\label{eq:P_sigma_spectral}
  P_{\sigma} = E(\sigma) = \int_{a\in\sigma}dE(a)
\end{equation}
and $dE(a)$ is
---informally speaking---
infinitesimal if $a$ belongs to the continous spectrum,
finite if $a$ is a discrete eigenvalue
and zero otherwise.

\section{Density operator}\label{app:density}

The above results apply to \term{pure states}.
More generally, a quantum state (either pure or \term{mixed}),
is described by a \term{density} (or \term{state}) operator $\rho$.
It's a well known result (see, for example, \cite{open_systems})
that the expectation value of an observable represented by the operator $A$
is given in this description by
\begin{equation}\label{eq:expvalA_rho}
  \expval{A} = \tr(A\rho)\,.
\end{equation}

The \eqref{eq:expvalA_rho} is valid for any hermitean operator $A$,
hence we can replace it with the projector $P_{\sigma}$,
associated to the set of eigenvalues $\sigma$
according to \eqref{eq:pi_sigma} and \eqref{eq:P_sigma_spectral},
which is of particular interest
because its mean value equates the probability that the outcome of a measurement
falls in a given subset of the spectrum of a given observable.
Hence we can derive:

\begin{proposition}\label{probability_rho}
  The probability $\pi_{\sigma}$
  that the outcome of a measurement of the observable $A$
  on the system described by the state operator $\rho$
  falls in the set $\sigma$
  is given by:
  \begin{equation}\label{eq:probability_rho}
    \pi_{\sigma} = \expval{P_{\sigma}} = \tr(P_{\sigma}\rho)
  \end{equation}
  with $P_{\sigma}$ defined as in \eqref{eq:pi_sigma}
  or, more generally, in \eqref{eq:P_sigma_spectral}.
\end{proposition}

\subsection{Properties of the density operator}
\subsubsection{Additivity}
The general expression for a density operator $\rho$ is
$$
  \rho = \sum_{k}p_{k}\ketbra{\psi_{k}}
$$
with $p_{k}$ non negative and $\sum_{k}p_{k} = 1$.

The condition is not restrictive to prevent proving the following
\begin{proposition}
  Let $A$ be an hermitean operator such that
  $$
    \tr(A\rho) = 0 
  $$
  \emph{for any density operator} $\rho$.

  It follows that $A = 0$.

  \begin{proof}
    We can choose,
    as basis to explicit the expresion of the trace,
    a complete set of eigenvectors $\{\ket{m}\}$ of $A$ and,
    for each positive integer $n$,
    a particular density operator $\rho = \ketbra{n}$,
    corresponding to a (pure) eigenstate of $A$.

    With this particular choice,
    for each $n$,
    $$
      A\rho = \mel{n}{A}{n}\ketbra{n}\,.
    $$
    It then follows:
    $$
      0 = \tr(A\rho) = \sum_{m}\mel{m}{ ( \mel{n}{A}{n}\ketbra{n} ) }{m}
        = \sum_{m} \mel{n}{A}{n} \braket{m}{n} \braket{n}{m}
        = \mel{n}{A}{n}\,,
    $$
    hence the generic diagonal element of the (diagonal) matrix of $A$ is null,
    and so is consequently the operator $A$ itself.
  \end{proof}
\end{proposition}

\begin{corollary}
If $\tr(A\rho) + \tr(B\rho) = \tr(C\rho)$ for any density operator $\rho$,
then $A + B = C$.
\end{corollary}


\printbibliography[title=References]

\end{document}