\documentclass{book}

\usepackage{xcolor}
\usepackage[
  colorlinks,
  citecolor=green!75!black,
  linkcolor=red!75!black,
  urlcolor=magenta!75!black
]{hyperref}
\usepackage{amsmath}
\usepackage{amssymb}
\usepackage{amsthm}
\usepackage{physics}
\usepackage{breqn}
\usepackage{bbm}
\usepackage[utf8]{inputenc}
\usepackage{graphicx}
\usepackage[margin=1.5in]{geometry}
\usepackage[safeinputenc,authordate,backend=biber]{biblatex-chicago}

%%%%

%% Theorem-like environments

%% http://www.maths.tcd.ie/~dwilkins/LaTeXPrimer/Theorems.html
%% https://www.sharelatex.com/learn/Theorems_and_proofs
%% http://tex.stackexchange.com/a/46262
\newtheorem{theorem}{Theorem}[chapter]
\newtheorem{lemma}[theorem]{Lemma}
\newtheorem{proposition}[theorem]{Proposition}
\newtheorem{corollary}{Corollary}[theorem]
\newtheorem{remark}{Remark}[chapter]
\newcommand{\remarkautorefname}{Remark}

%% Adapted from https://tex.stackexchange.com/q/45817
\theoremstyle{definition}
\newtheorem{definition}{Definition}[section]
\newtheorem{conjecture}{Conjecture}[section]
\newtheorem{example}{Example}[section]

%%%%

%% Misc

\newcommand{\term}[1]{\emph{#1}}
\newcommand{\idop}{\mathbbm{1}}           % Identity operator
\newcommand{\hilb}[1]{\mathcal{#1}}       % Hilbert space
\newcommand{\setof}[1]{\left\{#1\right\}}
\newcommand{\ox}{\otimes}

%% Allows better formatting than \underset
%% https://tex.stackexchange.com/a/130553
\DeclareMathOperator*{\repr}{\equiv}      % represented in a basis, or "has components..."

\renewcommand{\op}{\hat}                  % overwriting physics \op = \ketbra
%\newcommand{\eqbydef}{\coloneqq}
\newcommand{\eqbydef}{\triangleq}
\newcommand{\superop}{\mathcal}

\newcommand{\dket}[1]{\left.\left| #1 \right\rangle\right\rangle}
\newcommand{\Dket}[1]{\left.\left| #1 \right\rangle\!\right\rangle}
\newcommand{\dbra}[1]{\left\langle\left\langle #1 \right|\right.}
\newcommand{\Dbra}[1]{\left\langle\!\left\langle #1 \right|\right.}
\newcommand{\dbraket}[2]{\left\langle\left\langle #1 \middle| #2 \right\rangle\right.\!}
\newcommand{\bradket}[2]{\!\left.\left\langle #1 \middle| #2 \right\rangle\right\rangle}
\newcommand{\braDket}[2]{\!\left.\left\langle #1 \middle| #2 \right\rangle\!\right\rangle}
\newcommand{\dbradket}[2]{\left\langle\left\langle #1 \middle| #2 \right\rangle\right\rangle}
\newcommand{\dketdbra}[2]{\dket{#1}\dbra{#2}}

\newcommand{\pwspace}{\hilb{H}_T \ox \hilb{H}_S}

\NewEnviron{eqsplit}{\begin{equation}\begin{split}\BODY\end{split}\end{equation}}
\NewEnviron{eqsplit*}{\begin{equation*}\begin{split}\BODY\end{split}\end{equation*}}

% Imaginary unit (not used much this way here though)
% https://tex.stackexchange.com/a/303698
\newcommand{\iu}{\mathrm{i}\mkern1mu}
\newcommand{\E}{\mathrm{e}}

\author{Guido De Rosa \\ \small\tt{gderosa@umail.ucc.ie}}
\title{Time as a quantum observable: from Pauli objection to Page and Wootters model --- improvements and applications}

\addbibresource{biblio.bib}


\begin{document}

\maketitle

\tableofcontents

% main content
\chapter*{Abstract}
A proof is detailed of the (in)famous Pauli's ``theorem''~\parencite{PauliFootnote}
on the impossibility of a time observable in quantum mechanics. Some possible
approaches towards overcoming Pauli's objections are reviewd and expanded as well


\chapter{Motivation and experiments}
\begin{itemize}
  \item Time crystals?
  \item Time-resolved diffraction patters, temporal double slit experiment + Muga theory \url{https://arxiv.org/pdf/0812.3034.pdfß}
\end{itemize}

%\clearpage\includegraphics[width=\linewidth]{img/diffraction.jpg}


\chapter{Pauli's objection and historical developments}
\input{tex/hist/hist}

\chapter{Approaches}
\section{Approaches}

They are

\begin{itemize}
\item early attempts reviewed in \cite{TQM1, TQM2}, Aharonov-Bohm, Kijowski etc.
\item detector model (Ruschhaupt, Muga) \cite{TQM1, TQM2}
\item
    time ``tensor'' position Hilbert space or ``second'' Schr\"odinger equation (Prvanovic)
\item time and entanglement (Page and Wootters model, Leggett-Garg inequality as \emph{time} version of Bell inequalities, experiments by Moreva, Genovese et al.)
\item approachs where not only spacetime but causality itself is not fundamental (indefinite causal order: Oreshkov, Brukner et al)
\item event-based approaches: 
    \begin{itemize}
        \item ``event'' wavefunction square integrable in 4D (how does it relate rigourously to detector model?)
        \item event-enhanced quantum theory (EEQT, Ruschhaupt et al.)
        \item 
    \end{itemize}
\end{itemize}

\chapter{Decoherence and Measurement}
\section{Bipartite systems, separable and entangled states}

Consider a \term{bipartite quantum system}
i.e. a composite system $S$
made up of two parts, $A$ and~$B$,
described by their respective Hilbert spaces
$\hilb{H}_A$ and $\hilb{H}_B$.

In \cite{Haroche_Exploring},
a basis for $\hilb{H}_A$ and a basis for $\hilb{H}_B$
are indicated as
$$
  \setof{\ket{i_A}} \text{ and } \setof{\ket{\mu_B}}\,
$$
with Latin and Greek indices to label basis vectors in the two spaces.
This no-\\tation % \footnote{We will partly follow this convention.}
in\-di\-cates potentially different physical roles for the two spaces,
such as the \emph{system of interest} and the surrounding \emph{environment},
or the \emph{system being measured} and the \emph{measurement apparatus}.

If the two subsystems are prepared independently and do not interact with each other,
the system is in a \term{product state} or \term{separable state}:
\begin{equation}\label{eq:separableAB}
  \ket{\psi_S} = \ket{\psi_A}\otimes\ket{\psi_B} \,\text{.}
\end{equation}

It's worth recalling that in a tensor product space $\hilb{H}_A\otimes\hilb{H}_B$
not all vectors can be expressed as a tensor product as in \eqref{eq:separableAB}.
However, the following holds:

\begin{proposition}\label{TensorBase}
The set of tensor products of basis vectors of $\hilb{H}_A$ and $\hilb{H}_B$,
i.e. $$\setof{\ket{i} \otimes \ket{\mu}}_{i\mu},$$
is a basis for $\hilb{H}_A \otimes \hilb{H}_B$.
\end{proposition}
Therefore, we can express any
(generally not separable) state vector of $\hilb{H}_S$
as a \emph{superposition} of separable basis vectors
\begin{equation}\label{eq:bipartite_expansion}
  \ket{\psi_S} = \sum_{i, \mu}\alpha_{i\mu}\ket{i}\otimes\ket{\mu} \text{.}
\end{equation}

\begin{definition}
  Non separable states are defined as \term{entangled}.
\end{definition}

Physically,
$\ket{\psi_S}$ contains information
not only about the results of measurements on $A$ and $B$ separately,
but also on correlations between these measurements.

The simplest example of entangled system is given by two two-level systems,
namely two spin-$\frac{1}{2}$ particles in a singlet state:
\[
  \ket{\psi_{\text{singlet}}} = \frac{1}{\sqrt{2}} \left(\ket{\uparrow, \downarrow} - \ket{\downarrow, \uparrow}\right).
\]

\section{Partial trace and open systems}
\label{sec:p_tr}

In a bipartite system, one subsystem considered alone is an
\emph{open quantum system},
while the system as a whole is still a closed systems.

In this and the following sections, unless differently noted,
we shall stick with the convention of \eqref{eq:bipartite_expansion},
of using
Latin indices for subsystem $A$ and Greek ones for subsystem $B$.

Now, let's consider an observable $M_A$, acting only on subsystems $A$.
It can be expressed in $\hilb{H}_A \otimes \hilb{H}_B$ as
\[
  M_A \otimes \idop_B\, .
\]
Its expectation value is
(using the expansion in eq. \eqref{eq:bipartite_expansion},
and the notation of Latin indices for system $A$ and Greek ones for system $B$)
\begin{multline}\label{eq:exp_subsys}
  \expval{M_A} = \matrixel{\psi_S}{M_A\otimes\idop_B}{\psi_S} = \\[1em]
  \sum_{j,\nu}a^{*}_{j\nu}\left(\bra{j}\otimes\bra{\nu}\right)\left(M_A\otimes\idop_B\right)\sum_{i,\mu}a_{i\mu}\left(\ket{i}\otimes\ket{\mu}\right) = \\
  \sum_{j,\nu,i,\mu}a^{*}_{j\nu}a_{i\mu}\matrixel{j}{M}{i}\matrixel{\mu}{\idop}{\nu} =
  \sum_{j,\mu,i}a^{*}_{j\mu}a_{i\mu}\matrixel{j}{M}{i}
\end{multline}

The bra $\bra{\mu}$, when acting on a ket element of $\hilb{H}_{A} \ox \hilb{H}_{B}$,
can be defined
as a map from $\hilb{H}_A\otimes\hilb{H}_B$ to $\hilb{H}_A$.

A formal definition can be given in terms of how it acts upon a generic
basis ket of $\hilb{H}_A\otimes\hilb{H}_B$.

\begin{definition}\label{def:pBra}
\[
  \braket{\mu}{i\nu} = \bra{\mu}\Big(\ket{i}\otimes\ket{\nu}\Big) \eqbydef \delta_{\mu\nu}\ket{i} \text{.}
\]
\end{definition}

Intuitively, $\bra{\mu}$ is then a ``partial bra''
as it only maps the ``$\ket{\nu}$'' part of\\
``$\ket{i}\otimes\ket{\nu}$''
to a number:
\[
  \begin{array}{cccc}
    \bra{\mu}:  & \ket{i}\ox\ket{\nu}                   & \rightarrow & \delta_{\mu\nu}\ket{i}                \\
                & \rotatebox[origin=c]{270}{$\in$}      &             & \rotatebox[origin=c]{270}{$\in$}      \\
                & \hilb{H}_A\ox\hilb{H}_B               &             & \hilb{H}_A   \text{.}
  \end{array}
\]

This is consistent with the idea of
``tracing out the environment'' within an interpretation where
$\hilb{H}_B$ is the environment, and ultimately with the goal of
studying subsystem $A$ alone in spite of its entanglement with the 
``environment'' (or measurement apparatus etc.) modelled by
subsystem $B$.

Conversely,
the ket $\ket{\mu}$, when acting on a bra element of $\hilb{H}_{A} \ox \hilb{H}_{B}$,
can also be defined
as a map from $\hilb{H}_A\otimes\hilb{H}_B$ to $\hilb{H}_A$.
The following definition of $\ket{\mu}$ is simply \term{dual} to Definition \ref{def:pBra}:
\begin{definition}\label{def:pKet}
  \[
    \braket{i\nu}{\mu} = \Big(\bra{i}\ox\bra{\nu}\Big)\ket{\mu} \eqbydef \delta_{\mu\nu}\bra{i} \text{.}
  \]
\end{definition}
Similar considerations apply:
\[
  \begin{array}{cccc}
    \ket{\mu}:  & \bra{i}\ox\bra{\nu}                   & \rightarrow & \delta_{\mu\nu}\bra{i}               \\
                & \rotatebox[origin=c]{270}{$\in$}      &             & \rotatebox[origin=c]{270}{$\in$}      \\
                & \hilb{H}_A\ox\hilb{H}_B               &             & \hilb{H}_A     \text{.}
  \end{array}
\]

%%

We are now in the position to introduce the \term{partial trace}:
\begin{definition}\label{def:pTr}
  If $\setof{\ket{\mu}}$ is a basis of $\hilb{H}_B$,
  the \term{partial trace}
  is a linear map
  that takes an operator
  $M_{AB}$ on $\hilb{H}_A \otimes \hilb{H}_B$
  to an operator on $\hilb{H}_A$ defined as
  \[
    \tr_B M_{AB} = \sum_{\mu} \matrixel{\mu}{M_{AB}}{\mu}
    \, \text{.}
  \]
\end{definition}

The first thing to note is that the \emph{partial} trace is \emph{operator-valued},
its value is not a scalar as opposed to the \emph{trace}.

Next, we can introduce the \term{reduced density operator}
of system $A$, which is obtained by \emph{tracing out the environment} B,
via the partial trace:
\begin{definition}\label{def:density_A}
  If $\rho$ is the density operator of a whole bipartite system, composed of two parts $A$ and $B$,
  the \term{reduced density operator} related to subsystem $A$ is defined as
  \[
    \rho_A = \tr_B(\rho) \, \text{.}
  \]
\end{definition}

Using the definitions \ref{def:pBra} and \ref{def:pKet}
and the expansion \eqref{eq:bipartite_expansion}, we obtain
\begin{eqsplit}\label{eq:psiPartial}
  \braket{\mu}{\psi_S} &= \sum_i \alpha_{i\mu}    \ket{i} \\
  \braket{\psi_S}{\mu} &= \sum_j \alpha_{j\mu}^{*}\bra{j} \ \text{.}
\end{eqsplit}

which allows us to expand the Definition \ref{def:density_A}
as\footnote{
  Let's recall that the key point of this reasoning is showing how
  a pure state of the total system can correspond to mixed states
  in the two subsystems $A$ and $B$. Therefore, assuming, in what follows,
  that the total state $\rho = \ketbra{\psi}$ is pure, is not restrictive.
}

\begin{equation}\label{eq:density_A_expand}
  \rho_A = \tr_B(\ketbra{\psi_S}{\psi_S}) =
    \sum_{\mu}\braket{\mu}{\psi_S}\braket{\psi_S}{\mu} =
    \sum_{ij\mu} \alpha_{i\mu}\alpha_{j\mu}^{*}\ketbra{i}{j} \text{.}
\end{equation}

Finally, we can see that
\begin{multline*}
  \tr(M_A \rho_A) = \sum_k \mel{k}{M_A \rho_A}{k} =
    \sum_k \mel {k} {M_A \left(\sum_{ij\mu} \alpha_{i\mu}\alpha_{j\mu}^{*}\ketbra{i}{j}\right)} {k} = \\
    \sum_{ijk\mu} \alpha_{i\mu}\alpha_{j\mu}^{*} \mel{k}{M_A}{i} \braket{j}{k} =
    \sum_{ij\mu} \alpha_{i\mu}\alpha_{j\mu}^{*} \mel{j}{M_A}{i}
    \, \text{,}
\end{multline*}
which is the same result as \eqref{eq:exp_subsys}.

This allows us to conclude that
\begin{proposition}
  For an observable $M_A$, acting solely on one subsystem, the expectation value
  for state $\rho$ is given by
  \begin{equation}
    \expval{M_A} = \tr(M_A \rho_A)
  \end{equation}
  where $\rho_A$ is the reduced density operator obtained from $\rho$ by
  tracing out the environment $B$ (as per Definition \ref{def:density_A}).
\end{proposition}

It's clear from \eqref{eq:density_A_expand} that $\rho_A$ is self-adjoint,
hence it can be diagonalized:
\begin{equation}\label{eq:rho_diag}
  \rho_A = \sum_a p_a \ketbra{a}{a} \text{.}
\end{equation}

From \eqref{eq:density_A_expand} we can also derive that
\begin{equation}
  \tr\rho_A = \sum_k \mel{k}{\rho_A}{k} =
    \sum_{ijk\mu}\alpha_{i\mu}\alpha_{j\mu}^{*}\braket{k}{i}\braket{j}{k} =
    \sum_{k\mu}\abs{\alpha_{k\mu}}^2 = 1
\end{equation}
where the last equality is justified by the normalization of $\ket{\psi_S}$
in \eqref{eq:bipartite_expansion}.

The trace, i.e. the sum of matrix diagonal elements, is independent
from the chosen basis, hence $\tr\rho_A = 1$ is valid in particular
for the diagonal form \eqref{eq:rho_diag}, meaning
\[
  \sum_a p_a = 1
\]
and providing a necessary condition to interpret $p_a$ as a probability.

The \eqref{eq:rho_diag}, with $p_a$ as a probability, is often
given as a \emph{definition} of the density operator,
and the \eqref{eq:exp_subsys} can be derived from
such definition alone \parencite{open_systems}.
% TODO: do the calculation?
The probability distribution ${p_a}$
is described as an uncertainty in the preparation of the system
due to an unknown interaction with the environment.

We have seen that a bipartite system, as a whole in a pure state,
can have individual parts in mixed states, due to entanglement between them.

The reverse also holds true:
\begin{proposition}
A system with uncertainties in the preparation of the \emph{state}
(``mixed'') is equivalent to one part (``$A$'') of a bipartite system
entangled with the rest (``B'') such that the total system
$A+B$ is in a pure state.
\end{proposition}

The above is called \term{purification} (see e.g. \cite[sec.2.5]{NielsenChuang}).


\section{(Maximally) mixed state vs (maximally) coherent superposition; diagonal density matrices}

Given, for simplicity, a two-level system (a qubit), and a basis
$\{\ket{0}, \ket{1}\}$, it may be of interest to compare a \term{coherent superposition}
of the two pure states $\ket{0}$ and $\ket{1}$, with equal probability on the outcome of
a measurement (but in a well-defined quantum state); with a statistical mixture of
states $\ket{0}$ and $\ket{1}$, again with equal probability, but 
\emph{in the determination of the quantum state}.

A natural example of such coherent superposition is \parencite[Example 2.4]{Nakahara}
\[
  \ket{\psi} = \frac{1}{\sqrt{2}}\ket{0} + \frac{1}{\sqrt{2}}\ket{1}\text{,}
\]
and the corresponding density operator is
\begin{equation}\label{eq:matrix:pure}
  \rho = \ketbra{\psi}{\psi} =
  \frac{1}{2}\qty\Big(\ket{0} + \ket{1}) \qty\Big(\bra{0} + \bra{1}) =
  \frac{1}{2}\sum_{ij=0}^1\ketbra{i}{j} \repr
  \frac{1}{2}
    \begin{pmatrix}
      1 &1  \\
      1 &1  \\
    \end{pmatrix}
  \text{.}
\end{equation}
For the statistical mixture, the density operator is, by definition,
\begin{equation}\label{eq:matrix:mix}
  \rho = \frac{1}{2}\ketbra{0}{0} + \frac{1}{2}\ketbra{1}{1} \repr
  \frac{1}{2}
    \begin{pmatrix}
      1 &0  \\
      0 &1  \\
    \end{pmatrix}
  \text{.}
\end{equation}

By comparing matrices in \eqref{eq:matrix:pure} and \eqref{eq:matrix:mix}
we may conclude that the off-diagonal terms in the coherent superposition case
indicate \emph{coherence}; however, density operators are self-adjoint operators
(as linear combination of projectors) and can always be expressed in
diagonal form, including density operators of pure states: but it would be
a particular case of diagonal matrix, where all diagonal elements are zero
except one. Indeed, for a pure state:
\[
  \rho = \ketbra{\psi}{\psi}\text{,}
\]
and recalling the assumption that $\ket{\psi}$ is normalized,
we can always build a basis $\qty{\ket{e_j}}$ including $\ket{e_{j_0}} = \ket{\psi}$
among its elements, therefore the diagonal representation is
\[
  \begin{pmatrix}
    0           &       &       &       &       &           \\
                &\ddots &       &       &       &           \\
                &       &1      &       &       &           \\
                &       &       &\ddots &       &           \\
                &       &       &       &\ddots &           \\
                &       &       &       &       &0
  \end{pmatrix}\text{.}
\]

All the above allows us to conclude what follows:
\begin{remark}
  The density matrix of a pure state is either diag(000\dots1\dots000) or non-diagonal.

  So, except the special/trivial case above, a pure state must have off-diagonal terms.
\end{remark}

It's interesting, as well, to look at ensambles (mixtures)
which are not necessarily made up of orthonormal pure states.
In fact, an ensamble of orthonormal states has ``nothing special'',
and different ensambles may lead to the same density operator.
The diagonalized density matrix just shows the ``orthonormal''
ensamble which, again, has no special physical meaning.

For example, let's consider this orthogonal ensamble (diagonal density matrix):
\[
  \rho_1 = \frac{3}{4}\ketbra{0}{0} + \frac{1}{4}\ketbra{1}{1}
\]
and
\[
  \rho_2 = \frac{1}{2}\ketbra{a}{a} + \frac{1}{2}\ketbra{b}{b}\text{.}
\]
It's easy to prove that $\rho_1 = \rho_2$. This is just a particular
case of a theorem.
\begin{theorem}{(Unitary freedom in the ensemble for density matrices).}
  The sets $\setof{\tilde{\ket{\psi_i}}}$ and $\setof{\tilde{\ket{\phi_j}}}$
  generate the same density operator if and only if there exists
  a unitary matrix $\setof{u_{ij}}$ such that
  \[
    \tilde{\ket{\psi_i}} = \sum_j u_{ij}\tilde{\ket{\phi_j}}
  \]
  where $\tilde{\psi_i}$ and $\tilde{\phi_j}$ are not normalized and
  ``embed'' their probabilities $p_i$ and $q_j$ in their respective ensambles,
  i.e.
  \[
    \tilde{\ket{\psi_i}} = p_{i}\ket{\psi_i}
    \text{ and }\,
    \tilde{\ket{\phi_j}} = q_{j}\ket{\phi_j}
  \]
\end{theorem}
See \cite{NielsenChuang}{Theorem 2.6 and introductory example}
for a more detailed description and a proof of the theorem.

\section{Quantum and the Environment: from Quantum to Classical}

Placeholder text.

\section{TODO}

Measurement destroys off-diagonal terms and turns a pure state into a mixed one.

Off-diagonal terms in the density matrix indicates ``quantum purity''.
(Or internal Entanglement as well, if the system being detected was bipartite itself?)

In some sense, measurement turned a q-statistical distribution of values for the observable,
into a classical probability distribution.

Recommended read: \cite{Zurek_Decoherence, Zurek_Fundamentals}.

Zurek above: DeWitt, Everett, gell_Mann, hartl, Many Worlds, consistent/decoherent historis:
idea: Lagrangian over a history? Principle of least action?

Zuerk: Von Neumann: measurement apparatus as quantum, not classical (as opposed to Bohr).
TODO: use this to expand Historical Intro as ``the book'' is fairly brief in relation to
Von Neumann.

Nakahara, Ch. 9 book, ``Tracing out the extra degrees of freedom makes it impossible to invert a quantum operation.''

The environment is watching, from ``Exploring the quantum''
\parencite[Ch. 4]{Haroche_Exploring}. Start with subsect. 2.4.1.

2.5.4 The quantum–classical boundary
Decoherence models versus Copenhagen interpretation.

Environment-Induced Decoherence and the Transition from Quantum to Classical,
\cite{Zurek_Fundamentals}.

Partial trace, which is not a number, check wikipedia $\rightarrow$ consistent histories.
J. P. Paz W. H. Zurek

Von Neumann and Shanon entropy.

MIKIO NAKAHARA: QUANTUM COMPUTING: AN OVERVIEW

See also \cite{Schlosshauer_Decoherence}.

Vedral, Brukner (Oreshkov\dots)
Necessary and sufficient condition for non-zero quantum discord
\url{https://arxiv.org/pdf/1004.0190.pdf},
cites another paper by Zurek \url{https://arxiv.org/pdf/quant-ph/0105072.pdf}.


\chapter{Critical review of Page and Wootters mechanism and proposed improvements}
\chaptermark{Critical review of Page and Wootters \dots}
\section{Blah blah}

Three papers: \cite{Lloyd:Time}, \cite{Marletto:Evolution}, \cite{Prvanovic}.

And an experiment: \cite{Moreva:synthetic,Moreva:illustration}.

The basic idea is an additional Hilbert space $\mathcal{H}_T$ where time is an observable
corresponding to
a self-adjoint operator whose mathematical properties are the same of position in the
ordinary Hilbert space of a quantum particle in one dimension.

In this language, the ordinary Hilbert space can be labeled $\mathcal{H}_S$;
and we consider the product space $\mathcal{H}_T \otimes \mathcal{H}_S$ as
the space in which both time and position are observables, and they act as
$\hat{t} \otimes \idop_S$ and $\idop_T \otimes \hat{x}$
respectively.

\section{Questions and observations on Lloyd's paper}

With regards to \cite{Lloyd:Time}, section B, \textit{Measurement}.

\begin{enumerate}
  \item What is a memory system in quantum mechanics? 
  \begin{itemize}
    \item Do they mean a non-Markovian system? Look at \cite{MeasurementMarkovian}\dots
    \item Memory in the sense of the Maxwell daemon?
    \item
      Any general theory of quantum memory systems? Fiducial states?
      Yes, see \ref{sec:qmemory} or most importantly \cite[Ch.~3]{PreskillMeasurement}.
    \item memory kernels \cite{CarmichaelOQS2017}
  \end{itemize}
  \item ``\emph{the case where a measurement is performed at time t1}'' suggests they are back to time as an external parameter\dots
\end{enumerate}

\section{Memory systems (and fiducial state?)}\label{sec:qmemory}
Besides, and perhaps better than the below, see \cite[Ch.~3]{PreskillMeasurement},
which cover and expands topics discussed in \cite{open_systems}. Chapter 2
of Preskill \cite{PreskillMeasurement} also is insightful, paeritcularly
subsection 2.3.1 The bipartite quantum system --- ancillary systems. Chapter 3
also defines Kraus operators. We believe that \cite{Lloyd:Time} should
have cited \cite{PreskillMeasurement} --- although they're just \emph{lecture notes}.

\clearpage\includegraphics[width=\linewidth]{img/pw/qmem/1.jpg}
\clearpage\includegraphics[width=\linewidth]{img/pw/qmem/2.jpg}
\clearpage\includegraphics[width=\linewidth]{img/pw/qmem/3.jpg}
\clearpage\includegraphics[width=\linewidth]{img/pw/qmem/4.jpg}

\clearpage\includegraphics[width=\linewidth]{img/pw/qmem/nonmarkov.jpg}

\section{Entanglement and decoherence}
See also \cite{EntanglementVsDecoherence}.

Decoherence is an irreversible process, it also happens in measurement.

According to Marletto and Vedral, arrow of time is increase in Entanglement
between the clock and the rest.

So, there seems to be a contradiction: is entanglement ``decreasing''
(i.e. destroyed by decoherence) with time
or increasing?

Perhaps we can avoid the contradiction saying that
entanglement between two finite systems is
destroyed while the entanglement of each of them with the universe
is increasing?

Or in other words, the entanglement of two clocks with the rest of universe
is increasing at the expenses of the entanglement between them?

And maybe we can conclude that two systems  of which we successfully
protect the entanglement from decoherence are not good clocks?

Or perhaps there's no relation\dots
\section{In relation to Leggett-Garg inequalities (TODO: move to another file)}
Ref \cite{LeggettGarg+PageWootters}.

But also Lloyd: \url{https://arxiv.org/abs/1608.05672},
\emph{Decoherent histories approach to the cosmological measure problem}.
Lindbladt, Markov, Open Systems.

Halliwell, \url{https://arxiv.org/abs/1604.01659}. Ancilla, decay (spontaneous emission?).

A phylosophical object to decoherent histories / Everett (Everett mentioned in Marletto/ref)
is at
\url{https://arxiv.org/abs/1603.04845}.



\chapter{Reconciling the Detector Model with the Page and Wootters mechanism?}
\section{Detector model}

\cite{TQM2} (Kijowski and Detector) is cited and summarized well in
\cite{Halliwell_Detector}.

\iftodo

\section{Can a POVM on a system, if then we see it as part of a bipartite one,
equivalent to a PVM on the other, entangled, system?}

TODO: backflow effect in both models.

This will show an equivalence of models based on POVM with the Page and Wootters...?

Well, yes.

From \cite{PreskillNotes}, Ch.3 
\begin{quotation}
We have seen that
a pure state of the bipartite system AB may behave like a mixed state
when we observe subsystem A alone, and that an orthogonal measurement
of the bipartite system can realize a (nonorthogonal) POVM on A alone.
\end{quotation}

and

\begin{quotation}
A POVM in $H_A$ can be realized as a unitary transformation on the tensor
product $H_A \otimes H_B$, followed by an orthogonal measurement in $H_B$.
\end{quotation}

The same chapter talks about quantum operations, quantum channels and Kraus opertators.

We might want to look at exponential decay from \url{https://arxiv.org/abs/1704.07236},
then compare with exponential decay with P and W using Lloyd Giovannetti and Maccone (ref).

\subsection{Purification}

See https://arxiv.org/pdf/quant-ph/0512125.pdf, P-W time as a purifying ancilla
of the (Kijowski?) time.

\subsection{4-partite universe?}
\begin{itemize}
  \item{The system being measured/detected}
  \item{The Ruschhaupt detector --- which does not measure time, but whose detection happens at a certain time}
  \item{The Page and Wootters clock, entangled with the system and/or the detector}
  \item{The rest of the Universe, aka the Environment, aka the Termal Bath or Reservoir}
\end{itemize}

Can any of the above be identified? If the lab is isolated enough,
the detector is the only macro object and can act as a Universe/bath/environment/reservoir\dots?

\fi

\chapter{When Time Crystals come into play}
\section{Intro}

References: \cite{crystal2,crystal3,crystal2012}.


\chapter{Thermal time hypothesis}
\section{Connes and Rovelli}

Reference \parencite{ConnesRovelliThermo}.

\section{More}

Relate with John Goold's works? The ancilla as a clock? --- Topical Review

Markovianity, histories.

Lloyd on arXiv: from clock to cloners; erasing; scrambling (as in Goold).

Lloyd on decoherent histories (Gellman, Hartle?).

Dechoerence / irreversibility / measurement.

Vedral / Lloyd. Discord.

Measuring entanglement: Quantification of Concurrence via Weak Measurement: 1611.00149.

Marletto/Vedral on Arrow of time. Arrow of time as increasing entanglement.

Arrow of time: 

\url{https://www.wired.com/2014/04/quantum-theory-flow-time/}

\url{https://en.wikipedia.org/wiki/Loschmidt%27s_paradox}

\url{https://www.quantamagazine.org/20160119-time-entanglement/}

\section{Misc}

\url{https://arxiv.org/pdf/1702.07706.pdf} \textit{The second law of thermodynamics at the microscopic scale}
Thibaut Josset,
Aix Marseille Univ. (David).

Maxwell's demon: https://arxiv.org/pdf/1702.05161.pdf

\chapter{Gravitation and High Energy}
\section{Feynmann path stuff}

Sokolovski 1703.01966, Feynmann paths...
\section{(dis)entanglement under gravity, decoherence, event formalism}

1703.08036 An experiment to test decoherence under gravity aka entangled photons undergoing different paths and how their entanglement is affected.

Are they getting entangled with the environment instead? (Merletto and Vedral).

The theoretical paper behind the space experiment: \url{https://arxiv.org/pdf/1406.3677.pdf}. Interestingly, it mentions 
\emph{event formalism}, and we thought about that: is an event something
representable as a proper vector in $\mathcal{L}^2(\mathbb{R}^4)$ --- where one of the dimensions is time?
TODO: deepen the event formalism if it's quantum.

Closed timelike curves are also the subject of a paper by Lloyd (cite!).

``Deutsch argued that
the usual paradoxes associated with such solutions of general
relativity can be resolved by quantum mechanics''



\chapter{Extras}
\section{An intro chap on motivation and experiments?}
\begin{itemize}
  \item Time crystals?
  \item Time-resolved diffraction patters, temporal double slit experiment + Muga theory \url{https://arxiv.org/pdf/0812.3034.pdfß}
\end{itemize}

%\clearpage\includegraphics[width=\linewidth]{img/diffraction.jpg}


\section{Time crystals}

References: \cite{crystal2,crystal3,crystal2012}.


\section{Irrev}
\subsection{Use open quantum systems theory in ``Decoherence and measurement'' chapter}
Schmidt decompositions, spatial and temporal states in
$\hilb{H}_T$ and $\hilb{H}_S$
are described as density operators
(mixed states). What if there isn't a ``perfect entanglment'' between space and time.

\section{Entanglement and decoherence (Arrow of time)}
See also \cite{EntanglementVsDecoherence}.

Decoherence is an irreversible process, it also happens in measurement.

According to Marletto and Vedral, arrow of time is increase in Entanglement
between the clock and the rest.

So, there seems to be a contradiction: is entanglement ``decreasing''
(i.e. destroyed by decoherence) with time
or increasing?

We can avoid the contradiction saying that
entanglement between two finite systems is
destroyed while the entanglement of each of them with the universe
is increasing.

\subsection{``Harmonic clocks''}

TODO: use the harmonic oscillator in \cite{HarmonicClocks}
as a PaW clock for the same packet that is measured in
Ruschhaupt's detector model.

Therein, fading wave function: is minus derivative an event?
L4 normalized?


\subsection{Misc}

Idea: use section B ``Measurement'' of \cite{Lloyd:Time}: detector as (binary) measument device.

``Philospher'': \url{https://arxiv.org/abs/1704.07236}.

Time of arrival and clocks: again, \cite{YearsleyHalliwell_Clocks}.
Which maybe suggests we should not wory too much of $H\ket{\Psi} = 0$. 

We don't. 

BUT please note \cite{YearsleyHalliwell_Clocks} uses a clock that is
\emph{coupled} with the system, while in PaW they are ``only'' entangled.
So their calculation may be unnecessarily complicated.
Maybe the weakjly coupling case can be used?

Other systems of interest: decays. Prvanovic new.

Reference \cite{ConnesRovelliThermo}.

Relate with John Goold's works? The ancilla as a clock? --- Topical Review

Markovianity, histories.

Lloyd on arXiv: from clock to cloners; erasing; scrambling (as in Goold).

Lloyd on decoherent histories (Gellman, Hartle?).

Dechoerence / irreversibility / measurement.

Vedral / Lloyd. Discord.

Measuring entanglement: Quantification of Concurrence via Weak Measurement: 1611.00149.

Marletto/Vedral on Arrow of time. Arrow of time as increasing entanglement.

Arrow of time: 

\url{https://www.wired.com/2014/04/quantum-theory-flow-time/}

\url{https://en.wikipedia.org/wiki/Loschmidt%27s_paradox}

\url{https://www.quantamagazine.org/20160119-time-entanglement/}

\url{https://arxiv.org/pdf/1702.07706.pdf} \textit{The second law of thermodynamics at the microscopic scale}
Thibaut Josset,
Aix Marseille Univ. (David).

Maxwell's demon: https://arxiv.org/pdf/1702.05161.pdf

\subsection{and paths}

Both \cite{YearsleyHalliwell_Clocks} and \cite{Gambini_PW}
reason in terms of paths and actions, maybe Feynmann stuff
in following chapter... and maybe conistent historiesapproach can help
towards linking PaW and ToA?

Also \url{http://quantum.phys.cmu.edu/CHS/CHS_transp.pdf}.

\subsection{Decays?}

We might want to look at exponential decay from \url{https://arxiv.org/abs/1704.07236},
then compare with exponential decay with P and W using Lloyd Giovannetti and Maccone (ref).

\subsubsection{Purification}

See https://arxiv.org/pdf/quant-ph/0512125.pdf, P-W time as a purifying ancilla
of the (Kijowski?) time.

\section{Misc/Multi/Extras/TODO/Outlook}

\url{https://arxiv.org/abs/1703.05876}
--- \emph{comment}: time measured and stored here
may be all classical information
so this paper may or may not be relevant for the topic.

But
``prototypes of clocks based on quantum principles,
such as entanglement and squeezing''
may make this interesting again, see reference therein.
They also cite Lloyd, Giovannetti and Maccone,
but a paper quite older than \cite{Lloyd:Time}.

\url{https://arxiv.org/abs/1603.02522}
\emph{Decoherence by spontaneous emission: a single-atom analog of superradiance}.
Decoherent histories, non-markovianity, open quantum systems.

\url{https://arxiv.org/abs/1007.2615} Time travel / Quantum CTC.

Carmichael et al. \cite{CarmichaelOQS2017} (Andreas's reading)
(non-markovianity).

Non-markovian, quantum-to-classical, open systems, David,
\url{https://arxiv.org/pdf/1703.09428.pdf}.

In his works, Zurek mentions:
DeWitt, Everett, gell_Mann, hartl, Many Worlds, consistent/decoherent histories:
idea: Lagrangian over a history? Principle of least action?

Zurek: ``Reduction of the Wavepacket: How Long Does it Take?'' (arxiv),
``quantum''' time? \cite{Zurek_Einselect} also mentions
``decoherence timescale''.

Von Neumann/Shannon entropy in measurement? Mention information problems
in quantum cosmology (where a quantum time is necessary)? Etc. etc.


\chapter{Conceptual remarks (TODO: move to Intro? move to Conclusions?)}
Benefits of a computation/information approach for a better understanding of quantum
problems. Von Neumann/Shannon entropy in measurement? Mention information problems
in quantum cosmology (where a quantum time is necessary)? Etc. etc.

% appendices
\appendix
\chapter{Commutator properties}
\section{Commutator properties}
\subsection{Power}
\begin{lemma}\label{CommProp}
If the commutator $[T, H]$ commutes with $T$ i.e.
$$[T, H]T~=~T[T, H]\,,$$ then the following holds:
\begin{equation}\label{eq:tkh}
[T^k, H] = kT^{k-1}[T, H]\,.
\end{equation}
\end{lemma}
This is particularly true when $[T, H]$ is a \emph{number} as in \eqref{THcommutator} where
$T$ and $H$ are the time and energy operator respectively.
\begin{proof}
First of all, the \eqref{eq:tkh} is trivially valid for $k = 1$.

For an arbitrary positive integer $k$ there has:
\begin{dmath}\label{tkhrecur}
[T^k, H] = T^{k-1}TH - HT^{k-1}T = T^{k-1}TH - T^{k-1}HT + T^{k-1}HT - HT^{k-1}T \\
    = T^{k-1}[T, H] + [T^{k-1}, H]T
\end{dmath}
Now, iterating the result in \eqref{tkhrecur},
\begin{dmath}\label{tkhrecurplus}
[T^k, H] = T^{k-1}[T, H] + [T^{k-1}, H]T
= T^{k-1}[T, H] + (T^{k-2}[T, H] + [T^{k-2}, H]T)T
= T^{k-1}[T, H] +  T^{k-1}[T, H] + [T^{k-2}, H]T^2
= 2T^{k-1}[T, H] + [T^{k-2}, H]T^2
= \hdots
= nT^{k-1}[T, H] + [T^{k-n}, H]T^n = \hdots
\end{dmath}
where the commutativity hypothesis $[T, H]T = T[T, H]$ has been used to obtain $T^{k-2}[T, H]T = T^{k-1}[T, H]$.

Now, \eqref{tkhrecurplus} can be continued until it reaches $n=k$ when the term
$[T^{k-n}, H]T^n$ vanishes and a result of $kT^{k-1}[T, H]$ follows.
\end{proof}

\chapter{Density operators, projectors and measurement}
\section{Projectors}

We shall assume, unless otherwise stated, that all vectors are
normalized to unity.

For a given observable represented by an operator $A$ with a
\emph{discrete, non degenerate spectrum}, the probability of the
outcome $a$ of a
measurement on a system in the pure state $\ket{\psi}$ is given by
$$
\pi_{a} = \norm{\braket{a}{\psi}}^2
        = \bra{\psi}\ket{a}\bra{a}\ket{\psi}
        = \mel{\psi}{P_{a}}{\psi}
$$
where $P_{a} = \ketbra{a}$ appears as the \term{projector} on the
one-dimensional eigenspace of  $a$.

If $a$ is degenerate, and $j = 1, \dots, J$ its degeneracy index,
such probability shall sum over it:
$$
\pi_{a} = \sum_{j=1}^{J}\norm{\braket{aj}{\psi}}^2
        = \sum_{j=1}^{J}\bra{\psi}\ket{aj}\bra{aj}\ket{\psi}
        = \mel{\psi}{P_{a}}{\psi}
$$
where $P_{a} = \sum_{j=1}^{J}\ketbra{aj}$
still is the projector on the
($J$-dimensional) eigenspace of $a$.

Generally, the probability that the outcome of a measuremen falls in
the set of eigenvalues $\sigma = \{a_{1}, \dots, a_{S}\}$ is
\begin{equation}\label{eq:pi_sigma}
\pi_{\sigma}  = \sum_{s=1}^{S}\sum_{j=1}^{J_{s}}\norm{\braket{sj}{\psi}}^2
              = \sum_{s=1}^{S}\sum_{j=1}^{J_{s}}\bra{\psi}\ket{sj}\bra{sj}\ket{\psi}
              = \mel{\psi}{P_{\sigma}}{\psi}
              ,
\end{equation}
where $P_{\sigma} = \sum_{s=1}^{S}\sum_{j=1}^{J_{s}}\ketbra{sj}$
is once again a projector --- on the ``generalized eigenspace'' spanned by all
eigenvectors $\{\ket{sj}\}$ above.

\section[Measure]{Measure\footnote{Not to be confused with \emph{measurement}.}}

\begin{remark}\label{measure_properties}
  Being $\pi_{\sigma}$ the \emph{probability} of the outcome of a measurement to
  fall in a given set $\sigma$, it has to be:
  \begin{enumerate}
    \item \label{measure_properties:first} $0$ on the empty set
    \item non negative
    \item \label{measure_properties:last} \term{additive} on disjoint sets
    \item equal to $1$ if $\sigma$ includes the whole spectrum of $A$.
  \end{enumerate}
\end{remark}

\begin{remark}
  Properties \ref{measure_properties:first}\dots\ref{measure_properties:last}
  are the defining properties of a \term{measure} \parencite{EncMath_Measure}.
\end{remark}

The probability $\pi_{\sigma}$ in \eqref{eq:pi_sigma} is linear with respect to the projector
$P_{\sigma}$ hence it's not difficult to derive that the same properties in
\autoref{measure_properties} applies to $P_{\sigma}$, \emph{in the operator sense}.
In fact, the map $\sigma \subseteq \mathbb{R} \rightarrow P_{\sigma}$
implicitly defined in~\eqref{eq:pi_sigma} is a \term{projector-valued measure}.

The result is generalized,
in such a way to include both discrete and continous spectra,
by the following \cite{VonNeumann, Ballentine}
\begin{theorem}[spectral resolution]
  If $A$ is a self-adjoint operator,
  there is a unique projector-valued measure $E$
  defined on the Borel sets of $\mathbb{R}$
  such that
  \footnote{
    In \eqref{eq:spectral}, $a$ is a real number (not a set),
    but it's intended $E$ to be evaluated
    on the~\emph{interval}~from $-\infty$ to $a$.
  }
  \begin{equation}\label{eq:spectral}
    A=\int_{-\infty}^{\infty}a\, dE(a)
  \end{equation}
  and satisfying:
  \begin{align*}
    E(\mathbb{R})       & =\mathbf{1},\\
    E(B_{1}\cap B_{2}) & =E(B_{1})E(B_{2})\,.
  \end{align*}
\end{theorem}

In terms of this theorem, the projector in \eqref{eq:pi_sigma} is
\begin{equation}\label{eq:P_sigma_spectral}
  P_{\sigma} = E(\sigma) = \int_{a\in\sigma}dE(a)
\end{equation}
and $dE(a)$ is
---informally speaking---
infinitesimal if $a$ belongs to the continous spectrum,
finite if $a$ is a discrete eigenvalue
and zero otherwise.

\section{Density operator}\label{app:density}

The above results apply to \term{pure states}.
More generally, a quantum state (either pure or \term{mixed}),
is described by a \term{density} (or \term{state}) operator $\rho$.
It's a well known result (see, for example, \cite{open_systems})
that the expectation value of an observable represented by the operator $A$
is given in this description by
\begin{equation}\label{eq:expvalA_rho}
  \expval{A} = \tr(A\rho)\,.
\end{equation}

The \eqref{eq:expvalA_rho} is valid for any hermitean operator $A$,
hence we can replace it with the projector $P_{\sigma}$,
associated to the set of eigenvalues $\sigma$
according to \eqref{eq:pi_sigma} and \eqref{eq:P_sigma_spectral},
which is of particular interest
because its mean value equates the probability that the outcome of a measurement
falls in a given subset of the spectrum of a given observable.
Hence we can derive:

\begin{proposition}\label{probability_rho}
  The probability $\pi_{\sigma}$
  that the outcome of a measurement of the observable $A$
  on the system described by the state operator $\rho$
  falls in the set $\sigma$
  is given by:
  \begin{equation}\label{eq:probability_rho}
    \pi_{\sigma} = \expval{P_{\sigma}} = \tr(P_{\sigma}\rho)
  \end{equation}
  with $P_{\sigma}$ defined as in \eqref{eq:pi_sigma}
  or, more generally, in \eqref{eq:P_sigma_spectral}.
\end{proposition}

\subsection{Properties of the density operator}
\subsubsection{Additivity}
The general expression for a density operator $\rho$ is
$$
  \rho = \sum_{k}p_{k}\ketbra{\psi_{k}}
$$
with $p_{k}$ non negative and $\sum_{k}p_{k} = 1$.

The condition is not restrictive to prevent proving the following
\begin{proposition}
  Let $A$ be an hermitean operator such that
  $$
    \tr(A\rho) = 0 
  $$
  \emph{for any density operator} $\rho$.

  It follows that $A = 0$.

  \begin{proof}
    We can choose,
    as basis to explicit the expresion of the trace,
    a complete set of eigenvectors $\{\ket{m}\}$ of $A$ and,
    for each positive integer $n$,
    a particular density operator $\rho = \ketbra{n}$,
    corresponding to a (pure) eigenstate of $A$.

    With this particular choice,
    for each $n$,
    $$
      A\rho = \mel{n}{A}{n}\ketbra{n}\,.
    $$
    It then follows:
    $$
      0 = \tr(A\rho) = \sum_{m}\mel{m}{ ( \mel{n}{A}{n}\ketbra{n} ) }{m}
        = \sum_{m} \mel{n}{A}{n} \braket{m}{n} \braket{n}{m}
        = \mel{n}{A}{n}\,,
    $$
    hence the generic diagonal element of the (diagonal) matrix of $A$ is null,
    and so is consequently the operator $A$ itself.
  \end{proof}
\end{proposition}

\begin{corollary}
If $\tr(A\rho) + \tr(B\rho) = \tr(C\rho)$ for any density operator $\rho$,
then $A + B = C$.
\end{corollary}


% references
\printbibliography[heading=bibintoc]

\end{document}
