\section{Detector models}

Based on \cite[Sec. 1.4.3]{TQM1}.

Allock introduced the absorbing detector model based on a in ``imaginary'''' term $-iV_0$
(anti-hermitian) (which makes the whole Hamiltonian non-hermitian) in a series of three papers
\parencite{Allcock-1, Allcock-2, Allcock-3},
particularly in \cite[Sec. 2-4]{Allcock-2}.

He noticed that when $V_0$ is large the particle is not absorbed but reflected.

When it's small, the particle is absorbed, but ``in a very large length'' (duration?)

Muga and others responded to this concern by finding ``better potentials'' that absorb essentially the
whole wave packet in a short space interval (and time?)
\parencite{Muga_TOAQM, Muga_CompositeAbsPot, ComplexAbsPot}.

\subsection{TODO / ``backlog''}

\begin{itemize}
  \item Andreas: ``Allcock pessimistic conclusions''
  \begin{itemize}
    \item \cite[3, 6]{Leavens_TOA} or better \cite{Leavens_backflow} ?
  \end{itemize}

  \item ``Book vol. 2'' \cite[Ch. 4]{TQM2}
\end{itemize}

Cite \cite{Muga_ArrTimeOpNormal, Damborenea, Sudarshan_Zeno, Echanobe, Savvidou-1, Savvidou-2}.
