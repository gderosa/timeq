\section{Detector models}

As we have seen in the previous sections,
proper elements of $\pwspace$,
localized in both space and time,
can be interpreted as ``events''
(as opposed to full history vectors, which instead describe the unitary evolution
of a quantum system \emph{at all times}).

The arrival of a particle at a detector
is a kind of event
of particular interest from the perspective of time
in quantum mechanics, for several reasons.

The first reason
is the direct connection with experiments,
``given the appalling evidence that time is also a random variable in the laboratories''
\parencite{TQM2:Detector};
and because,
``{[as]} a matter of fact, a number of time observables are already routinely measured in laboratories,
for example arrival times in time-of-flight experiments,
but the theoretical foundation of these measurements is still being discussed''
\parencite[Preface to the First Ed.]{TQM1}.

The second reason is the existence of studies
about detector models which also investigate
time-of-arrival as a quantum observable.
In particular we will consider \cite{RuschhauptAbsorption},
which in turn traces back its basic ideas from the seminal work by Allcock
in the 1960s \parencite{Allcock-1, Allcock-2, Allcock-3},
while a detailed contemporary formulation can be found in
\cite{TQM2:Detector}.
None of these works are explicitly
based on Page--Wootters relational model ---of which, to our knowledge,
there are no working examples of application to
time-of-detection problems in the current literature.
Section \ref{sec:absorption+pw} and the remainder of the chapter
will be devoted to bridging such gap
by implementing such application
and comparing
the results from the different models.
Some emphasis will be given to
\emph{discrete} relational time
using the techniques developed in Chapter \ref{ch:pw}.

\subsection{The papers by Allcock and following developments}

With three papers dated 1969 \parencite{Allcock-1, Allcock-2, Allcock-3}, G. R. Allcock
introduced a time-of-arrival quantum observable,
along with a first formulation of a detector model based on a non-Hermitian
Hamiltonian (by including a \term{complex potential}).
See in particular \cite[\S II-IV]{Allcock-2}, where an anti-Hermitian term
$-iV_0\theta(x)$ is introduced in the Hamiltonian,
to model an absorber aim at detecting
the arrival of a particle in the region $x>0$
(here $\theta$ is the Heaviside \term{step function}).
The particle state evolves
in non-unitary manner, with a transfer of probability
from an ``incident channel''
into ``orthogonal and time-labelled measurement channels''.
Unfortunately Allcock came to negative conclusions regarding this approach.
Among other objections, he noticed that when $V_0$ is large the particle is not absorbed but reflected;
while, when it's small (i.e. the detector has a low absorption rate), the particle is eventually absorbed but
with an indetermination $\delta_T \sim \frac{1}{2}V_{0}^{-1}$
that becomes, in such case, impractically large.
Also, Allcock did not directly tackle the Pauli objection (see Section \ref{proof})
but rather maintained that argument.

The Pauli objection can be overcome by the Page--Wootters mechanism
in that a time observable is defined in a separate Hilbert space
than the one where the Hamiltonian is defined,
thus dropping a fundamental assumption of the Pauli argument.
One of the goals of this chapter
is indeed combining the two models. Another route is based on POVM
(a more general, ``unsharp'' measurement, as opposed to projective measurement; a concept we have reviewed in Chapter~\ref{ch:decohere}).
Quantum time measurement can be formulated as a POVM
(see ```Standard' Quantum-Mechanical Approach to Times of Arrival'', \cite{TQM1:Standard}).
This ``standard'' approach does not require an extra Hilbert space \emph{for the clock}.

Nonetheless, we know that a projective measurement on a bipartite system
can be regarded as a POVM
if we look at
on one part only (e.g. \cite{Paris2012}),
thus the clock space of the Page--Wootters mechanism, $\hilb{H}_T$,
can be regarded as a purification space \parencite{Paris2012} for the ``standard''
Hilbert space where the time-of-arrival POVM is defined,
and an interesting line of further research would be a more formal
study of the logical connection between the two approaches, based on this principle.
Both models respond to the issues raised by Pauli
by giving up
the pursuit of
a \emph{projective} measurement in the \emph{standard} Hilbert space,
and rather ``opening'' that Hilbert space in some sense.

The limitations of complex potentials either causing reflection
or not absorbing sufficiently has been resolved
in following literature
by noting that Allcock results were not general, and
potentials can be constructed that absorb the whole wave packet
and avoid reflection \parencite{Muga_TOAQM, Muga_CompositeAbsPot, ComplexAbsPot}.

\subsection{How a complex potential emerges}

Among the methods to define a time-of-arrival quantum observable,
the detector model by Allcock ---and all the following enhancements--- is an
\emph{operational} one,
in that it models a measurement procedure \parencite[\S 9]{Leavens_TOA}
rather than focusing on constructing an operator per se ---which much satisfy certain requirements,
like, for example, for the Kijowski distribution \parencite[\S 8]{Leavens_TOA}.

TODO cite \cite{Damborenea, Damborenea_atomic}.

\subsection{Then I would go straight to Kiukas article}

Duh.

\subsection*{TODO??}

Use the detector chapter Ruschhaupt book for both getting more references above and description following.

Maybe\dots? \cite{ProbCurrent}, \cite{Ruschhaupt_QMoT}.

``Book vol. 2'' \cite{TQM2:Detector}.

Cite \cite{Muga_ArrTimeOpNormal, Sudarshan_Zeno, Echanobe, Savvidou-1, Savvidou-2}.

