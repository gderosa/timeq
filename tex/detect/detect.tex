\section{Detector models}

As we have seen in the previous sections,
proper elements of $\pwspace$,
localized in both space and time,
can be interpreted as ``events''
(as opposed to full history vectors, which instead describe the unitary evolution
of a quantum system \emph{at all times}).

The arrival of a particle at a detector
is a kind of event
of particular interest from the perspective of time
in quantum mechanics, for several reasons.

The first reason
is the direct connection with experiments,
``given the appalling evidence that time is also a random variable in the laboratories''
\parencite{TQM2:Detector};
and because,
``{[as]} a matter of fact, a number of time observables are already routinely measured in laboratories,
for example arrival times in time-of-flight experiments,
but the theoretical foundation of these measurements is still being discussed''
\parencite[Preface to the First Ed.]{TQM1}.

The second reason is the existence of studies
about detector models which also investigate
time-of-arrival as a quantum observable.
In particular we will consider \cite{RuschhauptAbsorption},
which in turn traces back its basic ideas from the seminal work by Allcock
in the 1960s \parencite{Allcock-1, Allcock-2, Allcock-3},
while a detailed contemporary formulation can be found in
\cite{TQM2:Detector}.
None of these works are explicitly
based on Page--Wootters relational model ---of which, to our knowledge,
there are no working examples of application to
time-of-detection problems in the current literature.
Section \ref{sec:absorption+pw} and the remainder of the chapter
will be devoted to bridging such gap
by implementing such application
and comparing
the results from the different models.
Some emphasis will be given to
\emph{discrete} relational time
using the techniques developed in Chapter \ref{ch:pw}.

\subsection{The papers by Allcock and following developments}

With three papers dated 1969 \parencite{Allcock-1, Allcock-2, Allcock-3}, G. R. Allcock
introduced a time-of-arrival quantum observable,
along with a first formulation of a detector model based on a non-Hermitian
Hamiltonian (by including a \term{complex potential}).
See in particular \cite[\S II-IV]{Allcock-2}, where an anti-Hermitian term
$-iV_0\theta(x)$ is introduced in the Hamiltonian,
to model an absorber aimed at detecting
the arrival of a particle in the region $x>0$
(here $\theta$ is the Heaviside \term{step function}).
The particle state evolves
in non-unitary manner, with a transfer of probability
from an ``incident channel''
into ``orthogonal and time-labelled measurement channels''.
Unfortunately Allcock came to negative conclusions regarding this approach.
Among other objections, he noticed that when $V_0$ is large the particle is not absorbed but reflected;
while, when it's small (i.e. the detector has a low absorption rate), the particle is eventually absorbed but
with an indetermination $\delta_T \sim \frac{1}{2}V_{0}^{-1}$
that becomes, in such case, impractically large.
Also, Allcock did not directly tackle the Pauli objection (see Section \ref{proof})
but rather maintained that argument.

The Pauli objection can be overcome by the Page--Wootters mechanism
in that a time observable is defined in a separate Hilbert space
than the one where the Hamiltonian is defined,
thus dropping a fundamental assumption of the Pauli argument.
One of the goals of this chapter
is indeed combining the two models. Another route is based on POVM
(a more general, ``unsharp'' measurement, as opposed to projective measurement; a concept we have reviewed in Chapter~\ref{ch:decohere}).
Quantum time measurement can be formulated as a POVM
(see ```Standard' Quantum-Mechanical Approach to Times of Arrival'', \cite{TQM1:Standard}).
This ``standard'' approach does not require an extra Hilbert space \emph{for the clock}.

Nonetheless, we know that a projective measurement on a bipartite system
can be regarded as a POVM
if we look at
on one part only (e.g. \cite{Paris2012}),
thus the clock space of the Page--Wootters mechanism, $\hilb{H}_T$,
can be regarded as a purification space \parencite{Paris2012} for the ``standard''
Hilbert space where the time-of-arrival POVM is defined,
and an interesting line of further research would be a more formal
study of the logical connection between the two approaches, based on this principle.
Both models respond to the issues raised by Pauli
by giving up
the pursuit of
a \emph{projective} measurement in the \emph{standard} Hilbert space,
and rather ``opening'' that Hilbert space in some sense.

The limitations of complex potentials either causing reflection
or not absorbing sufficiently have been resolved
in the following literature
by noting that the results by Allcock were not general, and
potentials can be constructed that absorb the whole wave packet
and avoid reflection \parencite{Muga_TOAQM, Muga_CompositeAbsPot, ComplexAbsPot}.

\subsection{How a complex potential emerges}

Among the methods to define a time-of-arrival quantum observable,
the detector model by Allcock
---including the following enhancements---
is an \emph{operational} one,
in that it models a measurement procedure \parencite[\S 9]{Leavens_TOA}
rather than focusing on constructing an operator that satisfies certain requirements
(like the Kijowski distribution, see for example \cite[\S 8]{Leavens_TOA})
or on building an extended, general framework
like the Page and Wootters model, at least in its original formulation.

The introduction of complex potentials, a non-Hermitian Hamiltonian,
the non-unitary evolution (of a pure state!) may appear artificial to some,
and in contradiction with the fundamental rules of quantum mechanics.

\citereset
In fact, as pointed out well in \cite{Halliwell_Irreversible},
an \emph{effective} complex potential and the consequent loss of normalization
only emerge after ``tracing out'' a portion of the whole system,
which includes the particle, the detector and the environment altogether.
Such whole tripartite system \emph{is} closed, in a pure state, and evolves unitarily.
The particle wavefunction is also an ``effective'' one derived from its density operator.

See also
  \cite{Wave-function_approach, Hegerfeldt_WignerSymposium, TheQuantumJumpApproach};
in
  \cite{TQM2:Jump};
%in
%  \cite[\S 8.5.2 ``The `quantum jump' approach to damping: The wave function Monte Carlo approach'']{ScullyZubairy};
and in
  \cite[\S 6.7.1 ``Simulating Quantum Trajectories'']{WallsMilburn}.

Not surprisingly, absorption is an \term{irreverersible} process.
While a complex potential can be \emph{postulated} in order to obtain
phenomenological laws governing arrival times and detectors,
it can also be \emph{derived} from first principles and within the established framework
of open quantum systems, namely the master equation
for an irreversible detector.
The core ideas behind this derivation
can be ---very briefly--- summarized as follows.

The detector is modeled by a two-level system with $\ket{1}$ being the state of no-detection
and $\ket{0}$ the state of detection.
We also introduce the raising and lowering operators $\sigma_{+}=\ketbra{1}{0}$, $\sigma_{-}=\ketbra{0}{1}$.
The Hamiltonian of the detector is such to have $\ket{0}$
at a lower energy, so the detector ``decays'' when it ``clicks''. It's an irreversible transition because
of the coupling with the environment.
The Hamiltonian encompassing the particle, the detector, the environment and the interaction reads
\begin{equation}
  H = H_s + H_d + H_{E} + V(x)  H_{dE} \,\text{,}
\end{equation}
with $H_s$ being the Hamiltonian of a free particle and
\begin{subequations}\begin{align}
  H_d     &= \frac{1}{2}\hbar\omega \qty(\ketbra{1} - \ketbra{0}) \\
  H_{E}   &= \sum_n \hbar \omega_n a_n^{\dag} a_n \\
  H_{dE}  &= \sum_n \hbar \left( \kappa^*_n \sigma_{-} a_n^{\dag} + \kappa_n \sigma_{+} a_n \right) \\
  V(x)    &= \theta(-x) \,\text{,}
\end{align}\end{subequations}
where $a$ and $a^{\dagger}$ are the creation and annihilation operator for the electromagnetic field,
which constitutes the ``environment''; $V(x)$ is chosen to be a step function
i.e. the simplest function that makes the detector respond
when the particle reaches the region ($x<0$);
and the coupling constants $\kappa_{n}$ are to be intended in the same sense of the Jaynes--Cummings model
---in fact, the expressions of $H_d$ and $H_E$ also recall it
(\cite[\S 10.2]{WallsMilburn}, \cite{JCM} and many others).
This model essentially describes the detector as a Jaynes-Cummings atom
with the peculiarity that its coupling with the environment
is only activated by the position distribution
of another particle.

Tracing out the environment, and with some ``standard'' assumptions
(initial environment at zero temperature,
initial separable state ``undetected'' $\rho(0) = \ketbra{\psi_0}\ox\ketbra{1}$,
Markov approximation,
weak coupling),
the \emph{reduced} dynamics of the particle and the detector
is governed by the \emph{master equation}\footnote{
  Most details, calculation steps and possible generalizations are clearly skipped here, and can be found
  in the original article by J. J. Halliwell \parencite{Halliwell_Irreversible}
  and references therein.
}
\begin{equation}
  \dot{\rho} = -\frac{\iu}{\hbar} [ H_s + H_d, \rho] 
- { \gamma \over 2} \left( V^2 (x ) \sigma_{+} \sigma_{-}  \rho \ +  \rho
\sigma_{+} \sigma_{-}   V^2 (x)  \ -  \ 2 V (x) \sigma_{-}  \rho \sigma_{+} V (x )
\right)
\,\text{,}
\end{equation}
with $\gamma$ being ``a phenemonological constant determined by the distribution of oscillators in the
environment and underlying coupling constants'' \parencite{Halliwell_Irreversible}.

A general solution is in the form
\begin{equation}
  \rho =
  \rho_{11} \ox \ketbra{1}{1}
+ \rho_{01} \ox \ketbra{0}{1}
+ \rho_{10} \ox \ketbra{1}{0}
+ \rho_{00} \ox \ketbra{0}{0}
\,\text{,}
\end{equation}
with
\begin{subequations}\begin{align}
  \dot{\rho}_{11} =& -\frac{\iu}{\hbar} [ H_s, \rho_{11} ] -\frac{\gamma}{2}\left(V(x)\rho_{11} + \rho_{11}V(x)\right) \label{eq:drho11} \\
  \dot{\rho}_{01} =& -\frac{\iu}{\hbar} [ H_s, \rho_{01} ] -\frac{\gamma}{2}\rho_{01}V(x) + \iu \frac{\hbar\omega'}{2} \rho_{01}\\
  \dot{\rho}_{10} =& -\frac{\iu}{\hbar} [ H_s, \rho_{10} ] -\frac{\gamma}{2}V(x)\rho_{10} - \iu \frac{\hbar\omega'}{2} \rho_{10}\\
  \dot{\rho}_{00} =& -\frac{\iu}{\hbar} [ H_s, \rho_{00} ] +\gamma V(x)\rho_{11}V(x) \,\text{,}
\end{align}\end{subequations}
and $\omega'$ a ``renormalized'' value for the characteristic frequency $\omega$ in $H_d$.

Just to give an idea of how the complex potential emerges, let's now focus on $\rho_{11}$
and the corresponding probability of not being detected, which is
\begin{equation}\label{eq:p_nd}
  p_{nd} = \Tr \rho_{11}(\tau) = \int_{-\infty}^{\infty} \dd{x} \rho_{11}(x, x, \tau) \,\text{,}
\end{equation}
where $\rho_{11}(x, x, \tau)$
-- or, more generally, $\rho_{ij}(x, x, \tau)$ for each $i, j = 0, 1$ --
are (diagonal) density matrix elements such that, in general:
\begin{equation}
  \rho_{ij}(\tau) = \int_{-\infty}^{\infty} \dd{x_1} \int_{-\infty}^{\infty} \dd{x_2} \rho_{ij}(x_1, x_2, \tau) \ketbra{x_1}{x_2}
  \, \text{.}
\end{equation}

Now, the solution to the \eqref{eq:drho11} can be written
\begin{equation}\label{eq:rho11.evol}
  \rho_{11}(t) =  e^{-\frac{\iu}{\hbar}H_{s}t-\frac{\gamma}{2}Vt}
  \rho_{11}(0)    e^{ \frac{\iu}{\hbar}H_{s}t-\frac{\gamma}{2}Vt} \, \text{.}
\end{equation}
If $\rho_{11}$ were in a pure state, say $\rho_{11}(t) = \ketbra{\psi(t)}$
(the ``undetected wavefunction''),
and recalling that $\rho_{11}(0) = \ketbra{\psi_0}$,
eq. \eqref{eq:rho11.evol} can be reformulated as
\begin{equation}
  \ket{\psi(t)} =
  e^{-\frac{\iu}{\hbar}H_{s}t-\frac{\gamma}{2}Vt} \ket{\psi_0} =
  e^{-\frac{\iu}{\hbar} \qty( H_{s} -\iu\frac{\hbar}{2}\gamma V ) t} \ket{\psi_0} \,\text{,}
\end{equation}
showing that the particle evolves as if an ``imaginary'' (anti-Hermitian) term
$-\iu\frac{\hbar}{2}\gamma V$ was added to its otherwise free Hamiltonian $H_s$.
This ``proves'' the validity of the complex potential model as an \emph{effective} theory,
and the consistency
with the unitary evolution of the ``bigger picture'' quantum system.

Another consequence of treating $\rho_{11}$ as a pure state is that the probability of no detection
\eqref{eq:p_nd} can be expressed as
\begin{equation}
  p_{nd}(\tau) = \int_{-\infty}^{\infty} \dd{x} \qty|\psi(x, \tau)|^2 \, \text{,}
\end{equation}
which is the (non conserved) norm $\norm{\psi}$ of this effective state vector.
Therefore, the probability that the detection \emph{has} happened
up to time $\tau$, which is $1 - p_{nd}(\tau)$, is equal to the
\emph{loss of normalization} $1 - \norm{\psi}$ at that time.
This is the total probability that the detection has happened from the initial time
($t=0$ in our chosen settings)
to $t = \tau$. In terms of probability density (per unit of time)
$\mathbb{P}(t)$, it is therefore
\begin{equation}
  \mathbb{P}(t) = - \dv{\norm{\psi}}{t} \,\text{.}
\end{equation}
This is the other important result that will be used in the following Sections
with the related literature.

\subsection{2-level example}

The detection-by-absorption model in \cite{RuschhauptAbsorption}
is based on a complex potential that, plugged into the Schr\"odinger equation,
leads to a non-unitary evolution of the state vector
(with loss of normalization). We will particularly focud on the
two-level example in the paper.

Specifically, the Hamiltonian $\hat{H}$ is replaced by a $\hat{H} - i\hat{D}$
(with $\hat{D}$ self-adjoint, bounded, positive)
and, consequently:
\begin{equation}\label{eq:schrod_complex_pot}
  \hat{H} \ket{\psi(t)} = i\hbar\dv{t}\ket{\psi(t)} +i\hat{D}\ket{\psi(t)} \text{.}
\end{equation}

\citereset
In the detector model of \cite{RuschhauptAbsorption}, 
as we have seen  already in this Section,
the detection
by absorption
corresponds to the \emph{decrease} in norm of the wavefunction.

\citereset
Eq. 9 in \cite{RuschhauptAbsorption}
equates the squared norm of a ``time representation'' wavefunction
to the opposite derivative of the squared norm of the ``absorbed wavefunction''.
It reads:
\begin{quote}
  We will associate with any wave function $\psi \in \hilb{H}$
  another wave function $\hat{\psi}$,
  which is a function of time, so that
  $\abs{\hat{\psi}(t)}^2$
  is the arrival probability density. In other words,
  $\hat{\psi}$ is a wave function in a time representation. For each
  $t$, $\hat{\psi}(t)$ lies in the original Hilbert space $H$.
\end{quote}
Therefore we ``translate'' $\hat{\psi}(t)$ into $\ket{\phi(t)}_S$
and, consequently, $\abs{\hat{\psi}(t)}^2$ into $\braket{\phi(t)}$,
in the language of the Page--Wootters model and within the notation
adopted.

\citereset
Using eq. 8 in \cite{RuschhauptAbsorption} and translating into our notation we have:
\begin{equation}\label{eq:phi_psi_kiukas}
  \hat{\psi}(t) \eqbydef
  \ket{\phi(t)}_S =
  \begin{cases}
    \sqrt{\frac{2}{\hbar}} \hat{D}^{1/2} \ket{\psi_{\text{Kiukas}}(t)}_S &\text{ if } t > 0 \\
    0 &\text{ otherwise. }
  \end{cases}
\end{equation}
Where at $t \le 0$ the interaction with the detector is yet to come,
but so it is, as a limit, for small values of $t>0$,
in other terms
$\lim_{t \to 0^{+}} \norm{\hat{D} \ket{\psi_{\text{Kiukas}}(t)}} = 0$, thus avoiding the apparent discontinuity.

% \subsection*{\color{red} TODO}

% {
%   \color{red}
%   Figure out how to connect the above to what follows (or drop or separate if does not apply).

%   \scriptsize{
%     Hint: rather then the ``conspiracy theory''
%     (``I find an imaginary term both here and there'')
%     consider (time of) arrival as in \cite{Maccone:QMOT}.
%     This should bring to a normalized element of $\pwspace$ \dots
%   }
% }

\citereset\subsection{Application: two-level system}

In \cite{RuschhauptAbsorption}, an example application of the detector model
is provided for a two-level system.
In Page and Wootters terms,
this would corrspond to a bi-dimensional $\hilb{H}_S$, but a continuous
spectrum of $\hat{T}$ in $\hilb{H}_T$. The paper is \emph{not} based on
the Page--Wootters model, indeed the purpose of this section is a comparison
with such model, using the results of Section \ref{sec:absorption+pw}.

By setting, out of convenience, $\hbar = \omega = 1$
(with $\omega$ the characteristic frequency of the system),
and directly considering the parameters
that minimize the time--energy uncertainty product \parencite{RuschhauptAbsorption},
we have a non-Hermitian ``Hamiltonian''
$\mathit{K} = \hat{H} - i\hat{D}$ with
\begin{equation}\label{eq:complexpot}
  \mathit{K} = \hat{H} - i\hat{D} \repr
    \hbar\omega\left\{
      \left[\begin{matrix}0 & 1\\1 & 0\end{matrix}\right] -
      i \left[\begin{matrix}0 & 0\\0 & \gamma \end{matrix}\right]
    \right\}
\end{equation}
and $\gamma = 2\sqrt{2}$.

We take an initial state of $\ket{0}$
(or $\mqty[1\\0]$ in matrix form).

We then compute, symbolically, the non-unitary evolution
$\ket{\psi(t)} = e^{-i\mathit{K}t}\ket{0}$
with the aid of \term{SymPy} \parencite{comp:sympy} within a \term{Jupyter} \parencite{comp:jupyter} notebook
(see Appendix \ref{detector-model-kiukas-ruschhaupt-schmidt-werner} for all the details of the calculation).

Simplifying the result in eq. \eqref{eq:sympy:non-unitary-evol}, we have:
\begin{equation}
  \ket{\psi(t)} \repr e^{-\frac{\sqrt{2}}{t}} \mqty[
    \cos(\frac{\sqrt{2}}{2}t) + \sin(\frac{\sqrt{2}}{2}t)& \\
                     -i\sqrt{2} \sin(\frac{\sqrt{2}}{2}t)&
  ] \,\text{.}
\end{equation}

\begin{figure} %% https://tex.stackexchange.com/a/165730
  \centering
  \begin{subfigure}{0.49\textwidth}
    \includegraphics[width=\linewidth]{img/2ldetect/re_psi0_t.png}
    \subcaption{}\label{fig:absorbed-qubit-components:re0}
  \end{subfigure}
  % \hspace*{\fill} % separation between the subfigures
  \begin{subfigure}{0.49\textwidth}
    \includegraphics[width=\linewidth]{img/2ldetect/im_psi1_t.png}
    \subcaption{}\label{fig:absorbed-qubit-components:im1}
  \end{subfigure}
  % \hspace*{\fill} % separation between the subfigures
  \caption{
    Non-unitary evolution of the absorbed qubit
    according to the model in
    ref. \cite[\S ``Emission from a two-level system'']{RuschhauptAbsorption}.
    The component along $\ket{0}$ is purely real,
    and the one along $\ket{1}$ is purely imaginary,
    therefore only their their respective parts are plotted.
  }\label{fig:absorbed-qubit-components}
\end{figure}

\begin{figure}
  \centering
  \begin{subfigure}[b]{0.49\textwidth}
    \includegraphics[width=\linewidth]{img/2ldetect/qubit_normalization_loss.png}
    \subcaption{}\label{fig:absorbed-qubit-normalization-loss:t}
  \end{subfigure}
  \begin{subfigure}[b]{0.49\textwidth}
    \includegraphics[width=\linewidth]{img/2ldetect/P_omega.png}
    \subcaption{}\label{fig:absorbed-qubit-normalization-loss:omega}
  \end{subfigure}
  \caption{
    Non-unitary evolution of absorbed qubit.
    \subref{fig:absorbed-qubit-normalization-loss:t}
      Detection probability in time. It's equal to the
      loss of normalization $-\dv{\norm{\psi}^2}{t}$
      (but also to the squared norm of $\hat{\psi}(t)$ as of eq. \eqref{eq:analytic:hatpsi}.
    \subref{fig:absorbed-qubit-normalization-loss:omega}
      Detection probability in the frequency domain.
  }
  \label{fig:absorbed-qubit-normalization-loss}
\end{figure}

The ``lossy'' evolution, with the two components of the qubit, is shown in Fig.~\ref{fig:absorbed-qubit-components}.
The loss of normalization $-\dv{\norm{\psi}^2}{t}$, indicating the probability of detection by absorption,
is then derived directly and shown in Fig.~\ref{fig:absorbed-qubit-normalization-loss}.

This yields the \emph{probability} of detection.
One may wonder whether it is possible to derive a corresponding \emph{probability amplitude} vector,
whose squared norm across time is equal to the said probability distribution.\footnote{
  The solution is of course not unique, but one may ask whether such functions would lead
  to quantum interference patterns and other phenomenology which may be subject of further study.
}
Within the framework of \cite{RuschhauptAbsorption}, a ``wavefunction in time'' in such sense
is the $\hat{\psi}$, as in \eqref{eq:phi_psi_kiukas}.
It is computed in detail within the
notebook in Appendix \ref{detector-model-kiukas-ruschhaupt-schmidt-werner}, eq. \eqref{eq:sympy:hatpsi},
simplifying which we obtain:
\begin{equation}\label{eq:analytic:hatpsi}
  \hat{\psi}(t) =
    i 2^{\frac{5}{4}} e^{-\frac{\sqrt{2}}{2}t}\sin(\frac{\sqrt{2}}{2}t) \theta(t)
    \ket{1}
    \text{,}
\end{equation}
with $\theta(t)$ the Heaviside step function.

\begin{remark}\label{remark:detection_area}
In general, the operator $\hat{D}$ as in \eqref{eq:schrod_complex_pot}
is such that the eigenspace corresponding to its zero eigenvalue
is the ``area'' where the detector is not sensitive. Or, in other words,
the linear span of states with zero probability of triggering the detector.
Therefore, when $\hat{D}$ (or its square root) is applied to a state vector,
for example in \eqref{eq:phi_psi_kiukas},
the components in such eigenspace are cut off and the resulting
$\hat{\psi}$, eq. \eqref{eq:analytic:hatpsi} in the example, lies at all times in the ``area of detection''
i.e. it's a multiple of $\ket{1}$ in this case.
\end{remark}

The corresponding Page--Wootters (proper) vector of $\pwspace$ is
\begin{equation}\label{eq:hatpsi:pw}
  \dket{\Phi} = \int \dd{t} \ket{t} \ox \hat{\psi}(t) \,\text{,}
\end{equation}
to which the considerations of Section \ref{sec:for-normalized-elements}
and Section \ref{sec:pure-state-approach} in terms of time--frequency
(or time--energy) uncertainty relation apply, with some analogy
to what \cite{RuschhauptAbsorption} does within its own framework
in relation to $\hat{\psi}$ and its Fourier transform.

In that regard, the \eqref{eq:hatpsi:pw} can be reformulated
\begin{equation}
  \dket{\Phi} = \int \dd{\omega} \ket{\omega} \ox \mathcal{F} \hat{\psi} (\omega) \,\text{,}
\end{equation}
where it's
\begin{equation}
  \mathcal{F} \hat{\psi} (\omega) = - \frac{\sqrt[4]{2} i}{\sqrt{\pi} \left(- \omega^{2} + \sqrt{2} i \omega + 1\right)} \ket{1}
\end{equation}
---see notebook up to eq. \eqref{eq:fhatpsi1_omega} for details.

Taking the squared modulus, a probability distribution over angular frequency
(or, equivalently, energy) is obtained:
\[
  P(\omega) = \frac{\sqrt{2}}{\pi \left(\omega^{4} + 1\right)}
  \,\text{.}
\]
See Fig. \ref{fig:absorbed-qubit-normalization-loss:omega}.

\citereset
\subsubsection{
  (Non-unitary) evolution without evolution:
  plugging the complex potential in the \emph{discrete} Page--Wootters model
}

Similarly to what seen in Section \ref{sec:building-the-discrete-pw-clock}, we build
the clock by defining ---and representing in a convenient basis---
the time operator and the corresponding frequency operator:
\begin{align}
  \hat{T} \repr \frac{2\pi}{N}
  \begin{pmatrix}
    0           &       &       &       \\
                &1      &       &       \\
                &       &\ddots &       \\
                &       &       &N-1
  \end{pmatrix}
  &&
  \hat{\Omega} = \frac{N}{2\pi} F^{}_{N} \hat{T} F^{\dagger}_{N} \, \text{,}
\end{align}
where $F$ is, again, the discrete Fourier operator of order $N$.

Next we define the Wheeler--DeWitt operator $\mathbb{J}$ as in
\eqref{eq:pwHamiltonian}, but $\hat{H}_S$ is replaced by the non-hermitian
Hamiltonian of the detector model
$\mathit{K}_S = \hat{H}_S - \iu \hat{D}_S$
\parencite{RuschhauptAbsorption},
where the subscript $_S$ has been added to stress
that they would act on the $\hilb{H}_S$ part
of the Page--Wootters' $\pwspace$ ``spacetime'':
\begin{equation}
  \mathbb{J} = \hbar\hat{\Omega}\ox\idop_S + \idop_T\ox\qty(\hat{H}_S -\iu \hat{D}_S) \,\text{.}
\end{equation}
$\hat{H}_S$ and $\hat{D}_S$ are the same as $\hat{H}$ and $\hat{D}$
respectively from the \eqref{eq:complexpot}.

We then compute eigenvalues and eigenvectors of $\mathbb{J}$.
Eigenvectors associated to the eigenvalue $0$ will include
the history of the qubit over a ``period''
(of what would have been a periodic evolution, without the absorptive detector).

Eigenvectors associated to non-zero eigenvalues will need a ``phase correction'',
or energy shift\footnote{ Again, see also ref. \cite[\S ``The Zero-eigenvalue'']{Lloyd:Time}. }
as seen, for example, in \eqref{eq:comparison0} and \eqref{eq:comparison1}.
Or, in a more compact form:
\begin{equation}
  \dket{n}_{\text{hist.}} = e^{-\iu \epsilon_n \hat{T}} \ox \idop_S \dket{n}
  \,\text{,}
\end{equation}
where $\epsilon_n$'s are eigenalues of $\mathbb{J}$ and
$\dket{n}$'s their corresponding eigenvectors.

\citereset
Linear combination of histories $\sum_n \alpha_n \dket{n}_{\text{hist.}}$
are also physically possible\footnote{
  As opposed to (linear combinations of) mere, uncorrected eigenstates of $\mathbb{J}$.
  Indeed, one may observe,
  at least in the case where $\mathbb{J}$ was hermitian,
  that $\setof{\dket{n}}$ is a basis
  of $\pwspace$ therefore
  $\sum_n \alpha_n \dket{n}$ would span \emph{all elements}
  of $\pwspace$ and the theory would not predict anything i.e.
  it would not discriminate unphysical histories.
}
and in fact we pick
one that ensure that the initial state ${}_{T}\bradket{0}{\Psi}$ is equal to $\ket{0}_S$
so to allow a comparison with the example in \cite{RuschhauptAbsorption}.
Such linear combination is not unique. For reasons of numerical stability,
a linear combination with coefficients in the order of 1 would be ideal if it exists.
In this particular problem, it does.
We simply scan, as a first attempt, all possible values ${}_{T}\bradket{0}{n} + {}_{T}\bradket{0}{m}$
to find a combination that is equal to $\ket{0}_S$
-- or maximizes the fidelity with that respect.\footnote{
  See the definition, and invocation, of the function \texttt{find_best()} in Appendix \ref{discrete-page-wootters-model}.
}

All the above numerical computation is implemented with \emph{NumPy} \parencite{comp:numpy},
particularly in appendix
\ref{discrete-page-wootters-model}.
The results are visualized in
Fig. \ref{fig:absorbed-qubit-components_pwlattice},
and are \emph{compatible} with the non-Page--Wootters
solution previously shown in Fig. \ref{fig:absorbed-qubit-components}.

\begin{figure}
  \centering
  \begin{subfigure}[b]{0.49\textwidth}
    \includegraphics[width=\linewidth]{img/2ldetect/re_psi0_t_pwlattice.png}
    \subcaption{}\label{fig:absorbed-qubit-components_pwlattice:re0}
  \end{subfigure}
  \begin{subfigure}[b]{0.49\textwidth}
    \includegraphics[width=\linewidth]{img/2ldetect/im_psi1_t_pwlattice.png}
    \subcaption{}\label{fig:absorbed-qubit-components_pwlattice:im1}
  \end{subfigure}
  \caption{
    Non-unitary ``evolution without evolution''
    using the discrete Page--Wootters model,
    plugging in
    the complex potential of \cite{RuschhauptAbsorption}.
    The result is compatible with the continuous
    ``Schr\"odinger evolution''
    shown in Fig. \ref{fig:absorbed-qubit-components}.
    Note the different notation e.g.
    $\mathrm{Re}{\;}_{T}\hspace{-.2em}\left\langle t | {}_{S}\hspace{-.2em}\left\langle 0 | \Psi \right\rangle\hspace{-.17em}\right\rangle$
    to fit within the framework of the P--W space-time $\pwspace$.
  }
  \label{fig:absorbed-qubit-components_pwlattice}
\end{figure}

\subsubsection{Detection probability amplitude, time of arrival distribution}

In \cite{RuschhauptAbsorption} the probaility of detection in time is given
by how fast the norm decreases i.e. $-\dv{\norm{\psi}^2}{t}$.
A ``wavefunction in time'', whose squared modulus equates the detection probability,
is introduced in eq. (8) therein:\footnote{
  Please note we are importing some notation from \cite{RuschhauptAbsorption} as per the symbol $\hat{\psi}$,
  at variance to what generally adhered to in this work,
  where the ``hat'' ($\hat{\;}$) denotes quantum observables and their corresponding operators.
}
\begin{equation}
  \hat{\psi} = \theta(t) \sqrt{2/\hbar}\,\hat{D}^{1/2}\,\psi
\end{equation}
In Page--Wooters terms, this translates into applying
$\theta(\hat{T}) \ox \sqrt{2/\hbar}\,\hat{D}^{1/2}_S$
to the each history vector $\dket{n}_{\text{hist.}}$
(and their linear combinations).
The result is a detection-event \emph{proper} element of ${\pwspace}$
(that would be normalizable in a continuous, infinite-time model too):
\begin{equation}\label{eq:qubit_detection_wavefunction}
  \dket{\Phi_n} =
    \theta(\hat{T}) \ox \sqrt{2/\hbar}\,\hat{D}^{1/2}_S \dket{n}_{\text{hist.}} =
    \theta(\hat{T}) e^{-i\epsilon_{n}\hat{T}} \ox \sqrt{2\hat{D}} \dket{n} \, \text{.}
\end{equation}
Given we are considering, in our discrete model, only an interval of time within $0$ and $2\pi$,
i.e. all non-negative times (or, more physically, only times when the detector is active),
the Heaviside function $\theta$ (of operator) can be omitted, and simply replaced
by the identity in time $\idop_T$.

Of course, for our problem, we take a linear combination
\begin{equation}
  \dket{\Phi} = \sum_n \alpha_n \dket{\Phi_n}
\end{equation}
where the coefficients $\alpha_n$ are the same that verified
(not necessarily uniquely)
the initial condition of the problem
$\ket{0}_S = {}_T\bradket{0}{\Psi} = \sum_n \alpha_n \, {}_T\bradket{0}{n}_{\text{hist.}} = \sum_n \alpha_n \, {}_T\bradket{0}{n}$.

\citereset
The vector $\dket{\Phi}$, proper element of $\pwspace$,
is the Page--Wootters counterpart of the function
``in time representation'' $\hat{\psi}$,
as
described in \cite{RuschhauptAbsorption}. In formulas:
\begin{equation}
  \bradket{t}{\Phi} = \hat{\psi}(t) \, \text{.}
\end{equation}

As a consequence of Remark \ref{remark:detection_area},
the ``detection wavefunction''~$\dket{\Phi}$ will always lie,
\emph{spatially},
in the space generated by $\ket{1}_S$, or in other terms:
\begin{equation}\label{eq:detect_prob_ampl_only_on_1}
  \prescript{}{T}{\bra{t}}\prescript{}{S}{\bradket{0}{\Phi}} = 0 \text{,} \quad \forall t \, \text{.}
\end{equation}
The linear space of $\ket{1}_S$ is the ``detectable area'' of the qubit
in this problem, and
$\dket{\Phi}$ is essentially the result of the action of the operator $\hat{D}$,
that filters non-detectable components out from any vector of $\hilb{H}_S$.

We derive the components of $\dket{\Phi}$ numerically in Section \ref{detection-event},
where they are encoded in the array \texttt{prob_detect_v}.
We observe therein that the \eqref{eq:detect_prob_ampl_only_on_1} is confirmed,
and that ${}_T\!\bra{t} {}_S\!\bradket{1}{\Phi}$ is proportional
to the $\ket{1}$-component of the evolution of the qubit (history $\Dket{\Psi}$)
at any time $t$
(%
  which is not surprising, being $\sqrt{2/\hbar}\hat{D}^{1/2}$ of the form
  $\scriptsize \left[\begin{matrix}0 & 0\\0 & \kappa \end{matrix}\right]$%
).
In more formal terms:
\begin{equation}\label{eq:detect_prob_amplitude_proportional}
  \exists \kappa: \: {}_T\!\bra{t} {}_S\!\bradket{1}{\Phi} = \kappa \cdot {}_T\!\bra{t} {}_S\!\bradket{1}{\Psi} \; \forall t \text{,}
\end{equation}
which, algbebraically, is almost obvious, given the considerations above.
However, it's worth noting that the expression on the right side
(up to a factor $\kappa$) is, at each $t$,
an ordinary quantum mechanical probability amplitude
(probability amplitude of being $\ket{1}$ rather than $\ket{0}$);
while the expression on the left side, as a function of $t$,
is a probability amplitude \emph{in time}.
Even when normalized (in the probabilistic sense), the two expressions
are proportional but not equal (hence the factor $\kappa$).
This is because the expression on the right must be normalized
so that, at each $t$,
${ \abs{ {}_T\!\bra{t} {}_S\!\bradket{0}{\Psi} }^2 + \abs{ {}_T\!\bra{t} {}_S\!\bradket{1}{\Psi} }^2 = 1 }$
(or, actually, equal to $ \braket{\psi(t)} \leq 1$, the ``lossy norm'' due to absorption).
Whereas the expression on the left must be normalized so that
$\sum_t \norm{ {}_T\!\bradket{t}{\Phi} }_S^2 = P$,
where $P$ is the total probability that the detection/absorption happens at any time,
and it's equal to $1$ for an ideal detection.

\citereset
The probability of detection in time is shown in Fig. \ref{fig:2l_pw_detect_prob_t}
and is \emph{consistent} with the result of
the contuinuous and not explicitly Page--Wooters model of
\cite{RuschhauptAbsorption} (see Fig. \ref{fig:absorbed-qubit-normalization-loss:t}).

Switching to the frequency (and, therefore, \emph{energy}) domain,
the discrete Fourier transform yields another proper vector
$\Dket{\tilde{\Phi}}$ of $\pwspace$:
\begin{equation}
  \Dket{\tilde{\Phi}} = F \ox \idop_s \dket{\Phi} \, \text{.}
\end{equation}
This vector numerical values, taken pairwise
(or by groups of 3 for a qutrit etc.)
are the components in the computational basis
of the kets $\ket{\tilde{\phi}(\omega)} \in \hilb{H}_S$
such that
\begin{equation}
  \dket{\Phi} = \sum_n \ket{t}\bradket{t}{\Phi} = \sum_{\omega} \ket{\omega}\bradket{\omega}{\Phi}
    = \sum_{\omega} \ket{\omega}_T \ox \ket{\tilde{\phi}(\omega)}_S \, \text{.}
\end{equation}

As a numerical array, $\Dket{\tilde{\Phi}}$ is the representation of $\Dket{\Phi}$
under the basis $\ket{\omega} \ox \ket{0,1}$ instead of $\ket{t} \ox \ket{0,1}$,
where the $\ket{\omega}$ constitute an eigenbasis of angular frequency $\hat{\Omega}$.

\citereset
The squared norms $\braket{\tilde{\phi}(\omega)}_{\!S} = \norm{\braDket{\omega}{\Phi}}_S^2$
give the probability of detection in the frequency domain, which is plotted in
Fig. \ref{fig:2l_pw_detect_prob_omega} and, again, consistent
with Fig. \ref{fig:absorbed-qubit-normalization-loss:omega} i.e.
the result from the model of \cite{RuschhauptAbsorption}.

\begin{figure}
  \centering
  \begin{subfigure}[b]{0.49\textwidth}
    \includegraphics[width=\linewidth]{img/2ldetect/pw-detect-prob.png}
    \subcaption{}\label{fig:2l_pw_detect_prob_t}
  \end{subfigure}
  \begin{subfigure}[b]{0.49\textwidth}
    \includegraphics[width=\linewidth]{img/2ldetect/pw-detect-prob-ft.png}
    \subcaption{}\label{fig:2l_pw_detect_prob_omega}
  \end{subfigure}
  \caption{
    Arrival at the detector, probability distribution:
    \subref{fig:2l_pw_detect_prob_t} in the time domain
    and \subref{fig:2l_pw_detect_prob_omega} in the frequency domain.
  }
  \label{fig:2l_pw_detect_prob}
\end{figure}

\subsubsection{Time--energy uncertainty relation}

In quantum systems described by a finite Hilbert space,
uncertainty relations are quantified in terms of \term{entropies}
of the canonically conjugate probability distributions
---see Section \ref{sec:finite_uncertainty} and references therein,
in particular \cite{FiniteHilb}.

In our example, the entropic uncertainty relation explicitly reads:
\begin{multline}
 S_T + S_{\Omega} =
  -\sum_t \norm{\bradket{t}{\Phi}}_S^2 \ln \norm{\bradket{t}{\Phi}}_S^2
  -\sum_{\omega} \norm{\bradket{\omega}{\Phi}}_S^2 \ln \norm{\bradket{\omega}{\Phi}}_S^2 \\
%
  = -\sum_{n=0}^{N-1} \left(\abs{\Phi_{2n}}^2 + \abs{\Phi_{2n+1}}^2\right) \ln \left(\abs{\Phi_{2n}}^2 + \abs{\Phi_{2n+1}}^2\right) \\
    -\sum_{m=0}^{N-1}
          \left(\abs{\tilde{\Phi}_{2m}}^2 + \abs{\tilde{\Phi}_{2m+1}}^2\right)
      \ln \left(\abs{\tilde{\Phi}_{2m}}^2 + \abs{\tilde{\Phi}_{2m+1}}^2\right) \\
%
  \geq \ln N
\end{multline}
---where we had computed previously the probability values in parenthesis.

In Section \ref{jupy:entropic-uncertainties}, we numerically find that
\begin{equation*}
  S_{T} + S_{\Omega} \approx 1.14 \cdot \ln N
\end{equation*}
i.e. circa $14\%$ more than the theoretical minimal uncertainty.

In terms of uncertainty relation based on standard deviations,
which is the standard formulation, particularly for continuous distributions,
and allows a comparison with \cite{RuschhauptAbsorption}, we compute numerically
\begin{equation}\label{eq:uncertainty-us}
  \sigma_{T} \sigma_{\Omega} \approx 0.716 \, \text{,}
\end{equation} \citereset
while \cite{RuschhauptAbsorption} finds
\begin{equation}\label{eq:uncertainty-them}
  \sigma_{T} \sigma_{\Omega} \approx 0.707 \, \text{.}
\end{equation}

It's worth observing that the example in the paper
sets the optimal parameters to minimize the time--energy uncertainty
\emph{within the given constraints} of the particular physical system,
they don't necessarily achieve the theoretical Heisenberg minimum of
$\frac{\hbar}{2}$
(or $0.5$ in our numerical problem where $\hbar$ has been adimendsionalized).
More notably, parameters therein have been chosen so to minimize
an alternative definition of time--energy uncertainty, $\expval{T}\sigma_E$,
deemed more meaningful within the physics of the particular system.

\citereset
Nonetheless, with minimal or not minimal uncertainty states (or ``histories''),
a comparison of such uncertainties between the discrete Page--Wooters model implemented
here and the model in \cite{RuschhauptAbsorption} can be performed:
the difference between the results in \eqref{eq:uncertainty-us}
and \eqref{eq:uncertainty-them} may be large enough to be not simply
due to numerical approximation. A further line of investigation
will involve increasing the resolution (number of levels) of the quantum clock:
if the discrepancy persists, an experimental verification
---like those proposed in the paper itself---
will then be of
particular interest.

