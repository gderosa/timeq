\section{Detector models}

As we have seen in the previous sections,
proper elements of $\pwspace$,
localized in both space and time,
can be interpreted as ``events''
(as opposed to full history vectors, which instead describe the unitary evolution
of a quantum system \emph{at all times}).

The arrival of a particle at a detector
is a kind of event
of particular interest from the perspective of time
in quantum mechanics, for several reasons.

The first reason
is the direct connection with experiments,
``given the appalling evidence that time is also a random variable in the laboratories''
\parencite[Sec. 4.1]{TQM2};
and because,
``{[as]} a matter of fact, a number of time observables are already routinely measured in laboratories,
for example arrival times in time-of-flight experiments,
but the theoretical foundation of these measurements is still being discussed''
\parencite[Preface to the First Ed.]{TQM1}.

The second reason is the existence of studies
about detector models which also investigate
time-of-arrival as a quantum observable.
In particular we will consider \cite{RuschhauptAbsorption},
which in turn traces back its basic ideas from the seminal work by Allcock
in the 1960s \parencite{Allcock-1, Allcock-2, Allcock-3},
while a detailed contemporary formulation can be found in
\cite{TQM2:Detector}.

These works are not ---at least explicitly---
based on Page--Wootters relational model,
of which, to our knowledge,
there are no working examples of application to
time-of-detection problems in the current literature.
Section \ref{sec:absorption+pw} and the remainder of the chapter
will be devoted to bridging this gap,
by implementing such application through a few examples
and comparing
the results from the different models.
Some emphasis will be given to
\emph{discrete} relational time
using the techniques developed in Chapter \ref{ch:pw}.

\subsection*{TODO / ``backlog''}

Use the detector chapter Ruschhaupt book for both getting more references above and description following.
Also the history chapter, section on Allcock.

Allcock introduced the absorbing detector model based on a in ``imaginary'''' term $-iV_0$
(anti-Hermitian) (which makes the whole Hamiltonian non-Hermitian) in a series of three papers
\parencite{Allcock-1, Allcock-2, Allcock-3},
particularly in \cite[sec. II-IV]{Allcock-2}.

He noticed that when $V_0$ is large the particle is not absorbed but reflected.

When it's small, the particle is absorbed, but ``in a very large length'' (duration?)

Muga and others responded to this concern by finding ``better potentials'' that absorb essentially the
whole wave packet in a short space interval (and time?)
\parencite{Muga_TOAQM, Muga_CompositeAbsPot, ComplexAbsPot} (\emph{verify}).

For plane waves, this interval can be made arbitrarily small.

\begin{quotation}
  For wave packets the length can be small but not zero because of a
  peculiar property of quantum perfect absorbers; they exactly reproduce
  the behaviour of the wave function defined in the absence of the absorber.
  Therefore, they must accumulate and give back norm to reproduce the
  ``backflow effect'' \parencite{Bracken_bf, Bracken_ProbTransport}
  (\emph{where authors just talk about backflow per se})
  by which wave functions without negative momentum components
  have negative current density at certain time and position intervals
  \parencite{Leavens_backflow}. (\emph{where thay actually talk about perfect absobers})
\end{quotation}

\cite{ComplexAbsPot}

\cite{Werner_ArrivalTime} (less/old)

Maybe\dots? \cite{ProbCurrent}, \cite{Ruschhaupt_QMoT}.

Andreas: ``Allcock pessimistic conclusions''.

``Book vol. 2'' \cite{TQM2:Detector}.

Cite \cite{Muga_ArrTimeOpNormal, Damborenea, Sudarshan_Zeno, Echanobe, Savvidou-1, Savvidou-2}.

\cite{Damborenea_atomic}, for some interesting conceptual considerations. Together with the
intro of the detector chapter of the book.