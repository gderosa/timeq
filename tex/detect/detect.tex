\section{Detector models}

\subsection*{TODO / ``backlog''}

Based on \cite[sec. 1.4.3]{TQM1}.

Allock introduced the absorbing detector model based on a in ``imaginary'''' term $-iV_0$
(anti-hermitian) (which makes the whole Hamiltonian non-hermitian) in a series of three papers
\parencite{Allcock-1, Allcock-2, Allcock-3},
particularly in \cite[sec. II-IV]{Allcock-2}.

He noticed that when $V_0$ is large the particle is not absorbed but reflected.

When it's small, the particle is absorbed, but ``in a very large length'' (duration?)

Muga and others responded to this concern by finding ``better potentials'' that absorb essentially the
whole wave packet in a short space interval (and time?)
\parencite{Muga_TOAQM, Muga_CompositeAbsPot, ComplexAbsPot} (\emph{verify}).

For plane waves, this interval can be made abritrarily small.

\begin{quotation}
  For wave packets the length can be small but not zero because of a
  peculiar property of quantum perfect absorbers; they exactly reproduce
  the behaviour of the wave function defined in the absence of the absorber.
  Therefore, they must accumulate and give back norm to reproduce the
  ``backflow effect'' \parencite{Bracken_bf, Bracken_ProbTransport}
  (\emph{where authors just talk about backflow per se})
  by which wave functions without negative momentum components
  have negative current density at certain time and position intervals
  \parencite{Leavens_backflow}. (\emph{where thay actually talk about perfect absobers})
\end{quotation}

\cite{ComplexAbsPot}

\cite{Werner_ArrivalTime} (less/old)

Maybe\dots? \cite{ProbCurrent}, \cite{Ruschhaupt_QMoT}.

Andreas: ``Allcock pessimistic conclusions''.

``Book vol. 2'' \cite{TQM2:Detector}.

Cite \cite{Muga_ArrTimeOpNormal, Damborenea, Sudarshan_Zeno, Echanobe, Savvidou-1, Savvidou-2}.

\cite{Damborenea_atomic}?