\section{Time and the Quantum: An Invitation}\label{sec:intro}

Time is a fundamental concept in physics, together with space (or \term{position} within space),
matter (or \emph{mass}) and energy.

In quantum mechanics, at least in first quantization, defining the \emph{position} as
an observable is unproblematic and somewhat basic.
Indeed, representing the quantum state of a particle
in terms of coefficients with respect to eigenstates of position is extremely common.
Almost always when a \emph{wavefunction} $\psi$
for a state $\ket{\psi}$
is written down, the {position representation}
is assumed, unless otherwise (explicitly) noted:
\begin{equation}\label{eq:positionrepr}
  \ket{\psi} = \int \dd{x'} \psi(x') \ket{x'} \text{,}
\end{equation}
where\footnote{
  Here a one-dimensional system is considered, for simplicity:
  it can be easily verified that more dimensions,
  and degeneracy, may be taken into account without altering the key concepts.
}
$\psi(x') = \braket{x'}{\psi}$.

As per \emph{mass}, a quantum mechanism that explains how particle acquire mass was theorized by
Peter Higgs and others in 1964, and confirmed experimentally in 2012
\parencite{Higgs, EnglertBrout, Kibble+, HiggsATLAS, HiggsCMS}.

\emph{Energy}, as an obervable, is associated with arguably the most important operator in quantum dynamics: the Hamiltonian.

\emph{Time} is missing from this picture. As of writing of the present work,
there is no general consensus in the physics community on
the definition of a quantum operator associated to a time observable,
or even only on a particular quantum mechanism that would allow time to ``emerge'',
for example, as an effective quantity.

One may argue that the urge of a consistent quantum description for \emph{all} fundamental
measurable quantities, including time,
is essentially (if not purely)
philosophical,
until experiments show (or at least are proposed to show)
``wavelike'' or other nonclassical features of time,
possibly with analogies to what is already observed along the ``position axis''.
Interestingly, accounts exist of experiments, with atomic beams through temporal double-slits,
showing diffraction patterns in time \parencite{TemporalSlits}.
Moreover, signatures of \term{time-crystalline order}
in solids and trapped ions have been recently observed
\parencite{crystal.exp.ordered, crystal.exp.disordered, crystal.exp.nmr, crystal.exp}.

With all the above in mind, the relation between time and other quantities
(whether they are ``quantum observables'' in the formal sense or not)
is explored in the rest of this section,
with reminders about some basic properties that will be useful for the rest of the
discussion.

\subsection{Time and position}

Rigorously speaking, the position observable has no special role in quantum mechanics.
Eq.~\eqref{eq:positionrepr} can be also expanded to
\begin{equation}\label{eq:qprepr}
  \ket{\psi} = \int \dd{x'} \psi(x') \ket{x'} = \int \dd{p'} \tilde{\psi}(p') \ket{p'} \,\text{,}
\end{equation}
with $p$ and $\ket{p}$ being respectively eigenvalues and eigenvectors of the momentum operator.

By multiplying on the left by $\bra{x}$, Eq.~\eqref{eq:positionrepr} yields
\begin{equation}\label{eq:diracdeltax}
  \psi(x) = \int \dd{x'} \psi(x') \delta(x-x') \,\text{.}
\end{equation}
In general, for an observable with continuous spectrum,
in its own eigenbasis, eigenstates are represented as a Dirac deltas
(for discrete spectrums, the integral is replaced by a discrete sum,
and Dirac deltas by Kronecker deltas).

Applying the position operator $\hat{X}$ to the~\eqref{eq:positionrepr},
then taking the inner product on the left by $\bra{x}$,
there has
\begin{equation}
  \qty[\hat{X}\psi](x) \eqbydef \mel{x}{\hat{X}}{\psi} =
    \int \dd{x'} \psi(x') x' \mel{x}{\hat{X}}{x'} =
    \int \dd{x'} \psi(x') \delta(x-x') =
    x\psi(x)
  \,\text{.}
\end{equation}
The position
(and an observable in general) is represented, in its own eigenbasis,
simply as the multiplication of the original wavefunction by the respective variable.

\subsection{Time and evolution}

As a matter of experience, if not anything else,
the state of a physical system, either classical or quantum,
with all its observable quantities,
can be regarded as a function of the instant in \emph{time}
(let's call that independent variable $t$).
The same cannot be stated, in general, for any other pair of measurable quantities:
the $x$ coordinate is not a function of the $y$ coordinate
(or their respective statistical amplitudes, in the quantum realm),
and so on.
This status of time as the universally independent variable
has shaped classical mathematical physics: Galilean transformations
do not change time, which remains as an absolute quantity.
The Hamiltonian formalism itself is based on this assumption,
and the Hamiltonian formalism is at the foundation of quantum mechanics too,
which thus inherits ---one may argue--- all the consequences of
treating time as ``external'' with respect to the other quantities under study.
``[The] Hamiltonian method [\dots] marks out a particular time variable
as the canonical conjugate of the Hamiltonian function'' \parencite{DiracLagrangian}.

Emphasizing time as a parameter, the value of a wavefunction
(in position representation, for example)
can therefore be expressed as $\psi_{t}(x)$ at each position $x$
and at each time $t$.
Still, it can be regarded as a function of two variables
and one may wonder why (after an inessential change of notation),
in the identity
\begin{equation}\label{eq:diracdeltaxt}
  \psi(x; t) = \int \dd{x'}\dd{t'} \delta(x-x')\delta(t-t') \psi(x';\, t') \,\text{,}
\end{equation}
the term $\delta(t-t')$ cannot be interpreted as the eigenfunction of some time operator,
which in turn acts as a simple multiplication by $t$ on this
wavefunction $\psi(x; t)$ ---which would be, therefore, a wavefunction in ``time--position representation''.

The fact that $\delta(t-t')$ would be an \emph{improper} eigenfunction
is not the blocking issue in this case (it is not, for the position eigenfunction),
and the mathematical intricacies related to the continuous spectrum are
elegantly resolved by the spectral theory, mainly by Von Neumann
\parencite{VonNeumann}, which is an integral part of the standard, commonly accepted
formulation of quantum mechanics, almost since the early years of the theory.

The main obstacle is instead the \term{Pauli's objection},
which is only indirectly related to the special status of time
as independent variable of ``evolution'', and
directly related to the properties of the spectrum of the {Hamiltonian},
as will be illustrated in Section~\ref{proof}.

\subsection{Time and energy}\label{sec:T--H}

In the quest for a quantum time observable,
it is tempting to leverage
the dualism between position and momentum,
evoked in Eq.~\eqref{eq:qprepr}.
It is well known that the momentum is the infinitesimal generator of spatial translations:
\begin{equation}\label{eq:genspacetransl}
  \psi(x-\Delta x; \, t) = \E^{-\iu \Delta x \hat{P} / \hbar} \psi(x; t) \,\text{,}
\end{equation}
where the momentum operator is $\hat{p} \repr -\iu\hbar \pdv{x}$, in position representation.

It is also well known that the Schr\"odinger equation is equivalent to
stating that the Hamiltonian is the infinitesimal generator of
\emph{temporal} translation:
\begin{equation}\label{eq:gentimetransl}
  \psi(x; t+\Delta t) = \E^{-\iu \Delta t \hat{H} / \hbar} \psi(x; t) \,\text{.}
\end{equation}

The analogy between the two equations above,
among other considerations,
motivates the requirement that
an hypothetical \emph{self-adjoint} time operator should be the canonically conjugate of the Hamiltonian,
just like the momentum operator is canonically conjugate to the position one.
This would involve
a commutation relation between time and energy that is analogous to
the one between position and momentum, $[\hat{X}, \hat{P}] = \iu \hbar$,
and consequently a time--energy uncertainty relation.

Several formulations of such time--energy uncertainty relations have been proposed,
although most of them do not rigorously refer to
a time operator $\hat{T}$ in their definition of $\Delta T$.

On another note, it's worth stressing that,
while equations \eqref{eq:genspacetransl} and \eqref{eq:gentimetransl}
do offer some conceptual motivation towards defining a time observable that is conjugate to the Hamiltonian,
they are not sufficient to logically imply its existence.
Both $\Delta{x}$ and $\Delta{t}$ ---which quantify the spatial and temporal translations at the core of this discussion---
are, once again, mere \emph{parameters} for the \emph{displacement} and \emph{time evolution} operators respectively:
\begin{align}
  \mel{x}{\mathcal{D}_{\Delta{x}}}{\psi(t)}        = \braket{x-\Delta{x}}{\psi(t)}\!\text{, }
    &\text{ with } \mathcal{D}_{\Delta{x}}    = \E^{-\iu \Delta x \hat{P} / \hbar} \\
  \mel{x}{\mathcal{U}_{\Delta{t}}}{\psi(t)}        = \braket{x}{\psi(t+\Delta{t})}\!\text{, }
    &\text{ with } \mathcal{U}_{\Delta{t}}    = \E^{-\iu \Delta t \hat{H} / \hbar}
  \, \text{.}
\end{align}
Even in terms of \emph{position},
$\Delta x$ is not to be intended as the the spread of the $\hat{X}$ operator in this case.
Hence, the analogy between the roles of \emph{parameters} $\Delta x$ and $\Delta t$,
with respect to \emph{operators} $\hat{P}$ and $\hat{H}$,
cannot be used, alone, to rigorously prove the same analogy between the position operator $\hat{X}$
and some time operator $\hat{T}$, although it is somewhat suggestive.

Whether a quantum time observable can be in principle defined as a self-adjoint
operator, or other mathematical frameworks exist that are suitable,
possibly satisfying the above conjugation properties,
is indeed one of the fundamental questions within the entire topic
of the quantization of time.

\subsubsection{Uncertainty relations}

The Heisenberg uncertainty principle is at the core of quantum mechanics.
It was first introduced in 1927 \parencite{Heisenberg:Uncertainty},
in the form of a position--momentum relation: $p_{1}q_{1} \sim h$, in his notation.
Interestingly, in the same paper, Heisenberg also introduced
a \emph{time--energy} uncertainty relation, $E_{1}t_{1} \sim h$,
related to the time at which a transition between two energy levels occurs.

What did Heisenberg mean by $t_1$? Was that an early formalization of (the ``spread'' of)
a time observable (or ``matrix'', in the language of Heisenberg's first formulation of quantum mechanics)?
In fact, the same question can be raised for the position--momentum
relation too. The original formulation of the uncertainty principle
was not expressed in terms of standard deviations and mean values of Hermitian operators
as we know it today. Heisenberg approach was semi-empirical and,
while it turned unsatifactory from a formal perspective, in some aspects,
it had the merit of explicitly dealing with the physics of the disturbance
of a measurement device to the measurement itself. Therefore,
regardless of the mathematical maturity of its model,
the paper did provide convincing phenomenological arguments towards
the existence of an uncertainty principle for both the position--momentum
\emph{and} the time--energy pair.

As observed in \cite[sec.1.1.3]{TQM1}, Heisenberg introduced the equation
$\mathbf{Et}-\mathbf{tE} = -\iu\hbar$
as ``familiar'', without really specifying the nature of the matrix $\mathbf{t}$,
thus leaving the problem of defining a quantum time observable actually unresolved.
Therein, it is argued that the idea of ``familiarity'' may have originated from the
relation between
the time span and the frequency\footnote{
  Of course, energy and frequency will be often used interchangeably,
  being
  simply related by the the Planck relation $E = h\nu = \hbar\omega$;
  formulated as early as 1900, and incorporated in all the following research
  that brought to the foundation of quantum mechanics, including Heisenberg's work.
}
width of a signal. Regardless of the status
of signal theory in 1927, it is well known, at least in more recent times, that
methods like the Fourier analysis are used in Information and Communication engineering to associate a function in the time domain
to another in the frequency domain; while, in quantum mechanics, the same mathematical operation
is generally used to associate a wavefunction in position representation to
the corresponding one in momentum representation.
This is another suggestive analogy.\footnote{
  In Chapter \ref{ch:pw}, a discrete Fourier transformation will be used indeed to
  associate a time operator to a frequency operator,
  in a modified model of quantum mechanics.
}

Later in 1927, the uncertainty principle was derived in terms of
statistical interpretation of quantum observables
by Kennard and others (see e.g. \cite{Kennard1927}).
The clarification of all formal issues has continued until recently \parencite{Appleby}.
However, all this formalization process has regarded position and momentum only
---not time and energy.

\subsubsection{Mandelstam and Tamm}

TODO: move to Ch. \ref{ch:hist}?

The formulation of the time--energy uncertainty relations by Mandelstam and Tamm
\parencite{MandelstamTamm} is considered the least controversial, to the point
of being regularly mentioned in standard textbooks. However,
they refer to the characteristic times of each specific observable $A$ i.e.
the time at which its mean value changes significantly (where ``significantly''
here means: in the order of its spread $\Delta{A}$). So this notion of time
is \emph{observable-dependent} and does not refer to the time (or the statistical spread of time)
at which a particular event happens
---as opposed to the Heisenberg paper, which aimed at answering the question: when does
a ``quantum jump'' occur?

In formulas, the characteristic time is defined by Mandelstam and Tamm as
\begin{equation}
  \tau_{A} = \frac{ \Delta{A} } { \dv*{\langle\hat{A}\rangle}{t} } \, \text{,}
\end{equation}
and the uncertainty relation reads:
\begin{equation}
  \tau_{A}\Delta{E} \ge \frac{\hbar}{2} \, \text{.}
\end{equation}

Some limitations of the Mandelstam--Tamm
approach
will be tackled
in
Chapter \ref{ch:detect}
by further developing
models where time--energy relations have to be intended in the sense
of ``time of occurrence'' (of particle detection)
and time is represented
by a self-adjoint operator.

\subsection{Time evolution, relativity, photons}\label{sec:trel}

The idea of a perfect analogy between
position--momentum and time--energy uncertainty relations 
might naturally lead one to ask whether position and time
(or space and time coordinates), as well as momentum and energy, can be trated on
equal footing. This is what (classically) happens within Einstein's theory of relativity.
Thus, one may further wonder:
would a theory combining quantum mechanics and
relativity be the suitable framework which would allow a rigorous and logically consistent
definition of a time observable,
possibly exposing some analogy with the position operator?

None of the proposed models comprehending both quantum theory and \emph{general} relativity
have reached general consensus,
or the remote possibility of experimental verification \parencite{QGravIntro};
but \term{quantum field theory} (QFT) is an established framework that does combine
some principles
of quantum mechanics with the \emph{special} theory of relativity.

Unfortunately, QFT does not promote time $t$ to some quantum operator $\hat{t}$;
instead, it achieves equal treatment of space and time only by
``demoting'' the \emph{position} operator to a mere classical parameter.\footnote{
  More on this in Section \ref{sec:KG}, where the Klein-Gordon equation is reformulated
  with time as an Hermitian operator in an appropriate Hilbert space
  ---based on the Page--Wootters model, which is introduced in Chapter~\ref{ch:pw}.
}
It treats time and space equally only in the sense that they are all classical,
which appears as the necessary tradeoff in order to quantize other quantities
in some mathematically tractable way. \parencite[sec.I.1]{SrednickiQFT}.

It is no surprise that one of the most successful applications
of QFT, the Standard Model, describes all fundamental interactions except gravity:
while electroweak and strong interactions can be described within a field theory
which treats time and space as a ``background'' (or \emph{labels} to mark the relevant operators),
time and space are expected to be the main mathematical objects (observables),
and transform appropriately,
in a theory of gravity
which has general relativity as its classical counterpart.
Whether the definition of a quantum time operator is an useful step towards
a quantum theory of gravity is, of course, beyond the scope of the present work.

The fact that QFT ``externalizes'' time and position to the rank of classical parameters
is particularly true in quantum electrodynamics, also within the
phenomenology of quantum optics.
Thus it is impossible to define a \emph{photon} position
within standard quantum optics (see, for example, \cite{ScullyZubairy}, sec. 1.5.4 `Wave function for photons'),
and the problem of defining a quantum position observable for a photon
shows an interesting analogy with time for a quantum massive particle.
It is, in fact, a matter of active research \parencite{HawtonPhotonPosition, Hawton2019}.

While there isn't a wave function for photons, the wave function for massive particles,
as seen in the Shr\"{o}dinger equation, only exists as a function of position (or momentum, etc.)
but not as a function of time,
i.e. the variable $t$ in Eq.~\eqref{eq:diracdeltaxt}
cannot be regarded as spanning the spectrum of a time operator.

Nevertheless, we have provided an explanation as to why (even without considering the Pauli objection)
relativistic field theories, as currently accepted, cannot provide a solution
to the problem of quantum time, in spite of treating time and space on equal footing,
and despite the fact that position in quantum mechanics is an observable
with its associated self-adjoint operator.
