\section{Structure of the thesis}\label{sec:struct}

After explaining the motivation behind this work in Chapter \ref{ch:intro},
a review of open quantum systems and non-projective measurement theory
is outlined in Chapter \ref{ch:decohere}, as a necessary ``toolbox''
to understand relational models and other theoretical formulations
of time as a quantum observable.

The \emph{Pauli objection} itself is detailed in Chapter \ref{ch:hist},
together with some models that have been proposed to overcome it,
with the exception of relation models, and in particular the Page--Wootters model
of ``evolution without evolution'',
which is presented in Chapter \ref{ch:pw},
where it is also compared to some detector models.
Therein, results are compared in relation to some systems of particular interest,
and posssible analogies are explored.

Further areas of study and applications are suggested in Chapter \ref{ch:outlook}.