\section{Uncertainty}

For canonical pairs of operators with a continuous, unbounded spectrum i.e.
$\hat{q}$ and $\hat{p} \eqdef -i\hbar\hat{\partial}_{q}$,
it is in general straightforward to prove that
$\qty[\hat{q}, \hat{p}] = i\hbar$ and therefore
$\Delta q \Delta p \geq {\frac{1}{2} \qty|\ev{\qty[\hat{q}, \hat{p}]}| = \frac{\hbar}{2}}$.

In finite $d$-dimensional Hilbert spaces, the above commutation relation doesn't hold
in general, and is less essential.
Canonically conjugate operators are related
through the (discrete) Fourier transform ($\hat{p} = F\hat{q}F^{\dagger}$)
rather then differentiation,
and uncertainty relations are based on
the properties of Fourier transformation
rather than commutation relations.

Particularly, the entropic uncertainty relation holds
(\cite[\S 2.4]{FiniteHilb}; \cite{Deutsch:Uncertainty}):
\begin{equation}
  S_q + S_p \geq \ln d
\end{equation}
where the quantities $S_q$ and $S_p$ are the \term{R\'enyi}-\term{Shannon} entropies
\parencite[\S {\it I}.A]{Wehner:Uncertainty}; in this case:
\begin{align}
  S_q &= -\sum_n \qty|\lambda_n |^2  \ln\qty|\lambda_n|^2 \\
  S_p &= -\sum_n \qty|\mu_n     |^2  \ln\qty|\mu_n    |^2
  \,\text{,}
\end{align}
with $\lambda_n$ and $\mu_n$ being the discrete ``wave functions'' in the
(generalized) position and momentum basis.

\if{todo}
TODO Lloyd recent on entropic uncertainty (for time\dots).
From state vectors to density matrices and/or mixed states: Von Neumann entroy (wikipedia).
Subadditivity. 
\fi

In a continuous-time Page-Wootters universe,
one is tempted to observe that in $\hilb{H}_T$,
in the time representation, it is $E \eqdef \hbar\hat{\Omega} \repr i\hbar\partial_{t}$,
in analogy with $\hat{p} \repr -i\hbar\partial_{q}$ in $\hilb{H}_S$,
therefore a time-energy uncertainty relation can be derived
with a proof that is formally identical to the well known
position-momentum uncertainty relation. However, it's worth
noting that the time evolution of a system is never expressed
in terms of a separable state (and therefore a pure state in $\hilb{H}_T$)
but always a maximally entangled state in the product space
\begin{equation*}
  \int dt \ket{t}_T \ox \ket{\psi(t)}_S \, \text{,}
\end{equation*}
therefore the spatial degrees of freedom should be \emph{traced out},
before deriving a time-energy relation 
---in terms of the reduced density operator and partial traces.

TODO: depending on Ruschhaupt paper, we'll pick the right type of uncertainty
relation ---still taking the above into account as much as relevant.
