\section{Uncertainty}

In a continuous-time Page-Wootters universe,
one is tempted to observe that in $\hilb{H}_T$,
in the time representation, it is $E \eqbydef -\hbar\hat{\Omega} \repr i\hbar\partial_{t}$,
in analogy with $\hat{p} \repr -i\hbar\partial_{q}$ in $\hilb{H}_S$,
therefore a time-energy uncertainty relation can be derived
with a proof that is formally identical to the well known
position-momentum uncertainty relation. However, the analogy of
$\hat{\Omega}$ with momentum $\hat{p}$,
and thus of $\hat{T}$ with position $\hat{q}$,
would be correct only if there has a probability distribution \emph{in time},
which is not the case: unitary evolution means a particle exists with
certainty at all times. Therefore, a proof of a time-energy uncertainty
relation based on the formal analogy with position and momentum
(and, of course, the fact that time is finally an operator bypassing the Pauli objection)
can only apply to \emph{events}, in the sense of \eqref{eq:pwphi}.

One may argue that the unitary evolution
\[
  \dket{\Psi} = \int \dd{t} \ket{t}_T \ox \ket{\psi(t)}_S
\]
is a special case of the normalized element of $\hilb{H}_T\ox\hilb{H}_S$ (\term{event}):
\[
  \dket{\Phi} = \int \dd{t} \phi(t) \ket{t}_T \ox \ket{\psi(t)}_S
\]
with $\abs{\phi(t)}=1 \; \forall t$ (i.e. an event that is indefinitely spread in time),
but it is not the case, as in the ``Fourier transform''
\[
  \int \dd \omega \, \tilde{\phi}(\omega) \ket{\omega}_T \ox \ket{\tilde{\psi}(\omega)}_S
\]
$\tilde{\phi(\omega)}$ should then be infinitely concentrated (a Dirac delta),
meaning that all unitary evolutions are the evolution of an eigenstate of energy (frequency),
which is not true in general.

$\dket{\Phi}$ and $\dket{\Psi}$ encode two physically distinct concepts.

\subsection{Time-energy uncertainty relation for \emph{events}}
Towards proving a time-energy uncertainty relation in $\hilb{H}_T$
based on formal analogy with position and momentum in $\hilb{H}_S$,
a normalized element $\dket{\Phi}$ of $\hilb{H}_T\ox\hilb{H}_S$ 
is not not affected by the difficulties mentioned above.
Still, it is not generally
a separable state,
and thus
the problem of time and energy relation cannot be reduced to that of
a pure state in $\hilb{H}_T$.
It is in general an entangled state in the product space.
Thus the spatial degrees of freedom should be \emph{traced out},
before deriving a time-energy relation.

Namely, from
\begin{equation}
  \dket{\Phi} =
    \int \dd{t} \phi(t) \ket{t}_T \ox \ket{\psi(t)}_S =
    \int \dd \omega \, \tilde{\phi}(\omega) \ket{\omega}_T \ox \ket{\tilde{\psi}(\omega)}_S \, \text{,}
\end{equation}
with use of \eqref{eq:density_A_expand}, the reduced density operator can be computed
via partial trace:
\[
  \hat{\rho}^T = \Tr_S\qty(\dketdbra{\Phi}{\Phi}) = \int \dd t \abs{\phi(t)}^2 \ketbra{t}{t} 
\]

\[
  \hat{\rho}^T = \Tr_S\qty(\dketdbra{\Phi}{\Phi}) = \int \dd \omega \abs{\tilde{\phi}(\omega)}^2 \ketbra{\omega}{\omega}
  \,\text{.} 
\]
Here the probabilty distributions $\phi$ and $\tilde{\phi}$
are ``classical'' in the sense of a mixed state.

The relation $\sigma_T\sigma_{\hbar\Omega} = \hbar \sigma_{\phi} \sigma_{\tilde{\phi}} \geq \frac{\hbar}{2}$
can then be derived from the properties of the Fourier transform.

\subsection{TODO}
\cite{Maccone:Pauli} \S IV.  UNBOUNDED-ENERGY CLOCKS?