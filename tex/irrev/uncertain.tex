\section{Uncertainty}

In a continuous-time Page-Wootters universe,
one is tempted to observe that in $\hilb{H}_T$,
in the time representation, it is $E \eqbydef -\hbar\hat{\Omega} \repr i\hbar\partial_{t}$,
in analogy with $\hat{p} \repr -i\hbar\partial_{q}$ in $\hilb{H}_S$,
therefore a time-energy uncertainty relation can be derived
with a proof that is formally identical to the well known
position-momentum uncertainty relation. However, the analogy of
$\hat{\Omega}$ with momentum $\hat{p}$,
and thus of $\hat{T}$ with position $\hat{q}$,
would be correct only if there has a probability distribution \emph{in time},
which is not the case: unitary evolution means a particle exists with
certainty at all times. Therefore, a proof of a time-energy uncertainty
relation based on the formal analogy with position and momentum
(and, of course, the fact that time is finally an operator bypassing the Pauli objection)
can only apply to \emph{events}, in the sense of \eqref{eq:pwphi}.

One may argue that the unitary evolution
\[
  \dket{\Psi} = \int \dd{t} \ket{t}_T \ox \ket{\psi(t)}_S
\]
is a special case of the normalized element of $\hilb{H}_T\ox\hilb{H}_S$ (\term{event}):
\[
  \dket{\Phi} = \int \dd{t} \phi(t) \ket{t}_T \ox \ket{\psi(t)}_S
\]
with $\abs{\phi(t)}=1 \; \forall t$ (i.e. an event that is indefinitely spread in time),
but it is not the case, as in the ``Fourier transform''
\[
  \int \dd \omega \, \tilde{\phi}(\omega) \ket{\omega}_T \ox \ket{\tilde{\psi}(\omega)}_S
\]
$\tilde{\phi(\omega)}$ should then be infinitely concentrated (a Dirac delta),
meaning that all unitary evolutions are the evolution of an eigenstate of energy (frequency),
which is not true in general.

In fact, $\dket{\Phi}$ and $\dket{\Psi}$ encode two physically distinct concepts,
and a full unitary evolution $\dket{\Psi}$ can be obtained from a time-normalized
$\dket{\Phi}$ with the ``Page-Wootters evolution operator'' $\mathbb{U}$~\parencite{Lloyd:Time}: 












However, it's worth
noting that the time evolution of a system is never expressed
in terms of a separable state (and therefore a pure state in $\hilb{H}_T$)
but always a maximally entangled state in the product space
\begin{equation}
  \int \dd t \ket{t}_T \ox \ket{\psi(t)}_S \, \text{.}
\end{equation}


therefore the spatial degrees of freedom should be \emph{traced out},
before deriving a time-energy relation 
---in terms of the reduced density operator and partial traces.
This is not necessarily an obstacle

As also stated in \cite{Lloyd:Time},
the above can be formulated with respect to an eigenbasis of $\Omega$ (frequency)
instead of $t$ (time):
\begin{equation}
  \int \dd \omega \, \tilde{\phi}(\omega) \ket{\omega}_T \ox \ket{\tilde{\psi}(\omega)}_S \, \text{,}
\end{equation}


\subsection{TODO ME}

But there's hope: use \eqref{eq:density_A_expand}

\[
  \rho^T = \Tr_S\qty(\dketdbra{\Phi}{\Phi}) = \int \dd t \abs{\phi(t)}^2 \ketbra{t}{t} 
\]

\[
  \rho^T = \Tr_S\qty(\dketdbra{\Phi}{\Phi}) = \int \dd \omega \abs{\tilde{\phi}(\omega)}^2 \ketbra{\omega}{\omega} 
\]

They're clasical prob. ; proerties of Fourier etc.

\subsection{NO, NO, even better:}
\cite{Maccone:Pauli} \S IV.  UNBOUNDED-ENERGY CLOCKS?