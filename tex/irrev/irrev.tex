Similarly to Chapter \ref{ch:pw},
there will be a particular emphasis on finite-dimensional systems,
for their relatively simple implementation, experimentally,
and for the computational feasibility and interest.

\section{Uncertainty}

For canonical pairs of operators with a continuous, unbounded spectrum i.e.
$\hat{q}$ and $\hat{p} \eqdef -i\hbar\hat{\partial}_{q}$,
it is in general straightforward to prove that
$\qty[\hat{q}, \hat{p}] = i\hbar$ and therefore
$\Delta q \Delta p \geq {\frac{1}{2} \qty|\ev{\qty[\hat{q}, \hat{p}]}| = \frac{\hbar}{2}}$.

In finite $d$-dimensional Hilbert spaces, the above commutation relation doesn't hold
in general, and is less essential.
Canonically conjugate operators are related
through the (discrete) Fourier transform ($\hat{p} = F\hat{q}F^{\dagger}$)
rather then differentiation,
and uncertainty relations are based on
the properties of Fourier transformation
rather than commutation relations.

Particularly, the entropic uncertainty relation holds
(\cite[\S 2.4]{FiniteHilb}; \cite{Deutsch:Uncertainty}):
\begin{equation}
  S_q + S_p \geq \ln d
\end{equation}
where the quantities $S_q$ and $S_p$ are the \term{R\'enyi}-\term{Shannon} entropies
\parencite[\S {\it I}.A]{Wehner:Uncertainty}; in this case:
\begin{align}
  S_q &= -\sum_n \qty|\lambda_n |^2  \ln\qty|\lambda_n|^2 \\
  S_p &= -\sum_n \qty|\mu_n     |^2  \ln\qty|\mu_n    |^2
  \,\text{,}
\end{align}
with $\lambda_n$ and $\mu_n$ being the discrete ``wave functions'' in the
(generalized) position and momentum basis.

\if{todo}
TODO Lloyd recent on entropic uncertainty (for time\dots).
From state vectors to density matrices and/or mixed states: Von Neumann entroy (wikipedia).
Subadditivity. 
\fi

In a continuous-time Page-Wootters universe,
one is tempted to observe that in $\hilb{H}_T$,
in the time representation, it is $E \eqdef \hbar\hat{\Omega} \repr i\hbar\partial_{t}$,
in analogy with $\hat{p} \repr -i\hbar\partial_{q}$ in $\hilb{H}_S$,
therefore a time-energy uncertainty relation can be derived
with a proof that is formally identical to the well known
position-momentum uncertainty relation. However, it's worth
noting that the time evolution of a system is never expressed
in terms of a separable state (and therefore a pure state in $\hilb{H}_T$)
but always a maximally entangled state in the product space
\begin{equation*}
  \int dt \ket{t}_T \ox \ket{\psi(t)}_S \, \text{,}
\end{equation*}
therefore the spatial degrees of freedom should be \emph{traced out},
before deriving a time-energy relation 
---in terms of the reduced density operator and partial traces.

TODO: depending on Ruschhaupt paper, we'll pick the right type of uncertainty
relation ---still taking the above into account as much as relevant.

\section{Detection, absoption and complex potentials}

Quantum Jumps / trajectories / Monte Carlo Wave Function.

Ch. 6 Vol. 2 of Time in QM (G.C. Hegerfeldt). Or similarly
\begin{itemize}
  \item \url{https://arxiv.org/pdf/quant-ph/9710027.pdf}
  \item Gerhard C Hegerfeldt and Dirk G Sondermann 1996 Quantum Semiclass. Opt. 8 121
  \item  Almut Beige et al 1996 Quantum Semiclass. Opt. 8 999
  \item VOLUME
  \item 68,NUMBER 5 PRL 1992 Wave-Function Approach to Dissipative Processes in Quantum Optics
  \item ``How to reset an atom after photon detection'' 
  \item BEST: \url{http://www.theorie.physik.uni-goettingen.de/~hegerf/trieste.pdf}
  \item OR: textbooks and monographs: Carmichael, Scully, Milburn, Gerry
\end{itemize}


\section{Absorption by a detector, time of arrival}
Ref: \cite{RuschhauptAbsorption}.

Qubit PW vs Ruschhaupt detector on 2-level system,
particularly section “EMISSION FROM A TWO-LEVEL SYSTEM”.

Idea: the ``deviation'' from continuity equation as ``absorption event'',
normalizable in $L^2(\mathbb{R}^4)$.

\cite{TQM2} (Kijowski and Detector) is cited and summarized well in
\cite{Halliwell_Detector}.

\subsection{Use open quantum systems theory in ``Decoherence and measurement'' chapter}
Schmidt decompositions, spatial and temporal states in
$\hilb{H}_T$ and $\hilb{H}_S$
are described as density operators
(mixed states). What if there isn't a ``perfect entanglment'' between space and time.

\section{Entanglement and decoherence (Arrow of time)}
See also \cite{EntanglementVsDecoherence}.

Decoherence is an irreversible process, it also happens in measurement.

According to Marletto and Vedral, arrow of time is increase in Entanglement
between the clock and the rest.

So, there seems to be a contradiction: is entanglement ``decreasing''
(i.e. destroyed by decoherence) with time
or increasing?

We can avoid the contradiction saying that
entanglement between two finite systems is
destroyed while the entanglement of each of them with the universe
is increasing.




\section{``Harmonic clocks''}

TODO: use the harmonic oscillator in \cite{HarmonicClocks}
as a PaW clock for the same packet that is measured in
Ruschhaupt's detector model.

Therein, fading wave function: is minus derivative an event?
L4 normalized?


\section{Misc}

Time of arrival and clocks: again, \cite{YearsleyHalliwell_Clocks}.
Which maybe suggests we should not wory too much of $H\ket{\Psi} = 0$. 

We don't. 

BUT please note \cite{YearsleyHalliwell_Clocks} uses a clock that is
\emph{coupled} with the system, while in PaW they are ``only'' entangled.
So their calculation may be unnecessarily complicated.
Maybe the weakjly coupling case can be used?

Other systems of interest: decays. Prvanovic new.

Reference \cite{ConnesRovelliThermo}.

Relate with John Goold's works? The ancilla as a clock? --- Topical Review

Markovianity, histories.

Lloyd on arXiv: from clock to cloners; erasing; scrambling (as in Goold).

Lloyd on decoherent histories (Gellman, Hartle?).

Dechoerence / irreversibility / measurement.

Vedral / Lloyd. Discord.

Measuring entanglement: Quantification of Concurrence via Weak Measurement: 1611.00149.

Marletto/Vedral on Arrow of time. Arrow of time as increasing entanglement.

Arrow of time: 

\url{https://www.wired.com/2014/04/quantum-theory-flow-time/}

\url{https://en.wikipedia.org/wiki/Loschmidt%27s_paradox}

\url{https://www.quantamagazine.org/20160119-time-entanglement/}

\url{https://arxiv.org/pdf/1702.07706.pdf} \textit{The second law of thermodynamics at the microscopic scale}
Thibaut Josset,
Aix Marseille Univ. (David).

Maxwell's demon: https://arxiv.org/pdf/1702.05161.pdf

\subsection{and paths}

Both \cite{YearsleyHalliwell_Clocks} and \cite{Gambini_PW}
reason in terms of paths and actions, maybe Feynmann stuff
in following chapter... and maybe conistent historiesapproach can help
towards linking PaW and ToA?

Also \url{http://quantum.phys.cmu.edu/CHS/CHS_transp.pdf}.

\subsection{Can a POVM on a system, if then we see it as part of a bipartite one,
equivalent to a PVM on the other, entangled, system?}

TODO: backflow effect in both models.

This will show an equivalence of models based on POVM with the Page and Wootters...?

Well, yes.

From \cite{PreskillNotes}, Ch.3 
\begin{quotation}
We have seen that
a pure state of the bipartite system AB may behave like a mixed state
when we observe subsystem A alone, and that an orthogonal measurement
of the bipartite system can realize a (nonorthogonal) POVM on A alone.
\end{quotation}

and

\begin{quotation}
A POVM in $H_A$ can be realized as a unitary transformation on the tensor
product $H_A \otimes H_B$, followed by an orthogonal measurement in $H_B$.
\end{quotation}

The same chapter talks about quantum operations, quantum channels and Kraus opertators.

We might want to look at exponential decay from \url{https://arxiv.org/abs/1704.07236},
then compare with exponential decay with P and W using Lloyd Giovannetti and Maccone (ref).

\subsubsection{Purification}

See https://arxiv.org/pdf/quant-ph/0512125.pdf, P-W time as a purifying ancilla
of the (Kijowski?) time.

\subsubsection{4-partite universe?}
\begin{itemize}
  \item{The system being measured/detected}
  \item{The Ruschhaupt detector --- which does not measure time, but whose detection happens at a certain time}
  \item{The Page and Wootters clock, entangled with the system and/or the detector}
  \item{The rest of the Universe, aka the Environment, aka the Termal Bath or Reservoir}
\end{itemize}

Can any of the above be identified? If the lab is isolated enough,
the detector is the only macro object and can act as a Universe/bath/environment/reservoir\dots?
