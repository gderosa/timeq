\section{Unitary evolution, absorption, events}

A vector $\dket{\Psi}$ in $\hilb{H}_T \ox \hilb{H}_S$,
satisfying \eqref{eq:pwHamiltonian} and \eqref{eq:Wheeler-DeWitt},
encodes the whole (unitary) time evolution of a system.
\begin{equation}\label{eq:pwexpansion}
  \dket{\Psi} =
    \int \dd{t} \ket{t}_T \ox \ket{\psi(t)}_S =
    \int \dd{t}\dd[3]{\vec{r}} \ \Psi(t; \vec{r}) \; \ket{t}_T \ox \ket{\vec{r}}_S
    \,  \text{.}
\end{equation}
We know $\ket{t}_T \ox \ket{\vec{r}}_S$ is an othonormal basis of $\hilb{H}_T \ox \hilb{H}_S$, therefore
\begin{equation}
  \norm{\dket{\Psi}}^2 =
    \int \dd{t}\dd[3]{\vec{r}} \ \abs{\Psi(t; \vec{r})}^2 =
    \int \dd{t} \int \dd[3]{\vec{r}} \ \abs{\Psi(t; \vec{r})}^2 =
    \int \dd{t} 1 \rightarrow +\infty
    \,  \text{,}
\end{equation}
which means that such $\dket{\Psi}$ is an \term{improper} vector of $\hilb{H}_T \ox \hilb{H}_S$.

Proper (i.e. normalizable) states are described in \cite{Lloyd:Time} as well, by replacing (or generalizing)
the \eqref{eq:pwexpansion} with
\begin{equation}\label{eq:pwphi}
  \dket{\Psi} =
    \int \dd{t} \phi(t) \ket{t}_T \ox \ket{\psi(t)}_S \, \text{.}
\end{equation}
If the function $\phi \in \mathcal{L}^2(\mathbb{R})$,
then $\dket{\Psi}$ is a proper element of the product space,
and $\norm{\dket{\Psi}}^2 = \int \dd{t} \abs{\phi(t)}^2$.

We will consider some even more general case than \cite{Lloyd:Time},
for example when $\phi(t)$ is not a constant, but it's not square-integrable either,
or it's such only in the half-line $\mathbb{R}^{+}$,
which will be useful in some problems.

The case of non-normalizable $\dket{\Psi}$ in \eqref{eq:pwphi},
with normalized $\ket{\psi(t)}_S$ $\forall t \in \mathbb{R}$,
describes the unitary evolution, as seen throughout Chapter \ref{ch:pw}.
As observed in \cite{Maccone:QGR},
``%
  Quantum mechanics is formulated in terms of \emph{systems},
  typically limited in space but infinitely extended in time%
''.
If the state vector is \emph{conditioned} at a particular time $t$,
it holds $\norm{_{T}\bradket{t}{\Psi}}_S = \norm{\ket{\psi(t)}}_S = 1$,
meaning that, at each $t$,
\emph{the particle must certainly be in some (one) point in space}.

A normalized $\dket{\Psi}$ in the whole $\hilb{H}_T \ox \hilb{H}_S$,
instead,
can be interpreted as a total probability of~$1$ in both space and time combined.
It's an \term{event} that, as such, must certainly be in some point in space
\emph{and} at some individual time (in terms of outcome of an idealized measurement).
A ``typical'' example of localized ``event wave packet'' would be
therefore represented by
a \emph{4-dimensional} gaussian wave function,
in analogy to well known examples of purely spatial gaussian states
in quantum mechanics and quantum optics.

As an intermediate case, we may want to consider e.g.
$\phi(t) = 1$ for $t < 0 $ and a definitive decreasing behavior
in the positive half-line with $\lim_{t \to +\infty} \phi(t) = 0$.
In terms of ``spatial quantum mechanics'', we will compare this case with
models of detection by absoption \parencite{RuschhauptAbsorption},
where the evolution is not unitary and
the norm of $\ket{\psi(t)}$ vanishing with $t \to +\infty$
as a consequence of a \term{complex potential}
i.e.
an Hamiltonian corrected by a an anti-hermitian term
that models the detector.

A non hermitian Hamiltonian is justified as a computation method
to simplify the study of some open systems: the evolution of mixed
states is derived without explicit reference to density operators
or master equations, but resolving equations that are formally
identical to those of pure states,
i.e. in terms of
Schr{\"o}dinger equations and wave functions,
with the non-hermitian term in the Hamiltonian
to account for the non-unitarity of the evolution
(%
  \cite[Ch. 6]{TQM2};
  \cite{Wave-function_approach};
  \cite{HowToResetAnAtom};
  \cite{TheQuantumJumpApproach};
  \cite[\S 8.5 ``The `quantum jump' approach to damping'' and particularly \S 8.5.2 ``The wave function Monte Carlo approach to damping'']{ScullyZubairy};
  \cite[\S 6.7 ``Stochastic Unravellings'']{WallsMilburn}%
).

\section{Detection, absoption and complex potentials}

Quantum Jumps / trajectories / Monte Carlo Wave Function.

Ch. 6 Vol. 2 of Time in QM (G.C. Hegerfeldt). Or similarly
\begin{itemize}
  \item \url{https://arxiv.org/pdf/quant-ph/9710027.pdf}
  \item Gerhard C Hegerfeldt and Dirk G Sondermann 1996 Quantum Semiclass. Opt. 8 121
  \item  Almut Beige et al 1996 Quantum Semiclass. Opt. 8 999
  \item VOLUME
  \item 68,NUMBER 5 PRL 1992 Wave-Function Approach to Dissipative Processes in Quantum Optics
  \item ``How to reset an atom after photon detection'' 
  \item BEST: \url{http://www.theorie.physik.uni-goettingen.de/~hegerf/trieste.pdf}
  \item OR: textbooks and monographs: Carmichael, Scully, Milburn, Gerry
\end{itemize}

Qubit PW vs Ruschhaupt detector on 2-level system,
particularly section “EMISSION FROM A TWO-LEVEL SYSTEM”.

Idea: use section B ``Measurement'' of \cite{Lloyd:Time}: detector as (binary) measument device.

Idea: the ``deviation'' from continuity equation as ``absorption event'',
normalizable in $L^2(\mathbb{R}^4)$.

\cite{TQM2} (Kijowski and Detector) is cited and summarized well in
\cite{Halliwell_Detector}.

``Philospher'': \url{https://arxiv.org/abs/1704.07236}.

\subsection{Use open quantum systems theory in ``Decoherence and measurement'' chapter}
Schmidt decompositions, spatial and temporal states in
$\hilb{H}_T$ and $\hilb{H}_S$
are described as density operators
(mixed states). What if there isn't a ``perfect entanglment'' between space and time.
