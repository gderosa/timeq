\section{Unitary evolution \emph{versus} absorption}

A vector $\dket{\Psi}$ in $\hilb{H}_T \ox \hilb{H}_S$
satisfying \eqref{eq:pwHamiltonian} and \eqref{eq:Wheeler-DeWitt}
encodes the unitary time evolution of a system.
\begin{equation}\label{eq:pwexpansion}
  \dket{\Psi} =
    \int \dd{t} \ket{t}_T \ox \ket{\psi(t)}_S
    \,  \text{.}
\end{equation}
Clearly, it is $\braket{\Psi(t)}_S = 1$ at every time $t$.

\if{todo}

\subsection{TODO: Events (there are some redundancies)}

A vector $\dket{\Psi}$ in $\hilb{H}_T \ox \hilb{H}_S$,
satisfying \eqref{eq:pwHamiltonian} and \eqref{eq:Wheeler-DeWitt},
encodes the whole (unitary) time evolution of a system.
\begin{equation}\label{eq:pwexpansion}
  \dket{\Psi} =
    \int \dd{t} \ket{t}_T \ox \ket{\psi(t)}_S =
    \int \dd{t}\dd[3]{\vec{r}} \ \Psi(t; \vec{r}) \; \ket{t}_T \ox \ket{\vec{r}}_S
    \,  \text{.}
\end{equation}
We know $\ket{t}_T \ox \ket{\vec{r}}_S$ is an othonormal basis of $\hilb{H}_T \ox \hilb{H}_S$, therefore
\begin{equation}
  \norm{\dket{\Psi}}^2 =
    \int \dd{t}\dd[3]{\vec{r}} \ \abs{\Psi(t; \vec{r})}^2 =
    \int \dd{t} \int \dd[3]{\vec{r}} \ \abs{\Psi(t; \vec{r})}^2 =
    \int \dd{t} 1 \rightarrow +\infty
    \,  \text{,}
\end{equation}
which means that such $\dket{\Psi}$ is an \term{improper} vector of $\hilb{H}_T \ox \hilb{H}_S$.

Proper (i.e. normalizable) states are described in \cite{Lloyd:Time} as well, by replacing (or generalizing)
the \eqref{eq:pwexpansion} with
\begin{equation}\label{eq:pwphi}
  \dket{\Phi} =
    \int \dd{t} \phi(t) \ket{t}_T \ox \ket{\psi(t)}_S \, \text{.}
\end{equation}
If the function $\phi \in \mathcal{L}^2(\mathbb{R})$,
then $\dket{\Psi}$ is a proper element of the product space,
and $\norm{\dket{\Psi}}^2 = \int \dd{t} \abs{\phi(t)}^2$.

We will consider some even more general case than \cite{Lloyd:Time},
where $\phi(t)$ is replaced by a $f(t)$ that is not a constant,
but it's not square-integrable either,
which will be useful in some problems.

The case of non-normalizable $\dket{\Psi}$ in \eqref{eq:pwphi},
with normalized $\ket{\psi(t)}_S$ $\forall t \in \mathbb{R}$,
describes the unitary evolution, as seen throughout Chapter \ref{ch:pw}.
As observed in \cite{Maccone:QGR},
``%
  Quantum mechanics is formulated in terms of \emph{systems},
  typically limited in space but infinitely extended in time%
''.
If the state vector is \emph{conditioned} at a particular time $t$,
it holds $\norm{_{T}\bradket{t}{\Psi}}_S = \norm{\ket{\psi(t)}}_S = 1$,
meaning that, at each $t$,
\emph{the particle must certainly be in some (one) point in space}.

A normalized $\dket{\Psi}$ in the whole $\hilb{H}_T \ox \hilb{H}_S$,
instead,
can be interpreted as a total probability of~$1$ in both space and time combined.
It's an \term{event} that, as such, must certainly be in some point in space
\emph{and} at some individual time (in terms of outcome of an idealized measurement).
A ``typical'' example of localized ``event wave packet'' would be
therefore represented by
a \emph{4-dimensional} gaussian wave function,
in analogy to well known examples of purely spatial gaussian states
in quantum mechanics and quantum optics.

\fi

As an intermediate case, we may want to consider
\begin{equation}\label{eq:pwf}
  \dket{\mathsf{F}} =
    \int \dd{t} f(t) \ket{t}_T \ox \ket{\psi(t)}_S \, \text{.}
\end{equation}
formally identical to \eqref{eq:pwphi},
but $f$ is not square-integrable, it is instead:
\begin{equation}
  \lim_{t\to -\infty} f(t) = 1 \, \text{;} \quad
  \lim_{t\to +\infty} f(t) = 0
\end{equation}
In terms of ``spatial quantum mechanics'', we will compare this case with
models of detection by absorption \parencite{RuschhauptAbsorption},
where the evolution is not unitary and
the norm of
$\ket{\psi(t)}_{\mathrm{Kiukas}} \eqbydef f(t)\ket{\psi(t)}$
is decreasing and vanishing with time
as a consequence of a \term{complex potential}
i.e.
an Hamiltonian corrected by a an anti-hermitian term
that models the detector.

A non hermitian Hamiltonian is justified as a computation method
to simplify the study of some open systems: the evolution of mixed
states is derived without explicit reference to density operators
or master equations, but resolving equations that are formally
identical to those of pure states,
i.e. in terms of
Schr{\"o}dinger equations and wave functions,
with the non-hermitian term in the Hamiltonian
to account for the non-unitarity of the evolution
(%
  \cite[Ch. 6]{TQM2};
  \cite{Wave-function_approach};
  \cite{HowToResetAnAtom};
  \cite{TheQuantumJumpApproach};
  \cite[\S 8.5 ``The `quantum jump' approach to damping'' and particularly \S 8.5.2 ``The wave function Monte Carlo approach to damping'']{ScullyZubairy};
  \cite[particularly \S 6.7.1 ``Simulating Quantum Trajectories'']{WallsMilburn}%
).

In comparing the results in (or attempting to draw some common conclusion from)
\cite{Lloyd:Time} and \cite{RuschhauptAbsorption}
---or matching the \eqref{eq:pwphi} and \eqref{eq:pwf} if one will---
different approaches are conceivable:
\begin{enumerate}
  \item
    Setting
    \begin{equation}
      \phi = f \, \text{,}
    \end{equation}
    after checking that relevant results in \cite{Lloyd:Time} are valid in more general cases than
    $\phi \in \mathcal{L}^2(\mathbb{R})$
  \item
    Honoring the ``event'' semantics proposed above, and the notion of \term{absorption event}:
    something that has its peak probability (density)
    when the norm of $\ket{\psi(t)}$ decreases most rapidly; therefore if we can assume
    $\dv{\abs{f}^2}{t} \leq 0 \; \forall t$, the probability amplitude density
    (or the wavefunction of absorption event) will be a $\phi$ such that
    \begin{equation}\label{eq:absorptionEvent}
      \abs{\phi(t)}^2 = -\dv{\abs{f}^2}{t}
    \end{equation} 
\end{enumerate}

The requirement, implicit  in \eqref{eq:absorptionEvent},
that $f$ never locally increases in modulus
is not restrictive, or it's not more restrictive than simply stating
that the detector, at most, absorbs, but never \emph{emits}
(in the model of \cite{RuschhauptAbsorption}).
