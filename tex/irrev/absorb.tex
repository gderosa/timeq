\section{Unitary evolution, absorption, events}

A vector $\dket{\Psi}$ in $\hilb{H}_T \ox \hilb{H}_S$,
satisfying \eqref{eq:pwHamiltonian} and \eqref{eq:Wheeler-DeWitt},
encodes the whole (unitary) time evolution of a system.

\begin{equation}
  \dket{\Psi} = \int \dd{t} \ket{t}_T \ox \ket{\psi(t)}_S = \int \dd{t}\dd[3]{\va{r}} \psi_t(\va{r}) \ket{t}_T \ox \ket{\va{r}}
\end{equation}

AND the norm does not converge

\section{Detection, absoption and complex potentials}

Quantum Jumps / trajectories / Monte Carlo Wave Function.

Ch. 6 Vol. 2 of Time in QM (G.C. Hegerfeldt). Or similarly
\begin{itemize}
  \item \url{https://arxiv.org/pdf/quant-ph/9710027.pdf}
  \item Gerhard C Hegerfeldt and Dirk G Sondermann 1996 Quantum Semiclass. Opt. 8 121
  \item  Almut Beige et al 1996 Quantum Semiclass. Opt. 8 999
  \item VOLUME
  \item 68,NUMBER 5 PRL 1992 Wave-Function Approach to Dissipative Processes in Quantum Optics
  \item ``How to reset an atom after photon detection'' 
  \item BEST: \url{http://www.theorie.physik.uni-goettingen.de/~hegerf/trieste.pdf}
  \item OR: textbooks and monographs: Carmichael, Scully, Milburn, Gerry
\end{itemize}


\section{Absorption by a detector, time of arrival}
Ref: \cite{RuschhauptAbsorption}.

Qubit PW vs Ruschhaupt detector on 2-level system,
particularly section “EMISSION FROM A TWO-LEVEL SYSTEM”.

Idea: use section B ``Measurement'' of \cite{Lloyd:Time}: detector as (binary) measument device.

Idea: the ``deviation'' from continuity equation as ``absorption event'',
normalizable in $L^2(\mathbb{R}^4)$.

\cite{TQM2} (Kijowski and Detector) is cited and summarized well in
\cite{Halliwell_Detector}.

``Philospher'': \url{https://arxiv.org/abs/1704.07236}.

\subsection{Use open quantum systems theory in ``Decoherence and measurement'' chapter}
Schmidt decompositions, spatial and temporal states in
$\hilb{H}_T$ and $\hilb{H}_S$
are described as density operators
(mixed states). What if there isn't a ``perfect entanglment'' between space and time.
