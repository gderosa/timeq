\section{Non-unitary dynamics in $\hilb{H}_S$, in Page-Wootters terms}

First of all,
when the unproper vector of $\hilb{H}_T \ox \hilb{H}_S$
\begin{equation}
  \dket{\Psi} = \int dt \ket{t}_{T} \ox \ket{\psi(t)}_{S}
\end{equation}
is replaced by a \emph{proper}, normalized $\dket{\Phi}$ (eq. \ref{eq:pwphi}),
what should the equation
\begin{equation}
  \qty(\hbar\hat{\Omega} \ox \idop_S + \idop_T \ox \hat{H}_S)\dket{\Psi} = 0
\end{equation}
be replaced with?

As $\setof{\ket{t}_T}$ is an eigenbasis of $\hat{T}$, the \eqref{eq:pwphi}
contains the definition of an operator function in $\hilb{H}_T$,
and can then be reformulated as:\footnote{
  Or, more precisely, $\dket{\Psi} = \qty( \phi(\hat{T}) \ox \idop_S ) \dket{\Psi}$,
  but we will omit, in some cases,
  tensor product by identity operators
  when it's obvious.
}
\begin{equation}
  \dket{\Psi} = \phi(\hat{T})\dket{\Psi} \, \text{.}
\end{equation}

\begin{equation}
  \dket{\Psi} = \phi(\hat{T})\Dket{\frac{1}{2}} \, \text{.}
\end{equation}

\begin{equation}
  \dket{\Psi} = \phi(\hat{T})\left.\left| \frac{1}{2} \right\rangle\right\rangle \, \text{.}
\end{equation}
