\section{Page--Wootters and detector absorption models}\label{sec:absorption+pw}

Within the Page--Wootters framework, a non-hermitian term in the ``Hamiltonian''
is a consequence of non-unitary ``evolution''.
On the contrary, in detection models based on absorption and complex potentials
\parencite{RuschhauptAbsorption}, non-unitary evolution is a consequence
of such non-hermitian term in the modified Hamiltonian.
Specifically, the Hamiltonian $\hat{H}$ is replaced by a $\hat{H} - i\hat{D}$
(with $\hat{D}$ self-adjoint, bounded, positive ---\cite{RuschhauptAbsorption})
and, consequently:
\begin{equation}\label{eq:schrod_complex_pot}
  \hat{H} \ket{\psi(t)} = i\hbar\dv{t}\ket{\psi(t)} +i\hat{D}\ket{\psi(t)} \text{.}
\end{equation}
Similarly to the Page--Wootters model, $\ket{\psi(t)}$ does not conserve its norm in time.

However, the two models are conceptually different, and relating them will require some additional steps.

A normalized Page--Wootters ``position-time wavepacket'' may be used to describe the event
of being detected (or being absorbed). It's expected to be peaked around the time when
the absorption by the detector is maximum.

In the detector model of \cite{RuschhauptAbsorption}, the detection
by absorption
corresponds to the \emph{decrease} in norm of the wavefunction.

Therefore we expect the following relation to be true:
\begin{equation}\label{eq:pwkiukas}
  \abs{\phi(t)}^2 = -\dv{t}\norm{\psi_{\text{Kiukas}}(t)}^2 \text{,}
\end{equation}
both sides of which indicate probability of arrival at time $t$.
Here the function (of time) $\phi$ is to be intended in the sense of
\eqref{eq:pwphi} and \S \ref{non-unitary-pw}.

Interestingly, \cite{RuschhauptAbsorption} provides a solution of \eqref{eq:pwkiukas}.
Despite being not based on the Page--Wootters model, eq. 9 therein
equates the squared norm of a ``time representation'' wavefunction
to the antiderivative of the squared norm of the ``absorbed wavefunction''.
It reads:
\begin{quote}
  We will associate with any wave function $\psi \in \hilb{H}$
  another wave function $\hat{\psi}$,
  which is a function of time, so that
  $\abs{\hat{\psi}(t)}^2$
  is the arrival probability density. In other words,
  $\hat{\psi}$ is a wave function in a time representation. For each
  $t$, $\hat{\psi}(t)$ lies in the original Hilbert space $H$.
\end{quote}
Therefore we ``translate'' $\hat{\psi}(t)$ into $\phi(t)\ket{\psi(t)}_S$
and, consequently, $\abs{\hat{\psi}(t)}^2$ into $\abs{\phi(t)}^2$,
in the language of the Page--Wootters model and within the notation
adopted.

Using eq. 8 in \cite{RuschhauptAbsorption} and translating into our notation we have:
\begin{equation}
  \phi(t)\ket{\psi(t)} =
  \begin{cases}
    \sqrt{\frac{2}{\hbar}} \hat{D}^{1/2} \ket{\psi_{\text{Kiukas}}(t)}_S &\text{ if } t > 0 \\
    0 &\text{ otherwise. }
  \end{cases}
\end{equation}
Where at $t \le 0$ the interaction with the detector is yet to come, and therefore
$\hat{D} \ket{\psi_{\text{Kiukas}}(t)} = 0$, thus avoiding the apparent discontinuity.


\subsection{Application: two-level system}

The detector model of \cite{RuschhauptAbsorption} has been studied therein
with a simple application: a two-level system. In Page and Wootters terms,
this would corrspond to a bi-dimensional $\hilb{H}_S$, but a continuous
spectrum of $\hat{T}$ in $\hilb{H}_T$. The paper is \emph{not} based on
the Page--Wootters model, indeed the purpose of this section is a comparison
with such model, using the results of \S \ref{sec:absorption+pw}.

By setting, out of convenience, $\hbar = 1$, and directly considering the parameters
that minimize the uncertainty, we have a non-hermitian ``hamiltonian''
$K = \hat{H} - i\hat{D}$ with
\begin{equation}
  H = ; D =
\end{equation}