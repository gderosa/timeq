\section{Page-Wootters and detector absorption models}\label{sec:absorption+pw}

Within the Page-Wootters framework, a non-hermitian term in the ``Hamiltonian''
is a consequence of non-unitary ``evolution''.
On the contrary, in detection models based on absorption and complex potentials
\parencite{RuschhauptAbsorption}, non-unitary evolution is a consequence
of such non-hermitian term in the modified Hamiltonian.
Specifically, the Hamiltonian $\hat{H}$ is replaced by a $\hat{H} - i\hat{D}$
(with $\hat{D}$ self-adjoint, bounded, positive ---\cite{RuschhauptAbsorption})
and, consequently:
\begin{equation}\label{eq:schrod_complex_pot}
  \hat{H} \ket{\phi(t)} = i\hbar\dv{t}\ket{\phi(t)} +i\hat{D}\ket{\phi(t)} \text{.}
\end{equation}
Similarly to the Page-Wootters model, $\ket{\phi(t)}$ does not conserve its norm in time.

However, a comparison requires some extra caution.
In the P-W model, $\ket{\psi(t)} = \ket{\phi(t)}/\phi(t)$
is the corresponding unitary evolution vector. In the detector model,
it is generally false
that the unitarily evolved state, at a given $t$, is proportional to
the non-unitary one. In other words,
the detector (i.e. replacing $\hat{H}$ with $\hat{H}-i\hat{D}$)
affects the evolved state non-trivially:
it doesn't only change the norm (and possibly a phase factor)
with respect to what would be the evolution without the detector
(i.e. only due to $\hat{H}$).

Namely, we will show that 
\[
  \ket{\phi(t)} = \phi(t) \ket{\psi}
\]
will need to be replaced with
\[
  \ket{\phi(t)} = e^{-\hat{D}t/\hbar}\ket{\psi(t)} \, \text{.}
\]



\subsection{Application: two-level system}

The detector model of \cite{RuschhauptAbsorption} has been studied therein
with a simple application: a two-level system. In Page and Wootters terms,
this would corrspond to a bi-dimensional $\hilb{H}_S$, but a continuous
spectrum of $\hat{T}$ in $\hilb{H}_T$. The paper is \emph{not} based on
the Page-Wootters model, indeed the purpose of this section is a comparison
with such model, using the results of \S \ref{sec:absorption+pw}.

By setting, out of convenience, $\hbar = 1$, and directly considering the parameters
that minimize the uncertainty, we have a non-hermitian ``hamiltonian''
$K = \hat{H} - i\hat{D}$ with
\begin{equation}
  H = ; D =
\end{equation}