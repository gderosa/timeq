\section{Comparison with detector absorption models}

By ``conditioning'' the \eqref{eq:nonu_pwwdw} at each time $t$
---in other words: by inner-multiplying it on the left with the partial bra\footnote{
  Partial in the sense of open quantum systems, or as a part of a bipartite system.
  See Definition \ref{def:pBra}.
}
$\prescript{}{T}{\bra{t}}$~---
a modified Schr{\"o}dinger equation
can be derived,
as a variant of \eqref{eq:schrod_from_pw}.

In the position representation of $\hilb{H}_S$ (and ``tracing out'' time):
\begin{equation}
  \hat{H} \ket{\phi(t)} = i\hbar\dv{t}\ket{\phi(t)} -i\hbar\frac{\dv{\phi}{t}}{\phi(t)}\ket{\phi(t)} \text{.}
\end{equation}

Here the subscript $S$ has been removed and it's assumed.
$\ket{\phi(t)} = \prescript{}{T}{\bradket{t}{\Phi}}$
is a state vector that does not conserve its norn in its time evolution.
In a sense, its norm is ``absorbed'' in the detection process.

Within the Page-Wootters framework, a non-hermitian term in the ``Hamiltonian''
is a consequence of non-unitary ``evoution''.
On the contrary, in detection models based on absorption and complex potentials
\parencite{RuschhauptAbsorption}, non-unitary evolution is a consequence
of such non-hermitian term in the modified Hamiltonian.
Specifically, the Hamiltonian $\hat{H}$ is replaced by a $\hat{H} - i\hat{D}$
and, consequently:
\begin{equation}
  \hat{H} \ket{\phi(t)} = i\hbar\dv{t}\ket{\phi(t)} +iD\ket{\phi(t)} \text{.}
\end{equation}
Similarly to the Page-Wootters model, $\ket{\phi(t)}$ does not conserve its norm in time.

In the P-W model, $\ket{\psi(t)} = \ket{\phi(t)}/\phi(t)$
is the corresponding unitary evolution vector
(or ``what would have happened without an absorbing detector'').