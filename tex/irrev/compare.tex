\section{Page-Wootters and detector absorption models}

By ``conditioning'' the \eqref{eq:nonu_pwwdw_f} at each time $t$
---in other words: by inner-multiplying it on the left with the partial bra\footnote{
  Partial in the sense of open quantum systems
  i.e. as a part of a bipartite system.
  See Definition \ref{def:pBra}.
}
$\prescript{}{T}{\bra{t}}$~---
a modified Schr{\"o}dinger equation
can be derived,
as a variant of \eqref{eq:schrod_from_pw}.

In the position representation of $\hilb{H}_S$ (and ``tracing out'' time):
\begin{equation}\label{eq:schrod_pw_nonunitary}
  \hat{H} \ket{f(t)} = i\hbar\dv{t}\ket{f(t)} -i\hbar\frac{\dot{f}(t)}{f(t)}\ket{f(t)} \text{.}
\end{equation}

Here the subscript $S$ has been removed and it's assumed.
$\ket{f(t)} = \prescript{}{T}{\bradket{t}{f}}$
is a state vector that does not conserve its norn in its time evolution.
In a sense, its norm is ``absorbed'' in the detection process.

Within the Page-Wootters framework, a non-hermitian term in the ``Hamiltonian''
is a consequence of non-unitary ``evolution''.
On the contrary, in detection models based on absorption and complex potentials
\parencite{RuschhauptAbsorption}, non-unitary evolution is a consequence
of such non-hermitian term in the modified Hamiltonian.
Specifically, the Hamiltonian $\hat{H}$ is replaced by a $\hat{H} - i\hat{D}$
(with $\hat{D}$ self-adjoint, bounded, positive ---\cite{RuschhauptAbsorption})
and, consequently:
\begin{equation}\label{eq:schrod_complex_pot}
  \hat{H} \ket{f(t)} = i\hbar\dv{t}\ket{f(t)} +i\hat{D}\ket{f(t)} \text{.}
\end{equation}
Similarly to the Page-Wootters model, $\ket{f(t)}$ does not conserve its norm in time.

In the P-W model, $\ket{\psi(t)} = \ket{f(t)}/f(t)$
is the corresponding unitary evolution vector
(or ``what would have happened without an absorbing detector'').
This relation, along with \eqref{eq:schrod_complex_pot} and \eqref{eq:schrod_pw_nonunitary},
yields a differential equation
\begin{equation}\label{eq:phi_diffeq}
  -\hbar\dot{f}(t)\ket{\psi(t)} = f(t) \hat{D} \ket{\psi(t)} \text{,}
\end{equation}
where $f(t)$ and $\hat{D}$ commute, as $\hat{D}$ only acts on the spatial degrees of freedom,
while $f(t)$ is, with respect to $\hilb{H}_S$, in fact a constant, albeit parametrized in $t$.

One may find $\ket{\psi(t)}$ first (by simply computing the unitary evolution),
then compute the effect of the detector by resolving the \eqref{eq:phi_diffeq}
which is then an ordinary differential equation in $f$. This is, of course, valid
under the assumption that the detector term only affects norm and phase,
but doesn't change the ``ray'' of the state vector in the Hilbert space. 

Multiplying on the left by $\bra{\psi(t)}$, rearranging, and with the same observations
on commutativity:
\begin{equation}\label{eq:phi_diffeq_simpler}
  -\hbar\frac{\dot{f}(t)}{f(t)} = \ev{D(t)} \text{,}
\end{equation}
where we recognize the logaritmic derivative of $f$: in particular,
if $\hat{D}$ is a constant, $f$ would undergo an exponential decay law.
A general solution is:
\begin{equation}\label{eq:f-evol}
  f(t) = f(t_0) \exp[ -\frac{1}{\hbar} \int_{t_0}^{t} \dd{t'} \ev{D(t')} ]
\end{equation}
where:
\begin{enumerate*}[label=\emph{\alph*})]
  \item
    the physics of the problem may tell us, for example, a time when $f(t_0) = 1$;
  \item
    a suitable interval $\qty[t_0, t]$ may be chosen where it is $f(t') \ne 0$
    to avoid singularities in \eqref{eq:phi_diffeq_simpler};
  \item
    again, $\ev{D(t)} = \mel{\psi(t)}{\hat{D}}{\psi(t)}$ is a known function
    once the unitarily evolved $\ket{\psi(t)}$ is determined.
\end{enumerate*}

The above provides thus a convenient method to separately resolve
the unitary evolution and the loss of normalization.
   