\section{Misc/Multi/Extras}

\url{https://arxiv.org/abs/1703.05876}
--- \emph{comment}: time measured and stored here
may be all classical information
so this paper may or may not be relevant for the topic.

But
``prototypes of clocks based on quantum principles,
such as entanglement and squeezing''
may make this interesting again, see reference therein.
They also cite Lloyd, Giovannetti and Maccone,
but a paper quite older than \cite{Lloyd:Time}.

\url{https://arxiv.org/abs/1603.02522}
\emph{Decoherence by spontaneous emission: a single-atom analog of superradiance}.
Decoherent histories, non-markovianity, open quantum systems.

\url{https://arxiv.org/abs/1007.2615} Time travel / Quantum CTC.

Carmichael et al. \cite{CarmichaelOQS2017} (Andreas's reading)
(non-markovianity).

Non-markovian, quantum-to-classical, open systems, David,
\url{https://arxiv.org/pdf/1703.09428.pdf}.

In his works, Zurek mentions:
DeWitt, Everett, gell_Mann, hartl, Many Worlds, consistent/decoherent histories:
idea: Lagrangian over a history? Principle of least action?

Zurek: ``Reduction of the Wavepacket: How Long Does it Take?'' (arxiv),
``quantum''' time? \cite{Zurek_Einselect} also mentions
``decoherence timescale''.

Von Neumann/Shannon entropy in measurement? Mention information problems
in quantum cosmology (where a quantum time is necessary)? Etc. etc.

\section{Time crystals}

References: \cite{crystal2,crystal3,crystal2012}.
