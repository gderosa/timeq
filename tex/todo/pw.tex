\section{Non-unitary dynamics in $\hilb{H}_S$, in Page--Wootters terms}
\label{non-unitary-pw}

First of all,
when the unproper vector of $\hilb{H}_T \ox \hilb{H}_S$
\begin{equation}
  \dket{\Psi} = \int dt \ket{t}_{T} \ox \ket{\psi(t)}_{S}
\end{equation}
is replaced by a \emph{proper}, normalized $\dket{\Phi}$ (eq. \ref{eq:pwphi}),
what should the equation
\begin{equation}\label{eq:pwwd}
  \qty(\hbar\hat{\Omega} \ox \idop_S + \idop_T \ox \hat{H}_S)\dket{\Psi} = 0
\end{equation}
be replaced with?

As $\setof{\ket{t}_T}$ is an eigenbasis of $\hat{T}$, the \eqref{eq:pwphi}
contains the definition of an operator function in $\hilb{H}_T$,
and can then be reformulated as:\footnote{
  Or, more precisely, $\dket{\Psi} = \qty( \phi(\hat{T}) \ox \idop_S ) \dket{\Psi}$,
  but we will omit, in some cases,
  tensor product by identity operators
  when it's obvious.
}
\begin{equation}
  \dket{\Phi} = \phi(\hat{T})\dket{\Psi} \, \text{.}
\end{equation}

We will also need the relation
\begin{multline}\label{eq:fcomm}
  \comm{\phi(\hat{T})}{\hat{\Omega}} = \dot{\phi}(\hat{T}) \comm{\hat{T}}{\hat{\Omega}}, \,
    \\
    \forall \, \hat{T}, \hat{\Omega} \text{ self-adjoint in } \hilb{H}_T
    \text{, with }
    \comm{\hat{T}}{\comm{\hat{T}}{\hat{\Omega}}} = 0
    \text{ and }
    \phi \in C^{\infty}(\mathbb{R}) \, \text{,} 
\end{multline}
where $\dot{\phi}$ is the first derivative of the function $\phi$.
In fact, this is proven in Appendix \ref{CommProp} for when $\phi$ is a polynomial,
and the proof can be easily extended to a more general case by series expansion.
For the canonical pair, this will just reduce to $\comm{\phi(\hat{T})}{\hat{\Omega}} = i \dot{\phi} (\hat{T})$.

Therefore:
\begin{multline}\label{eq:complex_dynam_deriv}
  \qty( \hbar\hat{\Omega} \ox \idop_S + \idop_T \ox \hat{H}_S ) \dket{\Phi} =
  \hat{\mathbb{J}}\dket{\Phi} =
  \hat{\mathbb{J}} \qty( \phi(\hat{T}) \ox \idop_S) \dket{\Psi} =
  \\
  \qty( \hbar\hat{\Omega}\phi(\hat{T}) \ox \idop_S + \phi(\hat{T}) \ox \hat{H}_S )\dket{\Psi} =
  \qty{
    \hbar\qty( \phi(\hat{T})\hat{\Omega} - \comm{\phi(\hat{T})}{\hat{\Omega}} ) \ox \idop_S +
    \phi(\hat{T}) \ox \hat{H}_S
  }\dket{\Psi} =
  \\
  \qty(
    \phi(\hat{T}) \ox \idop_S
  )
  \qty(
    - i \hbar \frac{\dot{\phi}}{\phi} (\hat{T}) \ox \idop_S
    + \hbar\hat{\Omega} \ox \idop_S
    + \idop_T \ox \hat{H}_S
  ) \dket{\Psi} =
  \\
  - \qty( i \hbar \frac{\dot{\phi}}{\phi} (\hat{T}) \ox \idop_S ) \dket{\Phi}
  \text{,}
\end{multline}
where we have used the \eqref{eq:pwwd} first, then the fact that $\phi(\hat{T})$ 
and $-i\hbar\frac{\dot{\phi}}{\phi}(\hat{T})$ commute.

Comparing the first and last term of \eqref{eq:complex_dynam_deriv} we have:
\begin{equation}\label{eq:nonu_pwwdw}
  \qty[
    \hbar\hat{\Omega} \ox \idop_S +
    i \hbar \frac{\dot{\phi}}{\phi}(\hat{T}) \ox \idop_S +
    \idop_T \ox \hat{H}_S
   ] \dket{\Phi} = 0
  \text{,}
\end{equation}
showing that a non-hermitian\footnote{
  Unless $\phi$  is a pure imaginary function.
}
term
$i \hbar \frac{\dot{\phi}}{\phi} (\hat{T})$
emerges as a correction to the dynamic constraint~$\hat{\mathbb{J}}$.
This is the same as eq. (27) of \cite{Lloyd:Time}, of which 
\eqref{eq:complex_dynam_deriv} is an analytical proof.
\section{Page--Wootters and detector absorption models}\label{sec:absorption+pw}

The detection-by-absoption model in \cite{RuschhauptAbsorption}
is based on a complex potential that, plugged into the Schr\"odinger equation,
leads to a non-unitary evolution of the state vector
(with loss of normalization).

Specifically, the Hamiltonian $\hat{H}$ is replaced by a $\hat{H} - i\hat{D}$
(with $\hat{D}$ self-adjoint, bounded, positive ---\cite{RuschhauptAbsorption})
and, consequently:
\begin{equation}\label{eq:schrod_complex_pot}
  \hat{H} \ket{\psi(t)} = i\hbar\dv{t}\ket{\psi(t)} +i\hat{D}\ket{\psi(t)} \text{.}
\end{equation}

Similarly to what seen in \S\ref{non-unitary-pw}, $\ket{\psi(t)}$ does not conserve its norm in time.
However, the two models are conceptually different.
One may wonder whether a normalized Page--Wootters ``position-time wavepacket''
can be used to describe the event
of being detected (or being absorbed). It's expected to be peaked around the time when
the absorption by the detector is maximum.

In the detector model of \cite{RuschhauptAbsorption}, the detection
by absorption
corresponds to the \emph{decrease} in norm of the wavefunction.

Therefore we expect the following relation to be true:
\begin{equation}\label{eq:pwkiukas}
  \abs{\phi(t)}^2 = -\dv{t}\norm{\psi_{\text{Kiukas}}(t)}^2 \text{,}
\end{equation}
both sides of which indicate probability of arrival at time $t$.
Here the function (of time) $\phi$ is to be intended in the sense of
\eqref{eq:pwphi} and \S \ref{non-unitary-pw}.

Interestingly, \cite{RuschhauptAbsorption} provides a solution of \eqref{eq:pwkiukas}.
Despite being not based on the Page--Wootters model, eq. 9 therein
equates the squared norm of a ``time representation'' wavefunction
to the antiderivative of the squared norm of the ``absorbed wavefunction''.
It reads:
\begin{quote}
  We will associate with any wave function $\psi \in \hilb{H}$
  another wave function $\hat{\psi}$,
  which is a function of time, so that
  $\abs{\hat{\psi}(t)}^2$
  is the arrival probability density. In other words,
  $\hat{\psi}$ is a wave function in a time representation. For each
  $t$, $\hat{\psi}(t)$ lies in the original Hilbert space $H$.
\end{quote}
Therefore we ``translate'' $\hat{\psi}(t)$ into $\phi(t)\ket{\psi(t)}_S$
and, consequently, $\abs{\hat{\psi}(t)}^2$ into $\abs{\phi(t)}^2$,
in the language of the Page--Wootters model and within the notation
adopted.

Using eq. 8 in \cite{RuschhauptAbsorption} and translating into our notation we have:
\begin{equation}\label{eq:phi_psi_kiukas}
  \phi(t)\ket{\psi(t)} =
  \begin{cases}
    \sqrt{\frac{2}{\hbar}} \hat{D}^{1/2} \ket{\psi_{\text{Kiukas}}(t)}_S &\text{ if } t > 0 \\
    0 &\text{ otherwise. }
  \end{cases}
\end{equation}
Where at $t \le 0$ the interaction with the detector is yet to come,
but so it is, as a limit, for small values of $t>0$,
in other terms
$\lim_{t \to 0^{+}} \norm{\hat{D} \ket{\psi_{\text{Kiukas}}(t)}} = 0$, thus avoiding the apparent discontinuity.

% \subsection*{\color{red} TODO}

% {
%   \color{red}
%   Figure out how to connect the above to what follows (or drop or separate if does not apply).

%   \scriptsize{
%     Hint: rather then the ``conspiracy theory''
%     (``I find an imaginary term both here and there'')
%     consider (time of) arrival as in \cite{Maccone:QMOT}.
%     This should bring to a normalized element of $\pwspace$ \dots
%   }
% }

\subsection{Application: two-level system}

In \cite{RuschhauptAbsorption}, an example application of the detector model
is provided for a two-level system.
In Page and Wootters terms,
this would corrspond to a bi-dimensional $\hilb{H}_S$, but a continuous
spectrum of $\hat{T}$ in $\hilb{H}_T$. The paper is \emph{not} based on
the Page--Wootters model, indeed the purpose of this section is a comparison
with such model, using the results of \S \ref{sec:absorption+pw}.

By setting, out of convenience, $\hbar = \omega = 1$
(with $\omega$ the characteristic frequency of the system),
and directly considering the parameters
that minimize the time--energy uncertainty product \parencite{RuschhauptAbsorption},
we have a non-hermitian ``hamiltonian''
$\mathcal{K} = \hat{H} - i\hat{D}$ with
\begin{equation}\label{eq:complexpot}
  \mathcal{K} = \hat{H} - i\hat{D} \repr
    \hbar\omega\left\{
      \left[\begin{matrix}0 & 1\\1 & 0\end{matrix}\right] -
      i \left[\begin{matrix}0 & 0\\0 & \gamma \end{matrix}\right]
    \right\}
\end{equation}
and $\gamma = 2\sqrt{2}$.