\section{Non-unitary dynamics in $\hilb{H}_S$, in Page--Wootters terms}
\label{non-unitary-pw}

First of all,
when the unproper vector of $\hilb{H}_T \ox \hilb{H}_S$
\begin{equation}
  \dket{\Psi} = \int dt \ket{t}_{T} \ox \ket{\psi(t)}_{S}
\end{equation}
is replaced by a \emph{proper}, normalized $\dket{\Phi}$ (eq. \ref{eq:pwphi}),
what should the equation
\begin{equation}\label{eq:pwwd}
  \qty(\hbar\hat{\Omega} \ox \idop_S + \idop_T \ox \hat{H}_S)\dket{\Psi} = 0
\end{equation}
be replaced with?

As $\setof{\ket{t}_T}$ is an eigenbasis of $\hat{T}$, the \eqref{eq:pwphi}
contains the definition of an operator function in $\hilb{H}_T$,
and can then be reformulated as:\footnote{
  Or, more precisely, $\dket{\Psi} = \qty( \phi(\hat{T}) \ox \idop_S ) \dket{\Psi}$,
  but we will omit, in some cases,
  tensor product by identity operators
  when it's obvious.
}
\begin{equation}
  \dket{\Phi} = \phi(\hat{T})\dket{\Psi} \, \text{.}
\end{equation}

We will also need the relation
\begin{multline}\label{eq:fcomm}
  \comm{\phi(\hat{T})}{\hat{\Omega}} = \dot{\phi}(\hat{T}) \comm{\hat{T}}{\hat{\Omega}}, \,
    \\
    \forall \, \hat{T}, \hat{\Omega} \text{ self-adjoint in } \hilb{H}_T
    \text{, with }
    \comm{\hat{T}}{\comm{\hat{T}}{\hat{\Omega}}} = 0
    \text{ and }
    \phi \in C^{\infty}(\mathbb{R}) \, \text{,} 
\end{multline}
where $\dot{\phi}$ is the first derivative of the function $\phi$.
In fact, this is proven in Appendix \ref{CommProp} for when $\phi$ is a polynomial,
and the proof can be easily extended to a more general case by series expansion.
For the canonical pair, this will just reduce to $\comm{\phi(\hat{T})}{\hat{\Omega}} = i \dot{\phi} (\hat{T})$.

Therefore:
\begin{multline}\label{eq:complex_dynam_deriv}
  \qty( \hbar\hat{\Omega} \ox \idop_S + \idop_T \ox \hat{H}_S ) \dket{\Phi} =
  \hat{\mathbb{J}}\dket{\Phi} =
  \hat{\mathbb{J}} \qty( \phi(\hat{T}) \ox \idop_S) \dket{\Psi} =
  \\
  \qty( \hbar\hat{\Omega}\phi(\hat{T}) \ox \idop_S + \phi(\hat{T}) \ox \hat{H}_S )\dket{\Psi} =
  \qty{
    \hbar\qty( \phi(\hat{T})\hat{\Omega} - \comm{\phi(\hat{T})}{\hat{\Omega}} ) \ox \idop_S +
    \phi(\hat{T}) \ox \hat{H}_S
  }\dket{\Psi} =
  \\
  \qty(
    \phi(\hat{T}) \ox \idop_S
  )
  \qty(
    - i \hbar \frac{\dot{\phi}}{\phi} (\hat{T}) \ox \idop_S
    + \hbar\hat{\Omega} \ox \idop_S
    + \idop_T \ox \hat{H}_S
  ) \dket{\Psi} =
  \\
  - \qty( i \hbar \frac{\dot{\phi}}{\phi} (\hat{T}) \ox \idop_S ) \dket{\Phi}
  \text{,}
\end{multline}
where we have used the \eqref{eq:pwwd} first, then the fact that $\phi(\hat{T})$ 
and $-i\hbar\frac{\dot{\phi}}{\phi}(\hat{T})$ commute.

Comparing the first and last term of \eqref{eq:complex_dynam_deriv} we have:
\begin{equation}\label{eq:nonu_pwwdw}
  \qty[
    \hbar\hat{\Omega} \ox \idop_S +
    i \hbar \frac{\dot{\phi}}{\phi}(\hat{T}) \ox \idop_S +
    \idop_T \ox \hat{H}_S
   ] \dket{\Phi} = 0
  \text{,}
\end{equation}
showing that a non-hermitian\footnote{
  Unless $\phi$  is a pure imaginary function.
}
term
$i \hbar \frac{\dot{\phi}}{\phi} (\hat{T})$
emerges as a correction to the dynamic constraint~$\hat{\mathbb{J}}$.
This is the same as eq. (27) of \cite{Lloyd:Time}, of which 
\eqref{eq:complex_dynam_deriv} is an analytical proof.

