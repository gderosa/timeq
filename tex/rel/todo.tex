\section{TODOs}

TODO: \cite{RealisticClocks}.

\cite{HarmonicClocks} concludes ``Classical clock can be described by an Hamiltonian linear in momentum''\dots
like in relativity?

TODO: \cite{Lloyd:Time} does not only deal with improper eigenstates in $\hilb{H}_T$
(full evolutions?)
but also normalized ones (in $\hilb{H}_T$! \emph{``Events''}?)

A Page and Wootters time of arrival is mentioned in \cite{Gambini_PW}.

\subsection{Time-of-arrival for a Klein-Gordon free particle}

See \cite{Galapon_KG}.

\subsection{Feynmann path stuff}

Sokolovski 1703.01966, Feynmann paths...\subsection{(dis)entanglement under gravity, decoherence, event formalism}

1703.08036 An experiment to test decoherence under gravity aka entangled photons undergoing different paths and how their entanglement is affected.
``Space QUEST mission proposal: Experimentally testing decoherence due to gravity''.

Are they getting entangled with the environment instead? (Merletto and Vedral).

The theoretical paper behind the space experiment: \url{https://arxiv.org/pdf/1406.3677.pdf}. Interestingly, it mentions 
\emph{event formalism}, and we thought about that: is an event something
representable as a proper vector in $\mathcal{L}^2(\mathbb{R}^4)$ --- where one of the dimensions is time?
TODO: deepen the event formalism if it's quantum.

Resume ``Quantum Statistical Gravity''? \url{https://arxiv.org/abs/1602.05707}.

``Fundamental decoherence from quantum gravity: a pedagogical review''
\url{https://arxiv.org/abs/gr-qc/0603090} ---
``fundamental loss of unitarity
that appears in quantum mechanics
due to the use of a physical apparatus to measure time''.

Closed timelike curves are also the subject of a paper by Lloyd (cite!).

``Deutsch argued that
the usual paradoxes associated with such solutions of general
relativity can be resolved by quantum mechanics''  in the reference above. But in the even formalism
\emph{spacetime is still a classical background!}. Event operators a parametrized by $t$\dots

\subsection{Prvanovic and P\&W}
In \cite{Prvanovic}, essentially the clock observable is the Hamiltonian.
The two example clocks are an harmonic oscillator and a free particle.
The harmonic oscillator features discrete time. Generally a time which is
{bounded from below}
is consistent with the Big Bang...

Prvanovic uses ``relativisitc'' constants...



