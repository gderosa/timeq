\section{Relativistic formulation: Klein-Gordon}\label{sec:KG}

The Klein-Gordon equation is the relativistic extension of the
(\emph{square of}) a wave equation for a free spinless particle.
Indeed, both sides have the dimension of
the square of an energy, and finding their square root
(in the operator sense) is non trivial task which was only resolved
with the Dirac equation, from which the
intrinsic \term{spin} (particularly spin $\hbar/2$) logically emerges,
rather then being artificially introduced in the theory
to comply with the phenomenology
(\cite{Greiner_Rel}, \cite[Ch. 8]{Sakurai2}, \cite{DiracEquation}, \cite[handout 2]{Webber_notes}).

Both equations did not have much fortune as first-quantized equations,
for conceptual difficulties and historical reasons, including the rise
of Quantum Field Theory and second quantization. They do have a fundamental
role though as field equations (before quantization methods are enacted).
\parencite{PeskinSchroeder}.

The purpose of the present work is, in a sense, promoting time to a quantum
observable, or formally to a linear, self-adjoint operator in some Hilbert space.
The passage from quantum mechanics to field theory does not represent any progress
towards such goal, in that not only it does not ``promote'' time to an operator $\hat{t}$,
but it also ``demotes'' the three position operators $\hat{x}$, $\hat{y}$ and $\hat{z}$
to mere (classical) \emph{parameters}. However, consistently to relativistic covariance,
time and space coordinates are treated on equal footing. The Page and Wootters model
offers the opportunity to do the same, but both time and position are operators!

\subsection{Page and Wootters squared}

From the \eqref{eq:pwHamiltonian} and \eqref{eq:Wheeler-DeWitt}:
\begin{equation}\label{eq:pw_to_be_squared}
  -\hbar\hat{\Omega}\ox\idop_S \dket{\Psi} = \idop_T\ox\hat{H}_S \dket{\Psi} \,\text{.}
\end{equation}
Squaring the operators on both sides, we have:
\begin{equation}\label{eq:pw_squared}
  \hbar^{2}\hat{\Omega}^{2}\ox\idop_S \, \dket{\Psi} = \idop_T\ox\hat{H}_S^{2} \,\, \dket{\Psi} 
    = \idop_T \ox \left( \hat{p}^2 c^2 + m_0^2 c^4 \right) \dket{\Psi} \,\text{,}
\end{equation}
where the last equality is due to the ``quantization'' of the relation $E^2 = p^2 c^2 + m_0^2 c^4$
into $\hat{H}_S^{2} = \hat{p}^2 c^2 + m_0^2 c^4$.

We know from previous sections that in the $\qty{\ket{t}}_T$ basis $\hat{\Omega}$ is represented as
$-i\hbar\pdv{t}$. We know from standard quantum mechanics that in the $\qty{\ket{x}\ket{y}\ket{z}}_S$
basis is $\hat{p} \repr -i\hbar\nabla$. Therefore in the product basis the \eqref{eq:pw_squared} yields:
\begin{equation}\label{eq:proto_kg}
  -\hbar^2\pdv[2]{t}\Psi = \qty(-\hbar^2 c^2 \nabla^2 + m_0^2 c^4)\Psi
\end{equation}
which, up to some elementary algebra, is the well known \term{Klein-Gordon equation}.

In standard quantum mechanics (first quantization, relativistic or not)
\begin{equation}
  \ket{\psi(t)} = \int \dd{x}\dd{y}\dd{z} \Psi(x,y,z;t) \ket{x}\ox\ket{y}\ox\ket{z} \,\text{.}
\end{equation}
In this formulation
\begin{equation}
  \dket{\Psi} = \int \dd{x}\dd{y}\dd{z}\dd{t} \Psi(x,y,z,t) \ket{x}\ox\ket{y}\ox\ket{z}\ox\ket{t} \,\text{,}
\end{equation}
and $t$ is no longer a classical parameter, but an eigenvalue of the time operator,
or a label for time basis elements. Please note that, as a function of four variables,
$\Psi$ is not normalizable i.e. it's not a proper element of
$\mathscr{L}^2(\mathbb{R}^4)$
(normalization issues are tackled in \cite[eq. 23 and following]{Lloyd:Time}).

\subsection{Covariant notation}

The \eqref{eq:proto_kg} can be rearranged and expressed in explicitly covariant form:
\begin{equation}
  \qty[\partial_{\mu}\partial^{\mu} + \qty(\frac{mc}{\hbar})^2]\Psi = 0
\end{equation}
with standard relativistic notation.

Similarly
---at operator level, in the tensor product space---
the \eqref{eq:pw_squared} can be rearranged:
\begin{equation}\label{eq:pwkg}
  \qty[- \hat{\mathrm{P}}_{\mu} \hat{\mathrm{P}}^{\mu} + \qty(\frac{mc}{\hbar})^2]\dket{\Psi} = 0
\end{equation}
with
\begin{equation}
\begin{aligned}\label{eq:p0123}
  \hat{\mathrm{P}}_0 &=& \frac{\hbar\hat{\Omega}}{c}  \ox \idop_x   \ox \idop_y   \ox \idop_z \\
  \hat{\mathrm{P}}_1 &=& \idop_T                      \ox \hat{p}_x \ox \idop_y   \ox \idop_z \\
  \hat{\mathrm{P}}_2 &=& \idop_T                      \ox \idop_x   \ox \hat{p}_y \ox \idop_z \\
  \hat{\mathrm{P}}_3 &=& \idop_T                      \ox \idop_x   \ox \idop_y   \ox \hat{p}_z
\end{aligned}
\end{equation}
and the usual relation between contravariant and covariant 4-vectors:
\begin{equation}
  a_\mu = \eta_{\nu}^{\mu} a^{\nu} \text{;} \quad
  \eta = \mathrm{diag}(+1, -1, -1, -1)
  \text{.}
\end{equation}

It's tempting to name the \eqref{eq:pwkg} and \eqref{eq:p0123} the \term{Page--Wootters-Klein-Gordon}
equation.\footnote{
  Or \term{Klein-Gordon-Wheeler-DeWitt-Page--Wootters-Giovannetti-Lloyd-Maccone}
  equation, thus giving
  credits to all authors, particularly if we consider that the idea of an extra Hilbert space
  was conceived originally in \cite{Lloyd:Time}.
}

As anticipated, this resolves the asymmetry in Tables \ref{op_comparison_alg} and \ref{op_comparison_J},
whereas we can write
{
  \begin{table}[h!]
    \centering
    \begin{tabular}{p{0.3\linewidth}||c|c|c}
                                                                                &
        $\hilb{H}_T$                                                            &
        $\hilb{H}_S$                                                            &
        {\footnotesize Mass term}                                               \\
      \hline
      \hline
        {\footnotesize Spatio-temporal ``positions''}                           &
        $\hat{t}$                                                               &
        $\hat{x}$                                                               &
                                                                                \\
      \hline
        {\footnotesize Canonically conjugate}                                   &
        $\hat{P_0} = \frac{\hbar\hat{\Omega}}{c} \repr -i\partial_{0}$          &
        $\hat{P}_{1,2,3} \repr -i\nabla$                                        &
                                                                                \\
      \hline
        {\footnotesize Dynamics (terms in the Wheeler-DeWitt equation)}         &
        $\hat{P}_{0}\hat{P}^{0} \repr -\hbar^2 \partial_{0}\partial^{0}$        &
        $\hat{P}_{j}\hat{P}^{j} \repr -\hbar^2 \partial_{j}\partial^{j}$        &
        $\qty(\frac{mc}{\hbar})^2$
    \end{tabular}
    \caption{
      Operators in the space-time Hilbert spaces.
    }
  \end{table}
}