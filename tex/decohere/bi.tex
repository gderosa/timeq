\section{Bipartite systems, separable and entangled states}

Consider a \term{bipartite quantum system}
i.e. a composite system $S$
made up of two parts, $A$ and~$B$,
described by their respective Hilbert spaces
$\hilb{H}_A$ and $\hilb{H}_B$.

In \cite{Haroche_Exploring},
a basis for $\hilb{H}_A$ and a basis for $\hilb{H}_B$
are indicated as
$$
  \setof{\ket{i_A}} \text{ and } \setof{\ket{\mu_B}}\,
$$
with Latin and Greek indices to label basis vectors in the two spaces.
This no-\\tation % \footnote{We will partly follow this convention.}
in\-di\-cates potentially different physical roles for the two spaces,
such as the \emph{system of interest} and the surrounding \emph{environment},
or the \emph{system being measured} and the \emph{measurement apparatus}.

If the two subsystems are prepared independently and do not interact with each other,
the system is in a \term{product state} or \term{separable state}:
\begin{equation}\label{eq:separableAB}
  \ket{\psi_S} = \ket{\psi_A}\otimes\ket{\psi_B} \,\text{.}
\end{equation}

It is worth recalling that in a tensor product space $\hilb{H}_A\otimes\hilb{H}_B$
not all vectors can be expressed as a tensor product as in \eqref{eq:separableAB}.
However, the following holds:

\begin{proposition}\label{TensorBase}
The set of tensor products of basis vectors of $\hilb{H}_A$ and $\hilb{H}_B$,
i.e. $$\setof{\ket{i} \otimes \ket{\mu}}_{i\mu},$$
is a basis for $\hilb{H}_A \otimes \hilb{H}_B$.
\end{proposition}
Therefore, we can express any
(generally not separable) state vector of $\hilb{H}_S$
as a \emph{superposition} of separable basis vectors
\begin{equation}\label{eq:bipartite_expansion}
  \ket{\psi_S} = \sum_{i, \mu}\alpha_{i\mu}\ket{i}\otimes\ket{\mu} \text{.}
\end{equation}

\begin{definition}
  Non separable states are defined as \term{entangled}.
\end{definition}

Physically,
$\ket{\psi_S}$ contains information
not only about the results of measurements on $A$ and $B$ separately,
but also on correlations between these measurements.

The simplest example of entangled system is given by two two-level systems,
namely two spin-$\frac{1}{2}$ particles in a singlet state:
\[
  \ket{\psi_{\text{singlet}}} = \frac{1}{\sqrt{2}} \left(\ket{\uparrow, \downarrow} - \ket{\downarrow, \uparrow}\right).
\]
