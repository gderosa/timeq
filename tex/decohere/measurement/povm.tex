\subsection{General measurement and POVM}
\label{subsec:POVM}

In a similar fashion to
Section~\ref{sec:qmt:projm},
let us again consider
a system under measurement
and the measurement apparatus,
but, for further simplicity, let both their Hibert spaces have finite dimension $N=2$.

Let $\ket{0}_A, \ket{1}_A \in \hilb{H}_A$ be the eigenstates 
of the observable subject of measurement,
and $P_0$, $P_1$ their respective projectors.
Let $\qty{\ket{0}_B, \ket{1}_B}$ be the corresponding orthonormal basis in the pointer space $\hilb{H}_B$.

Let the initial state in $\hilb{H}_A$ be $\ket{\psi}_A = \alpha\ket{0}_A + \beta\ket{1}_A$,
and the initial state of the measurement device be $\ket{0}_B$.

Expanding the sums in eqs.~\eqref{eq:explicit_measurement_evolution}
and \eqref{eq:measurement_entangled}
---with indices $a$ and $b$ both spanning over values $0$, $1$ only---
the state after measurement will be:
\begin{multline}\label{eq:qubit_ortho_measurement}
        \ket{\Psi'}
    % = U(\ket{\Psi})
    = U( \ket{\psi}_A \ox \ket{0}_B )
    = U( \qty\big{\alpha\ket{0}_A + \beta\ket{1}_A} \ox \ket{0}_B )
    \\
    = \alpha\ket{0}_A \ox \ket{0}_B + \beta\ket{1}_A  \ox \ket{1}_B
    = P_0\ket{\psi}_A \ox \ket{0}_B + P_1\ket{\psi}_A \ox \ket{1}_B \text{.}
\end{multline}

When we observe the pointer, let us assume we are not
able to ``measure'' it with respect to the basis
$\setof{\ket{0}, \ket{1}}$,
but with respect to another basis, say,
\begin{equation}\label{eq:pmbasis}
\setof{\ket{\pm}_B = \frac{1}{\sqrt{2}} \qty\Big(\ket{0}_B \pm \ket{1}_B)} \,\text{.}
\end{equation}

A natural question is whether the above (projective) measurement
of the pointer device
will correspond to a projective
measurement in $\hilb{H}_A$,
with respect the corresponding basis
$\setof{\ket{\pm}_A = \frac{1}{\sqrt{2}} \qty\Big(\ket{0}_A \pm \ket{1}_A)}$.
The answer is negative, as it will be shown.

Indeed, we can rewrite \eqref{eq:qubit_ortho_measurement} as
\begin{multline}\label{eq:qubit_gen_measurement}
  \ket{\Psi'}                                               =
  U( \ket{\psi}_A \ox \ket{0}_B )                           =         \\
  \frac{1}{\sqrt{2}} \qty\Big(
    \alpha\ket{0}_A \ox \qty(\ket{+}+\ket{-}) +
    \beta \ket{1}_A \ox \qty(\ket{+}-\ket{-})
  )                                                         =         \\
  \frac{1}{\sqrt{2}} \qty\Big(
    \qty(\alpha\ket{0}_A + \beta\ket{1}_A) \ox \ket{+} +
    \qty(\alpha\ket{0}_A - \beta\ket{1}_A) \ox \ket{-}
  )                                               \eqbydef            \\
  M_+\ket{\psi}_A \ox \ket{+} + M_-\ket{\psi}_A \ox \ket{-}
  \,\text{,}
\end{multline}
where we have defined:
\begin{align}\label{eq:qmt:Mpm}
  &
  M_+ \repr \frac{1}{\sqrt{2}}\mqty(\imat{2})
  &
  M_- \repr \frac{1}{\sqrt{2}}\mqty(1& 0\\0& -1)
\end{align}
with respect to the basis $\setof{\ket{0}_A, \ket{1}_A}$.

The most important difference between equations \eqref{eq:qubit_gen_measurement}
and \eqref{eq:qubit_ortho_measurement} is that
the o\-per\-a\-tors $M_{\pm}$ are \emph{not} projectors onto $\ket{\pm}_A$.
Thus the unitary operation $U$ is not a projective measurement
with respect to such basis!

Moreover, $M_-$ in the example is not even idempotent,
therefore it is not a projector onto \emph{any} linear subspace of $\hilb{H}_A$.
The lack of idempotency means, physically,
that if we repeat the measurement we don't generally
obtain the same result (and don't leave the system $A$ in the same state).

On the other hand,
while $M_{\pm}$ are not (a complete set of orthogonal) projectors,
the operators $E_{\pm} \eqbydef M_{\pm}M_{\pm}^{\dagger}$
satisfy the
\term{decomposition of the identity}:
\[
  E_{+} + E_{-} = M_{+}^{\dagger}M_{+} + M_{-}^{\dagger}M_{-} = \idop \text{,}
\]
similarly to the projectors $P_n$ in eq. \eqref{eq:qmt:decomposition-id-proj}.

Operators $\setof{E_{+}, E_{-}}$ are an example of \term{POVM}. TODO Possibly NAKAHARA.

% \citereset
% All the following holds \parencite[sec.3.1]{PreskillNotes}:
% \begin{enumerate}
%   \item 
%     Hermiticity: \[E_a = E_a^{\dagger}\,\text{;}\]
%   \item
%     \term{Positivity}: \[\expval{E_a}{\psi} \geq 0\,\text{;}\]  
%   \item\label{listitem:POVM}
%     Decomposition of the identity (\term{completeness}):
%     \begin{equation}\label{eq:POVM}
%       \sum_a E_a = \sum_a M_a^{\dagger}M_a = \idop \,\text{;}
%     \end{equation}
%   \item= 
%     Probability of outcome $a$ of a measurement:
%     \[\Pr(a) = \norm{M_a\ket{\psi}}^2 = \expval{E_a}{\psi}\,\text{;}\]
%   \item
%     Probability of obtaining $b$ in a second measurement:
%     \[\Pr(b|a) = \frac{\norm{M_bM_a\ket{\psi}}^2}{\norm{M_a\ket{\psi}}^2}\]
%     (would be $\delta_{ab}$ if orthogonal);
%   \item
%     Probability of outcome, density operator form:
%     \[\Pr(a) = \tr(\rho E_a)\,\text{,}\]
%     also valid for mixed states.
% \end{enumerate}

\noindent\hrulefill

A more abstract and general definition of POVM is as
follows:\footnote{
  See e.g. \cite{BeneduciPhD, Berberian} --- the level of generalization may vary.
}
\begin{definition}
  Given a (Borel) $\sigma$-algebra $\mathcal{B}(\mathbb{R})$ of subsets of $\mathbb{R}$,
  and the space $\mathcal{F}(\hilb{H})$ of positive, self-adjoint operators on a Hilbert space,
  a \term{positive operator-valued measure} (\term{POVM})
  is a map $E: \mathcal{B}(\mathbb{R}) \rightarrow \mathcal{F}(\hilb{H})$
  such that
  \begin{itemize}
    \item $E(\mathbb{R}) = \idop$ (completeness)
    \item $E\qty(\bigcup\limits_{n} \Delta_n) = \sum\limits_{n} E\qty(\Delta_n)$ (additivity) 
  \end{itemize}
  where $\setof{\Delta_n}$ is a countable family of disjoint sets in
  $\mathcal{B}(\mathbb{R})$.
\end{definition}

\citereset
It's worth noting that,
given a POVM $E$, operators $M$
suitable for \eqref{eq:POVM} and \eqref{eq:gen_measurement}
and following generalization
can always be found \parencite[sec.3.1]{PreskillNotes},
but not uniquely, in the sense that
$UM$ are also valid
for any unitary operator $U$.

Therefore, this formulation leaves the state after a measurement
with outcome $a$
\[
  \frac{M_a\ket{\psi}}{\norm{M_a\ket{\psi}}^2}
  \leftrightarrow
  \frac{UM_a\ket{\psi}}{\norm{M_a\ket{\psi}}^2}
\]
in fact \emph{undetermined}.
