\subsection{General measurement and POVM}
\label{subsec:POVM}

In a similar fashion to
Section~\ref{sec:qmt:projm},
let us again consider
a system under measurement
and the measurement apparatus,
but, for further simplicity, let both their Hilbert spaces have finite dimension $N=2$.
%
The examples in this Subsection are based on \citereset\cite[\s 3.1.2]{PreskillNotes},
with the details of some calculations expanded further. 

Let $\ket{0}_A, \ket{1}_A \in \hilb{H}_A$ be the eigenstates
of the observable subject of measurement,
and $P_0$, $P_1$ their respective projectors.
Let $\qty{\ket{0}_B, \ket{1}_B}$ be the corresponding orthonormal basis in the pointer space $\hilb{H}_B$.

Let the initial state in $\hilb{H}_A$ be $\ket{\psi}_A = \alpha\ket{0}_A + \beta\ket{1}_A$,
and the initial state of the measurement device be $\ket{0}_B$.

Expanding the sums in Eqs.~\eqref{eq:explicit_measurement_evolution}
and \eqref{eq:measurement_entangled}
---with indices $a$ and $b$ both spanning over values $0$, $1$ only---
the state after measurement will be:
\begin{multline}\label{eq:qubit_ortho_measurement}
        \ket{\Psi'}
    % = U(\ket{\Psi})
    = U( \ket{\psi}_A \ox \ket{0}_B )
    = U( \qty\big{\alpha\ket{0}_A + \beta\ket{1}_A} \ox \ket{0}_B )
    \\
    = \alpha\ket{0}_A \ox \ket{0}_B + \beta\ket{1}_A  \ox \ket{1}_B
    = P_0\ket{\psi}_A \ox \ket{0}_B + P_1\ket{\psi}_A \ox \ket{1}_B \text{.}
\end{multline}

When we observe the pointer, let us assume we are not
able to ``measure'' it with respect to the basis
$\setof{\ket{0}, \ket{1}}$,
but with respect to another basis, say,
\begin{equation}\label{eq:pmbasis}
\setof{\ket{\pm}_B = \frac{1}{\sqrt{2}} \qty\Big(\ket{0}_B \pm \ket{1}_B)} \,\text{.}
\end{equation}

A natural question is whether the above (projective) measurement
of the pointer device
will correspond to a projective
measurement in $\hilb{H}_A$,
with respect the corresponding basis
$\setof{\ket{\pm}_A = \frac{1}{\sqrt{2}} \qty\Big(\ket{0}_A \pm \ket{1}_A)}$.
The answer is negative, as it will be shown.

Indeed, we can rewrite \eqref{eq:qubit_ortho_measurement} as
\begin{multline}\label{eq:qubit_gen_measurement}
  \ket{\Psi'}                                               =
  U( \ket{\psi}_A \ox \ket{0}_B )                           \\
  =
  \frac{1}{\sqrt{2}} \qty\Big(
    \alpha\ket{0}_A \ox \qty(\ket{+}+\ket{-}) +
    \beta \ket{1}_A \ox \qty(\ket{+}-\ket{-})
  )                                                         \\
  =
  \frac{1}{\sqrt{2}} \qty\Big(
    \qty(\alpha\ket{0}_A + \beta\ket{1}_A) \ox \ket{+} +
    \qty(\alpha\ket{0}_A - \beta\ket{1}_A) \ox \ket{-}
  )                                                         \\
  \eqbydef
  M_+\ket{\psi}_A \ox \ket{+} + M_-\ket{\psi}_A \ox \ket{-}
  \,\text{,}
\end{multline}
where we have defined:
\begin{align}\label{eq:qmt:Mpm}
  &
  M_+ \repr \frac{1}{\sqrt{2}}\mqty(\imat{2})
  &
  M_- \repr \frac{1}{\sqrt{2}}\mqty(1& 0\\0& -1)
\end{align}
with respect to the basis $\setof{\ket{0}_A, \ket{1}_A}$.

The most important difference between equations \eqref{eq:qubit_gen_measurement}
and \eqref{eq:qubit_ortho_measurement} is that
the o\-per\-a\-tors $M_{\pm}$ are \emph{not} projectors onto $\ket{\pm}_A$.
Thus the unitary operation $U$ is not a projective measurement
with respect to such basis!

Moreover, $M_-$ in the example is not even idempotent,
therefore it is not a projector onto \emph{any} linear subspace of $\hilb{H}_A$.
The lack of idempotency means, physically,
that, if we repeat the measurement, we do not generally
obtain the same result (and do not leave the system $A$ in the same state).

On the other hand,
while $M_{\pm}$ are not (a complete set of) orthogonal projectors,
the operators $E_{\pm} \eqbydef M_{\pm}M_{\pm}^{\dagger}$
still satisfy the
\term{decomposition of the identity}:
\begin{equation}\label{eq:qmt:qubit:povm}
  E_{+} + E_{-} = M_{+}^{\dagger}M_{+} + M_{-}^{\dagger}M_{-} = \idop \text{,}
\end{equation}
similarly to the projectors $P_n$ in Eq.~\eqref{eq:qmt:decomposition-id-proj}.

Operators $\setof{E_{+}, E_{-}}$ are an example of \term{POVM}
(positive operator-valued measure) i.e. a measure whose
values are \term{positive operators}.

Positive operators are a more general class of operators than projectors.
They are defined as Hermitian operators $E$ in an Hilbert space $\hilb{H}$ satisfying
\[
  \ev{E}{\psi} \ge 0 \text{,}\quad \forall \psi \in \hilb{H}.
\]

The example we have considered only allows two possible outcomes.
More generally, for a discrete set of possible outcomes,
Eq.~\eqref{eq:qmt:qubit:povm} can be written:
\begin{equation}
  \sum_m E_m = \sum_m M_m^{\dagger} M_m = \idop \text{.}
\end{equation}

Looking back at the general measurement postulate (Section \ref{sec:qmt:postulate}),
we see that the statistical distribution of measurement outcomes
is determined by the product $M_m^{\dagger} M_m = E_m$ rather than the
measurement operator $M_m$ itself. So one can define the positive operators $E_m$ directly.
However, the state \emph{after} the measurement, as in Eq.~\eqref{eq:qmt:postulate:after}
\emph{does} depend on $M_m$.

It is important to note that, given a positive operator $E_m$, the operator
$M_m$ such that
\begin{equation}\label{eq:qmt:positive-measurement}
  E_m = M_{m}^{\dagger} M_m
\end{equation}
is not uniquely determined.
Indeed, if one replaces $M_m$ with $UM_m$
---where $U$ is any unitary operator---
it is easily verified that
Eq.~\eqref{eq:qmt:positive-measurement} is still satisfied.

One conclusion that can be drawn is that,
taking into consideration \eqref{eq:qmt:postulate:prob} and
\eqref{eq:qmt:postulate:after}, and the above observation,
for a  given POVM defined by the positive operators $\setof{E_m}$,
one can predict the probability of measurement outcomes,
but not the state of the system after the measurement.

A more abstract and general definition of POVM is as
follows:\footnote{
  See e.g. \cite{BeneduciPhD, Berberian} --- the level of generalization may vary.
}
\begin{definition}
  Given a (Borel) $\sigma$-algebra $\mathcal{B}(\mathbb{R})$ of subsets of $\mathbb{R}$,
  and the space $\mathcal{F}(\hilb{H})$ of positive, self-adjoint operators on a Hilbert space,
  a \term{positive operator-valued measure} (\term{POVM})
  is a map $E: \mathcal{B}(\mathbb{R}) \rightarrow \mathcal{F}(\hilb{H})$
  such that
  \begin{itemize}
    \item $E(\mathbb{R}) = \idop$ (completeness)
    \item $E\qty(\bigcup\limits_{n} \Delta_n) = \sum\limits_{n} E\qty(\Delta_n)$ (additivity)
  \end{itemize}
  where $\setof{\Delta_n}$ is a countable family of disjoint sets in
  $\mathcal{B}(\mathbb{R})$.
\end{definition}
