\subsection{Generalized measurement (POVM)}
\label{subsec:POVM}

Let us consider
the same system and measurement apparatus of
Sec.~\ref{sec:qmt:projm}, with similar notation
but setting, for further simplicity, $N=2$,
i.e. the system being measured and its pointer space are both
two-level systems.

The unitary transformation describing the measurement process:
\begin{multline}\label{eq:qubit_ortho_measurement}
  U:
    \ket{\psi}_A \ox \ket{0}_B
    = \qty\big{\alpha\ket{0}_A + \beta\ket{1}_A} \ox \ket{0}_B
  \rightarrow \\
    \alpha\ket{0}_A \ox \ket{0}_B + \beta\ket{1}_A  \ox \ket{1}_B =
    \:
    E_0\ket{\psi}_A \ox \ket{0}_B + E_1\ket{\psi}_A \ox \ket{1}_B
\end{multline}
with subscripts $A$ and $B$ designating the system of interest and
the measurement apparatus (pointer space) respectively.

When we observe the pointer, let's assume we're not
able to ``measure'' it with respect to the fiducial basis
$\setof{\ket{0}, \ket{1}}$,
but with respect to another basis, say,
\begin{equation}\label{eq:pmbasis}
\setof{\ket{\pm} = \frac{1}{\sqrt{2}} \qty\Big(\ket{0}_B \pm \ket{1}_B)} \,\text{.}
\end{equation}

We can rewrite \eqref{eq:qubit_ortho_measurement} as
\begin{multline}\label{eq:qubit_gen_measurement}
  U: \ket{\psi}_A \ox \ket{0}_B                   \rightarrow \\
  \frac{1}{\sqrt{2}} \qty\Big(
    \alpha\ket{0}_A \ox \qty(\ket{+}+\ket{-}) +
    \beta \ket{1}_A \ox \qty(\ket{+}-\ket{-})
  )                                               =           \\
  \frac{1}{\sqrt{2}} \qty\Big(
    \qty(\alpha\ket{0}_A + \beta\ket{1}_A) \ox \ket{+} +
    \qty(\alpha\ket{0}_A - \beta\ket{1}_A) \ox \ket{-}
  )                                               \eqbydef      \\
  M_+\ket{\psi}_A \ox \ket{+} + M_-\ket{\psi}_A \ox \ket{-}
  \,\text{,}
\end{multline}
where we have defined:
\begin{align*}
  &
  M_+ \repr \frac{1}{\sqrt{2}}\mqty(\imat{2})
  &
  M_- \repr \frac{1}{\sqrt{2}}\mqty(1& 0\\0& -1)
\end{align*}
with respect to the basis $\setof{\ket{0}_A, \ket{1}_A}$.

After measurement, system $A$ is ``prepared''
(up to a normalization factor)
in one of the states $M_{\pm}\ket{\psi}$,
that are not necessarily orthogonal.

Moreover, $M_+$ and $M_-$ are not generally idempotent,
therefore if we repeat the measurement we don't generally
obtain the same result (and don't leave the system $A$ in the same state).
This is a fundamental difference with projective measurement.

Besides this particular example, $M_+$ and $M_-$ are not even necessarily
self-adjoint, and we can see that, while $M$ generalise the projector $P$
in terms of ``collapsing'' (or ``preparing'') the system under measurement,
in some sense a better generalization is in fact $M^{\dagger}M$\footnote{
  Such distinction is inessential for a projector $P$,
  as $P^{\dagger}P = P^2 = P$.
}, particularly in the sense of \term{decomposition of the identity}:
$\sum_a M_a^{\dagger}M_a = \idop$ \parencite[sec.3.1]{PreskillNotes}.

The example of $\ket{\pm}$ is a particularly ``unsharp'' measurement,
or a particularly ``overlapping'' decomposition of the identity,
because
$$M_+^{\dagger}M_+ = M_-^{\dagger}M_- = \frac{1}{2}\idop\text{.}$$

Generalizing to $N$-dimensional Hilbert spaces, we can replace
$\setof{\ket{\pm}}$ basis with
\[
  \setof{\ket{a}, a = 0\ldots N-1}
\]
and \eqref{eq:qubit_gen_measurement} with
\begin{equation}\label{eq:gen_measurement}
  U: \ket{\psi}_A \ox \ket{0}_B \rightarrow \sum_a M_a\ket{\psi}_A \ox \ket{a}_B
  \,\text{.}
\end{equation}

So, let's define
\[
  E_a = M^{\dagger}M \,\text{.}
\]

\citereset
All the following holds \parencite[sec.3.1]{PreskillNotes}:
\begin{enumerate}
  \item 
    Hermiticity: \[E_a = E_a^{\dagger}\,\text{;}\]
  \item
    \term{Positivity}: \[\expval{E_a}{\psi} \geq 0\,\text{;}\]  
  \item\label{listitem:POVM}
    Decomposition of the identity (\term{completeness}):
    \begin{equation}\label{eq:POVM}
      \sum_a E_a = \sum_a M_a^{\dagger}M_a = \idop \,\text{;}
    \end{equation}
  \item
    Probability of outcome $a$ of a measurement:
    \[\Pr(a) = \norm{M_a\ket{\psi}}^2 = \expval{E_a}{\psi}\,\text{;}\]
  \item
    Probability of obtaining $b$ in a second measurement:
    \[\Pr(b|a) = \frac{\norm{M_bM_a\ket{\psi}}^2}{\norm{M_a\ket{\psi}}^2}\]
    (would be $\delta_{ab}$ if orthogonal);
  \item
    Probability of outcome, density operator form:
    \[\Pr(a) = \tr(\rho E_a)\,\text{,}\]
    also valid for mixed states.
\end{enumerate}

This partition of the identity by positive operators
as expressed in \eqref{eq:POVM} is called
\term{positive operator-valued measure}, or \term{POVM}.
It generalises the \term{projector-valued measure} (PVM)
found in Von Neumann's theory.

For the sake if simplicity, we're referring to the finite dimensional case,
but the above can be extended to infinite dimensions and continuous spectra,
where the decomposition of the identity can be expressed as
$\int_{-\infty}^{\infty} dx M^{\dagger}(x)M(x) = \idop $.
A more abstract and general definition of POVM is as
follows:\footnote{
  See e.g. \cite{BeneduciPhD, Berberian} --- the level of generalization may vary.
}
\begin{definition}
  Given a (Borel) $\sigma$-algebra $\mathcal{B}(\mathbb{R})$ of subsets of $\mathbb{R}$,
  and the space $\mathcal{F}(\hilb{H})$ of positive, self-adjoint operators on a Hilbert space,
  a \term{positive operator-valued measure} (\term{POVM})
  is a map $E: \mathcal{B}(\mathbb{R}) \rightarrow \mathcal{F}(\hilb{H})$
  such that
  \begin{itemize}
    \item $E(\mathbb{R}) = \idop$ (completeness)
    \item $E\qty(\bigcup\limits_{n} \Delta_n) = \sum\limits_{n} E\qty(\Delta_n)$ (additivity) 
  \end{itemize}
  where $\setof{\Delta_n}$ is a countable family of disjoint sets in
  $\mathcal{B}(\mathbb{R})$.
\end{definition}

\citereset
It's worth noting that,
given a POVM $E$, operators $M$
suitable for \eqref{eq:POVM} and \eqref{eq:gen_measurement}
and following generalization
can always be found \parencite[sec.3.1]{PreskillNotes},
but not uniquely, in the sense that
$UM$ are also valid
for any unitary operator $U$.

Therefore, this formulation leaves the state after a measurement
with outcome $a$
\[
  \frac{M_a\ket{\psi}}{\norm{M_a\ket{\psi}}^2}
  \leftrightarrow
  \frac{UM_a\ket{\psi}}{\norm{M_a\ket{\psi}}^2}
\]
in fact \emph{undetermined}.
