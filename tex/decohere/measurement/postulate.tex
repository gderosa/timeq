\subsection{General requirements for quantum measurement}\label{sec:qmt:postulate}

One of the postulates of quantum mechanics \parencite[\s 2.2.3]{NielsenChuang}
states that
\begin{enumerate}
  \item
    quantum measurements are described by a collection
    $\setof{M_m}$ of \term{measurement operators}.
  \item
    The index $m$ refers to the measurement outcomes that
    may occur in the experiment.
  \item
    If the state of the quantum system is $\ket{\psi}$
    immediately before the measurement then the probability that result $m$ occurs is
    given by
    \begin{equation}\label{eq:qmt:postulate:prob}
      \pi_{m} = \ev{M^{\dagger}_{m} M_{m}}{\psi} \text{.}
    \end{equation}
  \item
    The state of the system after the measurement is
    \begin{equation}\label{eq:qmt:postulate:after}
      \frac{ M_m \ket{\psi} }{ \sqrt{\ev{M^{\dagger}_{m} M_{m}}{\psi}} } \text{.}
    \end{equation}
  \item
    The \term{completeness} property or \term{decomposition of the identity} is satisfied:
    \begin{equation}\label{eq:qmt:postulate:completeness}
      \sum_m M_m^{\dagger} M_m = \idop \text{.}
    \end{equation}
\end{enumerate}

It's easily shown that the completeness is equivalent to the condition that
$\sum_m \pi_m = 1$, i.e. the sum of the probabilities for \emph{all} measurement outcomes is equal to 1.

Both the projective measurement and positive-operator measurement (POVM),
which will be discussed in Sections \ref{sec:qmt:projm} and \ref{subsec:POVM},
satisfy the above postulate.