\subsection{Review on projectors}

A brief review of \term{projection operators} or \term{projectors}
is useful before introducing the \term{projective measurement}
in Section \ref{sec:qmt:projm}.

We shall assume, unless otherwise stated, that all vectors are
normalized to unity.
Consider an observable represented by a self-adjoint operator $A$
with a 
\emph{discrete, non degenerate spectrum}.
Let $a_i$ by an eigenvalue of $A$, and $\ket{i}$ its corresponding eigenvector.
The probability of a measurement
outcome $a_i$,
on a system in the pure state $\ket{\psi}$
is given by
$$
\pi_{i} = \norm{\braket{i}{\psi}}^2
        = \bra{\psi}\ket{i}\bra{i}\ket{\psi}
        = \mel{\psi}{P_{i}}{\psi}
$$
where $P_{i} = \ketbra{i}$ appears as the \term{projector} on the
one-dimensional ei\-gen\-space related to eigenvalue  $a_i$.

If $a_i$ is degenerate, $j = 1, \dots, J$ its degeneracy index, and $\ket{ij}$ the corresponding eigenvectors,
such probability shall sum over it:
$$
\pi_{i} = \sum_{j=1}^{J}\norm{\braket{ij}{\psi}}^2
        = \sum_{j=1}^{J}\bra{\psi}\ket{ij}\bra{ij}\ket{\psi}
        = \mel{\psi}{P_{i}}{\psi}
$$
where $P_{i} = \sum_{j=1}^{J}\ketbra{ij}$
still is the projector on the
($J$-dimensional) eigenspace of $a_i$.

Generally, the probability that the outcome of a measurement falls in
the set of eigenvalues $\sigma = \{a_{1}, \dots, a_{S}\}$ is
\begin{equation}\label{eq:pi_sigma}
\pi_{\sigma}  = \sum_{i=1}^{S}\sum_{j=1}^{J_{i}}\norm{\braket{ij}{\psi}}^2
              = \sum_{i=1}^{S}\sum_{j=1}^{J_{i}}\bra{\psi}\ket{ij}\bra{ij}\ket{\psi}
              = \mel{\psi}{P_{\sigma}}{\psi}
              ,
\end{equation}
where $P_{\sigma} = \sum_{i=1}^{S}\sum_{j=1}^{J_{i}}\ketbra{ij}$
is once again a projector --- on the ``generalized eigenspace'' spanned by all
eigenvectors $\{\ket{ij}\}$ above.

Looking back at Eq.~\eqref{eq:expvalA_rho},
it is valid for \emph{any} Hermitian operator,
therefore it is also valid for projectors.
Hence we can replace $A$ with the projector $P_{\sigma}$,
associated to the set of eigenvalues $\sigma$
according to \eqref{eq:pi_sigma} and \eqref{eq:P_sigma_spectral}.

This is of particular interest
because its mean value equates the probability that the outcome of a measurement
falls in a given subset of the spectrum of a given observable.
So we have:

\begin{proposition}\label{probability_rho}
  The probability $\pi_{\sigma}$
  that the outcome of a measurement of the observable $A$
  on the system described by the state operator $\rho$
  falls in the set $\sigma$
  is given by:
  \begin{equation}\label{eq:probability_rho}
    \pi_{\sigma} = \expval{P_{\sigma}} = \tr(P_{\sigma}\rho)
  \end{equation}
  with $P_{\sigma}$ defined as in \eqref{eq:pi_sigma}
  or, more generally, in \eqref{eq:P_sigma_spectral}.
\end{proposition}

Being $\pi_{\sigma}$ the \emph{probability} of the outcome of a measurement to
fall in a given set $\sigma$, it has to be:
\begin{enumerate}
  \item \label{measure_properties:first} $0$ on the empty set
  \item non negative
  \item \label{measure_properties:last} \term{additive} on disjoint sets
  \item \label{measure_properties:normalized} equal to $1$ if $\sigma$ includes the whole spectrum of $A$.
\end{enumerate}

Properties \ref{measure_properties:first}\dots\ref{measure_properties:last}
are the defining properties of a \term{measure}\footnote{
  Not to be confused with \emph{measurement}.
} \parencite{EncMath_Measure},
and
the map $\sigma \subseteq \mathbb{R} \rightarrow P_{\sigma}$
implicitly defined in~\eqref{eq:pi_sigma} is a \term{projector-valued measure}.

This notion of projector-valued measure can be generalized
in such a way to include both the cases of discrete and continuous spectra,
with the following \parencite{VonNeumann, Ballentine}
\begin{theorem}[spectral resolution]
  If $A$ is a self-adjoint operator,
  there is a unique projector-valued measure $E$
  defined on the Borel sets of $\mathbb{R}$
  such that
  \footnote{
    In \eqref{eq:spectral}, $a$ is a real number (not a set),
    but it's intended $E$ to be evaluated
    on the~\emph{interval}~from $-\infty$ to $a$.
  }
  \begin{equation}\label{eq:spectral}
    A=\int_{-\infty}^{\infty}a\, dE(a)
  \end{equation}
  and satisfying:
  \begin{align*}
    E(\mathbb{R})       & =\mathbf{1},\\
    E(B_{1}\cap B_{2}) & =E(B_{1})E(B_{2})\,.
  \end{align*}
\end{theorem}

In terms of this theorem, the projector in \eqref{eq:pi_sigma} is
\begin{equation}\label{eq:P_sigma_spectral}
  P_{\sigma} = E(\sigma) = \int_{a\in\sigma}dE(a)
\end{equation}
and $dE(a)$ is
---informally speaking---
infinitesimal if $a$ belongs to the continuous spectrum,
finite if $a$ is a discrete eigenvalue
and zero otherwise.
