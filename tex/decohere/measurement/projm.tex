\subsection{Projective measurement}\label{sec:qmt:projm}

Let $\hilb{H}_A$ and $\hilb{H}_B$ be the Hilbert spaces describing
a system under measurement and the measurement device respectively.
For simplicity, let both have finite di\-mension~$N$.
Let $\ket{\Psi}, \ket{\Psi'} \in \hilb{H}_{A} \ox \hilb{H}_{B}$
be the overall state of the
system \emph{plus} measurement apparatus,
respectively before and after the measurement.

Let us also consider a \term{complete}, \term{orthogonal} set of $N$ projectors
$P_0 \dots P_{N-1}$ on the Hilbert space $\hilb{H}_A$,
i.e. the projectors have the properties
\begin{align}\label{eq:qmt:decomposition-id-proj}
  \sum_n P_n  &= \idop \text{,}\\
  P_i P_j     &= \delta_{ij}P_i \, \text{.}
\end{align}

On $\hilb{H}_B$ instead, describing the measurement device, let's consider
the
% fiducial\footnote{
%   In the sense that it's assumed
%   that we can arbitrarily ``reset'' (prepare) the device,
%   for example at state $\ket{0}$,
%   and in general
%   we will find the device at one of the states
%   $\ket{0}\dots\ket{N-1}$
%   when we observe it (\term{pointer states}).
%   Therefore, what follows \emph{is not} a derivation of the Born's rule,
%   which is still a necessary postulate of quantum mechanics
%   (only \cite{Zurek_Decoherence, Zurek_Einselect, Zurek_Fundamentals} push
%   this point of view further). In other words, the Born rule for the system
%   will be ``proved''
%   only on the assumption that it is valid for the measurement apparatus already.
% }
basis:
\begin{equation}\label{eq:qmt:fiducial_basis_N}
  \setof{\ket{0} \dots \ket{N-1}} \text{.}
\end{equation}

The coupling between the system and the apparatus
can be modeled by
a unitary operator
transforming the state of system and apparatus before and after the measurement.
The following is a possible
example \parencite[\s 3.1.1 ``Orthogonal Measurements'']{PreskillNotes}:

\begin{equation}\label{eq:unitary_measurement}
  U = \sum_{a, b = 0}^{N-1} P_a \ox \ketbra{b+a}{b} \text{,}
\end{equation}
where
$\ket{b+a}$ and $\bra{b}$ are all elements of the basis \eqref{eq:qmt:fiducial_basis_N}
and
the sum $b+a$ is \emph{modulo N}:
intuitively, if the $a$-th eigenvalue of the observable is measured,
a sort of ``circular dial''
on the measurement device rotates of $a$ positions.

For this reason, $\hilb{H}_{B}$ is also called \term{pointer space}.

If the system and the measurement device are initially in the
state $\ket{\psi}$~and~$\ket{0}$ respectively
(therefore $\ket{\Psi} = \ket{\psi} \ox \ket{0}$),
using the Eq.~\eqref{eq:unitary_measurement},
the state after the measurement will be %
%
\begin{equation}\label{eq:explicit_measurement_evolution}
  \ket{\Psi^{\prime}} = U(\Psi) = U\Big(\ket{\psi} \ox \ket{0}\Big) =
    \sum_{a, b} P_a \ox \ketbra{b+a}{b} \big(\ket{\psi} \ox \ket{0}\big)
    \text{.}
\end{equation}
Indeed, in Eq.~\eqref{eq:explicit_measurement_evolution}, only terms with $b=0$ survive.
Therefore
\begin{eqsplit}\label{eq:measurement_entangled}
  \ket{\Psi^{\prime}} &= \sum_a \qty\Big(P_a\ox\ketbra{a}{0}) \qty\Big(\ket{\psi}\ox\ket{0}) \\
                      &= \sum_a P_a \ket{\psi} \ox \ket{a}
\end{eqsplit}

The probability of an outcome $a$ is
\begin{multline}\label{eq:measurement_probability}
  \pi_{a} = \expval{\qty\Big(\idop\ox\ketbra{a})}{\Psi^{\prime}} = \\
    \sum_{b,c}
      \qty(\bra{\psi}P_{b}\ox\bra{b})
      \qty\Big(\idop\ox\ketbra{a})
      \qty(P_{c}\ket{\psi}\ox\ket{c}) = \\
    \sum_{b,c}\qty\Big(\expval{P_bP_c}{\psi} \braket{b}{a} \braket{a}{c}) =
    \expval{P_a}{\psi} \text{.}
\end{multline}

Eq. \eqref{eq:measurement_entangled} clearly shows that the system
and the measurement device are completely correlated (entangled).
If the measurement apparatus is observed in state $\ket{a}$
---with probability $\pi_{a}$ as stated in Eq.~\eqref{eq:measurement_probability}---
then the system being measured is in state $P_{a}\ket{\psi}$
or, in normalized form:
\begin{equation}\label{eq:normalized_collapse}
  \ket{\psi^\prime_a} \eqbydef \frac{P_{a}\ket{\psi}}{\norm{P_{a}\ket{\psi}}}
    = \frac{P_{a}\ket{\psi}}{\sqrt{\expval{P_a}{\psi}}} \,\text{.}
\end{equation}

Indeed,
$\ket{\psi}$
is transformed
into its projection $\ket{\psi^\prime_a}$
onto the eigenspace
corresponding to the eigenvalue $a$ of the observable of interest.

Please note the above \emph{is not} a derivation of the Born (probability) rule altogether,
as it still needs to be postulated for the measurement apparatus.

Finally, if the measurement apparatus is not observed,
therefore an outcome $a$ is not known,
the system \emph{after} measurement is in a statistical mixture
of ``all possible wavefunction collapses'' weighted on the probability $\pi_{a}$.
By using both \eqref{eq:measurement_probability} and \eqref{eq:normalized_collapse},
and the definition of the density operator for the initial pure state
$\rho = \ketbra{\psi}$:
\[
  \rho^{\prime} = \sum_a \Pr(a) \ketbra{\psi^{\prime}_a} = \sum_a P_a \ketbra{\psi} P_a
    = \sum_a P_a \rho P_a \,\text{.}
\]
So, the initial pure state $\rho$ is transformed by the measurement process into a mixed one.
In other words, the initial, coherent superposition of eigenstates represented by $\rho = \ketbra{\psi}$
becomes a maximal statistical mixture $\rho^{\prime}$
(as seen in Section \ref{sec:mix}).

It can be proven that the transformation
\begin{equation}\label{eq:irreversible_measurement}
  \rho \rightarrow \sum_a P_a \rho P_a
\end{equation}
is also valid in the more general case of $\rho$ being a mixed state before the measurement
---in this case, it's transformed into another mixed state,
but still described by the \eqref{eq:irreversible_measurement}.
A generalization to observables with a continuous spectrum is also possible
(see e.g. \cite[\s 3.1.1]{PreskillNotes} for more details).
