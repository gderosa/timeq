Consider a \term{bipartite quantum system}
i.e. a composite system $S$
made up of two parts, $A$ and~$B$,
described by their respective Hilbert spaces
$\hilb{H}_A$ and $\hilb{H}_B$ \parencite{Haroche_Exploring}.

Notably, in \cite{Haroche_Exploring},
notation does not treat the two spaces
on perfectly equal footing,
in the sense that a basis for $\hilb{H}_A$ and a basis for $\hilb{H}_B$
are indicated as
$$
  \setof{\ket{i_A}} \text{ and } \setof{\ket{\mu_B}}\,
$$
with latin and greek indices to label base vectors in the two spaces.

This indicates different physical roles for the two spaces,
such as the \emph{system of interest} and the surrounding \emph{environment},
or the \emph{system being measured} and the \emph{measurement apparatus}.

If the two subsystems are independent, all states of $S$ are of the form
\begin{equation}\label{eq:separableAB}
  \ket{\psi_S} = \ket{\psi_A}\otimes\ket{\psi_A}
\end{equation}
i.e. they form a strict \emph{subspace} of $\hilb{H}_A\otimes\hilb{H}_B$.

It's worth recalling that in a tensor product space $\hilb{H}_A\otimes\hilb{H}_B$
not all vectors can be expressed as a tensor product as in \eqref{eq:separableAB}.
When a vector in a tensor product space
can be expressed in the form \eqref{eq:separableAB},
it's said \term{separable} \parencite{Nakahara}.

\begin{proposition}\label{TensorBase}
The set of tensor products of base vectors of $\hilb{H}_A$ and $\hilb{H}_B$,
i.e. $\setof{\ket{i_A} \otimes \ket{\mu_B}}$,
is a basis for $\hilb{H}_A \otimes \hilb{H}_B$
\end{proposition}

\section{The rest aka TODO}

The environment is watching, from ``Exploring the quantum''
\parencite[Ch. 4]{Haroche_Exploring}. Start with subsect. 2.4.1.

Environment-Induced Decoherence and the Transition from Quantum to Classical,
\cite{Zurek_Fundamentals}.

Partial trace, which is not a number, check wikipedia $\rightarrow$ consistent histories.
J. P. Paz W. H. Zurek

Von Neumann and Shanon entropy.

MIKIO NAKAHARA: QUANTUM COMPUTING: AN OVERVIEW

See also \cite{Schlosshauer_Decoherence}.

Vedral, Brukner (Oreshkov\dots)
Necessary and sufficient condition for non-zero quantum discord
\url{https://arxiv.org/pdf/1004.0190.pdf},
cites another paper by Zurek \url{https://arxiv.org/pdf/quant-ph/0105072.pdf}.

\section{Maximally mixed state of $\ket{e_1}$ and $\ket{e_1}$
Vs their maximal coherent superposition: density matrices}

See Nakahara notes, example II.2.

$$
\frac{1}{2}\begin{bmatrix}
  1 &0  \\
  0 &1  \\
\end{bmatrix}
\text{ Vs }
\frac{1}{2}\begin{bmatrix}
  1 &1  \\
  1 &1  \\
\end{bmatrix}
$$

The off-diagonal terms in the coherent superposition case indicate
``quantumness''. (explain) (Andreas: not really: change of basis\dots)

--- On non-othogonal mixture and non uniqueness of density matrix representations
\url{https://arxiv.org/pdf/quant-ph/0512125.pdf}

\begin{remark}
  The density matrix of a pure state is either diag(000\dots1\dots000) or non-diagonal.

  It can be easily proven reasoning on eigenvalues\dots

  So, except the special/trivial case above, a pure state must have off-diagonal terms.
\end{remark}

Measurement destroys off-diagonal terms and turns a pure state into a mixed one.

Off-diagonal terms in the density matrix indicates ``quantum purity''.
(Or internal Entanglement as well, if the system being detected was bipartite itself?)

In some sense, measurement turned a q-statistical distribution of values for the observable,
into a classical probability distribution.

Recommended read: \cite{Zurek_Decoherence, Zurek_Fundamentals}.
