\section{Bipartite systems, separable and entangled states}

Consider a \term{bipartite quantum system}
i.e. a composite system $S$
made up of two parts, $A$ and~$B$,
described by their respective Hilbert spaces
$\hilb{H}_A$ and $\hilb{H}_B$ \parencite{Haroche_Exploring}.

Notably, in \cite{Haroche_Exploring},
notation does not treat the two spaces
on perfectly equal footing,
in the sense that a basis for $\hilb{H}_A$ and a basis for $\hilb{H}_B$
are indicated as
$$
  \setof{\ket{i_A}} \text{ and } \setof{\ket{\mu_B}}\,
$$
with latin and greek indices to label base vectors in the two spaces.

This indicates different physical roles for the two spaces,
such as the \emph{system of interest} and the surrounding \emph{environment},
or the \emph{system being measured} and the \emph{measurement apparatus}.

If the two subsystems are independent, all states of $S$ are of the form
\begin{equation}\label{eq:separableAB}
  \ket{\psi_S} = \ket{\psi_A}\otimes\ket{\psi_A}
\end{equation}
i.e. they form a strict \emph{subspace} of $\hilb{H}_A\otimes\hilb{H}_B$.

It's worth recalling that in a tensor product space $\hilb{H}_A\otimes\hilb{H}_B$
not all vectors can be expressed as a tensor product as in \eqref{eq:separableAB}.
When a vector in a tensor product space
can be expressed in the form \eqref{eq:separableAB},
it's said \term{separable} \parencite{Nakahara}.

\begin{proposition}\label{TensorBase}
The set of tensor products of base vectors of $\hilb{H}_A$ and $\hilb{H}_B$,
i.e. $$\setof{\ket{i_A} \otimes \ket{\mu_B}}_{i\mu},$$
is a basis for $\hilb{H}_A \otimes \hilb{H}_B$
\end{proposition}

In general, given the above property yet the fact that not all states
in a tensor product space are separable, we can still express any
(generally not separable) state vector of $\hilb{H}_S$
as a \emph{superposition} of separable base vectors
\begin{equation}\label{eq:bipartite_expansion}
  \ket{\psi_S} = \sum_{i, \mu}\alpha_{i\mu}\ket{i_A}\otimes\ket{\mu_B} \neq \ket{\psi_A}\otimes\ket{\psi_B},
\end{equation}
meaning that the
``total'' $\ket{\psi_A}$ and $\ket{\psi_B}$ don't necessarily exist.

Non separable states are said \term{entangled}.
Physically,
$\ket{\psi_S}$ contains information
not only about the results of measurements on $A$ and $B$ separately,
but also on correlations between these measurements
\parencite{Haroche_Exploring}.

The simplest example of entangled system is given by two two-level systems,
namely two spin-$\frac{1}{2}$ particles in a singlet state:
\[
  \ket{\psi_{\text{singlet}}} = \frac{1}{\sqrt{2}} \left(\ket{\uparrow, \downarrow} - \ket{\downarrow, \uparrow}\right).
\]

\section{Partial trace and open systems}

In a bipartite (entangled) systems, one subsystem considered alone is an
\emph{open quantum system} \parencite{open_systems},
while the system as a whole is still a closed systems.

In this and following sections, unless differently noted,
we shall stick with the convetion of \eqref{eq:bipartite_expansion},
of using
latin indices for subsystem $A$ and greek ones for subsystem $B$.

As pointed out in
\cite[sec. 2.3.1]{PreskillNotes}, when we limit our attention to
just part of a larger system, contrary to the axioms of closed systems:
\begin{enumerate}
  \item States are not rays.
  \item Measurements are not orthogonal projections.
  \item Evolution is not unitary.
\end{enumerate}

Now, let's consider an observable $M_A$, acting only on subsystems $A$.
It can be expressed in $\hilb{H}_A \otimes \hilb{H}_B$ as
\[
  M_A \otimes \idop_B\, .
\]
Its expectation value is
(using the expansion in eq. \ref{eq:bipartite_expansion},
and the notation of latin indices for system $A$ and greek ones for system $B$)
\begin{multline}\label{exp_subsys}
  \expval{M_A} = \matrixel{\psi_S}{M_A\otimes\idop_B}{\psi_S} =
  \sum_{j,\nu}a^{*}_{j\nu}\left(\bra{j}\otimes\bra{\nu}\right)\left(M_A\otimes\idop_B\right)\sum_{i,\mu}a_{i\mu}\left(\ket{i}\otimes\ket{\mu}\right) = \\
  \sum_{j,\nu,i,\mu}a^{*}_{j\nu}a_{i\mu}\matrixel{j}{M}{i}\matrixel{\mu}{\idop}{\nu} =
  \sum_{j,\mu,i}a^{*}_{j\mu}a_{i\mu}\matrixel{j}{M}{i} =
  \tr(M_A\rho_a)
\end{multline}

To prove the last equality in \eqref{exp_subsys} we need to define the
\term{density operator} $\rho_A$ for an open system $A$ and to understand its
relation with the whole (closed) system $A+B$.\footnote{
  For some general properties of density operators, see \ref{app:density}.
}

Let's start with the following
\begin{definition}\label{def:pTr}
  The \term{partial trace} operation
  is a linear map
  that takes an operator
  $M_{AB}$ on $\hilb{H}_A \otimes \hilb{H}_B$
  to an operator on $\hilb{H}_A$ defined as
  \[
    \tr_B M_{AB} = \sum_{\mu} \matrixel{\mu}{M_{AB}}{\mu}
    \, .
  \]
\end{definition}

The first thing to note is that the \emph{partial} trace is \emph{operator-valued},
its value is not a scalar as opposed to the \emph{trace}.

Another problem with the \ref{def:pTr} is that $\bra{\mu}$
is a vector in $\hilb{H}_B$, not in $\hilb{H}_A\otimes\hilb{H}_B$,
so what is the meaning of $\matrixel{\mu}{M_{AB}}{\mu}$?

With a slight abuse of notation,
$\bra{\mu}$ can be re-defined as acting upon kets in
$\hilb{H}_A\otimes\hilb{H}_B$ by specifying its action on a basis element.

\begin{definition}
\[
  \braket{\mu}{i\nu} = \bra{\mu}\left(\ket{i}\otimes\ket{\nu}\right) := \delta_{\mu\nu}\ket{i}
\]
\end{definition}

Intuitively, $\bra{\mu}$ is a ``partial bra'', as it doesn't map a ket
of $\hilb{H}_A\otimes\hilb{H}_B$ to
a complex number, but to another ket (in $\hilb{H}_A$ though!).
In some sense, it only maps the ``$\ket{\nu}$'' part of
$\ket{i}\otimes\ket{\nu}$
to a number. This is consistent with the idea of
``tracing out the environment'' (TODO: cite anything?) (within an interpretation where
$\hilb{H}_B$ is the environment). And ultimately with the goal of
studying subsistem $A$ alone, in spite of its entanglement with the 
``environment'' (or measurement apparatus etc.) modeled by
subsystem $B$.
 
\section{The rest aka TODO}

The environment is watching, from ``Exploring the quantum''
\parencite[Ch. 4]{Haroche_Exploring}. Start with subsect. 2.4.1.

2.5.4 The quantum–classical boundary
Decoherence models versus Copenhagen interpretation.

Environment-Induced Decoherence and the Transition from Quantum to Classical,
\cite{Zurek_Fundamentals}.

Partial trace, which is not a number, check wikipedia $\rightarrow$ consistent histories.
J. P. Paz W. H. Zurek

Von Neumann and Shanon entropy.

MIKIO NAKAHARA: QUANTUM COMPUTING: AN OVERVIEW

See also \cite{Schlosshauer_Decoherence}.

Vedral, Brukner (Oreshkov\dots)
Necessary and sufficient condition for non-zero quantum discord
\url{https://arxiv.org/pdf/1004.0190.pdf},
cites another paper by Zurek \url{https://arxiv.org/pdf/quant-ph/0105072.pdf}.

\section{Maximally mixed state of $\ket{e_1}$ and $\ket{e_1}$
Vs their maximal coherent superposition: density matrices}

See Nakahara notes, example II.2.

$$
\frac{1}{2}\begin{bmatrix}
  1 &0  \\
  0 &1  \\
\end{bmatrix}
\text{ Vs }
\frac{1}{2}\begin{bmatrix}
  1 &1  \\
  1 &1  \\
\end{bmatrix}
$$

The off-diagonal terms in the coherent superposition case indicate
``quantumness''. (explain) (Andreas: not really: change of basis\dots)

--- On non-othogonal mixture and non uniqueness of density matrix representations
\url{https://arxiv.org/pdf/quant-ph/0512125.pdf}

\begin{remark}
  The density matrix of a pure state is either diag(000\dots1\dots000) or non-diagonal.

  It can be easily proven reasoning on eigenvalues\dots

  So, except the special/trivial case above, a pure state must have off-diagonal terms.
\end{remark}

Measurement destroys off-diagonal terms and turns a pure state into a mixed one.

Off-diagonal terms in the density matrix indicates ``quantum purity''.
(Or internal Entanglement as well, if the system being detected was bipartite itself?)

In some sense, measurement turned a q-statistical distribution of values for the observable,
into a classical probability distribution.

Recommended read: \cite{Zurek_Decoherence, Zurek_Fundamentals}.
