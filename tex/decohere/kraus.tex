\section{Quantum channels and Kraus operators}

We have seen so far that:
\begin{enumerate}
\item
  A pure state of the bipartite system $\hilb{H}_A\ox\hilb{H}_B$
  may behave like a mixed state when we observe $\hilb{H}_A$ alone
  (Section \ref{sec:p_tr}).
\item
  An orthogonal measurement of the bipartite system can realise a
  (non-orthogonal) POVM on $\hilb{H}_A$ alone (Subsection \ref{subsec:POVM}).
\end{enumerate}
Now, how about the evolution of subsystem $A$ alone,
when the bipartite system as a whole evolve unitarily?

In fact, the measurement described in Subsection \ref{subsec:POVM}
is a particular case of generalized evolution, and the unitary operator $U$
in \eqref{eq:gen_measurement} is a particular example of
evolution operator for the whole bipartite system.

The system $A$ alone
(initially described by the density operator $\rho=\ketbra{\psi}$)
can be studied by tracing out $B$
(in the sense of Section \ref{sec:p_tr}),
or equivalently by measuring $B$
but without recording an outcome, therefore leaving the system
in a statistical mixture.
In either ways, there has:
\begin{equation}\label{eq:channel}
  \rho \rightarrow \superop{E}(\rho) \eqbydef \sum_a M_a \rho M_a^{\dagger}
\end{equation}
which generalises the unitary evolution $\rho \rightarrow U \rho U^{\dagger}$.

The linear map, or ``superoperator'',\footnote{
  A \emph{superoperator} is a linear map that associates an operator
  in a Hilbert space to another operator (instead of a vector to another vector).
}
$\superop{E}$
is called \term{quantum channel}.\footnote{
  Another name for it is
  \term{trace-preserving completely positive map},
  or \term{TPCP map} for short. \parencite[\S 3.2]{PreskillNotes}
}
The word ``channel'' is drawn from communication theory,
as the state $\rho$ can be interpreted as being \emph{transmitted}
through
a communication link from a sender to another party
who receives it modified into the state $\superop{E}(\rho)$.

A quantum channel $\superop{E}$
\begin{enumerate}
  \item is linear:
    $\superop{E}(\alpha\rho_1+\beta\rho_2) = \alpha\superop{E}(\rho_1) + \beta\superop{E}(\rho_2)$;
  \item preserves hermiticity:
    $\rho = \rho^{\dagger} \implies \superop{E}(\rho) = \superop{E}(\rho)^{\dagger}$;
  \item preserves positivity:
    $\rho = \rho^{\dagger} \geq 0 \implies \superop{E}(\rho) \geq 0$;
  \item preserves trace:
    $\tr(\superop{E}(\rho)) = \tr(\rho)$.  
\end{enumerate}

Expressing $\superop{E}$ in terms of operators $M_a$
satisfying the partition of the identity \eqref{eq:POVM}
is called
\term{operator-sum representation} of the quantum channel.

The operators $\setof{M_a}$ are called the \term{Kraus operators}
of the channel.

Similarly to POVM $\setof{E_a}$, given a particular channel $\superop{E}$,
the set $\setof{M_a}$ is not uniquely determined, but
generally exists.

A fundamental comparison with unitary evolution is that
quantum channels can be \emph{composed} too, but an inverse
does not generally exists.
This mathematically corresponds to the concept of \term{semigroup},
and physically to the \emph{irreversibility} of the process
of entanglement of subsystem $A$ with the environment. In other words,
there isn't a quantum channel that will bring back an entangled,
mixed state back to its initial pure state;
but a generalized evolution from time $t_0$ to time $t_1$,
described by $\superop{E}_1$,
and another from $t_1$ to $t_2$ described by $\superop{E}_2$
can be combined to describe the evolution from $t_0$ to $t_2$:
\[
  \rho \rightarrow \qty(\superop{E}_1 \circ \superop{E}_2) (\rho) =
  \sum_{\mu,a} N_{\mu} M_a \rho M_a^{\dagger} N_{\mu}^{\dagger}
  \,\text{.}
\]
If we demand that $\superop{E}_1 \circ \superop{E}_2$
is the identity (or ``superidentity'', should we say),
in other words that $\superop{E}_2$
is the \term{inverse} of $\superop{E}_1$,
we can prove that the channel must be unitary i.e.
$\superop{E}_1(\rho) = U \rho U^{\dagger}$
for some unitary evolution operator $U$.
This excludes decoherence,
or entanglement with environment from an initial pure state,
from being reversible.
See \cite[\S 3.2]{PreskillNotes} for a detailed proof,
and general properties of quantum channels.
