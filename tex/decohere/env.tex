\section{Quantum and the Environment: from Quantum to Classical}
\label{sec:q2c}

In Section \ref{sec:mix} we have considered a simple orthonormal basis
and built a statistical mixture of their vectors. We have shown
that the same density operator can be generated by other ensembles,
not necessarily orthogonal. Nonetheless, the orthonormal ensemble,
corresponding to a diagonal representation of $\rho_A$, \emph{is}
special, only in the sense that an orthonormal, complete set of
vectors can be seen as the eigenbasis for some observable,
and such representation is particularly expressive if the
interaction with the environment we are considering
(or the entanglement with the other part of a bi-partite system, in other terms)
is a \emph{measurement}.

An ideal measurement (without knowing the outcome) turns
the (sub-)system $A$ from a coherent superposition into a
maximally mixed state like in the examples
\eqref{eq:matrix:pure} and \eqref{eq:matrix:mix}.
In other words, it transforms a quantum superposition into a classical probability
distribution of possible eigenstates (or possible \term{wavefunction collapses}, in other terms).

Historically,
the effort in overcoming the limitations of quantum measurement models,
treating the measurement apparatus as a quantum,
and not a classical system,
was pioneered by J. von Neumann \parencite{VonNeumann}.

As a self-adjoint operator, $\rho_A$ after the interaction
can always be diagonalized, so we can conclude that any
interaction with the environment is the measurement of some
observable $M_A$, at least in the mathematical sense: $M_A$ may or
may not be of any notable physical significance.

With this in mind, we can mention
a novel approach formulated in the last decades aimed at
\emph{deriving}
the Born rule
instead of requiring it as a postulate
\parencite{Zurek_Einselect}.

A key concept in \cite{Zurek_Einselect} is the
\term{environment-induced superselection} or \term{einselection},
i.e. the idea that particular \term{pointer states} are not
perturbed by the \term{environmental monitoring}.
Because of their stability, they can be regarded as
\term{memory states}.

Coherent superpositions of pointer states are not pointer states
themselves, and such \emph{coherence} is destroyed in the process.
Contrary,
or complementary to previous studies, which focused on the effect of
the environment on the system, Zurek's work
focuses on
accessibility of information spread throughout the environment,
noting that observers monitor systems indirectly, by intercepting
small fractions of their environments:
\begin{quote}
\emph{Relaxation and noise are caused by the environment perturbing
the system}, while \emph{decoherence and einselection
are caused by the system perturbing the environment}.
\end{quote}
