\section{Density operator}\label{app:density}

A quantum state (either pure or \term{mixed}),
is generally described by a \term{density} operator (or density matrix) $\rho$.

The general expression for a density operator $\rho$ is
$$
  \rho = \sum_{k}p_{k}\ketbra{\psi_{k}}
$$
with $p_{k}$ being non negative and $\sum_{k}p_{k} = 1$;
where $p_k$ indicates the ``classical'' probability of the system to be in state $\ket{\psi_{k}}$.

For a pure state (corresponding to ket $\ket{\psi}$) the density operator reduces to
\[
  \rho_{(\text{pure})} = \ketbra{\psi}\,\text{.}
\]

The expectation value of an observable represented by the operator $A$
is given by \parencite{open_systems}
\begin{equation}\label{eq:expvalA_rho}
  \expval{A} = \tr(A\rho)\,.
\end{equation}

Eq. \eqref{eq:expvalA_rho} is valid for any Hermitean operator $A$. Particularly,
it is also valid if $A$ is a \emph{projector} and we will use this
e.g. in Proposition \ref{probability_rho}.

We will now look at some properties of the density operator.

We can prove the following
\begin{proposition}
  Let $A$ be an Hermitean operator
  a complete set of eigenvectors $\{\ket{m}\}$
  and
  such that
  $$
    \tr(A\rho) = 0 
  $$
  \emph{for any density operator} $\rho$.
  It follows that $A = 0$.

  \begin{proof}
    We can choose
    ---to explicitly evaluate the expression of the trace---
    for each positive integer $n$,
    a particular density operator $\rho = \ketbra{n}$,
    corresponding to a (pure) eigenstate of $A$.

    With this particular choice,
    for each $n$,
    $$
      A\rho = \mel{n}{A}{n}\ketbra{n}\,.
    $$
    It then follows:
    \begin{multline}\label{eq:qmt:alldiagzero}
      0 = \tr(A\rho) = \sum_{m}\mel{m}{ ( \mel{n}{A}{n}\ketbra{n} ) }{m} = \\
          \sum_{m} \mel{n}{A}{n} \braket{m}{n} \braket{n}{m}
        = \mel{n}{A}{n} .
    \end{multline}
    With respect to the basis $\setof{\ket{n}}$ the operator $A$ is
    represented by a matrix whose elements are given by $a_{nm} = \mel{n}{A}{m}$;
    but $\setof{\ket{n}}$ is an eigenbasis, therefore we are only interested in the diagonal
    elements (the off-diagonal ones being zero).
    However, the diagonal elements are  $\mel{n}{A}{n} = 0$
    for all $n$, according to eq. \eqref{eq:qmt:alldiagzero}, therefore the operator itself is $A=0$.
  \end{proof}
\end{proposition}

The \emph{additivity} property follows immediately:

\begin{corollary}
If $\tr(A\rho) + \tr(B\rho) = \tr(C\rho)$ for any density operator $\rho$,
then $A + B = C$.
\end{corollary}
