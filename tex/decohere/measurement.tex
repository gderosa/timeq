\section{Standard measurement and generalizations}

Decoherence,
or more generally the interaction of a system with the environment,
can be seen as a process of information loss for the system
\parencite[Ch. 9]{Nakahara} or information storage
\parencite{Zurek_Einselect}, if the system under consideration
is \emph{the observer}.

A \term{quantum operation} \parencite[Ch. 9]{Nakahara} is the generalization
of unitary evolution to include open systems as well as closed ones.

The evolution of the density operator for a closed system is given by the map
\[
    \rho_{S} \rightarrow \mathcal{E}(\rho_{S}) = U(t)\rho_{S}U^{\dagger}(t) \, \text{.}
\] 
We are looking to describe a general more change
$\rho \rightarrow \mathcal{E}(\rho_{S})$ that may also include
changes due to measurement (or noise).

Furthermore, we are looking to describe a generalization
of projective measurement~\parencite{VonNeumann}:
when a projective, Von-Neumann
measurement is performed on a multipartite system,
it does not necessarily correspond to a projective measurement
on each subsystem~\parencite[Ch. 3]{PreskillNotes}.

\subsection{Review on projectors}

We shall assume, unless otherwise stated, that all vectors are
normalized to unity.

For a given observable represented by an operator $A$ with a
\emph{discrete, non degenerate spectrum}, the probability of the
outcome $a$ of a
measurement on a system in the pure state $\ket{\psi}$ is given by
$$
\pi_{a} = \norm{\braket{a}{\psi}}^2
        = \bra{\psi}\ket{a}\bra{a}\ket{\psi}
        = \mel{\psi}{P_{a}}{\psi}
$$
where $P_{a} = \ketbra{a}$ appears as the \term{projector} on the
one-dimensional eigenspace of  $a$.

If $a$ is degenerate, and $j = 1, \dots, J$ its degeneracy index,
such probability shall sum over it:
$$
\pi_{a} = \sum_{j=1}^{J}\norm{\braket{aj}{\psi}}^2
        = \sum_{j=1}^{J}\bra{\psi}\ket{aj}\bra{aj}\ket{\psi}
        = \mel{\psi}{P_{a}}{\psi}
$$
where $P_{a} = \sum_{j=1}^{J}\ketbra{aj}$
still is the projector on the
($J$-dimensional) eigenspace of $a$.

Generally, the probability that the outcome of a measuremen falls in
the set of eigenvalues $\sigma = \{a_{1}, \dots, a_{S}\}$ is
\begin{equation}\label{eq:pi_sigma}
\pi_{\sigma}  = \sum_{s=1}^{S}\sum_{j=1}^{J_{s}}\norm{\braket{sj}{\psi}}^2
              = \sum_{s=1}^{S}\sum_{j=1}^{J_{s}}\bra{\psi}\ket{sj}\bra{sj}\ket{\psi}
              = \mel{\psi}{P_{\sigma}}{\psi}
              ,
\end{equation}
where $P_{\sigma} = \sum_{s=1}^{S}\sum_{j=1}^{J_{s}}\ketbra{sj}$
is once again a projector --- on the ``generalized eigenspace'' spanned by all
eigenvectors $\{\ket{sj}\}$ above.




%%%%%%%%%%% MOVE TO SOMEWHERE HERE %%%%%%%%%%%%%%%%%%

Looking back at \eqref{eq:expvalA_rho},
it is valid for \emph{any} hermitean operator,
therefore it's also valid for projectors.

Hence, can replace $A$ with the projector $P_{\sigma}$,
associated to the set of eigenvalues $\sigma$
according to \eqref{eq:pi_sigma} and \eqref{eq:P_sigma_spectral}.

This is of particular interest
because its mean value equates the probability that the outcome of a measurement
falls in a given subset of the spectrum of a given observable.
So we can derive:

\begin{proposition}\label{probability_rho}
  The probability $\pi_{\sigma}$
  that the outcome of a measurement of the observable $A$
  on the system described by the state operator $\rho$
  falls in the set $\sigma$
  is given by:
  \begin{equation}\label{eq:probability_rho}
    \pi_{\sigma} = \expval{P_{\sigma}} = \tr(P_{\sigma}\rho)
  \end{equation}
  with $P_{\sigma}$ defined as in \eqref{eq:pi_sigma}
  or, more generally, in \eqref{eq:P_sigma_spectral}.
\end{proposition}

%%%%%%%%%%%%%%%%%%%%%%%%%%%%%%%%%%%%%%%%%%%%%%%%%%%%%





\subsection[Measure]{Measure\footnote{Not to be confused with \emph{measurement}.}}

\begin{remark}\label{measure_properties}
  Being $\pi_{\sigma}$ the \emph{probability} of the outcome of a measurement to
  fall in a given set $\sigma$, it has to be:
  \begin{enumerate}
    \item \label{measure_properties:first} $0$ on the empty set
    \item non negative
    \item \label{measure_properties:last} \term{additive} on disjoint sets
    \item equal to $1$ if $\sigma$ includes the whole spectrum of $A$.
  \end{enumerate}
\end{remark}

\begin{remark}
  Properties \ref{measure_properties:first}\dots\ref{measure_properties:last}
  are the defining properties of a \term{measure} \parencite{EncMath_Measure}.
\end{remark}

The probability $\pi_{\sigma}$ in \eqref{eq:pi_sigma} is linear with respect to the projector
$P_{\sigma}$ hence it's not difficult to derive that the same properties in
\autoref{measure_properties} applies to $P_{\sigma}$, \emph{in the operator sense}.
In fact, the map $\sigma \subseteq \mathbb{R} \rightarrow P_{\sigma}$
implicitly defined in~\eqref{eq:pi_sigma} is a \term{projector-valued measure}.

The result is generalized,
in such a way to include both discrete and continous spectra,
by the following \parencite{VonNeumann, Ballentine}
\begin{theorem}[spectral resolution]
  If $A$ is a self-adjoint operator,
  there is a unique projector-valued measure $E$
  defined on the Borel sets of $\mathbb{R}$
  such that
  \footnote{
    In \eqref{eq:spectral}, $a$ is a real number (not a set),
    but it's intended $E$ to be evaluated
    on the~\emph{interval}~from $-\infty$ to $a$.
  }
  \begin{equation}\label{eq:spectral}
    A=\int_{-\infty}^{\infty}a\, dE(a)
  \end{equation}
  and satisfying:
  \begin{align*}
    E(\mathbb{R})       & =\mathbf{1},\\
    E(B_{1}\cap B_{2}) & =E(B_{1})E(B_{2})\,.
  \end{align*}
\end{theorem}

In terms of this theorem, the projector in \eqref{eq:pi_sigma} is
\begin{equation}\label{eq:P_sigma_spectral}
  P_{\sigma} = E(\sigma) = \int_{a\in\sigma}dE(a)
\end{equation}
and $dE(a)$ is
---informally speaking---
infinitesimal if $a$ belongs to the continous spectrum,
finite if $a$ is a discrete eigenvalue
and zero otherwise.


\subsection{Projective measurement (Von Neumann)}

Let's consider a \term{complete}, \term{orthogonal} set of $N$ projectors
$P_0 \dots P_{N-1}$ on the Hilbert space $\hilb{H}_A$
of the system being measured;
by definition they have the properties:
\begin{align}
  \sum_n P_n  &= \idop \\
  P_i P_j     &= \delta_{ij}P_i \, \text{.}
\end{align}

On $\hilb{H}_B$ instead, describing the measurement device, let's consider
the fiducial\footnote{
  In the sense that it's assumed
  that we can arbitrarily ``reset'' (prepare) the device,
  for example at state $\ket{0}$,
  and in general
  we will find the device at one of the states
  $\ket{0}\dots\ket{N-1}$
  when we observe it (\term{pointer states}).
  Therefore, what follows \emph{is not} a derivation of the Born's rule,
  which is still a necessary postulate of quantum mechanics
  (only \cite{Zurek_Decoherence, Zurek_Einselect, Zurek_Fundamentals} push
  this point of view further). In other words, the Born rule for the system
  will be ``proved''
  only on the assumption that it is valid for the measurement apparatus already.
}
basis:
\[
  \setof{\ket{0} \dots \ket{N-1}} \text{.}
\]

The coupling between the system and the apparatus
can be modelled by
a unitary operator
transforming the state of system and apparatus before and after the measurement.

As we will show, the following is a possible example:
\begin{equation}\label{eq:unitary_measurement}
  U = \sum_{a, b = 0}^{N-1} P_a \ox \ketbra{b+a}{b} \text{,}
\end{equation}
where the sum $b+a$ is \emph{modulo N}.

\begin{eqsplit}\label{eq:explicit_measurement_evolution}
  U:  &\ket{\Psi}           =\ket{\psi} \ox \ket{0} \rightarrow \\
      &\ket{\Psi^{\prime}}  =\sum_{a, b} P_A \ox \ketbra{b+a}{b} \qty(\ket{\psi} \ox \ket{0})
\end{eqsplit}
Indeed, in the \ref{eq:explicit_measurement_evolution} only terms with $b=0$ survive.
Therefore
\begin{eqsplit}\label{eq:measurement_entangled}
  \ket{\Psi^{\prime}} &= \sum_a \qty\Big(P_a\ox\ketbra{a}{0}) \qty\Big(\ket{\psi}\ox\ket{0}) \\
                      &= \sum_a P_a \ket{\psi} \ox \ket{a}
\end{eqsplit}

If we measure the pointer in the fiducial basis
(Hilbert space of the measurement apparatus),
the probability of an outcome $a$ is
\begin{multline}\label{eq:measurement_probability}
  \Pr(a) = \expval{\qty\Big(\idop\ox\ketbra{a})}{\Psi^{\prime}} =
    \sum_{b,c}
      \qty(\bra{\psi}P_{b}\ox\bra{b})
      \qty\Big(\idop\ox\ketbra{a})
      \qty(P_{c}\ket{\psi}\ox\ket{c}) = \\
    \sum_{b,c}\qty\Big(\expval{P_bP_c}{\psi} \braket{b}{a} \braket{a}{c}) =
    \expval{P_a}{\psi}
\end{multline}
which shows that the Born's rule has been ``transferred'', from the system being
measured, to the measurement device and therefore the
\eqref{eq:unitary_measurement} is a correct description of measurement.

The \eqref{eq:measurement_entangled} clearly shows that the system
and the measurement device are completely correlated (entangled).
If the measurement apparatus is observed in state $\ket{a}$
---with probability $\Pr(a)$ as stated in \eqref{eq:measurement_probability}---
then the system being measured is in state $E_{a}\ket{\psi}$
or, in normalized form:
\begin{equation}\label{eq:normalized_collapse}
  \ket{\psi^\prime_a} \eqbydef \frac{P_{a}\ket{\psi}}{\norm{P_{a}\ket{\psi}}}
    = \frac{P_{a}\ket{\psi}}{\sqrt{\expval{P_a}{\psi}}} \,\text{.}
\end{equation}

This is the \term{wave function collapse} described in terms of a unitary
transformation \eqref{eq:unitary_measurement}
acting on the system + detector compound system and describing
the measurement process
(instead of just being postulated as part of the Born's rule).
See \cite[\S 2.5.4, \emph{Decoherence models versus Copenhagen interpretation}]{Haroche_Exploring},
for a closer conceptual examination.

Indeed,
$\ket{\psi}$
is transformed
into its projection $\ket{\psi^\prime_a}$
onto the eigenspace
corresponding to the eigenvalue $a$ of the observable of interest.

The above \emph{is not} a derivation of the Born (probability) rule altogether,
as it still needs to be postulated for the measurement apparatus.

Finally, if the measurement apparatus is not observed,
therefore an outcome $a$ is not known,
the system after measurement is in a statistical mixture
of ``all possible collapses'' weighted on the probability $\Pr(a)$.
By using both \eqref{eq:measurement_probability} and \eqref{eq:normalized_collapse},
and the definition of the density operator for the initial pure state
$\rho = \ketbra{\psi}$:
\[
  \rho^{\prime} = \sum_a \Pr(a) \ketbra{\psi^{\prime}_a} = \sum_a P_a \ketbra{\psi} P_a
    = \sum_a P_a \rho P_a \,\text{.}
\]
So, the initial pure state $\rho$ is transformed by the measurement process into a mixed one.
It is said that the initial, coherent superposition of eigenstates represented by $\rho = \ketbra{\psi}$
\term{decoheres} towards the maximal statistical mixture $\rho^{\prime}$
(as seen in Section \ref{sec:mix}).

It can be proven that the transformation
\begin{equation}\label{eq:irreversible_measurement}
  \rho \rightarrow \sum_a P_a \rho P_a
\end{equation}
is also valid in the more general case of $\rho$ being a mixed state before the measurement
---in this case, it's transformed into another mixed state,
but still described by the \eqref{eq:irreversible_measurement}.
A generalization to observables with a continuous spectrum is also possible
(see e.g. \cite[Section 3.1.1]{PreskillNotes} for more details).


\subsection{Generalized measurement (\term{POVM})}
\label{subsec:POVM}

Let's start considering, for simplicity, a 2-level system,
the corresponding 2-level pointer space,
and the unitary transformation describing the measurement process:
\begin{multline}\label{eq:qubit_ortho_measurement}
  U:
    \ket{\psi}_A \ox \ket{0}_B
    {\color{gray}= \qty\big{\alpha\ket{0}_A + \beta\ket{1}_A} \ox \ket{0}_B}
  \rightarrow \\
    {\color{gray}\alpha\ket{0}_A \ox \ket{0}_B + \beta\ket{1}_A  \ox \ket{1}_B =}
    \:
    E_0\ket{\psi}_A \ox \ket{0}_B + E_1\ket{\psi}_A \ox \ket{1}_B
\end{multline}
with subscripts $A$ and $B$ designating the system of interest and
the measurement apparatus (pointer space) respectively.

When we observe the pointer, let's assume we're not
able to ``measure'' it with respect to the fiducial basis
$\setof{\ket{0}, \ket{1}}$,
but with respect to another basis, say,
\begin{equation}\label{eq:pmbasis}
\setof{\ket{\pm} = \frac{1}{\sqrt{2}} \qty\Big(\ket{0}_B \pm \ket{1}_B)} \,\text{.}
\end{equation}

We can rewrite \eqref{eq:qubit_ortho_measurement} as
\begin{multline}\label{eq:qubit_gen_measurement}
  U: \ket{\psi}_A \ox \ket{0}_B                   \rightarrow \\
  \color{gray}
  \frac{1}{\sqrt{2}} \qty\Big(
    \alpha\ket{0}_A \ox \qty(\ket{+}+\ket{-}) +
    \beta \ket{1}_A \ox \qty(\ket{+}-\ket{-})
  )                                               =           \\
  \color{gray}
  \frac{1}{\sqrt{2}} \qty\Big(
    \qty(\alpha\ket{0}_A + \beta\ket{1}_A) \ox \ket{+} +
    \qty(\alpha\ket{0}_A - \beta\ket{1}_A) \ox \ket{-}
  )                                               \eqbydef      \\
  M_+\ket{\psi}_A \ox \ket{+} + M_-\ket{\psi}_A \ox \ket{-}
  \,\text{,}
\end{multline}
where we have defined:
\begin{align*}
  &
  M_+ \repr \frac{1}{\sqrt{2}}\mqty(\imat{2})
  &
  M_- \repr \frac{1}{\sqrt{2}}\mqty(\dmat[0]{1, -1})
\end{align*}
with respect to the basis $\setof{\ket{0}_A, \ket{1}_A}$.

After measurement, system $A$ is ``prepared''
(up to a normalization factor)
in one of the states $M_{\pm}\ket{\psi}$,
that are not necessarily orthogonal.

Moreover, $M_+$ and $M_-$ are not generally idempotent,
therefore if we repeat the measurement we don't generally
obtain the same result (and don't leave the system $A$ in the same state).
This is a fundamental difference with projective measurement.

Besides this particular example, $M_+$ and $M_-$ are not even necessarily
self-adjoint, and we can see that, while $M$ generalise the projector $P$
in terms of ``collapsing'' (or ``preparing'') the system under measurement,
in some sense a better generalization is in fact $M^{\dagger}M$\footnote{
  Such distinction is inessential for a projector $P$,
  as $P^{\dagger}P = P^2 = P$.
}, particularly in the sense of \term{decomposition of the identity}:
$\sum_a M_a^{\dagger}M_a = \idop$ \parencite[\S 3.1]{PreskillNotes}.

The example of $\ket{\pm}$ is a particularly ``unsharp'' measurement,
or a particularly ``overlapping'' decomposition of the identity,
because $M_+^{\dagger}M_+ = M_-^{\dagger}M_- = \frac{1}{2}\idop$.

Generalising to $N$-dimensional Hilbert spaces, we can replace
$\setof{\ket{\pm}}$ basis with
\[
  \setof{\ket{a}, a = 0\ldots N-1}
\]
and \eqref{eq:qubit_gen_measurement} with
\begin{equation}\label{eq:gen_measurement}
  U: \ket{\psi}_A \ox \ket{0}_B \rightarrow \sum_a M_a\ket{\psi}_A \ox \ket{a}_B
  \,\text{.}
\end{equation}

So, let's define
\[
  E_a = M^{\dagger}M \,\text{.}
\]

All the following holds \parencite[\S 3.1]{PreskillNotes}:
\begin{enumerate}
  \item 
    Hermiticity: \[E_a = E_a^{\dagger}\,\text{;}\]
  \item
    \term{Positivity}: \[\expval{E_a}{\psi} \geq 0\,\text{;}\]  
  \item\label{listitem:POVM}
    Decomposition of the identity (\term{completeness}):
    \begin{equation}\label{eq:POVM}
      \sum_a E_a = \sum_a M_a^{\dagger}M_a = \idop \,\text{;}
    \end{equation}
  \item
    Probability of outcome $a$ of a measurement:
    \[\Pr(a) = \norm{M_a\ket{\psi}}^2 = \expval{E_a}{\psi}\,\text{;}\]
  \item
    Probability of obtaining $b$ in a second measurement:
    \[\Pr(b|a) = \frac{\norm{M_bM_a\ket{\psi}}^2}{\norm{M_a\ket{\psi}}^2}\]
    (would be $\delta_{ab}$ if orthogonal);
  \item
    Probability of outcome, density operator form:
    \[\Pr(a) = \tr(\rho E_a)\,\text{,}\]
    also valid for mixed states.
\end{enumerate}

This partition of the identity by positive operators
as expressed in \eqref{eq:POVM} is called
\term{positive operator-valued measure}, or \term{POVM}.
It generalises the \term{projector-valued measure} (PVM)
found in Von Neumann's theory.

For the sake if simplicity, we're referring to the finite dimensional case,
but the above can be extended to infinite dimensions and continuous spectra,
where the decomposition of the identity can be expressed as
$\int_{-\infty}^{\infty} dx M^{\dagger}(x)M(x) = \idop $.
A more abstract and general definition of POVM is as
follows:\footnote{
  See e.g. \cite{BeneduciPhD, Berberian} --- the level of generalization may vary.
}
\begin{definition}
  Given a (Borel) $\sigma$-algebra $\mathcal{B}(\mathbb{R})$ of subsets of $\mathbb{R}$,
  and the space $\mathcal{F}(\hilb{H})$ of positive, self-adjoint operators on a Hilbert space,
  a \term{positive operator-valued measure} (\term{POVM})
  is a map $E: \mathcal{B}(\mathbb{R}) \rightarrow \mathcal{F}(\hilb{H})$
  such that
  \begin{itemize}
    \item $E(\mathbb{R}) = \idop$ (completeness)
    \item $E\qty(\bigcup\limits_{n} \Delta_n) = \sum\limits_{n} E\qty(\Delta_n)$ (additivity) 
  \end{itemize}
  where $\setof{\Delta_n}$ is a countable family of disjoint sets in
  $\mathcal{B}(\mathbb{R})$.
\end{definition}

It's worth noting that,
given a POVM $E$, operators $M$
suitable for \eqref{eq:POVM} and \eqref{eq:gen_measurement}
and following generalization
can always be found \parencite[\S 3.1]{PreskillNotes},
but not uniquely, in the sense that
$UM$ are also valid
for any unitary operator $U$.

Therefore, this formulation leaves the state after a measurement
with outcome $a$
\[
  \frac{M_a\ket{\psi}}{\norm{M_a\ket{\psi}}^2}
  \leftrightarrow
  \frac{UM_a\ket{\psi}}{\norm{M_a\ket{\psi}}^2}
\]
in fact \emph{undetermined}.
