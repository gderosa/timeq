\section{Bipartite systems, separable and entangled states}

Consider a \term{bipartite quantum system}
i.e. a composite system $S$
made up of two parts, $A$ and~$B$,
described by their respective Hilbert spaces
$\hilb{H}_A$ and $\hilb{H}_B$
(see, for example, \cite{Haroche_Exploring}).

Notably, in \cite{Haroche_Exploring},
notation does not treat the two spaces
on perfectly equal footing,
in the sense that a basis for $\hilb{H}_A$ and a basis for $\hilb{H}_B$
are indicated as
$$
  \setof{\ket{i_A}} \text{ and } \setof{\ket{\mu_B}}\,
$$
with latin and greek indices to label base vectors in the two spaces.

This indicates different physical roles for the two spaces,
such as the \emph{system of interest} and the surrounding \emph{environment},
or the \emph{system being measured} and the \emph{measurement apparatus}.

If the two subsystems are independent, all states of $S$ are of the form
\begin{equation}\label{eq:separableAB}
  \ket{\psi_S} = \ket{\psi_A}\otimes\ket{\psi_A}
\end{equation}
i.e. they form a strict \emph{subspace} of $\hilb{H}_A\otimes\hilb{H}_B$.

It's worth recalling that in a tensor product space $\hilb{H}_A\otimes\hilb{H}_B$
not all vectors can be expressed as a tensor product as in \eqref{eq:separableAB}.
When a vector in a tensor product space
can be expressed in the form \eqref{eq:separableAB},
it's said \term{separable} \parencite{Nakahara}.

\begin{proposition}\label{TensorBase}
The set of tensor products of base vectors of $\hilb{H}_A$ and $\hilb{H}_B$,
i.e. $$\setof{\ket{i_A} \otimes \ket{\mu_B}}_{i\mu},$$
is a basis for $\hilb{H}_A \otimes \hilb{H}_B$
\end{proposition}

In general, given the above property yet the fact that not all states
in a tensor product space are separable, we can still express any
(generally not separable) state vector of $\hilb{H}_S$
as a \emph{superposition} of separable base vectors
\begin{equation}\label{eq:bipartite_expansion}
  \ket{\psi_S} = \sum_{i, \mu}\alpha_{i\mu}\ket{i_A}\otimes\ket{\mu_B} \neq \ket{\psi_A}\otimes\ket{\psi_B},
\end{equation}
meaning that the
``total'' $\ket{\psi_A}$ and $\ket{\psi_B}$ don't necessarily exist.

Non separable states are said \term{entangled}.
Physically,
$\ket{\psi_S}$ contains information
not only about the results of measurements on $A$ and $B$ separately,
but also on correlations between these measurements
\parencite{Haroche_Exploring}.

The simplest example of entangled system is given by two two-level systems,
namely two spin-$\frac{1}{2}$ particles in a singlet state:
\[
  \ket{\psi_{\text{singlet}}} = \frac{1}{\sqrt{2}} \left(\ket{\uparrow, \downarrow} - \ket{\downarrow, \uparrow}\right).
\]
