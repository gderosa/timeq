\section[Channels, operations and generalized measurement]
  {Quantum channels, quantum operations and generalized measurement}

Decoherence,
or more generally the interaction of a system with the environment,
can be seen as a process of information loss for the system
\parencite[Ch. 9]{Nakahara} or information storage
\parencite{Zurek_Einselect}, if the system under consideration
is \emph{the observer}.

A \term{quantum operation} \parencite[Ch. 9]{Nakahara} is the generalization
of unitary evolution to include open systems as well as closed ones.

The evolution of the density operator for a closed system is given by the map
\[
    \rho_{S} \rightarrow \mathcal{E}(\rho_{S}) = U(t)\rho_{S}U^{\dagger}(t) \, \text{.}
\] 
We are looking to describe a general more change
$\rho \rightarrow \mathcal{E}(\rho_{S})$ that may also include
changes due to measurement (or noise).

Furthermore, we are looking to describe a generalization
of projective measurement~\parencite{VonNeumann}:
as when a projective, Von-Neumann
measurement is performed on a multipartite system,
it does not necessarily correspond to a projective measurement
on each subsystem~\parencite[Ch. 3]{PreskillNotes}.

\subsection{Projective measurement (Von Neumann)}

Let's consider a \term{complete}, \term{othogonal} set of $N$ projectors
on the Hilbert space
of the system being measured:
\[
  E_0 \dots E_{N-1} \,\text{;}
\]
a fiducial\footnote{
  In the sense that it's assumed
  that we can arbitrarily ``reset'' (prepare) the device,
  for example at state $\ket{0}$,
  and in general
  we will find the device at one of the states
  $\ket{0}\dots\ket{N-1}$
  when we observe it (\term{pointer states}).
  Therefore, what follows \emph{is not} a derivation of the Born's rule,
  which is still a necessary postulate of quantum mechanics
  (only \cite{Zurek_Decoherence, Zurek_Einselect, Zurek_Fundamentals} push
  this point of view further). In other words, the Born rule for the system
  will be ``proved''
  only on the assumption that it is valid for the measurement apparatus already.
}
basis for the Hilbert space of the measurement device:
\[
  \setof{\ket{0} \dots \ket{N-1}} \text{;}
\]
and a unitary transformation describing how the
system being measured and the measurement apparatus
are changed by their coupling (i.e. the performing of the measurement):
\begin{equation}\label{eq:unitary_measurement}
  U = \sum_{a, b = 0}^{N-1} E_a \ox \ketbra{b+a}{b} \text{,}
\end{equation}
where the sum $b+a$ is \emph{modulo N}.

\begin{eqsplit}\label{eq:explicit_measurement_evolution}
  U:  &\ket{\Psi}           =\ket{\psi} \ox \ket{0} \rightarrow \\
      &\ket{\Psi^{\prime}}  =\sum_{a, b} E_A \ox \ketbra{b+a}{b} \qty(\ket{\psi} \ox \ket{0})
\end{eqsplit}

In the \ref{eq:explicit_measurement_evolution} only terms with $b=0$ survive.
Therefore
\begin{eqsplit}\label{eq:measurement_entangled}
  \ket{\Psi^{\prime}} &= \sum_a \qty\Big(E_a\ox\ketbra{a}{0}) \qty\Big(\ket{\psi}\ox\ket{0}) \\
                      &= \sum_a E_a \ket{\psi} \ox \ket{a}
\end{eqsplit}

If we measure the pointer in the fiducial basis
(Hilbert space of the measurement apparatus),
the probability of an outcome $a$ is
\begin{multline}\label{eq:measurement_probability}
  \Pr(a) = \expval{\qty\Big(\idop\ox\ketbra{a})}{\Psi^{\prime}} =
    \sum_{b,c}
      \qty(\bra{\psi}E_{b}\ox\bra{b})
      \qty\Big(\idop\ox\ketbra{a})
      \qty(E_{c}\ket{\psi}\ox\ket{c}) = \\
    \sum_{b,c}\qty\Big(\expval{E_bE_c}{\psi} \braket{b}{a} \braket{a}{c}) =
    \expval{E_a}{\psi}
\end{multline}
which shows that the Born's rule has been ``transfered'', from the system being
measured, to the measurement device and therefore the
\eqref{eq:unitary_measurement} is a correct description of measurement.

The \eqref{eq:measurement_entangled} clearly shows that the system
and the measurement device are completely correlated (entangled).
If the measurement apparatus is observed in state $\ket{a}$
---with probability $\Pr(a)$ as stated in \eqref{eq:measurement_probability}---
then the system being measured is in state $E_{a}\ket{\psi}$
or, in normalized form:
\begin{equation}\label{eq:normalized_collapse}
  \ket{\psi^\prime_a} \eqdef \frac{E_{a}\ket{\psi}}{\norm{E_{a}\ket{\psi}}}
    = \frac{E_{a}\ket{\psi}}{\sqrt{\expval{E_a}{\psi}}} \,\text{.}
\end{equation}

This is the \term{wavefunction collapse} \emph{derived} in terms of a unitary
transformation \eqref{eq:unitary_measurement}
acting on the system + detector compound system and describing
the measurement process
(instead of just being postulated as part of the Born's rule).
See \cite[sec. 2.5.4, \emph{Decoherence models versus Copenhagen interpretation}]{Haroche_Exploring},
for a closer conceptual examination.

Indeed,
$\ket{\psi}$
is transformed
into its projection $\ket{\psi^\prime_a}$
onto the eigenspace
corresponding to the eigenvalue $a$ of the observable of interest.

The above \emph{is not} a derivation of the Born (probability) rule altogether,
as it still needs to be postulated for the measurement apparatus.

Finally, if the measurement apparatus is not observed,
therefore an outcome $a$ is not known,
the system after measurement is in a statistical mixture
of ``all possible collapses'' weighted on the probabilty $\Pr(a)$.
By using both \eqref{eq:measurement_probability} and \eqref{eq:normalized_collapse},
and the definition of the density operator for the initial pure state
$\rho = \ketbra{\psi}$:
\[
  \rho^{\prime} = \sum_a \Pr(a) \ketbra{\psi^{\prime}_a} = \sum_a E_a \ketbra{\psi} E_a
    = \sum_a E_a \rho E_a \,\text{.}
\]
So, the initial pure state $\rho$ is transformed by the measurement process into a mixed one.
It is said that the initial, coherent superposition of eigenstates represented by $\rho = \ketbra{\psi}$
\term{decoheres} towards the maximal statistical mixture $\rho^{\prime}$
(as seen in Section \ref{sec:mix}).

It can be proven that the transformation
\begin{equation}\label{eq:irreversible_measurement}
  \rho \rightarrow \sum_a E_a \rho E_a
\end{equation}
is also valid in the more general case of $\rho$ being a mixed state before the measurement
---in this case, it's transformed into another mixed state,
but still described by the \eqref{eq:irreversible_measurement}.
A generalization to observables with a continuous spectrum is also possible.
See e.g. \cite[Section 3.1.1]{PreskillNotes} for more details.


\subsection{Generalized measurement}

Let's start considering, for simplicity, a 2-level system,
the corresponding 2-level pointer space,
and the unitary transformation describing the measurement process:
\begin{equation*}
  U:
    \qty\big{\alpha\ket{0}_A + \beta\ket{1}_A} \ox \ket{0}_B
    \rightarrow
    \alpha\ket{0}_A \ox \ket{0}_B + \beta\ket{1}_A \ox \ket{1}_B
\end{equation*}
with subscripts $A$ and $B$ designating the system of interest and
the measurement apparatus (pointer space) repectively.

But when we observe the pointer, let's assume we're not
able to ``measure'' it with respect to the fiducial basis
$\setof{\ket{0}, \ket{1}}$,
but with respect to another basis, say,
$\setof{\ket{\pm} = \frac{1}{\sqrt{2}} \qty(\ket{0} \pm \ket{1})}$.

\iftodo
\section{TODO}
\texttt{img/kraus-andreas/} from sources,
then delete scans.
\fi
