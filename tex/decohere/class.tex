\section{Quantum and the Environment: from Quantum to Classical}
\label{sec:q2c}

In sec. \ref{sec:mix} we have considered a simple orthonormal basis
and built a statistical mixture of their vectors. We have shown
that the same density operator can be generated by other ensembles,
not necessarily orthogonal. Nonetheless, the orthonormal ensemble,
corresponding to a diagonal representation of $\rho$, \emph{is}
special, only in the sense that an orthonormal, complete set of
vectors can be seen as the eigenbasis for some observable,
and such representation is particularly expressive if the
interaction with the environment we are considering
(or the entanglement with the other part of a bi-partite system, in other terms)
is a \emph{measurement}.

A measurement turns
the (sub-)system $S$ from a coherent superposition into a
maximally mixed state like in the examples
\eqref{eq:matrix:pure} and \eqref{eq:matrix:mix}.
A quantum superposition into a classical probability
distribution of possible eigenstates.
Following this route, the Born rule, a fundamental concept in the
Copenhagen interpretation of quantum mechanics, can be
\emph{derived} instead of postulated
\parencite{Zurek_Einselect}.
Historically,
the effort in overcoming the \term{wavefunction collapse},
treating the measurement apparatus as a quantum,
and not a classical system,
was pioneered by J. von Neumann \parencite{VonNeumann}.

As a self-adjoint operator, $\rho_S$ after the interaction
can always be diagonalized, so we can conclude that any
interaction with the environment is the measurement of some
observable $A$, at least in the mathematical sense: $A$ may or
may not be of any notable physical significance.

\section{TODO}

Recommended read: \cite{Zurek_Decoherence, Zurek_Fundamentals}.

Zurek above: DeWitt, Everett, gell_Mann, hartl, Many Worlds, consistent/decoherent historis:
idea: Lagrangian over a history? Principle of least action?

EVEN BETTER ZUREK \url{http://public.lanl.gov/whz/images/decoherence.pdf} sec. D.2
+ video Trieste. And \url{http://public.lanl.gov/whz/images/Entanglement2.pdf}.

Zurek: ``Reduction of the Wavepacket: How Long Does it Take?'',
``quantum''' time?

Zurek \url{https://arxiv.org/abs/quant-ph/0408125} and above on
``memory'' systems?

With Santamato: \url{https://arxiv.org/pdf/1604.01471.pdf}

Nakahara, Ch. 9 book, ``Tracing out the extra degrees of freedom makes it impossible to invert a quantum operation.''

The environment is watching, from ``Exploring the quantum''
\parencite[Ch. 4]{Haroche_Exploring}. Start with subsect. 2.4.1.

2.5.4 The quantum–classical boundary
Decoherence models versus Copenhagen interpretation.

Environment-Induced Decoherence and the Transition from Quantum to Classical,
\cite{Zurek_Fundamentals}.

Partial trace, which is not a number, check wikipedia $\rightarrow$ consistent histories.
J. P. Paz W. H. Zurek

Von Neumann and Shanon entropy.

MIKIO NAKAHARA: QUANTUM COMPUTING: AN OVERVIEW

See also \cite{Schlosshauer_Decoherence}.

Vedral, Brukner (Oreshkov\dots)
Necessary and sufficient condition for non-zero quantum discord
\url{https://arxiv.org/pdf/1004.0190.pdf},
cites another paper by Zurek \url{https://arxiv.org/pdf/quant-ph/0105072.pdf}.
