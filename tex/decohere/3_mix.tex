\section{(Maximally) mixed Vs (maximally) coherent}
\label{sec:mix}

Given, for simplicity, a two-level system (a qubit), and an orthonormal basis
$\setof{\ket{0}, \ket{1}}$, it may be of interest to compare a \term{coherent superposition}
of the two pure states $\ket{0}$ and $\ket{1}$, with equal probability on the outcome of
a measurement (but in a well-defined quantum state); with a statistical mixture of
states $\ket{0}$ and $\ket{1}$, again with equal probability, but 
\emph{in the determination of the quantum state}.

A natural example of such coherent superposition is \parencite[Example 2.4]{Nakahara}
\[
  \ket{\psi} = \frac{1}{\sqrt{2}}\ket{0} + \frac{1}{\sqrt{2}}\ket{1}\text{,}
\]
and the corresponding density operator is
\begin{equation}\label{eq:matrix:pure}
  \rho = \ketbra{\psi}{\psi} =
  \frac{1}{2}\qty\Big(\ket{0} + \ket{1}) \qty\Big(\bra{0} + \bra{1}) =
  \frac{1}{2}\sum_{ij=0}^1\ketbra{i}{j} \repr
  \frac{1}{2}
    \begin{pmatrix}
      1 &1  \\
      1 &1  \\
    \end{pmatrix}
  \text{.}
\end{equation}
For the statistical mixture, the density operator is, by definition,
\begin{equation}\label{eq:matrix:mix}
  \rho = \frac{1}{2}\ketbra{0}{0} + \frac{1}{2}\ketbra{1}{1} \repr
  \frac{1}{2}
    \begin{pmatrix}
      1 &0  \\
      0 &1  \\
    \end{pmatrix}
  \text{.}
\end{equation}

By comparing matrices in \eqref{eq:matrix:pure} and \eqref{eq:matrix:mix}
we may conclude that the off-diagonal terms in the coherent superposition case
indicate \emph{coherence}, or ``'quantum purity'';
however, density operators are self-adjoint operators
(as linear combination of projectors) and can always be expressed in
diagonal form, including density operators of pure states: but it would be
a particular case of diagonal matrix, where all diagonal elements are zero
except one. Indeed, for a pure state:
\[
  \rho = \ketbra{\psi}{\psi}\text{,}
\]
and recalling the assumption that $\ket{\psi}$ is normalized,
we can always build a basis $\qty{\ket{e_j}}$ including $\ket{e_{j_0}} = \ket{\psi}$
among its elements, therefore the diagonal representation is
\[
  \begin{pmatrix}
    0           &       &       &       &       &           \\
                &\ddots &       &       &       &           \\
                &       &1      &       &       &           \\
                &       &       &\ddots &       &           \\
                &       &       &       &\ddots &           \\
                &       &       &       &       &0
  \end{pmatrix}\text{.}
\]

All the above allows us to conclude what follows:
\begin{remark}
  The density matrix of a pure state is either diag(000\dots1\dots000) or non-diagonal.

  So, except the special/trivial case above, a pure state must have off-diagonal terms.
\end{remark}

It's interesting, as well, to look at ensambles (mixtures)
which are not necessarily made up of orthonormal pure states.
In fact, an ensamble of orthonormal states has ``nothing special'',
and different ensambles may lead to the same density operator.
The diagonalized density matrix just shows the ``orthonormal''
ensamble which, again, has no special physical meaning.

For example, let's consider this orthogonal ensamble (diagonal density matrix):
\[
  \rho_1 = \frac{3}{4}\ketbra{0}{0} + \frac{1}{4}\ketbra{1}{1}
\]
and
\[
  \rho_2 = \frac{1}{2}\ketbra{a}{a} + \frac{1}{2}\ketbra{b}{b}\text{.}
\]
with
\begin{align*}
  \ket{a} &= \frac{4}{4}\ket{0} + \frac{1}{4}\ket{1} \\
  \ket{b} &= \frac{4}{4}\ket{0} - \frac{1}{4}\ket{1}
\end{align*}
It's easy to prove that $\rho_1 = \rho_2$. This is just a particular
case of a theorem.
\begin{theorem}{(Unitary freedom in the ensemble for density matrices).}
  The ensambles
  $\setof{p_i, \ket{\psi_i}}$ and $\setof{q_{j}, \ket{\phi_j}}$
  generate the same density operator
  $\rho = \sum_i p_i \ketbra{\psi_i} = \sum_j q_j \ketbra{\phi_j}$
  if and only if there exists
  a unitary matrix\footnote{
    The transformation represented by the unitary matrix $\setof{u_{ij}}$
    does not preserve the orthogonality of the vectors sets,
    which seems to contradict a fundamental property of unitary
    transofrmations: however, it should be noted that the coefficients
    $u_{ij}$ do not act on components of each state vector; in other
    words, $\setof{u_{ij}}$ represent a transformation acting
    on ``vectors of vectors'' and therefore is not really a unitary
    transformation in a Hilbert space.
  }
  $\setof{u_{ij}}$ such that
  \[
    \tilde{\ket{\psi_i}} = \sum_j u_{ij}\tilde{\ket{\phi_j}}
  \]
  where $\tilde{\ket{\psi_i}}$ and $\tilde{\ket{\phi_j}}$
  are the (not normalized) vectors defined as
  \[
    \tilde{\ket{\psi_i}} = p_{i}\ket{\psi_i}
    \text{ and }\,
    \tilde{\ket{\phi_j}} = q_{j}\ket{\phi_j}
  \]
\end{theorem}
See \cite[Theorem 2.6 and introductory example]{NielsenChuang}
for a more detailed description and a proof of the theorem.
