\section{(Maximally) mixed state vs (maximally) coherent superposition}

We have seen how a density 

See Nakahara notes, example II.2.

$$
\frac{1}{2}\begin{bmatrix}
  1 &0  \\
  0 &1  \\
\end{bmatrix}
\text{ Vs }
\frac{1}{2}\begin{bmatrix}
  1 &1  \\
  1 &1  \\
\end{bmatrix}
$$

The off-diagonal terms in the coherent superposition case indicate
``quantumness''. (explain) (Andreas: not really: change of basis\dots)

--- On non-othogonal mixture and non uniqueness of density matrix representations
\url{https://arxiv.org/pdf/quant-ph/0512125.pdf}. See also: relative states.

But also Nielsen and Chuang excercise  2.71.

\begin{remark}
  The density matrix of a pure state is either diag(000\dots1\dots000) or non-diagonal.

  It can be easily proven reasoning on eigenvalues\dots

  So, except the special/trivial case above, a pure state must have off-diagonal terms.
\end{remark}

Measurement destroys off-diagonal terms and turns a pure state into a mixed one.

Off-diagonal terms in the density matrix indicates ``quantum purity''.
(Or internal Entanglement as well, if the system being detected was bipartite itself?)

In some sense, measurement turned a q-statistical distribution of values for the observable,
into a classical probability distribution.

Recommended read: \cite{Zurek_Decoherence, Zurek_Fundamentals}.
