\section{(Maximally) mixed state vs (maximally) coherent superposition; diagonal density matrices}

Given, for simplicity, a two-level system (a qubit), and a basis
$\{\ket{0}, \ket{1}\}$, it may be of interest to compare a \term{coherent superposition}
of the two pure states $\ket{0}$ and $\ket{1}$, with equal probability on the outcome of
a measurement (but in a well-defined quantum state); with a statistical mixture of
states $\ket{0}$ and $\ket{1}$, again with equal probability, but 
\emph{in the determination of the quantum state}.

A natural example of such coherent superposition is \parencite[Example 2.4]{Nakahara}
\[
  \ket{\psi} = \frac{1}{\sqrt{2}}\ket{0} + \frac{1}{\sqrt{2}}\ket{1}\text{,}
\]
and the corresponding density operator is
\begin{equation}\label{eq:matrix:pure}
  \rho = \ketbra{\psi}{\psi} =
  \frac{1}{2}\qty\Big(\ket{0} + \ket{1}) \qty\Big(\bra{0} + \bra{1}) =
  \frac{1}{2}\sum_{ij=0}^1\ketbra{i}{j} \repr
  \frac{1}{2}
    \begin{pmatrix}
      1 &1  \\
      1 &1  \\
    \end{pmatrix}
  \text{.}
\end{equation}
For the statistical mixture, the density operator is, by definition,
\begin{equation}\label{eq:matrix:mix}
  \rho = \frac{1}{2}\ketbra{0}{0} + \frac{1}{2}\ketbra{1}{1} \repr
  \frac{1}{2}
    \begin{pmatrix}
      1 &0  \\
      0 &1  \\
    \end{pmatrix}
  \text{.}
\end{equation}

By comparing matrices in \eqref{eq:matrix:pure} and \eqref{eq:matrix:mix}
we may conclude that the off-diagonal terms in the coherent superposition case
indicate ``quantumness'' or \emph{coherence}; and diagonal



--- On non-othogonal mixture and non uniqueness of density matrix representations
\url{https://arxiv.org/pdf/quant-ph/0512125.pdf}. See also: relative states.

But also Nielsen and Chuang excercise  2.71.

\begin{remark}
  The density matrix of a pure state is either diag(000\dots1\dots000) or non-diagonal.

  It can be easily proven reasoning on eigenvalues\dots

  So, except the special/trivial case above, a pure state must have off-diagonal terms.
\end{remark}

Measurement destroys off-diagonal terms and turns a pure state into a mixed one.

Off-diagonal terms in the density matrix indicates ``quantum purity''.
(Or internal Entanglement as well, if the system being detected was bipartite itself?)

In some sense, measurement turned a q-statistical distribution of values for the observable,
into a classical probability distribution.

Recommended read: \cite{Zurek_Decoherence, Zurek_Fundamentals}.
