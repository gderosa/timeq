\section{Time evolution and relativity}\label{sec:trel}

Eq.~\eqref{eq:diracdeltax} is a mathematical identity
of general validity,
regardless of any particular physical meaning,
and can be easily generalized to functions
of more variables.

On a different note
---as a matter of experience if not anything else---
the state of a physical system, either classical or quantum,
with all observable quantities,
can be regarded as a function of the instant in \emph{time}
(let's call that independent variable $t$).
The same cannot be stated in general for any other pair of measurable quantities:
the $x$ coordinate is not a function of the $y$ coordinate
(or their respective statistical amplitudes, in the quantum realm),
and so on.
This status of time as the universally independent variable
has shaped classical mathematical physics: Galilean transformations
do not change time, which stays as an absolute quantity.
The Hamiltonian formalism itself is based on this assumption,
and the Hamiltonian formalism is at the foundation of quantum mechanics too,
which thus inherits ---one may argue--- the difficulties connected to
treating time as ``external'' with respect to the other quantities under study.
``[The] Hamiltonian method [\dots] marks out a particular time variable
as the canonical conjugate of the Hamiltonian function'' \parencite{DiracLagrangian}.

Emphasizing time as a parameter, the value of a wavefunction
(in position representation, for example)
can therefore be expressed as $\psi_{t}(x)$ at each position $x$
and at each time $t$.
Still, it can be regarded as a function of two variables
and one may wonder why (after an inessential change of notation),
in the identity
\begin{equation}\label{eq:diracdeltaxt}
  \psi(x; t) = \int \dd{x'}\dd{t'} \delta(x-x')\delta(t-t') \psi(x';\, t') \,\text{,}
\end{equation}
the term $\delta(t-t')$ cannot be interpreted as the eigenfunction of some time operator,
which in turn acts as a simple multiplication by $t$ on this
wavefunction $\psi(x; t)$ ---which would be, therefore, a wavefunction in ``time--position representation''.
In other words, why cannot the instant in \emph{time}
and the position in \emph{space}
be treated equally?

The fact that $\delta(t-t')$ would be an \emph{improper} eigenfunction
is not the blocking issue in this case (it is not, for the position eigenfunction),
and the mathematical intricacies related to the continuous spectrum are
elegantly resolved by the spectral theory, mainly by Von Neumann
\parencite{VonNeumann}, which is an integral part of the standard, commonly accepted
formulation of quantum mechanics, almost since the early years of the theory.

The idea of treating space and time on equal footing might naturally lead the reader to
ask another question: whether a theory combining quantum mechanics with Einstein's
theory of relativity is the suitable framework which would allow a rigorous and logically consistent
definition of a time observable,
possibly exposing an analogy with the spatial position.

None of the proposed models comprehending both quantum theory and \emph{general} relativity
have reached general consensus,
or the remote possibility of experimental verification to date \parencite{QGravIntro}.
However, \term{quantum field theory} (QFT) is an established framework that does combine
some principles
of quantum mechanics with the \emph{special} theory of relativity.
Unfortunately, QFT not only does not promote time $t$ to some quantum operator $\hat{t}$,
but it does also ``demote'' \emph{position} to the role of a classical parameter.\footnote{
  More on this in Section \ref{sec:KG}, where the Klein-Gordon equation is reformulated
  with time as an Hermitian operator in an appropriate Hilbert space
  ---based on the Page--Wootters model, which is introduced in Chapter~\ref{ch:pw}.
}
It treats time and space equally only in the sense that they are all classical,
which appears as the necessary tradeoff in order to quantize other quantities
in some mathematically tractable way. \parencite[\S I.1]{SrednickiQFT}.

It is no surprise that one of the most successful applications
of QFT, the Standard Model, describes all fundamental interactions except gravity:
while electroweak and strong interactions can be described within a field theory
which treats time and space as a ``background'' (or \emph{labels} to mark the relevant operators),
time and space are expected to be the main mathematical objects (observables),
and transform appropriately,
in a theory of gravity
which has general relativity as its classical counterpart.
Whether the definition of a quantum time operator is an useful step towards
a quantum theory of gravity is, of course, beyond the scope of the present work.
