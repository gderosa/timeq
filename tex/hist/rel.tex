\section{Time evolution and relativity}\label{sec:trel}

The idea of treating space and time on equal footing might naturally lead the reader to
ask another question: whether a theory combining quantum mechanics with Einstein's
theory of relativity is the suitable framework which would allow a rigorous and logically consistent
definition of a time observable,
possibly exposing an analogy with the spatial position.

None of the proposed models comprehending both quantum theory and \emph{general} relativity
have reached general consensus,
or the remote possibility of experimental verification to date \parencite{QGravIntro}.
However, \term{quantum field theory} (QFT) is an established framework that does combine
some principles
of quantum mechanics with the \emph{special} theory of relativity.
Unfortunately, QFT not only does not promote time $t$ to some quantum operator $\hat{t}$,
but it does also ``demote'' \emph{position} to the role of a classical parameter.\footnote{
  More on this in Section \ref{sec:KG}, where the Klein-Gordon equation is reformulated
  with time as an Hermitian operator in an appropriate Hilbert space
  ---based on the Page--Wootters model, which is introduced in Chapter~\ref{ch:pw}.
}
It treats time and space equally only in the sense that they are all classical,
which appears as the necessary tradeoff in order to quantize other quantities
in some mathematically tractable way. \parencite[sec.I.1]{SrednickiQFT}.

It is no surprise that one of the most successful applications
of QFT, the Standard Model, describes all fundamental interactions except gravity:
while electroweak and strong interactions can be described within a field theory
which treats time and space as a ``background'' (or \emph{labels} to mark the relevant operators),
time and space are expected to be the main mathematical objects (observables),
and transform appropriately,
in a theory of gravity
which has general relativity as its classical counterpart.
Whether the definition of a quantum time operator is an useful step towards
a quantum theory of gravity is, of course, beyond the scope of the present work.
