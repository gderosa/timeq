\section{Allcock and following developments}\label{sec:allcock}

With three papers dated 1969 \parencite{Allcock-1, Allcock-2, Allcock-3}, G. R. Allcock
introduced a time-of-arrival quantum observable,
along with a first formulation of a detector model based on a non-Hermitian
Hamiltonian (by including a \term{complex potential}). %
%See in particular \cite[\s II-IV]{Allcock-2}, where
An anti-Hermitian term
$-iV_0\theta(x)$ is introduced in the Hamiltonian,
to model an absorber aimed at detecting
the arrival of a particle in the region $x>0$
(here $\theta$ is the Heaviside \term{step function}).
The particle state evolves
in a non-unitary manner, with a transfer of probability
from an ``incident channel''
into ``orthogonal and time-labeled measurement channels''.

Allcock's initial proposal was in fact based on a \emph{discrete} model, where
the wavefunction in the detection region $x>0$
is periodically removed | i.e. transformed into $\psi(x) = 0 \; \forall x>0$,
while it is left unchanged for $x < 0$; then it evolves according to the Schr\"odinger equation
for a time $\delta t$ when the next ``removal'' occurs, and so forth.

This suffered mathematical difficulties, which led Allcock to formulate a continuous extension
based on the aforementioned complex potential,
where the wavefunction norm can continuously decrease as the particle is detected.
He showed that the discrete and continuous model
can be compared in terms of time resolution, which is $\delta t$ for the former,
and $\hbar/2V_0$ for the latter.

Allcock noticed that when $V_0$ is large the particle is
rather reflected
than absorbed;
while, when it is small (i.e. the detector has a low absorption rate),
the particle is eventually absorbed but
the indetermination $\delta_t \sim \hbar/2V_0$
is, in such case, impractically large.

Overall, Allcock came thus to a negative conclusion and did not overcome the Pauli objection.
However, the problem with this complex potential model, either causing reflection
or not absorbing sufficiently, was eventually resolved
by noting that his results were not general, and
potentials can be constructed which absorb the whole wavepacket
and avoid reflection \parencite{Muga_TOAQM, Muga_CompositeAbsPot, ComplexAbsPot}.

\subsection{How a complex potential emerges}\label{sec:halliwell}

Among the methods to define a time-of-arrival quantum observable,
the detector model by Allcock \parencite{Allcock-2},
including the following enhancements,
is an \emph{operational} one,
in that it models a measurement procedure \parencite[\s 9]{Leavens_TOA}
rather than focusing on constructing an operator that satisfies certain requirements
(like the Kijowski distribution, see for example \citereset\cite[\s 8]{Leavens_TOA}).

The introduction of complex potentials, a non-Hermitian Hamiltonian and
the non-unitary time evolution (of what appears formally a pure state)
seems artificial
and in contradiction with the fundamental rules of quantum mechanics.

% \citereset
% In fact, as pointed out well in \cite{Halliwell_Irreversible},
% an \emph{effective} complex potential and the consequent loss of normalization
% only emerge after ``tracing out'' a portion of the whole system,
% which includes the particle, the detector and the environment altogether.
% Such whole tripartite system \emph{is} closed, in a pure state, and evolves unitarily.
% The particle wavefunction is also an ``effective'' one derived from its density operator.
% (See also
%   \cite{Wave-function_approach, Hegerfeldt_WignerSymposium, TheQuantumJumpApproach};
%   \cite[\ch 6]{TQM2};
% and
%   \cite[\s 6.7.1 ``Simulating Quantum Trajectories'']{WallsMilburn}).

% Not surprisingly, absorption is an \term{irreverersible} process.

While a complex potential can be \emph{postulated} in order to obtain
phenomenological laws governing arrival times and detectors,
it can also be \emph{obtained} % from first principles and  % Andreas...
within the established framework
of open quantum systems, namely the master equation
for an irreversible detector.
The core ideas behind this derivation
can be ---very briefly--- summarized as follows \parencite{Halliwell_Irreversible}.

The detector is modeled by a two-level system with $\ket{1}$ being the state of no-detection
and $\ket{0}$ the state of detection.
We also introduce the raising and lowering operators $\sigma_{+}=\ketbra{1}{0}$, $\sigma_{-}=\ketbra{0}{1}$.
The Hamiltonian of the detector is such to have $\ket{0}$
at a lower energy, so the detector ``decays'' when it ``clicks''. It is an irreversible transition because
of the coupling with the environment.
The Hamiltonian encompassing the particle ($H_s$), the detector ($H_d$), the environment ($H_E$) and the interaction ($H_dE$) reads
\begin{equation}
  H = H_s + H_d + H_{E} + V(x)  H_{dE} \,\text{,}
\end{equation}
with $H_s$ being the Hamiltonian of a free particle and
\begin{subequations}\begin{align}
  H_d     &= \frac{1}{2}\hbar\omega \qty(\ketbra{1} - \ketbra{0}) \\
  H_{E}   &= \sum_n \hbar \omega_n a_n^{\dag} a_n \\
  H_{dE}  &= \sum_n \hbar \left( \kappa^*_n \sigma_{-} a_n^{\dag} + \kappa_n \sigma_{+} a_n \right) \\
  V(x)    &= \theta(-x) \,\text{,}
\end{align}\end{subequations}
where $a$ and $a^{\dagger}$ are the creation and annihilation operator for the electromagnetic field,
which constitutes the ``environment''; $V(x)$ is chosen to be a step function
i.e. the simplest function that makes the detector respond
when the particle reaches the region ($x<0$);
and the coupling constants $\kappa_{n}$ are to be intended in the same sense of the Jaynes--Cummings model
---in fact, the expressions of $H_d$ and $H_E$ also recall it
(\cite[\s 10.2]{WallsMilburn}, \cite{JCM} and many others).
This model essentially describes the detector as a Jaynes-Cummings atom
with the peculiarity that its coupling with the environment
is only activated by the position distribution
of another particle.

Tracing out the environment, and with some ``standard'' assumptions
(initial environment at zero temperature,
initial separable state ``undetected'' $\rho(0) = \ketbra{\psi_0}\ox\ketbra{1}$,
Markov approximation,
weak coupling),
the \emph{reduced} dynamics of the particle and the detector
is governed by the \emph{master equation}\footnote{
  Most details, calculation steps and possible generalizations are clearly skipped here, and can be found
  in the original article by J. J. Halliwell \parencite{Halliwell_Irreversible}
  and references therein.
}
\begin{equation}
  \dot{\rho} = -\frac{\iu}{\hbar} [ H_s + H_d, \rho]
- { \gamma \over 2} \left( V^2 (x ) \sigma_{+} \sigma_{-}  \rho \ +  \rho
\sigma_{+} \sigma_{-}   V^2 (x)  \ -  \ 2 V (x) \sigma_{-}  \rho \sigma_{+} V (x )
\right)
\,\text{,}
\end{equation}
with $\gamma$ being ``a phenomenological constant determined by the distribution of oscillators in the
environment and underlying coupling constants'' \parencite{Halliwell_Irreversible}.

A general solution is in the form
\begin{equation}
  \rho =
  \rho_{11} \ox \ketbra{1}{1}
+ \rho_{01} \ox \ketbra{0}{1}
+ \rho_{10} \ox \ketbra{1}{0}
+ \rho_{00} \ox \ketbra{0}{0}
\,\text{,}
\end{equation}
with
\begin{subequations}\begin{align}
  \dot{\rho}_{11} =& -\frac{\iu}{\hbar} [ H_s, \rho_{11} ] -\frac{\gamma}{2}\left(V(x)\rho_{11} + \rho_{11}V(x)\right) \label{eq:drho11} \\
  \dot{\rho}_{01} =& -\frac{\iu}{\hbar} [ H_s, \rho_{01} ] -\frac{\gamma}{2}\rho_{01}V(x) + \iu \frac{\hbar\omega'}{2} \rho_{01}\\
  \dot{\rho}_{10} =& -\frac{\iu}{\hbar} [ H_s, \rho_{10} ] -\frac{\gamma}{2}V(x)\rho_{10} - \iu \frac{\hbar\omega'}{2} \rho_{10}\\
  \dot{\rho}_{00} =& -\frac{\iu}{\hbar} [ H_s, \rho_{00} ] +\gamma V(x)\rho_{11}V(x) \,\text{,}
\end{align}\end{subequations}
and $\omega'$ a ``renormalized'' value for the characteristic frequency $\omega$ in $H_d$.

To give an idea of how the complex potential emerges, let us now focus on $\rho_{11}$
and the corresponding probability of not being detected, which is
\begin{equation}\label{eq:p_nd}
  p_{nd} = \Tr \rho_{11}(\tau) = \int_{-\infty}^{\infty} \dd{x} \rho_{11}(x, x, \tau) \,\text{,}
\end{equation}
where $\rho_{11}(x, x, \tau)$
-- or, more generally, $\rho_{ij}(x_1, x_2, \tau)$ for each $i, j = 0, 1$ --
are (diagonal) density matrix elements such that, in general:
\begin{equation}
  \rho_{ij}(\tau) = \int_{-\infty}^{\infty} \dd{x_1} \int_{-\infty}^{\infty} \dd{x_2} \rho_{ij}(x_1, x_2, \tau) \ketbra{x_1}{x_2}
  \, \text{.}
\end{equation}

Now, the solution to the \eqref{eq:drho11} can be written
\begin{equation}\label{eq:rho11.evol}
  \rho_{11}(t) =  e^{-\frac{\iu}{\hbar}H_{s}t-\frac{\gamma}{2}Vt}
  \rho_{11}(0)    e^{ \frac{\iu}{\hbar}H_{s}t-\frac{\gamma}{2}Vt} \, \text{.}
\end{equation}
If $\rho_{11}$ were in a pure state, say $\rho_{11}(t) = \ketbra{\psi(t)}$
(the ``undetected wavefunction''),
and recalling that $\rho_{11}(0) = \ketbra{\psi_0}$,
Eq.~\eqref{eq:rho11.evol} can be reformulated as
\begin{equation}
  \ket{\psi(t)} =
  e^{-\frac{\iu}{\hbar}H_{s}t-\frac{\gamma}{2}Vt} \ket{\psi_0} =
  e^{-\frac{\iu}{\hbar} \qty( H_{s} -\iu\frac{\hbar}{2}\gamma V ) t} \ket{\psi_0} \,\text{,}
\end{equation}
showing that the particle evolves as if an ``imaginary'' (anti-Hermitian) term
$-\iu\frac{\hbar}{2}\gamma V$ was added to its otherwise free Hamiltonian $H_s$.
This ``proves'' the validity of the complex potential model as an \emph{effective} theory,
and the consistency
with the unitary evolution of the ``bigger picture'' quantum system.

Another consequence of treating $\rho_{11}$ as a pure state is that the probability of no detection
\eqref{eq:p_nd} can be expressed as
\begin{equation}
  p_{nd}(\tau) = \int_{-\infty}^{\infty} \dd{x} \qty|\psi(x, \tau)|^2 \, \text{,}
\end{equation}
which is the (non conserved) norm $\norm{\psi}$ of this effective state vector.
Therefore, the probability that the detection \emph{has} happened
up to time $\tau$, which is $1 - p_{nd}(\tau)$, is equal to the
\emph{loss of normalization} $1 - \norm{\psi}$ at that time.
This is the total probability that the detection has happened from the initial time
($t=0$ in our chosen settings)
to $t = \tau$. In terms of detection time probability density
$\mathbb{P}(t)$, it is therefore
\begin{equation}
  \mathbb{P}(t) = - \dv{\norm{\psi}}{t} \,\text{.}
\end{equation}
