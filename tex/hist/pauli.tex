\section{Pauli's objection}\label{proof}

The existence of a time operator with the properties required in Section \ref{sec:T--H}
is hindered by a fundamental objection
which was raised by W.~Pauli for the first time
in his \term{General Principles of Quantum Mechanics}, dated 1933,
where he argued that the existence of a self-adjoint operator,
conjugate to the Hamiltonian,
would lead to absurd consequences on the energy spectrum.
He did not provide a formal proof of his argument,
and only relegated his observation into a footnote
of his work
\parencite{PauliFootnote}; but that was sufficient to
lead most physicists to insist that ``time is just a parameter''
in quantum mechanics
since then.

Here we aim at providing
a more technical proof of Pauli's ``theorem'',
based on some relatively recent work by
E. Galapon \parencite{Galapon2002},
while expanding some details further.
A preliminary Lemma on commutator algebra follows.

\begin{lemma}\label{CommProp}
  If the commutator $[T, H]$ commutes with $T$ i.e.
  $$[T, H]T~=~T[T, H]\,,$$ then the following holds:
  \begin{equation}\label{eq:tkh}
  [T^k, H] = kT^{k-1}[T, H]\,.
  \end{equation}
  \end{lemma}
  This is particularly true when $[T, H]$ is a \emph{number} as in \eqref{THcommutator} where
  $T$ and $H$ are the time and energy operator respectively.
  \begin{proof}
  First of all, the \eqref{eq:tkh} is trivially valid for $k = 1$.

  For an arbitrary positive integer $k > 1$ there has:
  \begin{dmath}\label{tkhrecur}
  [T^k, H] = T^{k-1}TH - HT^{k-1}T = T^{k-1}TH - T^{k-1}HT + T^{k-1}HT - HT^{k-1}T \\
      = T^{k-1}[T, H] + [T^{k-1}, H]T
  \end{dmath}
  Now, iterating the result in \eqref{tkhrecur},
  \begin{dmath}\label{tkhrecurplus}
  [T^k, H] = T^{k-1}[T, H] + [T^{k-1}, H]T
  = T^{k-1}[T, H] + (T^{k-2}[T, H] + [T^{k-2}, H]T)T
  = T^{k-1}[T, H] +  T^{k-1}[T, H] + [T^{k-2}, H]T^2
  = 2T^{k-1}[T, H] + [T^{k-2}, H]T^2
  = \hdots
  = nT^{k-1}[T, H] + [T^{k-n}, H]T^n = \hdots
  \end{dmath}
  where the commutativity hypothesis $[T, H]T = T[T, H]$ has been used to obtain
  $$T^{k-2}[T, H]T = T^{k-1}[T, H]\text{.}$$

  Now, \eqref{tkhrecurplus} can be continued until it reaches $n=k$ when the term
  $[T^{k-n}, H]T^n$ vanishes and a result of $kT^{k-1}[T, H]$ follows.
  \end{proof}
\normalsize

Let us now come to the Pauli argument; and
assume that there exists a self-adjoint time operator, $T$, canonically conjugate
to the Hamiltonian $H$, i.e.

\begin{equation}
\label{THcommutator}
[T, H] = i\hbar
\end{equation}
Since T is self-adjoint, then for all
$\beta\in\mathbb{R}$, $U_{\beta} = \exp(- i \beta T / \hbar)$
is unitary. A formal
expansion of the exponential yields the commutator

\begin{equation}\label{eq:pauli_sum}
[U_{\beta}, H]  =
\left[
  \sum_{k=0}^{\infty} \frac{1}{k!} \left(- \frac{i\beta T}{\hbar} \right)^k, H
\right]         =
\sum_{k=0}^{\infty} \frac{1}{k!} \left(- \frac{i\beta}{\hbar} \right)^k [T^k, H] \,\text{.}
\end{equation}

As the commutator $[T, H]$ itself commutes with its operator $T$,
the following identity holds (see Lemma \ref{CommProp}):

$$
[T^k, H] = kT^{k-1}[T, H]
$$
hence:

\begin{multline}\label{eq:pauli_sum_w_lemma}
[U_{\beta}, H]  =
\sum_{k=0}^{\infty} \frac{1}{k!} \left(- \frac{i\beta}{\hbar} \right)^k kT^{k-1}[T, H] \\ =
\beta\sum_{k=1}^{\infty} \frac{1}{(k-1)!} \left(- \frac{i\beta}{\hbar} \right)^{k-1} T^{k-1} =
\beta\sum_{\kappa=0}^{\infty} \frac{1}{\kappa!} \left(- \frac{i\beta T}{\hbar} \right)^{\kappa}  =
\beta U_{\beta}
\end{multline}
where the term for $k=0$ in the first sum clearly vanishes, hence we can start the sum from
$k=1$ then set $\kappa=k-1$ in the last step.

Now, given an eigenvector $\varphi_{E}$ of $H$ so that $H\varphi_{E}=E\varphi_{E}$,
using Eq.~\eqref{eq:pauli_sum_w_lemma}, we get

$$
HU_{\beta}\varphi_{E} = (U_{\beta}H - [U_{\beta}, H])\varphi_{E} =
EU_{\beta}\varphi_{E} - \beta U_{\beta}\varphi_{E} = (E-\beta)U_{\beta}\varphi_{E} \, \text{,}
$$
showing that $U_{\beta}\varphi_{E}$ is another eigenvector of $H$ with eigenvalue
$E-\beta$. But $\beta$ is an arbitrary real number and $H$ a \emph{generic} Hamiltonian,
hence the spectrum of a generic Hamiltonian $H$ should
be the whole real line, which contradicts the discrete and semi-bounded energy spectrum
in fact found in most physical systems.

%%

For several decades, Pauli's argument had prevented most theoretical attempts at
defining a self-adjoint
time operator with the required commutation and uncertainty properties.
However, some research effort has been invested
into weakening some of those requirements, 
thus rendering the Pauli objection no longer applicable.

Notable examples of possible approaches involve:
renouncing the self-adjointness of the operator~$\hat{T}$ (replacing it with a more general symmetric operator);
allowing for the corresponding spectral measurement not to be projective (replacing it with a POVM);
or including the case of an \emph{imaginary} potential in the Hamiltonian,
to model absorption by a detector via loss of normalization (non-unitary evolution).
