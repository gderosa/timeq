\section{Pauli's objection}\label{proof}

In a footnote in~\cite{PauliFootnote}, W. Pauli excluded the possibility
of a self-adjoint operator representing time as a quantum observable.
However, he did not provide an explicit proof.
Here a proof is given, based on~\cite{Galapon2002}, but expanded in more detail.

\subsection*{Proof}

Let's assume that there exists a self-adjoint time operator, $T$, canonically conjugate
to the Hamiltonian $H$, i.e.

\begin{equation}
\label{THcommutator}
[T, H] = i\hbar
\end{equation}
Since T is self-adjoint, then for all
$\beta\in\mathbb{R}$, $U_{\beta} = \exp(- i \beta T / \hbar)$
is unitary. A formal
expansion of the exponential yields the commutator

\begin{equation}
[U_{\beta}, H]  =
\left[
    \sum_{k=0}^{\infty} \frac{1}{k!} \left(- \frac{i\beta T}{\hbar} \right)^k, H
\right]         =
\sum_{k=0}^{\infty} \frac{1}{k!} \left(- \frac{i\beta}{\hbar} \right)^k [T^k, H] \,\text{.}
\end{equation}

As the commutator $[T, H]$ itself commutes with its operator $T$,
the following identity holds (See Lemma \ref{CommProp}):

$$
[T^k, H] = kT^{k-1}[T, H]
$$
hence:

\begin{multline}
[U_{\beta}, H]  =
\sum_{k=0}^{\infty} \frac{1}{k!} \left(- \frac{i\beta}{\hbar} \right)^k kT^{k-1}[T, H] \\ =
\beta\sum_{k=1}^{\infty} \frac{1}{(k-1)!} \left(- \frac{i\beta}{\hbar} \right)^{k-1} T^{k-1} =
\beta\sum_{\kappa=0}^{\infty} \frac{1}{\kappa!} \left(- \frac{i\beta T}{\hbar} \right)^{\kappa}  =
\beta U_{\beta}
\end{multline}
where the term for $k=0$ in the first sum clearly vanishes, hence we can start the sum from
$k=1$ then set $\kappa=k-1$.

Now, given an eigenvector $\varphi_{E}$ so that $H\varphi_{E}=E\varphi_{E}$, there has:

$$
HU_{\beta}\varphi_{E} = (U_{\beta}H - [U_{\beta}, H])\varphi_{E} =
EU_{\beta}\varphi_{E} - \beta U_{\beta}\varphi_{E} = (E-\beta)U_{\beta}\varphi_{E}
$$
showing that $U_{\beta}\varphi_{E}$ is another eigenvector of $H$ with eigenvalue
$E-\beta$. But $\beta$ is an arbitrary real number and $H$ a \emph{generic} Hamiltonian,
hence the spectrum of a generic Hamiltonian $H$ should
be the whole real line, which contradicts the discrete and semi-bounded energy spectrum
in fact found in most physical systems.

\section{Historical overview: Approaches}

\begin{itemize}
\item early attempts reviewed in \cite{TQM1, TQM2}, Aharonov-Bohm, Kijowski etc.
\item detector model \cite{TQM1, TQM2}
\item
    time $\otimes$ position Hilbert space or ``second'' Schr\"odinger equation (Prvanovic)
\item time and entanglement (Page and Wootters model, Leggett-Garg inequality as \emph{time} version of Bell inequalities, experiments by Moreva, Genovese et al.)
\item approachs where not only spacetime but causality itself is not fundamental (indefinite causal order: Oreshkov, Brukner et al)
\item event-based approaches: 
    \begin{itemize}
        \item ``event'' wavefunction square integrable in 4D (how does it relate rigourously to detector model?)
        \item event-enhanced quantum theory (EEQT, Ruschhaupt et al.)
        \item 
    \end{itemize}
\end{itemize}
