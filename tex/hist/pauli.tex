\section{Pauli's objection}\label{proof}

In the quest for a quantum time observable,
even in non-relativistic quantum mechanics,
it is tempting to leverage
the dualism between position and momentum,
evoked in Eq.~\eqref{eq:qprepr}.
It is well known that momentum is the infinitesimal generator of spatial translations:
\begin{equation}
  \psi(x-\Delta x; \, t) = \E^{-\iu \Delta x \hat{p} / \hbar} \psi(x; t) \,\text{,}
\end{equation}
where the momentum operator is $\hat{p} \repr -\iu\hbar \pdv{x}$ in position representation.

It is also well known that the Schr\"odinger equation is equivalent to
stating that the Hamiltonian is the infinitesimal generator of
\emph{temporal} translation:
\begin{equation}
  \psi(x; t+\Delta t) = \E^{-\iu \Delta t \hat{H} / \hbar} \psi(x; t) \,\text{.}
\end{equation}

The analogy between the two equations above,
among other considerations,
motivates the requirement that
the time (self-adjoint) operator is the canonically conjugate of the Hamiltonian,
just like linear momentum is canonically conjugate to the position operator.

However, the existence of such operator is hindered by a fundamental objection
which was raised by W.~Pauli for the first time
in his General Principles of Quantum Mechanics, dated 1933,
where he argued that the existence of a self-adjoint operator,
conjugate to the Hamilonian,
would lead to absurd consequences on the energy spectrum.
He did not provide a formal proof of his argument,
and only relegated his observation into a footnote
\parencite{PauliFootnote}; but that was sufficient to
lead most physicists to insist that ``time is just a parameter''
in quantum mechanics
since then.

Section~\ref{sec:pauliproof} aims at providing
a more technical proof of Pauli's ``theorem'',
it is based on some relatively recent work by
E. Galapon \parencite{Galapon2002},
while expanding some details further.
A preliminary Lemma on commutator algebra is also illustrated
towards such proof.


\subsection{Proof of the Pauli argument}\label{sec:pauliproof}

\small
\subsubsection{Preliminaries: commutator properties}
\begin{lemma}\label{CommProp}
  If the commutator $[T, H]$ commutes with $T$ i.e.
  $$[T, H]T~=~T[T, H]\,,$$ then the following holds:
  \begin{equation}\label{eq:tkh}
  [T^k, H] = kT^{k-1}[T, H]\,.
  \end{equation}
  \end{lemma}
  This is particularly true when $[T, H]$ is a \emph{number} as in \eqref{THcommutator} where
  $T$ and $H$ are the time and energy operator respectively.
  \begin{proof}
  First of all, the \eqref{eq:tkh} is trivially valid for $k = 1$.

  For an arbitrary positive integer $k$ there has:
  \begin{dmath}\label{tkhrecur}
  [T^k, H] = T^{k-1}TH - HT^{k-1}T = T^{k-1}TH - T^{k-1}HT + T^{k-1}HT - HT^{k-1}T \\
      = T^{k-1}[T, H] + [T^{k-1}, H]T
  \end{dmath}
  Now, iterating the result in \eqref{tkhrecur},
  \begin{dmath}\label{tkhrecurplus}
  [T^k, H] = T^{k-1}[T, H] + [T^{k-1}, H]T
  = T^{k-1}[T, H] + (T^{k-2}[T, H] + [T^{k-2}, H]T)T
  = T^{k-1}[T, H] +  T^{k-1}[T, H] + [T^{k-2}, H]T^2
  = 2T^{k-1}[T, H] + [T^{k-2}, H]T^2
  = \hdots
  = nT^{k-1}[T, H] + [T^{k-n}, H]T^n = \hdots
  \end{dmath}
  where the commutativity hypothesis $[T, H]T = T[T, H]$ has been used to obtain $T^{k-2}[T, H]T = T^{k-1}[T, H]$.

  Now, \eqref{tkhrecurplus} can be continued until it reaches $n=k$ when the term
  $[T^{k-n}, H]T^n$ vanishes and a result of $kT^{k-1}[T, H]$ follows.
  \end{proof}
\normalsize

\subsection*{The Pauli argument}

Let's assume that there exists a self-adjoint time operator, $T$, canonically conjugate
to the Hamiltonian $H$, i.e.

\begin{equation}
\label{THcommutator}
[T, H] = i\hbar
\end{equation}
Since T is self-adjoint, then for all
$\beta\in\mathbb{R}$, $U_{\beta} = \exp(- i \beta T / \hbar)$
is unitary. A formal
expansion of the exponential yields the commutator

\begin{equation}
[U_{\beta}, H]  =
\left[
  \sum_{k=0}^{\infty} \frac{1}{k!} \left(- \frac{i\beta T}{\hbar} \right)^k, H
\right]         =
\sum_{k=0}^{\infty} \frac{1}{k!} \left(- \frac{i\beta}{\hbar} \right)^k [T^k, H] \,\text{.}
\end{equation}

As the commutator $[T, H]$ itself commutes with its operator $T$,
the following identity holds (See Lemma \ref{CommProp}):

$$
[T^k, H] = kT^{k-1}[T, H]
$$
hence:

\begin{multline}
[U_{\beta}, H]  =
\sum_{k=0}^{\infty} \frac{1}{k!} \left(- \frac{i\beta}{\hbar} \right)^k kT^{k-1}[T, H] \\ =
\beta\sum_{k=1}^{\infty} \frac{1}{(k-1)!} \left(- \frac{i\beta}{\hbar} \right)^{k-1} T^{k-1} =
\beta\sum_{\kappa=0}^{\infty} \frac{1}{\kappa!} \left(- \frac{i\beta T}{\hbar} \right)^{\kappa}  =
\beta U_{\beta}
\end{multline}
where the term for $k=0$ in the first sum clearly vanishes, hence we can start the sum from
$k=1$ then set $\kappa=k-1$.

Now, given an eigenvector $\varphi_{E}$ so that $H\varphi_{E}=E\varphi_{E}$, there has:

$$
HU_{\beta}\varphi_{E} = (U_{\beta}H - [U_{\beta}, H])\varphi_{E} =
EU_{\beta}\varphi_{E} - \beta U_{\beta}\varphi_{E} = (E-\beta)U_{\beta}\varphi_{E}
$$
showing that $U_{\beta}\varphi_{E}$ is another eigenvector of $H$ with eigenvalue
$E-\beta$. But $\beta$ is an arbitrary real number and $H$ a \emph{generic} Hamiltonian,
hence the spectrum of a generic Hamiltonian $H$ should
be the whole real line, which contradicts the discrete and semi-bounded energy spectrum
in fact found in most physical systems.
