\section{Photon position}

The fact that QFT ``externalizes'' time and position to the rank of classical parameters
is particularly true in quantum electrodynamics, also within the
phenomenology of quantum optics.
Thus it is impossible to define a \emph{photon} position
within standard quantum optics (see, for example, \cite{ScullyZubairy}, sec. 1.5.4 `Wave function for photons'),
and the problem of defining a quantum position observable for a photon
shows an interesting analogy with time for a quantum massive particle.
It is, in fact, a current topic of active research \parencite{HawtonPhotonPosition, Hawton2019}.

While there isn't a wave function for photons, the wave function for massive particles,
as seen in the Shr\"{o}dinger equation, only exists as a function of position (or momentum, etc.)
but not as a function of time,
i.e. the variable $t$ in Eq.~\eqref{eq:diracdeltaxt} cannot be regarded as taking values among the eigenvalues of a time operator.

[TODO: ADD SOME ``GLUE'' HERE]

Nevertheless, we have provided an explanation as to why (even without considering the Pauli objection)
relativistic field theories, as currently accepted, cannot provide a solution
to the problem of quantum time, in spite of treating time and space on equal footing,
and in spite of the fact that position in quantum mechanics is an observable
with its associated self-adjoint operator.
