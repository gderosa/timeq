\section{Detector models}\label{sec:hist:detect}

TODO: merge with Allcock Section.

TODO: Detector model, Ch. 4 (not all of it!) Vol. 2 book? \parencite[Ch. 4]{TQM2}.

Levels.

\begin{enumerate}
  \item No explicit detector, POVM, Kijowski, consistency arguments e.g. a reasonable classical limit.
  \item Early attempts at incorporating the apparatus. Allcock, Imaginary potential: incident channel, irreversible process, final channel e.g. photon emission following excitation by laser
  \item Detector model; Allcock complex potential revisited (explained?); overlap with Hallywell?
\end{enumerate}

Notes:
\begin{itemize}
  \item time-of-arrival is stocastic just as other quantities, which are quantum
  \item preparation encode in the wavefunction
  \item avg or statistical moments | well-known prescription based on self-adjoint operators
\end{itemize}

\subsection{Basics of Atom-Laser Model}

|

* Halliwell is deemed as ``toy model''.

* b

\subsection{Ruschhaupt, Kiukas et al}\label{sec:hist:detect:kiukas}

Studies
about detector models which investigate
time-of-arrival as a quantum observable.
In particular we consider \cite{RuschhauptAbsorption},
which in turn traces back its basic ideas from the seminal work by Allcock
in the 1960s \parencite{Allcock-1, Allcock-2, Allcock-3},
while a detailed contemporary formulation can be found in
\cite[Ch. 4]{TQM2}.

A non hermitian Hamiltonian is justified as a computation method
to simplify the study of some open systems: the evolution of mixed
states is derived without explicit reference to density operators
or master equations, but resolving equations that are formally
identical to those of pure states,
i.e. in terms of
Schr{\"o}dinger equations and wave functions,
with the non-hermitian term in the Hamiltonian
to account for the non-unitarity of the evolution.

The detection-by-absoption model in \cite{RuschhauptAbsorption}
is based on a complex potential that, plugged into the Schr\"odinger equation,
leads to a non-unitary evolution of the state vector
(with loss of normalization).

Specifically, the Hamiltonian $\hat{H}$ is replaced by a $\hat{H} - i\hat{D}$
(with $\hat{D}$ self-adjoint, bounded, positive)
and, consequently:
\begin{equation}\label{eq:schrod_complex_pot}
  \hat{H} \ket{\psi(t)} = i\hbar\dv{t}\ket{\psi(t)} +i\hat{D}\ket{\psi(t)} \text{.}
\end{equation}

\citereset\subsection{Application: two-level system}

In \cite{RuschhauptAbsorption}, an example application of the detector model
is provided for a two-level system.
In Page and Wootters terms,
this would corrspond to a bi-dimensional $\hilb{H}_S$, but a continuous
spectrum of $\hat{T}$ in $\hilb{H}_T$. The paper is \emph{not} based on
the Page--Wootters model, indeed the purpose of this section is a comparison
with such model, using the results of Section \ref{sec:absorption+pw}.

By setting, out of convenience, $\hbar = \omega = 1$
(with $\omega$ the characteristic frequency of the system),
and directly considering the parameters
that minimize the time--energy uncertainty product \parencite{RuschhauptAbsorption},
we have a non-Hermitian ``Hamiltonian''
$\mathit{K} = \hat{H} - i\hat{D}$ with
\begin{equation}\label{eq:complexpot}
  \mathit{K} = \hat{H} - i\hat{D} \repr
    \hbar\omega\left\{
      \left[\begin{matrix}0 & 1\\1 & 0\end{matrix}\right] -
      i \left[\begin{matrix}0 & 0\\0 & \gamma \end{matrix}\right]
    \right\}
\end{equation}
and $\gamma = 2\sqrt{2}$.

We take an initial state of $\ket{0}$
(or $\mqty[1\\0]$ in matrix form).

We then compute, symbolically, the non-unitary evolution
$\ket{\psi(t)} = e^{-i\mathit{K}t}\ket{0}$
with the aid of \term{SymPy} \parencite{comp:sympy} within a \term{Jupyter} \parencite{comp:jupyter} notebook
(see Appendix \ref{detector-model-kiukas-ruschhaupt-schmidt-werner} for all the details of the calculation).

Simplifying the result in eq. \eqref{eq:sympy:non-unitary-evol}, we have:
\begin{equation}
  \ket{\psi(t)} \repr e^{-\frac{\sqrt{2}}{t}} \mqty[
    \cos(\frac{\sqrt{2}}{2}t) + \sin(\frac{\sqrt{2}}{2}t)& \\
                     -i\sqrt{2} \sin(\frac{\sqrt{2}}{2}t)&
  ] \,\text{.}
\end{equation}

\begin{figure} %% https://tex.stackexchange.com/a/165730
  \centering
  \begin{subfigure}{0.49\textwidth}
    \includegraphics[width=\linewidth]{img/2ldetect/re_psi0_t.png}
    \subcaption{}\label{fig:absorbed-qubit-components:re0}
  \end{subfigure}
  % \hspace*{\fill} % separation between the subfigures
  \begin{subfigure}{0.49\textwidth}
    \includegraphics[width=\linewidth]{img/2ldetect/im_psi1_t.png}
    \subcaption{}\label{fig:absorbed-qubit-components:im1}
  \end{subfigure}
  % \hspace*{\fill} % separation between the subfigures
  \caption{
    Non-unitary evolution of the absorbed qubit
    according to the model in
    ref. \cite[sec.``Emission from a two-level system'']{RuschhauptAbsorption}.
    The component along $\ket{0}$ is purely real,
    and the one along $\ket{1}$ is purely imaginary,
    therefore only their their respective parts are plotted.
  }\label{fig:absorbed-qubit-components}
\end{figure}

\begin{figure}
  \centering
  \begin{subfigure}[b]{0.49\textwidth}
    \includegraphics[width=\linewidth]{img/2ldetect/qubit_normalization_loss.png}
    \subcaption{}\label{fig:absorbed-qubit-normalization-loss:t}
  \end{subfigure}
  \begin{subfigure}[b]{0.49\textwidth}
    \includegraphics[width=\linewidth]{img/2ldetect/P_omega.png}
    \subcaption{}\label{fig:absorbed-qubit-normalization-loss:omega}
  \end{subfigure}
  \caption{
    Non-unitary evolution of absorbed qubit.
    \subref{fig:absorbed-qubit-normalization-loss:t}
      Detection probability in time. It's equal to the
      loss of normalization $-\dv{\norm{\psi}^2}{t}$
      (but also to the squared norm of $\hat{\psi}(t)$ as of eq. \eqref{eq:analytic:hatpsi}.
    \subref{fig:absorbed-qubit-normalization-loss:omega}
      Detection probability in the frequency domain.
  }
  \label{fig:absorbed-qubit-normalization-loss}
\end{figure}

The ``lossy'' evolution, with the two components of the qubit, is shown in Fig.~\ref{fig:absorbed-qubit-components}.
The loss of normalization $-\dv{\norm{\psi}^2}{t}$, indicating the probability of detection by absorption,
is then derived directly and shown in Fig.~\ref{fig:absorbed-qubit-normalization-loss}.

This yields the \emph{probability} of detection.
One may wonder whether it is possible to derive a corresponding \emph{probability amplitude} vector,
whose squared norm across time is equal to the said probability distribution.\footnote{
  The solution is of course not unique, but one may ask whether such functions would lead
  to quantum interference patterns and other phenomenology which may be subject of further study.
}
Within the framework of \cite{RuschhauptAbsorption}, a ``wavefunction in time'' in such sense
is the $\hat{\psi}$, as in \eqref{eq:phi_psi_kiukas}.
It is computed in detail within the
notebook in Appendix \ref{detector-model-kiukas-ruschhaupt-schmidt-werner}, eq. \eqref{eq:sympy:hatpsi},
simplifying which we obtain:
\begin{equation}\label{eq:analytic:hatpsi}
  \hat{\psi}(t) =
    i 2^{\frac{5}{4}} e^{-\frac{\sqrt{2}}{2}t}\sin(\frac{\sqrt{2}}{2}t) \theta(t)
    \ket{1}
    \text{,}
\end{equation}
with $\theta(t)$ the Heaviside step function.

\begin{remark}\label{remark:detection_area}
In general, the operator $\hat{D}$ as in \eqref{eq:schrod_complex_pot}
is such that the eigenspace corresponding to its zero eigenvalue
is the ``area'' where the detector is not sensitive. Or, in other words,
the linear span of states with zero probability of triggering the detector.
Therefore, when $\hat{D}$ (or its square root) is applied to a state vector,
for example in \eqref{eq:phi_psi_kiukas},
the components in such eigenspace are cut off and the resulting
$\hat{\psi}$, eq. \eqref{eq:analytic:hatpsi} in the example, lies at all times in the ``area of detection''
i.e. it's a multiple of $\ket{1}$ in this case.
\end{remark}

The corresponding Page--Wootters (proper) vector of $\pwspace$ is
\begin{equation}\label{eq:hatpsi:pw}
  \dket{\Phi} = \int \dd{t} \ket{t} \ox \hat{\psi}(t) \,\text{,}
\end{equation}
to which the considerations of Section \ref{sec:for-normalized-elements}
and Section \ref{sec:pure-state-approach} in terms of time--frequency
(or time--energy) uncertainty relation apply, with some analogy
to what \cite{RuschhauptAbsorption} does within its own framework
in relation to $\hat{\psi}$ and its Fourier transform.

In that regard, the \eqref{eq:hatpsi:pw} can be reformulated
\begin{equation}
  \dket{\Phi} = \int \dd{\omega} \ket{\omega} \ox \mathcal{F} \hat{\psi} (\omega) \,\text{,}
\end{equation}
where it's
\begin{equation}
  \mathcal{F} \hat{\psi} (\omega) = - \frac{\sqrt[4]{2} i}{\sqrt{\pi} \left(- \omega^{2} + \sqrt{2} i \omega + 1\right)} \ket{1}
\end{equation}
---see notebook up to eq. \eqref{eq:fhatpsi1_omega} for details.

Taking the squared modulus, a probability distribution over angular frequency
(or, equivalently, energy) is obtained:
\[
  P(\omega) = \frac{\sqrt{2}}{\pi \left(\omega^{4} + 1\right)}
  \,\text{.}
\]
See Fig. \ref{fig:absorbed-qubit-normalization-loss:omega}.