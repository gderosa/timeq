\section{Aharonov--Bohm}\label{sec:AharonovBohm}

In their 1961 paper, Aharonov and Bohm \parencite{AharonovBohm}
showed that ``energy can be measured
  reproducibly in an arbitrarily short time'',
thus apparently contradicting the time--energy indeterminacy theorized
by
Mandelstam and Tamm (see also Sec. \ref{sec:T--H}) and by other authors.
As observed by Aharonov and Bohm, in the Mandelstam--Tamm derivation, \emph{time} has a different meaning:
it is essentially the lifetime of a system in a particular state
and not the duration of the energy measurement process.
They also critically reviewed previous works by Landau and Peierls \parencite{LandauPeierls}
and by Fock and Krylov \parencite{FockKrylov}, discussed their level of generality and provided
counterexamples to their formulation of the time--energy uncertainty relation.

In their model, they considered a free particle as a ``clock''
and quantize (by symmetrizing the corresponding operators)
the classical relation
$t = y / v = m y / p$
for a particle that is at position $y = 0$ at time $t = 0$ and travels at velocity $v$ along the $y$ axis.
The corresponding time operator has then expression:
\begin{equation}
  \hat{T}_{AB} = \frac{m}{2} \qty( \hat{Y} \frac{1}{\hat{P}_y} + \frac{1}{\hat{P}_y} \hat{Y} ) \,\text{.}
\end{equation}
In the paper, this operator is claimed to be Hermitian, although with a singularity for $p_{y} = 0$.\footnote{
  This is stressed in a footnote of Aharonov--Bohm's paper.
  The reader might have noticed the irony of fundamental historical papers
  relegating important information at the level of footnotes
  ---and the present work, in many parts, not even trying to ``improve'' such a traditional trend.
}
With some simple algebra, a commutation relation $[\hat{H}_{c}, \hat{T}_{AB}] = \iu\hbar$
can be proved, where $\hat{H}_c = \hat{P}_y^2/2m$ is the Hamiltonian of the free particle that is used as a ``clock''.

This seems to contradict Pauli's argument: however, more recent studies
\parencite{MugaAB98, MugaAB99, MugaAB99Err}
have shown that
$\hat{T}_{AB}$ is not, strictly speaking, self-adjoint but ``only'' maximally symmetric
(a weaker and more general condition).
An analogy has been made therein with the momentum on the half-line,
a restriction of the well known momentum operator to a subdomain
that is defined as
$\qty{\psi \in \mathscr{L}^2(\mathbb{R}): \psi(x) = 0 \; \forall x < 0}$ in position representation.
% Removal suggested due to mathematical intricacies...
% As opposed to the momentum in the half-line, $\hat{T}_{AB}$ does not have a self-adjoint extension though.
It has also been shown that the associated spectral measure is not projector valued, but
positive-operator valued (POVM). Projectors are a particular subclass of \term{positive operators}.
Among other applications, suitable positive operators can be used to extend von Neumann decomposition
in order to describe
open quantum systems and ``unsharp'' measurements.

Another apparent contradiction is the commutator $[\hat{H}_{c}, \hat{T}_{AB}] = \iu\hbar$
(and the consequent uncertainty relation $\sigma_{H_{c}} \sigma_{T_{AB}} \geq \frac{\hbar}{2}$)
with Aharonov--Bohm's conclusion that
energy can be measured, in principle, with arbitrary precision in an arbitrarily short time;
but it should be noted that $\hat{H}_{c}$ is the energy of the clock, not of the system under energy-measurement,
thus the contradiction does not hold.

The Aharonov--Bohm model explicitly includes a clock in the description, i.e.,
a different system to the one under measurement;
and this separate system has one of its observables
on a well known dependency upon time, so the corresponding eigenvalues can be seen
as possible positions of the ``hand'' of the clock.
Aharonov and Bohm infer that $\hat{T}_{AB}$ must commute with all the observables
of the main system (and in particular the Hamiltonian, let's call it $\hat{H}_S$),
given that they are defined in different Hilbert spaces:
therefore, there is no
fundamental reason why the system energy represented by $\hat{H}_S$, and the clock time $\hat{T}_{AB}$,
could not have, in the same measurement,
arbitrarily
narrow statistical distributions around some of their eigenvalues.

This idea of a quantum description of time through modeling a ``clock''
(in other words, the notion of time as ``what is shown on a clock'' with some suitable properties)
has a certain historical relevance because later models, independently developed,
have been based on that idea too.
In particular, we can mention
the Page--Wootters model \parencite{PageWootters}, where a peculiar relation is in place between the clock and the system,
of which we will not explore the details at this stage though,
as those will be discussed extensively (with numerical applications, experiments, and comparisons with other models)
in Chapter \ref{ch:pw}.
