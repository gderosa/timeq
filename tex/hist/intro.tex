\section{Time and the Quantum: An Invitation}\label{sec:intro}

Time is a fundamental concept in physics, together with space (or \term{position} within space),
matter (or \emph{mass}) and energy.

In quantum mechanics, at least in first quantization, defining the \emph{position} as
an observable is unproblematic and somewhat basic.
Indeed, representing the quantum state of a particle
in terms of coefficients with respect to eigenstates of position is extremely common.
Almost always when a ``wavefunction'' $\psi$
for a state $\ket{\psi}$
is written down, the \emph{position representation}
is assumed, unless otherwise (explicitly) noted:
\begin{equation}\label{eq:positionrepr}
  \ket{\psi} = \int \dd{x'} \psi(x') \ket{x'} \text{,}
\end{equation}
where\footnote{
  Here a one-dimensional system is considered, for simplicity:
  it can be easily verified that more dimensions,
  and degeneracy, may be taken into account without altering the core rationale of the
  following arguments in this Section.
}
$\psi(x') = \braket{x'}{\psi}$.

As per \emph{mass}, a quantum mechanism that explains how particle acquire mass was theorized by
Peter Higgs and others in 1964, and confirmed experimentally in 2012
\parencite{Higgs, EnglertBrout, Kibble+, HiggsATLAS, HiggsCMS}.

\emph{Energy} is associated with arguably the most important operator in quantum dynamics: the Hamiltonian.

\emph{Time} is missing from this picture. As of writing of the present work,
there is no general consensus in the physics community on
the definition of a quantum operator associated to a time observable,
or even only on a particular quantum mechanism that would allow time to ``emerge'',
for example as an effective quantity.

\subsection{Position observable}

Rigorously speaking, the position observable has no special role in quantum mechanics.
Eq.~\eqref{eq:positionrepr} can be further expanded to
\begin{equation}\label{eq:qprepr}
  \ket{\psi} = \int \dd{x'} \psi(x') \ket{x'} = \int \dd{p'} \tilde{\psi}(p') \ket{p'} \,\text{,}
\end{equation}
with $p$ and $\ket{p}$ being respectively eigenvalues and eigenvectors of the momentum operator.

By multiplying on the left by $\bra{x}$, Eq.~\eqref{eq:positionrepr} yields
\begin{equation}\label{eq:diracdeltax}
  \psi(x) = \int \dd{x'} \psi(x') \delta(x-x') \,\text{.}
\end{equation}
In general, for an observable with continuous spectrum,
in its own eigenbasis, eigenstates are represented as a Dirac deltas
(for discrete spectrums, the integral is replaced by a discrete sum,
and Dirac deltas by Kronecker deltas).

Applying the position operator $\hat{X}$ to the~\eqref{eq:positionrepr},
then taking the inner product on the left by $\bra{x}$,
there has
\begin{equation}
  \qty[\hat{X}\psi](x) \eqbydef \mel{x}{\hat{X}}{\psi} =
    \int \dd{x'} \psi(x') x' \mel{x}{\hat{X}}{x'} =
    \int \dd{x'} \psi(x') \delta(x-x') =
    x\psi(x)
  \,\text{.}
\end{equation}
The position
(and an observable in general) is represented, in its own eigenbasis,
as a simple multiplication of the original wavefunction by the respective variable.

\subsection{Photon position, and a final remark}

The fact that QFT ``externalizes'' time and position to the rank of classical parameters
is particularly true in quantum electrodynamics, also within the
phenomenology of quantum optics.
Thus it is impossible to define a \emph{photon} position
within standard quantum optics (see, for example, \cite{ScullyZubairy}, \S 1.5.4 `Wave function for photons'),
and the problem of defining a quantum position observable for a photon
shows an interesting analogy with time for a quantum massive particle.
It is, in fact, a current topic of active research \parencite{HawtonPhotonPosition, Hawton2019}.

While there isn't a wave function for photons, the wave function for massive particles,
as seen in the Shr\"{o}dinger equation, only exists as a function of position (or momentum, etc.)
but not as a function of time,
i.e. the variable $t$ in Eq.~\eqref{eq:diracdeltaxt} cannot be regarded as taking values among the eigenvalues of a time operator.
The main reason for this is the \term{Pauli's objection}, as illustrated in Section~\ref{proof},
which is also ---and particularly--- valid in nonrelativistic quantum mechanics. Nevertheless, this Section
has provided an explanation as to why (even without considering the Pauli objection)
relativistic field theories, as currently accepted, cannot provide a solution
to the problem of quantum time, in spite of treating time and space on equal footing,
and in spite of the fact that position in quantum mechanics is an observable
with its associated self-adjoint operator.
