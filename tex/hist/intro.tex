\section{Time and the Quantum: An invitation}

Time is a fundamental concept in physics, together with space (or \term{position} within space),
matter (or \emph{mass}) and energy.

In quantum mechanics, at least in first quantization, defining the \emph{position} as
an observable is unproblematic and somewhat basic.
Indeed, representing the quantum state of a particle
in terms of coefficients with respect to eigenstates of its position is extremely common.
Almost always when a ``wavefunction'' $\psi$
for a state $\ket{\psi}$
is written down, the \emph{position representation}
is assumed, unless otherwise (explicitly) noted:
\begin{equation}\label{eq:positionrepr}
  \ket{\psi} = \int \dd{x} \psi(x) \ket{x}
\end{equation}
(here a one-dimensional system is considered, for simplicity:
it can be easily verified that more dimensions,
and degeneracy, may be taken into account without altering the core rationale of the
following arguments).

As per \emph{mass}, a quantum mechanism that explains how particle acquire mass was theorized by
Peter Higgs and others in 1964, and confirmed experimentally in 2012.

\emph{Energy} is associated with arguably the most important operator in quantum dynamics: the Hamiltonian.

\emph{Time} is missing from this picture.

Rigorously, the position observable has no special role in quantum mechanics.
Eq. \eqref{eq:positionrepr} can be further expanded to
\begin{equation}\label{eq:qprepr}
  \ket{\psi} = \int \dd{x} \psi(x) \ket{x} = \int \dd{p} \tilde{\psi}(p) \ket{p} \,\text{,}
\end{equation}
with $p$ and $\ket{p}$ being eigenvalues and eigenvectors of linear \emph{momentum}.
The fact that the symbol $\ket{\psi}$ instead of $\ket{\tilde{\psi}}$
is used for the state vector is a pure matter of convention.

Eq. \eqref{eq:diracdeltax} is really a simple mathematical identity.
It's a defining property of the Dirac delta.
%Likewise, the \eqref{eq:diracdeltax} in discrete form is $a_k = \sum_j \delta_{jk} a_j$.
There is nothing special about the position observable, from this point of view.
In the position representation it acts on the wavefunction by simply
multiplying it by $x$,
but the same would happen for another operator with continuous, or discrete, spectrum
(of course, changing the variable, and understanding that it may have a different physical meaning,
but the formal properties would be the same).
Switching to momentum representation,
for example,
is customary in the resolution of many quantum mechanical problems;
and the momentum representation of the momentum operator,
essentially,
acts on a wavefunction $\tilde{\psi}(p)$ by associating
$p\tilde{\psi}(p)$ for each $p$ to it. 

With this in mind one may wonder why, in the identity
\begin{equation}\label{eq:diracdeltaxt}
  \psi(x; t) = \int \dd{x'}\dd{t'} \delta(x-x')\delta(t-t') \psi(x';\, t') \,\text{,}
\end{equation}
the term $\delta(t-t')$ cannot be interpreted as the eigenfunction of some time operator,
which in turn acts as a simple multiplication by $t$ on this
wavefunction $\psi(x; t)$ ---which would be, therefore, a wavefunction in ``time--position representation''.
In other words, why cannot the instant in \emph{time} and the position in \emph{space} be treated
\emph{on equal footing}?

The fact that $\delta(t-t')$ would be an \emph{improper} eigenfunction
is not the blocking issue in this case (it is not, for the position eigenfunction),
and the mathematical delicacies related to the continuous spectrum are
elegantly resolved by the spectral theory, mainly by Von Neumann
\parencite{VonNeumann}, which is an integral part of the standard, commonly accepted
formulation of quantum mechanics, almost since the early years of the theory.

The idea of treating space and time on equal footing might naturally lead the reader to
ask another question: whether a theory combining quantum mechanics with Einstein's
relativity is the suitable framework which would allow a rigorous, logically consistent
definition of a time observable.

None of the proposed models comprehending both quantum theory and \emph{General} relativity
have reached general consensus or the remote possibility of experimental verification to date.
