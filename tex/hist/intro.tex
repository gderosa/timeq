\section{Time and the Quantum: an invitation}

Time is a fundamental concept in physics, together with space (or \term{position} within space),
matter (or \emph{mass}) and energy.

In quantum mechanics, at least in first quantization, defining the \emph{position} of a particle as
an observable is unproblematic. More strongly, representing the quantum state of a particle
in terms of coefficients with respect to eigenstates of position is extremely common.
Almost always when a ``wavefunction'' is written down, the \emph{position representation}
is assumed, unless otherwise (explicitly) noted.

\begin{equation}
  \psi(x) = \int \dd{x'} \delta(x-x') \psi(x') \,\text{.}
\end{equation}

A quantum mechanism that explains how particle acquire \emph{mass} was theorized by
Peter Higgs and others in 1964, and confirmed experimentally in 2012.

\emph{Energy} is associated with arguably the most important operator in quantum dynamics: the Hamiltonian.

Time is missing from this picture.