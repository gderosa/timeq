\section{Time and the Quantum: An Invitation}\label{sec:intro}

Time is a fundamental concept in physics, together with space (or \term{position} within space),
matter (or \emph{mass}) and energy.

In quantum mechanics, at least in first quantization, defining the \emph{position} as
an observable is unproblematic and somewhat basic.
Indeed, representing the quantum state of a particle
in terms of coefficients with respect to eigenstates of position is extremely common.
Almost always when a ``wavefunction'' $\psi$
for a state $\ket{\psi}$
is written down, the \emph{position representation}
is assumed, unless otherwise (explicitly) noted:
\begin{equation}\label{eq:positionrepr}
  \ket{\psi} = \int \dd{x'} \psi(x') \ket{x'} \text{,}
\end{equation}
where\footnote{
  Here a one-dimensional system is considered, for simplicity:
  it can be easily verified that more dimensions,
  and degeneracy, may be taken into account without altering the core rationale of the
  following arguments in this Section.
}
$\psi(x') = \braket{x'}{\psi}$.

As per \emph{mass}, a quantum mechanism that explains how particle acquire mass was theorized by
Peter Higgs and others in 1964, and confirmed experimentally in 2012
\parencite{Higgs, EnglertBrout, Kibble+, HiggsATLAS, HiggsCMS}.

\emph{Energy} is associated with arguably the most important operator in quantum dynamics: the Hamiltonian.

\emph{Time} is missing from this picture. As of writing of the present work,
there is no general consensus in the physics community on
the definition of a quantum operator associated to a time observable,
or even only on a particular quantum mechanism that would allow time to ``emerge'',
for example as an effective quantity.

The relation between time and other concepts and quantities
(whether they are ``quantum observables'' in the formal sense or not)
is explored in the rest of this Section,
with reminders about some basic properties that will be useful in the rest of the chapter.

\subsection{Time and position}

Rigorously speaking, the position observable has no special role in quantum mechanics.
Eq.~\eqref{eq:positionrepr} can be further expanded to
\begin{equation}\label{eq:qprepr}
  \ket{\psi} = \int \dd{x'} \psi(x') \ket{x'} = \int \dd{p'} \tilde{\psi}(p') \ket{p'} \,\text{,}
\end{equation}
with $p$ and $\ket{p}$ being respectively eigenvalues and eigenvectors of the momentum operator.

By multiplying on the left by $\bra{x}$, Eq.~\eqref{eq:positionrepr} yields
\begin{equation}\label{eq:diracdeltax}
  \psi(x) = \int \dd{x'} \psi(x') \delta(x-x') \,\text{.}
\end{equation}
In general, for an observable with continuous spectrum,
in its own eigenbasis, eigenstates are represented as a Dirac deltas
(for discrete spectrums, the integral is replaced by a discrete sum,
and Dirac deltas by Kronecker deltas).

Applying the position operator $\hat{X}$ to the~\eqref{eq:positionrepr},
then taking the inner product on the left by $\bra{x}$,
there has
\begin{equation}
  \qty[\hat{X}\psi](x) \eqbydef \mel{x}{\hat{X}}{\psi} =
    \int \dd{x'} \psi(x') x' \mel{x}{\hat{X}}{x'} =
    \int \dd{x'} \psi(x') \delta(x-x') =
    x\psi(x)
  \,\text{.}
\end{equation}
The position
(and an observable in general) is represented, in its own eigenbasis,
as a simple multiplication of the original wavefunction by the respective variable.

\subsection{Time and evolution}

As a matter of experience if not anything else,
the state of a physical system, either classical or quantum,
with all observable quantities,
can be regarded as a function of the instant in \emph{time}
(let's call that independent variable $t$).
The same cannot be stated in general for any other pair of measurable quantities:
the $x$ coordinate is not a function of the $y$ coordinate
(or their respective statistical amplitudes, in the quantum realm),
and so on.
This status of time as the universally independent variable
has shaped classical mathematical physics: Galilean transformations
do not change time, which stays as an absolute quantity.
The Hamiltonian formalism itself is based on this assumption,
and the Hamiltonian formalism is at the foundation of quantum mechanics too,
which thus inherits ---one may argue--- the difficulties connected to
treating time as ``external'' with respect to the other quantities under study.
``[The] Hamiltonian method [\dots] marks out a particular time variable
as the canonical conjugate of the Hamiltonian function'' \parencite{DiracLagrangian}.

Emphasizing time as a parameter, the value of a wavefunction
(in position representation, for example)
can therefore be expressed as $\psi_{t}(x)$ at each position $x$
and at each time $t$.
Still, it can be regarded as a function of two variables
and one may wonder why (after an inessential change of notation),
in the identity
\begin{equation}\label{eq:diracdeltaxt}
  \psi(x; t) = \int \dd{x'}\dd{t'} \delta(x-x')\delta(t-t') \psi(x';\, t') \,\text{,}
\end{equation}
the term $\delta(t-t')$ cannot be interpreted as the eigenfunction of some time operator,
which in turn acts as a simple multiplication by $t$ on this
wavefunction $\psi(x; t)$ ---which would be, therefore, a wavefunction in ``time--position representation''.

The fact that $\delta(t-t')$ would be an \emph{improper} eigenfunction
is not the blocking issue in this case (it is not, for the position eigenfunction),
and the mathematical intricacies related to the continuous spectrum are
elegantly resolved by the spectral theory, mainly by Von Neumann
\parencite{VonNeumann}, which is an integral part of the standard, commonly accepted
formulation of quantum mechanics, almost since the early years of the theory.

The main obstacle is instead the \term{Pauli's objection},
which is only indirectly related to the special status of time
as independent variable of ``evolution'', and
directly related to the properties of the spectrum of the {Hamiltonian},
as illustrated in Section~\ref{proof}.

\subsection{Time and Energy}\label{sec:T--H}

In the quest for a quantum time observable,
it is tempting to leverage
the dualism between position and momentum,
evoked in Eq.~\eqref{eq:qprepr}.
It is well known that momentum is the infinitesimal generator of spatial translations:
\begin{equation}\label{eq:genspacetransl}
  \psi(x-\Delta x; \, t) = \E^{-\iu \Delta x \hat{P} / \hbar} \psi(x; t) \,\text{,}
\end{equation}
where the momentum operator is $\hat{p} \repr -\iu\hbar \pdv{x}$ in position representation.

It is also well known that the Schr\"odinger equation is equivalent to
stating that the Hamiltonian is the infinitesimal generator of
\emph{temporal} translation:
\begin{equation}\label{eq:gentimetransl}
  \psi(x; t+\Delta t) = \E^{-\iu \Delta t \hat{H} / \hbar} \psi(x; t) \,\text{.}
\end{equation}

The analogy between the two equations above,
among other considerations,
motivates the requirement that
the time (self-adjoint) operator is the canonically conjugate of the Hamiltonian,
just like the momentum operator is canonically conjugate to the position one.
This involves ---if such a time operator exists---
a commutation relation between time and energy that is analogous to
the one between position and momentum, $[\hat{X}, \hat{P}] = \iu \hbar$,
and a consequent
time--energy uncertainty relation.

Incidentally, it's important to stress that, while equations \eqref{eq:genspacetransl} and \eqref{eq:gentimetransl}
do offer some conceptual motivation towards defining a time observable that is conjugate to the Hamiltonian,
they are not sufficient to logically imply its existence.
Both $\Delta{x}$ and $\Delta{t}$ ---which quantifies the spatial and temporal translations at the core of this discussion---
are, once again, mere \emph{parameters} for the displacement and time evolution operators respectively:
\begin{align}
  \mel{x}{\mathcal{D}_{\Delta{x}}}{\psi(t)}        = \braket{x-\Delta{x}}{\psi(t)}\!\text{, }
    &\text{ with } \mathcal{D}_{\Delta{x}}    = \E^{-\iu \Delta x \hat{P} / \hbar} \\
  \mel{x}{\mathcal{U}_{\Delta{t}}}{\psi(t)}        = \braket{x}{\psi(t+\Delta{t})}\!\text{, }
    &\text{ with } \mathcal{U}_{\Delta{t}}    = \E^{-\iu \Delta t \hat{H} / \hbar}
  \, \text{.}
\end{align}
The analogy between the roles of \emph{parameters} $\Delta x$ and $\Delta t$,
with respect to \emph{operators} $\hat{P}$ and $\hat{H}$,
cannot be used, alone, to rigorously prove the same analogy between the position operator $\hat{X}$
and an hypothetical time operator $\hat{T}$, although it is somewhat suggestive.

Whether a quantum time observable can be in principle defined as a self-adjoint
operator, or alternative mathematical frameworks exist that are suitable,
possibly satisfying the above conjugation properties,
is indeed one of the fundamental questions within the entire topic
of the quantization of time.

\subsubsection{Uncertainty relations}

The Heisenberg uncertainty principle is at the core of quantum mechanics.
It was first introduced in 1927 \parencite{Heisenberg:Uncertainty},
in the form of a position--momentum relation: $p_{1}q_{1} \sim h$, in his notation.
Interestingly, in the same paper, Heisenberg also introduced
a \emph{time--energy} uncertainty relation: $E_{1}t_{1} \sim h$.

What did Heisenberg mean by $t_1$? Was that an early formalization of (the ``spread'' of)
a time observable (or ``matrix'', in the language of Heisenberg's first formulation of quantum mechanics)?
In fact, the same question can be raised for the position--momentum
relation too. The original formulation of the uncertainty principle
was not expressed in terms of standard deviations and mean values of Hermitian operators
as we know it today. Heisenberg approach was semi-empirical and,
while it turned unsatifactory from a formal perspective, in some aspects,
it had the merit of explicitly dealing with the physics of the disturbance
of a measurement device on the measurement itself. Therefore,
regardless of the mathematical maturity of its model,
the paper did provide convincing phenomenological arguments towards
the existence of un uncertainty principle for both the position--momentum
and the time--energy pairs.

(1.1.3 in the Book TQM1). time-frequency signals / telcos. Later we do it.

TODO. Mandelstam--Tamm and all that.

There's also an entire chapter in the book on TE uncertainty rel.