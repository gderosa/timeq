\section{Time and the Quantum: An invitation}

Time is a fundamental concept in physics, together with space (or \term{position} within space),
matter (or \emph{mass}) and energy.

In quantum mechanics, at least in first quantization, defining the \emph{position} as
an observable is unproblematic and somewhat basic.
Indeed, representing the quantum state of a particle
in terms of coefficients with respect to eigenstates of its position is extremely common.
Almost always when a ``wavefunction'' $\psi$
for a state $\ket{\psi}$
is written down, the \emph{position representation}
is assumed, unless otherwise (explicitly) noted:
\begin{equation}\label{eq:positionrepr}
  \ket{\psi} = \int \dd{x'} \psi(x') \ket{x'} \text{,}
\end{equation}
where\footnote{
  Here a one-dimensional system is considered, for simplicity:
  it can be easily verified that more dimensions,
  and degeneracy, may be taken into account without altering the core rationale of the
  following arguments in this Section.
}
$\psi(x') = \braket{x'}{\psi}$.

As per \emph{mass}, a quantum mechanism that explains how particle acquire mass was theorized by
Peter Higgs and others in 1964, and confirmed experimentally in 2012.

\emph{Energy} is associated with arguably the most important operator in quantum dynamics: the Hamiltonian.

\emph{Time} is missing from this picture.

Rigorously speaking, the position observable has no special role in quantum mechanics.
Eq.~\eqref{eq:positionrepr} can be further expanded to
\begin{equation}\label{eq:qprepr}
  \ket{\psi} = \int \dd{x'} \psi(x') \ket{x'} = \int \dd{p'} \tilde{\psi}(p') \ket{p'} \,\text{,}
\end{equation}
with $p$ and $\ket{p}$ being respectively eigenvalues and eigenvectors of linear \emph{momentum}.
The fact that the symbol $\ket{\psi}$ instead of $\ket{\tilde{\psi}}$
is used for the state vector is a pure matter of convention.

By multiplying on the left by $\bra{x}$, Eq.~\eqref{eq:positionrepr} yields
\begin{equation}\label{eq:diracdeltax}
  \psi(x) = \int \dd{x'} \psi(x') \delta(x-x') \,\text{,}
\end{equation}
showing that for an observable with continuous spectrum,
in its own eigenbasis, eigenstates are represented as a Dirac deltas
(for discrete spectrums, the integral is replaced by a discrete sum,
and Dirac deltas by Kronecker ones).

Applying the position operator $\hat{x}$ to the~\eqref{eq:positionrepr},
then taking the inner product on the left by $\bra{x}$,
there has
\begin{equation}
  \qty[\hat{x}\psi](x) \eqbydef \mel{x}{\hat{x}}{\psi} =
    \int \dd{x'} \psi(x') x' \mel{x}{\hat{x}}{x'} =
    \int \dd{x'} \psi(x') \delta(x-x') =
    x\psi(x)
  \,\text{,}
\end{equation}
which
shows that the position
(and an observable in general) is represented, in its own eigenbasis,
as a simple multiplication of the original wavefunction by the respective variable.

\subsection{Time evolution and relativity}

Eq.~\eqref{eq:diracdeltax} is a mathematical identity
of general validity,
regardless of any particular physical meaning,
and can be easily generalized to functions
of more variables.

On a different note, as a matter of experience if not anything else,
the state of a physical system, either classical or quantum,
and all observable quantities
can be regarded as functions of the instant in \emph{time}
(let's call the related variable $t$).
The same cannot be generally stated for any other pair of measurable quantities:
the $x$ coordinate is not a function of the $y$ coordinate
(or their respective statistical amplitudes in the quantum realm),
and so on.

The value of a wavefunction (in position representation, for example)
can therefore be expressed as $\psi(x; t)$ at each position $x$
and at each time $t$.

With this in mind one may wonder why, in the identity
\begin{equation}\label{eq:diracdeltaxt}
  \psi(x; t) = \int \dd{x'}\dd{t'} \delta(x-x')\delta(t-t') \psi(x';\, t') \,\text{,}
\end{equation}
the term $\delta(t-t')$ cannot be interpreted as the eigenfunction of some time operator,
which in turn acts as a simple multiplication by $t$ on this
wavefunction $\psi(x; t)$ ---which would be, therefore, a wavefunction in ``time--position representation''.
In other words, why cannot the instant in \emph{time} and the position in \emph{space} be treated
\emph{on equal footing}?

The fact that $\delta(t-t')$ would be an \emph{improper} eigenfunction
is not the blocking issue in this case (it is not, for the position eigenfunction),
and the mathematical delicacies related to the continuous spectrum are
elegantly resolved by the spectral theory, mainly by Von Neumann
\parencite{VonNeumann}, which is an integral part of the standard, commonly accepted
formulation of quantum mechanics, almost since the early years of the theory.

The idea of treating space and time on equal footing might naturally lead the reader to
ask another question: whether a theory combining quantum mechanics with Einstein's
Relativity is the suitable framework which would allow a rigorous and logically consistent
definition of a time observable.

None of the proposed models comprehending both quantum theory and \emph{general} relativity
have reached general consensus or the remote possibility of experimental verification to date.
