\section{Time and the Quantum: An Invitation}\label{sec:intro}

Time is a fundamental concept in physics, together with space (or \term{position} within space),
matter (or \emph{mass}) and energy.

In quantum mechanics, at least in first quantization, defining the \emph{position} as
an observable is unproblematic and somewhat basic.
Indeed, representing the quantum state of a particle
in terms of coefficients with respect to eigenstates of position is extremely common.
Almost always when a ``wavefunction'' $\psi$
for a state $\ket{\psi}$
is written down, the \emph{position representation}
is assumed, unless otherwise (explicitly) noted:
\begin{equation}\label{eq:positionrepr}
  \ket{\psi} = \int \dd{x'} \psi(x') \ket{x'} \text{,}
\end{equation}
where\footnote{
  Here a one-dimensional system is considered, for simplicity:
  it can be easily verified that more dimensions,
  and degeneracy, may be taken into account without altering the core rationale of the
  following arguments in this Section.
}
$\psi(x') = \braket{x'}{\psi}$.

As per \emph{mass}, a quantum mechanism that explains how particle acquire mass was theorized by
Peter Higgs and others in 1964, and confirmed experimentally in 2012
\parencite{Higgs, EnglertBrout, Kibble+, HiggsATLAS, HiggsCMS}.

\emph{Energy} is associated with arguably the most important operator in quantum dynamics: the Hamiltonian.

\emph{Time} is missing from this picture. As of writing of the present work,
there is no general consensus in the physics community on
the definition of a quantum operator associated to a time observable,
or even only on a particular quantum mechanism that would allow time to ``emerge'',
for example as an effective quantity.

In particular, the relation between time and position is explored in what follows,
with reminders about basic properties that will be useful in the rest of the chapter.

\subsection{Time and position}

Rigorously speaking, the position observable has no special role in quantum mechanics.
Eq.~\eqref{eq:positionrepr} can be further expanded to
\begin{equation}\label{eq:qprepr}
  \ket{\psi} = \int \dd{x'} \psi(x') \ket{x'} = \int \dd{p'} \tilde{\psi}(p') \ket{p'} \,\text{,}
\end{equation}
with $p$ and $\ket{p}$ being respectively eigenvalues and eigenvectors of the momentum operator.

By multiplying on the left by $\bra{x}$, Eq.~\eqref{eq:positionrepr} yields
\begin{equation}\label{eq:diracdeltax}
  \psi(x) = \int \dd{x'} \psi(x') \delta(x-x') \,\text{.}
\end{equation}
In general, for an observable with continuous spectrum,
in its own eigenbasis, eigenstates are represented as a Dirac deltas
(for discrete spectrums, the integral is replaced by a discrete sum,
and Dirac deltas by Kronecker deltas).

Applying the position operator $\hat{X}$ to the~\eqref{eq:positionrepr},
then taking the inner product on the left by $\bra{x}$,
there has
\begin{equation}
  \qty[\hat{X}\psi](x) \eqbydef \mel{x}{\hat{X}}{\psi} =
    \int \dd{x'} \psi(x') x' \mel{x}{\hat{X}}{x'} =
    \int \dd{x'} \psi(x') \delta(x-x') =
    x\psi(x)
  \,\text{.}
\end{equation}
The position
(and an observable in general) is represented, in its own eigenbasis,
as a simple multiplication of the original wavefunction by the respective variable.

\subsection{Is time ``special''?}

As a matter of experience if not anything else,
the state of a physical system, either classical or quantum,
with all observable quantities,
can be regarded as a function of the instant in \emph{time}
(let's call that independent variable $t$).
The same cannot be stated in general for any other pair of measurable quantities:
the $x$ coordinate is not a function of the $y$ coordinate
(or their respective statistical amplitudes, in the quantum realm),
and so on.
This status of time as the universally independent variable
has shaped classical mathematical physics: Galilean transformations
do not change time, which stays as an absolute quantity.
The Hamiltonian formalism itself is based on this assumption,
and the Hamiltonian formalism is at the foundation of quantum mechanics too,
which thus inherits ---one may argue--- the difficulties connected to
treating time as ``external'' with respect to the other quantities under study.
``[The] Hamiltonian method [\dots] marks out a particular time variable
as the canonical conjugate of the Hamiltonian function'' \parencite{DiracLagrangian}.

Emphasizing time as a parameter, the value of a wavefunction
(in position representation, for example)
can therefore be expressed as $\psi_{t}(x)$ at each position $x$
and at each time $t$.
Still, it can be regarded as a function of two variables
and one may wonder why (after an inessential change of notation),
in the identity
\begin{equation}\label{eq:diracdeltaxt}
  \psi(x; t) = \int \dd{x'}\dd{t'} \delta(x-x')\delta(t-t') \psi(x';\, t') \,\text{,}
\end{equation}
the term $\delta(t-t')$ cannot be interpreted as the eigenfunction of some time operator,
which in turn acts as a simple multiplication by $t$ on this
wavefunction $\psi(x; t)$ ---which would be, therefore, a wavefunction in ``time--position representation''.

The fact that $\delta(t-t')$ would be an \emph{improper} eigenfunction
is not the blocking issue in this case (it is not, for the position eigenfunction),
and the mathematical intricacies related to the continuous spectrum are
elegantly resolved by the spectral theory, mainly by Von Neumann
\parencite{VonNeumann}, which is an integral part of the standard, commonly accepted
formulation of quantum mechanics, almost since the early years of the theory.

\subsection{Time and the Hamiltonian}\label{sec:T--H}

In the quest for a quantum time observable,
it is tempting to leverage
the dualism between position and momentum,
evoked in Eq.~\eqref{eq:qprepr}.
It is well known that momentum is the infinitesimal generator of spatial translations:
\begin{equation}
  \psi(x-\Delta x; \, t) = \E^{-\iu \Delta x \hat{p} / \hbar} \psi(x; t) \,\text{,}
\end{equation}
where the momentum operator is $\hat{p} \repr -\iu\hbar \pdv{x}$ in position representation.

It is also well known that the Schr\"odinger equation is equivalent to
stating that the Hamiltonian is the infinitesimal generator of
\emph{temporal} translation:
\begin{equation}
  \psi(x; t+\Delta t) = \E^{-\iu \Delta t \hat{H} / \hbar} \psi(x; t) \,\text{.}
\end{equation}

The analogy between the two equations above,
among other considerations,
motivates the requirement that
the time (self-adjoint) operator is the canonically conjugate of the Hamiltonian,
just like linear momentum is canonically conjugate to the position operator.

\subsection{Photon position, and a final remark}

The fact that QFT ``externalizes'' time and position to the rank of classical parameters
is particularly true in quantum electrodynamics, also within the
phenomenology of quantum optics.
Thus it is impossible to define a \emph{photon} position
within standard quantum optics (see, for example, \cite{ScullyZubairy}, \S 1.5.4 `Wave function for photons'),
and the problem of defining a quantum position observable for a photon
shows an interesting analogy with time for a quantum massive particle.
It is, in fact, a current topic of active research \parencite{HawtonPhotonPosition, Hawton2019}.

While there isn't a wave function for photons, the wave function for massive particles,
as seen in the Shr\"{o}dinger equation, only exists as a function of position (or momentum, etc.)
but not as a function of time,
i.e. the variable $t$ in Eq.~\eqref{eq:diracdeltaxt} cannot be regarded as taking values among the eigenvalues of a time operator.
The main reason for this is the \term{Pauli's objection}, as illustrated in Section~\ref{proof},
which is also ---and particularly--- valid in nonrelativistic quantum mechanics. Nevertheless, this Section
has provided an explanation as to why (even without considering the Pauli objection)
relativistic field theories, as currently accepted, cannot provide a solution
to the problem of quantum time, in spite of treating time and space on equal footing,
and in spite of the fact that position in quantum mechanics is an observable
with its associated self-adjoint operator.
