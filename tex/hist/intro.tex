\section{Time and the Quantum: an invitation}

Time is a fundamental concept in physics, together with space (or \term{position} within space),
matter (or \emph{mass}) and energy.

In quantum mechanics, at least in first quantization, defining the \emph{position} of a particle as
an observable is unproblematic. More strongly, representing the quantum state of a particle
in terms of coefficients with respect to eigenstates of position is extremely common.
Almost always when a ``wavefunction'' is written down, the \emph{position representation}
is assumed, unless otherwise (explicitly) noted:
\begin{equation}\label{eq:diracdeltax}
  \psi(x) = \int \dd{x'} \delta(x-x') \psi(x') \,\text{.}
\end{equation}

A quantum mechanism that explains how particle acquire \emph{mass} was theorized by
Peter Higgs and others in 1964, and confirmed experimentally in 2012.

\emph{Energy} is associated with arguably the most important operator in quantum dynamics: the Hamiltonian.

Time is missing from this picture.

The \eqref{eq:diracdeltax} is really a simple mathematical identity.
It's a defining property of the Dirac delta.
From a quantum mechanical perspective it's almost tautological
that the coefficients of an eigenvector $\ket{e_k}$ with respect to a basis
that include the eigenvector itself are simply $\delta_{jk}$.
Likewise, the \eqref{eq:diracdeltax} in discrete form is $a_k = \sum_j \delta_{jk} a_j$.
There is nothing special about the position observable, from this point of view.
In the position representation it acts on the wavefunction by simply
associating $x\psi(x)$ to $\psi(x)$, for each $x \in \mathbb{R}$,
but the same would happen for another operator with continuous spectrum
if the wavefunction is with respect to the continuous eigenbasis
of that operator; and a similar identity
would hold true in case of discrete spectrum
(with discrete sums and \emph{Kronecker} deltas in the expression);
and $\delta(\xi-\xi')$ (or $\delta_{jk}$) are the respective
eigenfunctions (albeit \emph{improper} for the continuous case).

With this in mind, one may wonder why in the identity
\begin{equation}\label{eq:diracdeltaxt}
  \psi(x; t) = \int \dd{x'}\dd{t'} \delta(x-x')\delta(t-t') \psi(x';\, t') \,\text{.}
\end{equation}
the term $\delta(t-t')$ cannot be interpreted as the eigenfunction of a time operator,
which in turn acts as a simple multiplication by $t$ on this
wavefunction $\psi(x; t)$ which is in ``time-position representation''.
