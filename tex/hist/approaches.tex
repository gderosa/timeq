\section{Overcoming Pauli objection through weaker requirements}

For several decades, Pauli's argument had prevented most theoretical attempts at
defining a self-adjoint time operator with the required commutation and uncertainty properties.
However, some research effort has been invested
into weakening some of those requirements, thus
introducing a notion of a time observable that did not satisfy
Pauli's ---explicit or implicit--- assumptions in the first place.

Notable examples of possible approaches involve:
renouncing the Hermiticity of the operator~$\hat{T}$ (replacing it with a more general symmetric operator);
allowing for the corresponding spectral measurement not to be projective (replacing it with a POVM);
or including the case of an \emph{imaginary} potential in the Hamiltonian,
to model absorption by a detector via loss of normalization (non-unitary evolution).

\subsection{Aharonov--Bohm}

In their 1961 paper, Aharonov and Bohm \parencite{AharonovBohm}
shown that ``energy can be measured
  reproducibly in an arbitrarily short time'',
thus apparently contradicting the time--energy indeterminacy theorized
by Mandelstam--Tamm and other authors.
It is worth observing though,
as they do in the article, that in the Mandelstam--Tamm derivation time has a different meaning:
it is essentially the lifetime of a system in a particular state
and not the duration of the energy measurement process.
They also critically reviewed previous work by Landau and Peierls \parencite{LandauPeierls}
and by Fock and Krylov \parencite{FockKrylov}, discussed their level of generality and provided
counterexamples to their formulation of the time--energy uncertainty relation.

\subsection*{TODO list}
\begin{itemize}
\item ``the fifties'': use it? Events/jumps (sec. 1.6.1 1.6.2 book?), lack of quantum description, need for classical etc.
\item Kijowski
\item detector model \cite{TQM1, TQM2}. Allcock. Move from chapter on pw+detectors.
\item also mention detector model with explicit detector in the picture (and no complex potentialm Ruschhaupt Ch. 4 vol. 2 book)
\item Zeno?
\item POVM 1.5.3 book, plus dedicated chapter 10 or P. Busch, M. Grabowski, P. Lahti: Operational Quantum Physics (Springer–
Verlag, Berlin 1997) (maybe move up and join with Kijowski? ---``Kijowski is esentially POVM''?)
\item consistent histories, Halliwell chapter in the book?
\item `some recent trends, book 1.6'
\item
    time $\otimes$ position Hilbert space or ``second'' Schr\"odinger equation (Prvanovic)
\item time and entanglement (Page and Wootters model, Leggett-Garg inequality as \emph{time} version of Bell inequalities, experiments by Moreva, Genovese et al.)
\item approaches where not only spacetime but causality itself is not fundamental (indefinite causal order: Oreshkov, Brukner et al)
\item event-based approaches:
  \begin{itemize}
    \item ``event'' wavefunction square integrable in 4D (how does it relate rigorously to detector model?)
    \item event-enhanced quantum theory (EEQT, Ruschhaupt et al.)
    \item
  \end{itemize}
\item J. Vaccaro
\item Nikolova (topological)
\item Noncommutative geometry?
\begin{itemize}
  \item \url{https://arxiv.org/pdf/1708.04769.pdf}
  \item \url{https://arxiv.org/pdf/1703.02470.pdf}
  \item \url{https://www.impan.pl/swiat-matematyki/notatki-z-wyklado~/connes_ncgp.pdf}
  \item Connes' 90s book.
  \item Balachandran is pessimistic
\end{itemize}
\end{itemize}
