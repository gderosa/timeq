This work relates to the
\emph{problem of time},
the difficulties of which
represent a classic problem in the foundations of quantum mechanics.
It can be traced back to
the argument, by Wolfgang Pauli,
on the impossibility of defining time as a quantum observable,
due to contradictions related to the spectrum of energy~(1933).
A number of models, aimed at overcoming such difficulties,
have been developed over the decades,
and are reviewed in the present work.

Particular attention is paid to 
the theoretical framework first proposed by Page and Wootters in 1983
(later improved by Lloyd, Giovannetti and Maccone and others),
where time and unitary evolution only emerge in
terms of \emph{entanglement} between non-interacting subsystems
of an otherwise stationary ``universe'',
and where one of the subsystems acts as a ``clock'' for the ``rest'' of it.

\emph{Discrete} clock examples, within the framework, are proposed and implemented,
using
existing results related to quantum systems described by finite-dimensional Hilbert spaces.

The formalism is then applied to some simple quantum systems,
and a numerical comparison is performed between the Page--Wootters model and the predictions
of ``ordinary'' quantum mechanics.

The Page and Wootters formalism is also applied to \emph{non-unitary} systems,
such as those modeled in absorption theories of time-of-arrival at a particle detector.
A comparison is made between the predictions of the two models as well.
The case of differing results, among them, potentially opens the way to a future experimental verification.