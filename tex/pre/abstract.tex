This work relates to the problem of time in quantum physics%
\footnote{
  See, for a review, \cite{TQM1, TQM2}.
},
from Wolfgang Pauli's argument
on the impossibility of defining a time observable \parencite{PauliFootnote},
to a number of models aimed at overcoming it,
and in particular \emph{relational} models
developed during the last decades.

According to the theoretical framework first proposed in \cite{PageWootters}
(and later improved in \cite{Lloyd:Time}),
time and unitary evolution only emerge in
terms of \emph{entanglement} between noninteracting subsystems
---one of which acts as a ``clock''
of an otherwise stationary universe \parencite{Marletto:Evolution}.

Historically, Pauli's ``no-go'' argument was simply a brief observation in a footnote
of his \textit{General Principles of Quantum Mechanics},
with no rigorous or complete proof. Recent works, sketching such proof, are
reviewed, giving some additional details.

One of the aims of this thesis
is contributing towards bridging a gap between the theoretical
and experimental literature on the topic.
To this end, examples and applications are developed on top of existing theoretical results.  
Conversely,
a conceptual critique and theoretical analysis
of recent experiments is carried out.
Numerical computation is also extensively employed in order to compare
predictions from different models. A particular attention is given
to finite-level clocks,
thus extending the Page--Wootter formalism ---which is based on continuous time---
to cases that are easier to implement numerically;
have no mathematical difficulties in terms of convergence;
and, in principle, can be realized experimentally through existing quantum technologies
(see e.g. \cite{FiniteHilb}). 