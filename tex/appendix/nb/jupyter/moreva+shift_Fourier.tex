\section{Analysis of the Moreva et al. experiment}\label{sec:moreva:shit_Fourier}

This notebook is, in its essence, an analysis of the experiment in \cite{Moreva:illustration, Moreva:synthetic}.
In addition to the findings therein, an explicit form of the time operator is derived.
Through change of basis, a suitable 
representation of the Page--Wootters history vector $\dket{\Psi}$
is obtained,
thus allowing a comparison with the time evolution
as predicted by ordinary quantum theory.

\begin{lstlisting}[language=Python]
from sympy import *
from sympy.matrices.expressions.fourier import DFT
from sympy.physics.quantum import TensorProduct
from sympy.physics.quantum.dagger import Dagger

init_printing()
\end{lstlisting}

\subsection*{Basic assumptions and units}

It is set $\hbar = \omega = 0$, hence $\hat{H}_T = \hat{\Omega}$.

\subsection*{Polarization eigensates (``computational'' basis) in $\mathcal{H}_T$}

\begin{lstlisting}[language=Python]
H_hv = Matrix([
    [1],
    [0]
])
V_hv = Matrix([
    [0],
    [1]
])
\end{lstlisting}

\subsection*{Page--Wootters history vector $\dket{\Psi}$, in polarization representation}

\begin{lstlisting}[language=Python]
Psi_hv = (1 / sqrt(2)) * (TensorProduct(H_hv, V_hv) - TensorProduct(V_hv, H_hv))
\end{lstlisting}

\subsection*{Hamiltonians in $\hilb{H}_T$ and $\hilb{H}_S$}

\begin{lstlisting}[language=Python]
HT_hv = Matrix([
    [ 0, I],
    [-I, 0]
])
HS_hv = Matrix([
    [ 0, I],
    [-I, 0]
])
\end{lstlisting}

\subsection*{Verification of the Wheeler--DeWitt condition}

\begin{lstlisting}[language=Python]
J_hv = TensorProduct(HT_hv, eye(2)) + TensorProduct(eye(2), HS_hv)

J_hv @ Psi_hv
\end{lstlisting}

$$
  \left[\begin{matrix}0\\0\\0\\0\end{matrix}\right]
$$

\subsection*{Eigenvalues and eigenvectors of $\op{H}_T$}

\begin{lstlisting}[language=Python]
F = DFT(2).as_explicit().as_mutable()

eigensys_H_T = HT_hv.eigenvects()

eigensys_H_T
\end{lstlisting}
$$
  \left[ \left( -1, \  1, \  \left[ \left[\begin{matrix}- i\\1\end{matrix}\right]\right]\right), \  \left( 1, \  1, \  \left[ \left[\begin{matrix}i\\1\end{matrix}\right]\right]\right)\right]
$$
\begin{lstlisting}[language=Python]
eigenvalues_H_T = list(map(lambda el: el[0], eigensys_H_T))
\end{lstlisting}

\subsection*{Change to $H_T$ representation}

\begin{lstlisting}[language=Python]
U = Matrix([eigensys_H_T[0][2][0].T, eigensys_H_T[1][2][0].T]).T / sqrt(2)

HT_HT = Dagger(U) @ HT_hv @ U
\end{lstlisting}

\subsection*{Clock frequency (or energy) resolution}

\begin{lstlisting}[language=Python]
delta_HT = abs(HT_HT[1,1] - HT_HT[0,0])
\end{lstlisting}

\subsection*{Time operator in clock frequency (or energy) eigenbasis}

\begin{lstlisting}[language=Python]
T_HT = (pi / (delta_HT**2)) * F @ HT_HT @ Dagger(F) 

T_HT
\end{lstlisting}
$$
  \left[\begin{matrix}0 & - \frac{\pi}{4}\\- \frac{\pi}{4} & 0\end{matrix}\right]
$$
\begin{lstlisting}[language=Python]
eigenvals_T = list(T_HT.eigenvals().keys())

eigenvals_T.sort()
\end{lstlisting}

\subsection*{Time resolution of the clock (difference between contiguous eigenvalues)}

\begin{lstlisting}[language=Python]
delta_T = eigenvals_T[1] - eigenvals_T[0]
\end{lstlisting}

\subsection*{(Double) change of basis}

Now, the $U^{\dagger}$ matrix translates components \emph{from Polarization into Energy/Frequency} eigenbasis representation
in the clock space $\hilb{H}_T$.

The (inverse) Fourier $F^{\dagger}$ does the same \emph{from Energy/Frequency into Time}.

As we are operating in the product space, we need the tensor product by $\idop$ in the ``system'' space $\mathcal{H}_S$.

\subsubsection*{Correction term due to $\omega_0 \neq 0$ (frequency shift of Fourier transform)}

See Eq.~\eqref{eq:IDFT:chrepr:tshift}:
\begin{equation}
  \langle{t_{m}}|{\psi}\rangle = e^{i\omega_{0}t_m} \sum_n F^{\dagger}_{mn} \langle{\omega_n}|{\psi}\rangle \text{.}
\end{equation}

Shift term: $e^{i\omega_{0}t_m} \text{,} \; \forall m = 0, 1$.

\begin{lstlisting}[language=Python]
omega_0 = eigenvalues_H_T[0]

shift = list(map(lambda t_m: exp(t_m * omega_0 * I), eigenvals_T))

# matrix form
Shift = diag(*shift)

Shift
\end{lstlisting}
$$
  \left[\begin{matrix}e^{\frac{i \pi}{4}} & 0\\0 & e^{- \frac{i \pi}{4}}\end{matrix}\right]
$$

\begin{lstlisting}[language=Python]
UU = TensorProduct(Shift @ Dagger(F) @ Dagger(U), eye(2))

UU
\end{lstlisting}
$$
  \left[\begin{matrix}0 & 0 & e^{\frac{i \pi}{4}} & 0\\0 & 0 & 0 & e^{\frac{i \pi}{4}}\\i e^{- \frac{i \pi}{4}} & 0 & 0 & 0\\0 & i e^{- \frac{i \pi}{4}} & 0 & 0\end{matrix}\right]
$$

\begin{lstlisting}[language=Python]
Psi_t = UU @ Psi_hv
\end{lstlisting}

\subsection*{Page--Wootters history vector in time $\otimes$ polarization basis}

\begin{lstlisting}[language=Python]
psi_0 = Matrix(Psi_t[0:2])
psi_1 = Matrix(Psi_t[2:])

psi_0 = psi_0 / psi_0.norm()
psi_1 = psi_0 / psi_1.norm()
\end{lstlisting}

\pagebreak\subsubsection*{``Initial'' and ``evolved'' normalized states extracted from the components of the Page--Wootters history vector $\dket{\Psi}$}

\begin{lstlisting}[language=Python]
psi_0
\end{lstlisting}
$$
  \left[\begin{matrix}- e^{\frac{i \pi}{4}}\\0\end{matrix}\right]
$$
\begin{lstlisting}[language=Python]
psi_1
\end{lstlisting}
$$
  \left[\begin{matrix}0\\i e^{- \frac{i \pi}{4}}\end{matrix}\right]
$$

\subsection*{Time evolution in standard quantum mechanics (for comparison)}

\begin{lstlisting}[language=Python]
U_evol = exp(-I*HS_hv*delta_T)

evolved_Schrod = U_evol @ psi_0

evolved_PW = psi_1

evolved_Schrod
\end{lstlisting}
$$
  \left[\begin{matrix}0\\e^{\frac{i \pi}{4}}\end{matrix}\right]
$$
\begin{lstlisting}[language=Python]
evolved_PW
\end{lstlisting}
$$
  \left[\begin{matrix}0\\i e^{- \frac{i \pi}{4}}\end{matrix}\right]
$$
Thanks to the shift term (``\verb|Shift|'')
(and simply recalling that $i=e^{\frac{i\pi}{2}}$),
it is seen that
results from the two theories coincide.
\begin{lstlisting}[language=Python]
simplify(evolved_Schrod - evolved_PW)
\end{lstlisting}
$$
  \left[\begin{matrix}0\\0\end{matrix}\right] \, \text{.} \; \qed
$$