\hypertarget{nb:moreva}{%
\section{Analysis of the Moreva et
al.~experiment}\label{nb:jupyter:moreva}}

\hypertarget{nb:jupyter:moreva:preliminaries}{%
\subsection{Preliminaries}\label{nb:jupyter:moreva:preliminaries}}

\begin{lstlisting}[language=Python]
# Symbolic computation
from sympy import *
from sympy.physics.matrices import mdft
from sympy.physics.quantum import TensorProduct
from sympy.physics.quantum.constants import hbar
\end{lstlisting}

\begin{lstlisting}[language=Python]
# Remember this to have LaTeX rendered output in Jupyter
init_printing()
\end{lstlisting}

\hypertarget{nb:jupyter:moreva:computation}{%
\subsection{Computation}\label{nb:jupyter:moreva:computation}}

\begin{lstlisting}[language=Python]
Omega = Symbol(r'\Omega')
omega = Symbol(r'\omega', real=True)
\end{lstlisting}

\begin{lstlisting}[language=Python]
F = mdft(2)
\end{lstlisting}

\begin{lstlisting}[language=Python]
Omega = I*omega*Matrix([
    [0, 1],
    [-1,0]
])
\end{lstlisting}

\begin{lstlisting}[language=Python]
Omega.eigenvects()
\end{lstlisting}

\[\left [ \left ( - \omega, \quad 1, \quad \left [ \left[\begin{matrix}- i\\1\end{matrix}\right]\right ]\right ), \quad \left ( \omega, \quad 1, \quad \left [ \left[\begin{matrix}i\\1\end{matrix}\right]\right ]\right )\right ]\]

\begin{lstlisting}[language=Python]
T = (pi / (2*omega)**2) * F.adjoint()*Omega*F
\end{lstlisting}

\begin{lstlisting}[language=Python]
T
\end{lstlisting}

\[\left[\begin{matrix}0 & - \frac{i \pi}{4 \omega}\\\frac{i \pi}{4 \omega} & 0\end{matrix}\right]\]

\begin{lstlisting}[language=Python]
T.eigenvects()
\end{lstlisting}

\[\left [ \left ( - \frac{\pi}{4 \omega}, \quad 1, \quad \left [ \left[\begin{matrix}i\\1\end{matrix}\right]\right ]\right ), \quad \left ( \frac{\pi}{4 \omega}, \quad 1, \quad \left [ \left[\begin{matrix}- i\\1\end{matrix}\right]\right ]\right )\right ]\]

\begin{lstlisting}[language=Python]
T_d = diag(-pi/(4*omega), pi/(4*omega))
\end{lstlisting}

\begin{lstlisting}[language=Python]
T_d
\end{lstlisting}

\[\left[\begin{matrix}- \frac{\pi}{4 \omega} & 0\\0 & \frac{\pi}{4 \omega}\end{matrix}\right]\]

Check: this is what we would obtain with matrix of cols egeinv

\begin{lstlisting}[language=Python]
R = (1/sqrt(2)) * Matrix([
    [I, -I],
    [1, 1]
])
\end{lstlisting}

\begin{lstlisting}[language=Python]
R.adjoint()*T*R
\end{lstlisting}

\[\left[\begin{matrix}- \frac{\pi}{4 \omega} & 0\\0 & \frac{\pi}{4 \omega}\end{matrix}\right]\]

\begin{lstlisting}[language=Python]
Omega_T_d = (pi/((pi/(2*omega))**2))*F*T_d*F.adjoint()
\end{lstlisting}

\begin{lstlisting}[language=Python]
Omega_T_d
\end{lstlisting}

\[\left[\begin{matrix}0 & - \omega\\- \omega & 0\end{matrix}\right]\]

\begin{lstlisting}[language=Python]
Hs = I*hbar*omega*Matrix([
    [0, 1],
    [-1,0]
])
\end{lstlisting}

\begin{lstlisting}[language=Python]
J = TensorProduct(hbar*Omega_T_d, eye(2)) + TensorProduct(eye(2), Hs)
\end{lstlisting}

\begin{lstlisting}[language=Python]
J
\end{lstlisting}

\[\left[\begin{matrix}0 & \hbar i \omega & - \hbar \omega & 0\\- \hbar i \omega & 0 & 0 & - \hbar \omega\\- \hbar \omega & 0 & 0 & \hbar i \omega\\0 & - \hbar \omega & - \hbar i \omega & 0\end{matrix}\right]\]

\begin{lstlisting}[language=Python]
J.eigenvects()
\end{lstlisting}

\[\left [ \left ( 0, \, 2, \, \left [ \left[\begin{matrix}0\\- i\\1\\0\end{matrix}\right], \, \left[\begin{matrix}i\\0\\0\\1\end{matrix}\right]\right ]\right ), \, \left ( - 2 \hbar \omega, \, 1, \, \left [ \left[\begin{matrix}- i\\1\\- i\\1\end{matrix}\right]\right ]\right ), \, \left ( 2 \hbar \omega, \, 1, \, \left [ \left[\begin{matrix}- i\\-1\\i\\1\end{matrix}\right]\right ]\right )\right ]\]

\hypertarget{nb:jupyter:moreva:qm}{%
\subsection{Ordinary quantum theory}\label{nb:jupyter:moreva:qm}}

\begin{lstlisting}[language=Python]
t = Symbol('t')
t0 = Symbol('t_0')
\end{lstlisting}

\begin{lstlisting}[language=Python]
exp(-I*Hs*(t-t0)/hbar)
\end{lstlisting}

\[\left[\begin{matrix}\frac{1}{2} e^{i \omega \left(t - t_{0}\right)} + \frac{1}{2} e^{- i \omega \left(t - t_{0}\right)} & - \frac{i}{2} e^{i \omega \left(t - t_{0}\right)} + \frac{i}{2} e^{- i \omega \left(t - t_{0}\right)}\\\frac{i}{2} e^{i \omega \left(t - t_{0}\right)} - \frac{i}{2} e^{- i \omega \left(t - t_{0}\right)} & \frac{1}{2} e^{i \omega \left(t - t_{0}\right)} + \frac{1}{2} e^{- i \omega \left(t - t_{0}\right)}\end{matrix}\right]\]

\begin{lstlisting}[language=Python]
exp(-I*Hs*(t-t0)/hbar) * Matrix([0, -I])
\end{lstlisting}

\[\left[\begin{matrix}- i \left(- \frac{i}{2} e^{i \omega \left(t - t_{0}\right)} + \frac{i}{2} e^{- i \omega \left(t - t_{0}\right)}\right)\\- i \left(\frac{1}{2} e^{i \omega \left(t - t_{0}\right)} + \frac{1}{2} e^{- i \omega \left(t - t_{0}\right)}\right)\end{matrix}\right]\]

\begin{lstlisting}[language=Python]
(exp(-I*Hs*(t-t0)/hbar) * Matrix([0, -I])).subs({t: pi/(4*omega), t0: -pi/(4*omega)})
\end{lstlisting}

\[\left[\begin{matrix}- i\\0\end{matrix}\right]\]

There is consistency in predicting the probability (square modulus), but
not probability amplitude: at \(t=\frac{\pi}{4\omega}\) P-W finds
\((1, 0)\) instead of \((-i, 0)\). But the Rabi oscillation in terms of
probability from 100\% \(\left|V\right>\) at
\(t=t_0=-\frac{\pi}{4\omega}\), to 100\% \(\left|H\right>\) at
\(t=\frac{\pi}{4\omega}\) is correctly predicted.
