This work relates to the problem of time in quantum physics,
from Wolfgang Pauli's argument
on the impossibility of a time observable \parencite{PauliFootnote},
to \emph{relational} models developend in the last decades and based
on entanglement of the ordinary Hilbert space of quantum mechanics
with an extra one where a time operator is introduced.

Pauli's objection is analyzed in possibly more details, as well as the models
aimed at overcoming it, in an effort of generalization on one side;
and at bridging a gap between the theoretical
and experimental literature on the other
---adding examples and applications to the former,
and a degree of conceptual critique and theoretical analysys to the latter.
