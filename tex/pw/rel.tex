\section{Relativistic formulation: Klein-Gordon}

The Klein-Gordon equation is the relativistic extension of the
(\emph{square of}) a wave equation for a free spinless particle
[TODO: REFERENCE]. Indeed, both sides have the dimension of
the square of an energy, and finding their square root
(in the operator sense) is non trivial task which was only resolved 
with the Dirac equation [TODO: REFERENCE], from which the
intrinsic \term{spin} (particularly spin $\hbar/2$) logically emerges,
rather then being artificially introduced in the theory
to comply with the phenomenology.

Both equations did not have much fortune as first-quantized equations,
for conceptual difficulties and historical reasons, including the rise
of Quantum Field Theory and second quantization. They do have a fundamental
role though as field equations (before quantization methods are enacted).

The purpose of the present work is, in a sense, promoting time to a quantum
observable, or formally to a linear, self-adjoint operator in some Hiblert space.
The passage from quantum mechanics to field theory does not represent any progress
towards such goal, in that not only it does not ``promote'' time to an operator $\hat{t}$,
but it also ``demotes'' the three position operators $\hat{x}$, $\hat{y}$ and $\hat{z}$
to mere (classical) \emph{parameters}. 