\section{Detector models}

As we have seen in the previous sections,
proper elements of $\pwspace$,
localized in both space and time,
can be interpreted as ``events''
(as opposed to full history vectors, which instead describe the unitary evolution
of a quantum system \emph{at all times}).

The arrival of a particle at a detector
is a kind of event
of particular interest from the perspective of time
in quantum mechanics, for several reasons.

The first reason
is the direct connection with experiments,
``given the appalling evidence that time is also a random variable in the laboratories''
\parencite[Ch. 4]{TQM2};
and because,
``{[as]} a matter of fact, a number of time observables are already routinely measured in laboratories,
for example arrival times in time-of-flight experiments,
but the theoretical foundation of these measurements is still being discussed''
\parencite[Preface to the First Ed.]{TQM1}.

The second reason is the existence of studies
about detector models which also investigate
time-of-arrival as a quantum observable,
as we have seen in Section \ref{sec:hist:detect}.
None of those works are explicitly
based on Page--Wootters relational model ---of which, to our knowledge,
there are no working examples of application to
time-of-detection problems in the current literature.
Section \ref{sec:absorption+pw} and the remainder of the chapter
will be devoted to bridging such gap
by implementing such application
and comparing
the results from the different models.
Some emphasis will be given to
\emph{discrete} relational time
using the techniques developed in Section \ref{sec:pw:qubit}.

One of the goals of this chapter
is indeed combining PW and detector models.

Another route is POVM, see Chap. \ref{ch:decohere} and \ref{ch:hist}.
Nonetheless, we know that a projective measurement on a bipartite system
can be regarded as a POVM
if we look at
on one part only (e.g. \cite{Paris2012}),
thus the clock space of the Page--Wootters mechanism, $\hilb{H}_T$,
can be regarded as a purification space \parencite{Paris2012} for the ``standard''
Hilbert space where the time-of-arrival POVM is defined,
and an interesting line of further research would be a more formal
study of the logical connection between the two approaches, based on this principle.
Both models respond to the issues raised by Pauli
by giving up
the pursuit of
a \emph{projective} measurement in the \emph{standard} Hilbert space,
and rather ``opening'' that Hilbert space in some sense.
