\section{TODOs}

\subsection{Quantum computing, IBM, decomposition, Trotter-SUzuki, Sine-cosine, qubiter dir}

Previous section(s) read(s):
\begin{quote}
  A benefit of finite-dimensional systems is the potential implementation on a finite array of
  qubits in a quantum computer.
\end{quote}
Make a ref from there to here, and talk about qubiter and stuff.

Also, \cite{Moreva:illustration} has FIG.1 gate array repr\dots

\subsection{Quantum Blockchain and Entanglement in Time}

\url{https://spectrum.ieee.org/tech-talk/computing/networks/quantum-blockchains-could-act-like-time-machines}
\url{https://arxiv.org/abs/1804.05979}


\subsection{Misc}

Extra: \cite{TimeAnyons}.

TQM Book: time of residence: applicable to qubits, therefore Moreva experiment.

\url{https://arxiv.org/abs/1708.04302} majorization, thermo.


\subsection{and dwell time}

\cite[\S 5]{TQM2}, \cite{YearsleyHalliwell_Clocks}.

\subsection{In relation with the \term{time of residence}}

Here we compare \cite{Moreva:synthetic, Moreva:illustration}
(a Page and Wootters problem)
with the ``standard'' \term{time of residence}
treated in \cite[\S 5.5.2]{TQM2},
the only usable concept since there is no notion of position
for a qubit.

\subsection{Comparison with (Kijowski/Aharonov-Bohm) time of arrival}

Time of arrival is derived for non-relativistic \parencite{Delgado_TOA, Delgado_TOA2}
and relativisitic \parencite{Leon_TOA_R}
particle.

Kijowski: \cite{Kijowski_Time, Kijowski_Comment}.

\subsection{In relation to Leggett-Garg inequalities}
(Also mentioned in \cite{Moreva_position}).

Ref \cite{LeggettGarg+PageWootters}.

But also Lloyd: \url{https://arxiv.org/abs/1608.05672},
\emph{Decoherent histories approach to the cosmological measure problem}.
Lindbladt, Markov, Open Systems.

Halliwell, \url{https://arxiv.org/abs/1604.01659}. Ancilla, decay (spontaneous emission?).

A phylosophical object to decoherent histories / Everett (Everett mentioned in Marletto/ref)
is at
\url{https://arxiv.org/abs/1603.04845}.



\subsection{On ``taking time seriously'' (time is real) and why it can still be real in Page and Wootters}

No, Smolin
(\term{Time Reborn}, \term{The Singular Universe and the Reality of Time})
pushes this too far, even Special Relativity's time is too ``virtual'' for him.
Similarly, the philosopher Tim Maudlin

\url{https://www.quantamagazine.org/a-defense-of-the-reality-of-time-20170516/}

Here instead we just argue that the clock can actually be really time, not any observable.
We use notation $T$ as in time instead of $C$ as in clock etc.

In this sense, the Moreva experiment is a quantum analogue simulation.
