
\section{Prvanovic and P\&W}
In \cite{Prvanovic}, essentially the clock observable is the Hamiltonian.
The two example clocks are an harmonic oscillator and a free particle.
The harmonic oscillator features discrete time. Generally a time which is
{bounded from below}
is consistent with the Big Bang...

Prvanovic uses ``relativisitc'' constants...

\section{Entanglement and decoherence (Arrow of time)}
See also \cite{EntanglementVsDecoherence}.

Decoherence is an irreversible process, it also happens in measurement.

According to Marletto and Vedral, arrow of time is increase in Entanglement
between the clock and the rest.

So, there seems to be a contradiction: is entanglement ``decreasing''
(i.e. destroyed by decoherence) with time
or increasing?

We can avoid the contradiction saying that
entanglement between two finite systems is
destroyed while the entanglement of each of them with the universe
is increasing.

\section{TODOs}

TODO: \cite{Lloyd:Time} does not only deal with improper eigenstates in $\hilb{H}_T$
(full evolutions?)
but also normalized ones (in $\hilb{H}_T$! \emph{``Events''}?)

TODO: \cite{RealisticClocks}.

\cite{HarmonicClocks} concludes ``Classical clock can be described by an Hamiltonian linear in momentum''\dots
like in relativity?

Extra: \cite{TimeAnyons}.

TODO: use the harmonic oscillator in \cite{HarmonicClocks}
as a PaW clock for the same packet that is measured in
Ruschhaupt's detector model.

Therein, fading wave function: is minus derivative an event?
L4 normalized?

Other systems of interest: decays. Prvanovic new.

TQM Book: time of residence: applicable to qubits, therefore Moreva experiment.

\url{https://arxiv.org/abs/1708.04302} majorization, thermo.

\section{and paths}

Both \cite{YearsleyHalliwell_Clocks} and \cite{Gambini_PW}
reason in terms of paths and actions, maybe Feynmann stuff
in following chapter... and maybe conistent historiesapproach can help
towards linking PaW and ToA?

Also \url{http://quantum.phys.cmu.edu/CHS/CHS_transp.pdf}.

\section{and dwell time}

\cite[\S 5]{TQM2}, \cite{YearsleyHalliwell_Clocks}.

\section{In relation with the \term{time of residence}}

Here we compare \cite{Moreva:synthetic, Moreva:illustration}
(a Page and Wootters problem)
with the ``standard'' \term{time of residence}
treated in \cite[\S 5.5.2]{TQM2},
the only usable concept since there is no notion of position
for a qubit.

\section{Comparison with (Kijowski/Aharonov-Bohm) time of arrival}

Time of arrival is derived for non-relativistic \parencite{Delgado_TOA, Delgado_TOA2}
and relativisitic \parencite{Leon_TOA_R}
particle.

Kijowski: \cite{Kijowski_Time, Kijowski_Comment}.

A Page and Wootters time of arrival is mentioned in \cite{Gambini_PW}.

Time of arrival and clocks: again, \cite{YearsleyHalliwell_Clocks}.
Which maybe suggests we should not wory too much of $H\ket{\Psi} = 0$. 

BUT please note \cite{YearsleyHalliwell_Clocks} uses a clock that is
\emph{coupled} with the system, while in PaW they are ``only'' entangled.
So their calculation may be unnecessarily complicated.
Maybe the weakjly coupling case can be used?

\section{In relation to Leggett-Garg inequalities}
(Also mentioned in \cite{Moreva_position}).

Ref \cite{LeggettGarg+PageWootters}.

But also Lloyd: \url{https://arxiv.org/abs/1608.05672},
\emph{Decoherent histories approach to the cosmological measure problem}.
Lindbladt, Markov, Open Systems.

Halliwell, \url{https://arxiv.org/abs/1604.01659}. Ancilla, decay (spontaneous emission?).

A phylosophical object to decoherent histories / Everett (Everett mentioned in Marletto/ref)
is at
\url{https://arxiv.org/abs/1603.04845}.
\section{Detector model}

\cite{TQM2} (Kijowski and Detector) is cited and summarized well in
\cite{Halliwell_Detector}.



\section{Can a POVM on a system, if then we see it as part of a bipartite one,
equivalent to a PVM on the other, entangled, system?}

TODO: backflow effect in both models.

This will show an equivalence of models based on POVM with the Page and Wootters...?

Well, yes.

From \cite{PreskillNotes}, Ch.3 
\begin{quotation}
We have seen that
a pure state of the bipartite system AB may behave like a mixed state
when we observe subsystem A alone, and that an orthogonal measurement
of the bipartite system can realize a (nonorthogonal) POVM on A alone.
\end{quotation}

and

\begin{quotation}
A POVM in $H_A$ can be realized as a unitary transformation on the tensor
product $H_A \otimes H_B$, followed by an orthogonal measurement in $H_B$.
\end{quotation}

The same chapter talks about quantum operations, quantum channels and Kraus opertators.

We might want to look at exponential decay from \url{https://arxiv.org/abs/1704.07236},
then compare with exponential decay with P and W using Lloyd Giovannetti and Maccone (ref).

\subsection{Purification}

See https://arxiv.org/pdf/quant-ph/0512125.pdf, P-W time as a purifying ancilla
of the (Kijowski?) time.

\subsection{4-partite universe?}
\begin{itemize}
  \item{The system being measured/detected}
  \item{The Ruschhaupt detector --- which does not measure time, but whose detection happens at a certain time}
  \item{The Page and Wootters clock, entangled with the system and/or the detector}
  \item{The rest of the Universe, aka the Environment, aka the Termal Bath or Reservoir}
\end{itemize}

Can any of the above be identified? If the lab is isolated enough,
the detector is the only macro object and can act as a Universe/bath/environment/reservoir\dots?
