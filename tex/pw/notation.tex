\addcontentsline{toc}{section}{A quick note on symbols and conventions}
\section*{\large\it A quick note on symbols and conventions}

\small

In this and following chapters, when treating quantum time models,
$\hilb{H}_S$ will indicate the Hilbert space of ordinary quantum mechanics.
$\hilb{H}_T$, an extra space where a time operator $\hat{T}$ is introduced. The product space $\pwspace$
will often be referred to as well.
Bras and kets in this product space will be indicated with a special double-angle-bracket
notation, as in $\dket{\Psi}$.

$\hat{A}$, with circumflex, denotes a self-adjoint operator, generally associated with a quantum observable.

No curcumflex accent on symbols indicating non self-adjoint operators and their matrices:
this includes unitary transformations such as
time evolution or discrete Fourier.
This also includes ``Hamiltonians'' embedding a \emph{complex potential}
and leading to a non-unitary evolution. The same rule applies to alternative typefaces too, when used.

$\hat{\mathbb{J}}$ (blackboard bold) indicates a self-adjoint operator defined in
a larger space than the one of standard quantum mechanics,
typically the product $\pwspace$.
No circumflex accent on the symbol ${\mathbb{J}}$ if the operator is
known to be not self-adjoint.

The symbol `$\repr$', as in $\hat{A} \repr \mqty(a&b\\c&d)$, $\ket{\psi} \repr \mqty[\alpha \\ \beta]$
means: representation with respect to a particular basis (as opposed to intrinsic equality `$=$').

The symbol `$\eqbydef$'
means: equal by definition, equal by settings i.e. postulated and not derived logically.

\normalsize

% % Yeah, \mathcal seems a good idea for unitaries...

% \begin{equation}
%   \hat{\Omega} = \mathcal{F} \hat{T} \mathrm{\mathcal{F}}^{\dagger}
% \end{equation}

% \begin{equation}
%   \ket{\psi} = \mathcal{U}_t \ket{0}
% \end{equation}
