\section{Finite-dimensional systems}\label{sec:finite-quantum}\label{sec:pw:theory_last}
\epigraph{\textelp{} discreteness in the world is simply the Fourier transform of compactness.}{%
  \emph{Physics and the Integers} \parencite{Tong_Integers}%
}

\noindent
Finite-dimensional systems are of interest, phenomenologically,
and in general easier to implement,
both experimentally and numerically.

An experimental illustration of the Page--Wootters model will be discussed in Sec. \ref{sec:pw:qubit}.
It is based on a clock
(defined through the corresponding frequency operator)
which has two discrete levels only.

However,
finite-dimensional formulas
present difficulties in identifying canonically conjugate
pairs, and one cannot use the same explicit formalism
that would be used for observables with a continuous spectrum.
Moreover, operators satisfying a canonical
commutation relation such as
\begin{equation}\label{eq:canonical_commutation_in_time}
  [\op{T}, \op{\Omega}] = i
\end{equation}
cannot be both bounded,
therefore they cannot exist
in a finite-dimensional space \parencite{Weyl:FiniteComm}.

This problem was studied in detail in
\cite{FiniteHilb}. It was shown therein
that satisfying the canonical commutation relation
is not essential to build operators representing physical observables
with the same role of position and momentum.\footnote{
  Or $\op{T}$ and $\hbar\op{\Omega}$
  in $\hilb{H}_T$ for a finite-dimensional Page and Wootters model.
}

Discrete, bounded, position-like and momentum-like operators can be obtained from
each other via
the \term{finite} or \term{discrete} \term{Fourier transform} (DFT).
In our case, we are particularly interested in relating the
time operator $\op{T}$ and the ``energy'' operator $\hbar\op{\Omega}$
in $\hilb{H}_T$ ---via the angular frequency operator $\op{\Omega}$ which, in the continuous limit, would satisfy
Eq.~\eqref{eq:canonical_commutation_in_time} exactly.

A benefit of finite-dimensional systems is the potential implementation on a finite array of
qubits in a quantum computer. The use of Discrete Fourier Transform extends the overlap
with technology and engineering to the domain of signal processing \citereset\parencite{FiniteHilb}.
In \emph{ordinary} quantum mechanics, the Fourier transform (discrete or continuous)
is generally used
to associate wavefunctions in position and momentum space
(whereas time and frequency are \emph{not} operators),
while in communication engineering it is used to convert signals
from the time to the frequency domain and vice versa.
Thanks to the introduction of the Hilbert space $\hilb{H}_T$,
the interpretation in terms of time and frequency
(or time and energy, up to a factor $\hbar$)
is applicable to quantum theory as well, not only formally
i.e. not in the sense of a mere operation among (``classical'') parameters;
but in the sense of conversion between representations of the
same quantum state vector with respect to different eigenbasis,
in full analogy with position and momentum in $\hilb{H}_S$.

\subsection{Discrete Fourier transform}

We will add some more details to the analysis of \citereset\cite{FiniteHilb} with
additional emphasis on sign and scaling conventions which will affect
our computation. Results will be interpreted in the context of
the Page--Wootters model, in particular with regards to the temporal subspace
$\hilb{H}_T$.

Let us start with a continuous system where the Fourier transform $\phi \eqbydef \mathcal{F}\psi$ of
an integrable function $\psi: \mathbb{R} \to \mathbb{C}$ is defined as
\begin{equation}\label{eq:fourier_transform:def}
  \phi(\omega) = \left(\mathcal{F}\psi\right) (\omega) =
    \frac{1}{\sqrt{2\pi}} \int \dd{t} \psi(t) \E^{-\iu t \omega} \text{,}
\end{equation}
and the well known \term{inversion} property
\begin{equation}\label{eq:inverse_fourier_transform:def}
  \psi(t) = \left(\mathcal{F}^{-1} \phi \right) (t) =
    \frac{1}{\sqrt{2\pi}} \int \dd{\omega} \phi(\omega) \E^{ \iu t \omega}
\end{equation}
can be can be proven (see e.g. \cite[Eq.~7.1]{Folland:Fourier}).

Eq.~\eqref{eq:inverse_fourier_transform:def} can be interpreted
as a function of time
expressed as a linear superposition
of ``pure frequencies'' $\E^{\iu \omega t}$,
with coefficients given by the Fourier transform $\phi(\omega)$.

In a discrete system of dimension $N$, a finite interval $(0, \Delta{T})$ is considered,
equally divided in $N$ sub-intervals. If we set $\delta{T} = \frac{\Delta{T}}{N}$,
$N$ discrete points in time can be considered
\begin{equation}\label{eq:DFT:t_spectrum}
  t_n \in \setof{0,\,\hdots,\,(N-1)\delta{T}} \text{.}
\end{equation}
%
The corresponding frequencies are multiples of the \term{fundamental} frequency $\nu_1 = \frac{1}{\Delta{T}}$.
The fundamental \emph{angular} frequency is therefore $\omega_1 = \frac{2\pi}{\Delta{T}}$ and the
$N$ sample values are:
\begin{equation}\label{eq:DFT:omega_spectrum}
  \omega_n \in \setof{0, \frac{2\pi}{\Delta{T}}, \dots, \frac{2\pi(N-1)}{\Delta{T}}} \text{.}
\end{equation}
%
It is easily seen that:
\begin{gather}\label{eq:DFT:deltas}
  \delta\Omega \delta T = \frac{2\pi}{N} \, \text{;} \quad
  \Delta\Omega \Delta T = 2\pi N \, \text{.}
\end{gather}
Also:
\begin{equation}\label{eq:DFT:eigenratio}
  \omega_{n} = \frac{2\pi}{N(\delta{T})^{2}} t_{n}  \, \text{.}
\end{equation}
%
The discretization
of \eqref{eq:fourier_transform:def} and \eqref{eq:inverse_fourier_transform:def}
then reads\footnote{
  See e.g.
  \cite{Oppenheim:Int1,Oppenheim:Int3,ProakisManolakis}.
}
\begin{equation}\label{eq:DFT:def}
  \phi_n \eqbydef \phi(\omega_n) = \frac{1}{\sqrt{N}} \sum_{m=0}^{N-1} \psi_m \E^{-\iu n m 2 \pi / N}
\end{equation}
and, respectively,
\begin{equation}\label{eq:IDFT:def}
  \psi_m \eqbydef \psi(t_m) = \frac{1}{\sqrt{N}} \sum_{n=0}^{N-1} \phi_n \E^{\iu n m 2 \pi / N} \text{.}
\end{equation}
%
Eq.~\eqref{eq:DFT:def} is the definition of the \term{Discrete Fourier Transform} (DFT);
Eq.~\eqref{eq:IDFT:def} defines the \term{Inverse Discrete Fourier Transform} (IDFT).
%
The factor $\frac{1}{\sqrt{N}}$
is chosen to guarantee \emph{unitarity} of the transformation:
$\sum_{n=0}^{N-1} \abs{\psi_n}^2 = \sum_{n=0}^{N-1} \abs{\phi_n}^2$.

From Eqs.~\eqref{eq:DFT:def} and~\eqref{eq:IDFT:def} one can identify the
\term{Discrete Fourier Matrix}, with elements $F_{mn}$
(and its inverse,
with elements $F_{mn}^{\dagger}$):
\begin{align}\label{eq:pw:F_mn:def}
  F_{mn}            &= \frac{1}{\sqrt{N}} \E^{-\iu n m 2 \pi / N} \;\text{;} &
  F_{mn}^{\dagger}  &= \frac{1}{\sqrt{N}} \E^{ \iu n m 2 \pi / N}
  \text{.}
\end{align}

% {\color{magenta}\hrulefill}

The (discrete) Fourier transform can be used to transform the time representation
of a vector $\ket{\psi}_T \in \hilb{H}_T$ into its angular frequency representation.
Namely, from Eq.~\eqref{eq:DFT:def}, the sequence\footnote{
  We will omit the subscript ``${}_{T}$'' from the following equations
  where there is no ambiguity that
  we are operating in the subspace $\hilb{H}_T$
  of the Page--Wootters model.
}
$\left\{\braket{t_n}{\psi}\right\}_{n=0, \dots, N-1}$ into
$\left\{\braket{\omega_m}{\psi}\right\}_{m=0, \dots, N-1}$:
\begin{equation}\label{eq:DFT:chrepr}
  \braket{\omega_{m}}{\psi} = \sum_n F_{mn} \braket{t_n}{\psi} \text{.}
\end{equation}
It is convenient to write down the ``complex conjugate'' of Eq.~\eqref{eq:DFT:chrepr} as well:
\begin{equation}\label{eq:DFT:chrepr:cconj}
  \braket{\psi}{\omega_{m}} = \sum_n F_{mn}^{\dagger} \braket{\psi}{t_n}
\end{equation}
(recalling that $F$ is symmetric hence
$F_{mn}^{*} = F_{mn}^{\dagger}$,
$\forall m,n \in \setof{0, \dots, N-1}$).

Equally, from Eq.~\eqref{eq:IDFT:def}:
\begin{align}
  \label{eq:IDFT:chrepr}
  \braket{t_{m}}{\psi} &= \sum_n F^{\dagger}_{mn} \braket{\omega_n}{\psi} \text{,} \\
  \label{eq:IDFT:chrepr:cconj}
  \braket{\psi}{t_{m}} &= \sum_n F_{mn} \braket{\psi}{\omega_n} \text{.}
\end{align}

As $\ket{\psi}$ (respectively: $\bra{\psi}$) is a generic ket (or bra) in the Hilbert space,
from Eqs.~\eqref{eq:DFT:chrepr} and~\eqref{eq:DFT:chrepr:cconj} it follows:
\begin{align}
  \label{eq:DFT:bra}  \bra{\omega_{m}} &= \sum_n F_{mn}           \bra{t_n}       \,\text{,}  \\
  \label{eq:DFT:ket}  \ket{\omega_{m}} &= \sum_n F_{mn}^{\dagger} \ket{t_n}       \,\text{;}
\end{align}
and, from Eqs.~\eqref{eq:IDFT:chrepr} and~\eqref{eq:IDFT:chrepr:cconj}:
\begin{align}
  \label{eq:IDFT:bra} \bra{t_{m}}      &= \sum_n F^{\dagger}_{mn} \bra{\omega_n}  \,\text{,} \\
  \label{eq:IDFT:ket} \ket{t_{m}}      &= \sum_n F_{mn}           \ket{\omega_n}  \,\text{.}
\end{align}
This shows that the Fourier matrix can be used, not only to transform the components
of a vector from one eigenbasis to the canonically conjugate eigenbasis,
but also to obtain the eigenvectors themselves (of the canonically conjugate operator).

Let us now introduce the \term{Fourier operator}: %\parencite[Sec. 2.1, Eq. 2]{FiniteHilb}
\begin{equation}\label{eq:pw:def_fourier_op}
  \op{F} \eqbydef \sum_{m,n=0}^{N-1} F_{mn} \ketbra{t_m}{t_n} \text{.}
\end{equation}
Multiplying each element of the sum in Eq.~\eqref{eq:DFT:bra}
by $\braket{t_m} = 1$
we immediately get:
\begin{equation}\label{eq:DFO:bra}
  \bra{\omega_{m}} = \bra{t_m} \op{F} \text{.}
\end{equation}
Similarly, Eq.~\eqref{eq:DFT:ket} can be expressed as
\begin{equation}\label{eq:DFO:ket}
  \ket{\omega_{m}} = \op{F}^{\dagger} \ket{t_m} \text{.}
\end{equation}
Also, applying $\op{F}^{\dagger}$ to the right of \eqref{eq:DFO:bra},
and $\op{F}$ to the left of \eqref{eq:DFO:ket}:
\begin{align}
  \label{eq:IDFO:bra} \bra{t_m} &= \bra{\omega_m}\op{F}^{\dagger} \, \text{;} \\
  \label{eq:IDFO:ket} \ket{t_m} &= \op{F} \ket{\omega_m} \, \text{.} 
\end{align}

Finally, we use the spectral decomposition of $\op{\Omega}$,
the results \eqref{eq:DFO:bra} and~\eqref{eq:DFO:ket} above,
and Eq.~\eqref{eq:DFT:eigenratio}
to obtain:
\begin{multline}\label{eq:DFT:OmegaFTF}
  \op{\Omega} = \sum_{m} \omega_{m} \ketbra{\omega_{m}} =
  \sum_{m} \omega_{m} \op{F}^{\dagger} \ketbra{t_{m}} \op{F} =
  \sum_{m} m \qty(\delta\Omega) \op{F}^{\dagger} \ketbra{t_{m}} \op{F}
  \\
  = \sum_{m} \frac{t_m}{\delta{T}} \op{F}^{\dagger} \ketbra{t_{m}} \op{F}
  = \frac{2\pi}{N(\delta{T})^{2}} \op{F}^{\dagger} \left(\sum_{m}t_{m}\ketbra{t_{m}}\right) \op{F}
  \\
  =
  \frac{2\pi}{N(\delta{T})^{2}} \op{F}^{\dagger} \op{T} \op{F} =
  \frac{2\pi N}{\qty(\Delta T)^2} \op{F}^{\dagger} \op{T} \op{F} = \frac{\delta\Omega}{\delta{T}} \op{F}^{\dagger} \op{T} \op{F}
  \text{.}
\end{multline}

In conclusion:
\begin{gather}
  \label{eq:SI_Fourier:Omega}
    \op{\Omega} =
      \frac{2\pi}{N(\delta T)^2}          \op{F}^{\dagger} \op{T} \op{F} =
      \frac{2\pi N}{\qty(\Delta T)^2}     \op{F}^{\dagger} \op{T} \op{F} =
      \op{F}^{\dagger} \qty(\delta{\Omega} \frac{\op{T}}{\delta{T}}) \op{F}
      \, \text{,}
      \\
  \label{eq:SI_Fourier:T}
    \op{T} =
      \frac{2\pi}{N(\delta\Omega)^2}      \op{F} \op{\Omega} \op{F}^{\dagger} =
      \frac{2\pi N}{\qty(\Delta\Omega)^2} \op{F} \op{\Omega} \op{F}^{\dagger} =
      \op{F} \qty(\delta{T} \frac{\op{\Omega}}{\delta{\Omega}}) \op{F}^{\dagger}
      \, \text{;}
\end{gather}
where the operator $\frac{\op{T}}{\delta{T}}$ in Eq.~\eqref{eq:SI_Fourier:Omega},
in $T$ representation, is diagonal with integer elements $\setof{0, \dots, N-1}$.

Conversely, Eq.~\eqref{eq:SI_Fourier:T} is obtained using
the spectral decomposition of $\op{T}$
and Eqs.~\eqref{eq:IDFO:bra} and~\eqref{eq:IDFO:ket};
and $\frac{\op{\Omega}}{\delta{\Omega}}$
is diagonal in $\Omega$ representation,
with elements $\setof{0, \dots, N-1}$.

Note that Eqs.~\eqref{eq:SI_Fourier:Omega} and~\eqref{eq:SI_Fourier:T}
differ from the results of \cite{FiniteHilb}.
Besides the sign convention\footnote{
  The paper, essentially, swaps the definition of Discrete Fourier Transform and its Inverse,
  while we follow e.g. \cite[Sec. 7.6]{Folland:Fourier},
  which is more consistent to the definition of continuous Fourier transform
  found in most quantum mechanics textbooks,
  and used to
  convert position into momentum wavefunctions.
},
this is due to the use of
``natural'' units therein,
in the sense that
both time and angular frequency
have adimensional, integer eigenvalues:
$t_n = \omega_n = 0, 1, \dots, N-1$
(in fact, the paper defines ``position'' $\op{X}$ and ``momentum'' $\op{P}$,
but in a similar fashion).
In other words, in \citereset\cite{FiniteHilb}, the sampling interval and the fundamental harmonic angular frequency
are set to $\delta{T} = \delta{\Omega} = 1$ (or their equivalent in terms of position and momentum).

\subsubsection{Fourier operator in frequency representation}

From Eqs.~\eqref{eq:IDFO:bra} and~\eqref{eq:IDFO:ket},
and the definition \eqref{eq:pw:def_fourier_op}:
\begin{equation}
  \op{F} = \sum_{mn} F_{mn} \op{F} \ketbra{\omega_m}{\omega_n} \op{F}^{\dagger} \, \text{;}
\end{equation}
but, from the unitarity $\op{F}^{\dagger}\op{F} = \idop$:
\begin{equation}\label{eq:pw:proof_same_repr}
  \op{F} = \op{F}^{\dagger}\op{F}\op{F} = \sum_{mn} F_{mn} \op{F}^{\dagger}\op{F} \ketbra{\omega_m}{\omega_n} \op{F}^{\dagger}\op{F}
  = \sum_{mn} F_{mn} \ketbra{\omega_m}{\omega_n} \, \text{.}
\end{equation}
Comparing \eqref{eq:pw:proof_same_repr} and \eqref{eq:pw:def_fourier_op}
it is seen that the Fourier operator $\op{F}$
has the same matrix representation $F_{mn}$
in both $\setof{\ket{t}}$ and $\setof{\ket{\omega}}$ basis. 

\subsection{Uncertainty in finite-dimensional systems}\label{sec:finite_uncertainty}
\citereset
For canonical pairs of operators with a continuous, unbounded spectrum i.e.
$\op{q}$ and $\op{p} \eqbydef -i\hbar\op{\partial}_{q}$,
it is in general straightforward to prove that
\begin{equation}\label{eq:commconstant}
  \qty[\op{q}, \op{p}] = i\hbar
\end{equation}
and therefore
the Robertson form of the uncertainty relation
\begin{equation}\label{eq:robertsonconstant}
  \Delta q \Delta p \geq { \frac{1}{2} \qty|\ev{\qty[\op{q}, \op{p}]}| }
\end{equation}
yields
\begin{equation}\label{eq:min_uncertain_constant}
  \Delta q \Delta p \geq { \frac{\hbar}{2} } \, \text{.}
\end{equation}

In finite $d$-dimensional Hilbert spaces, the commutator of canonically conjugate operators
is not a constant ---see \cite{Weyl:FiniteComm}.
Indeed, in an eigenbasis of $\op{q}$,
the latter is represented by a diagonal matrix Q,
with diagonal elements $q_1, \dots, q_d$.
Let also P  be the matrix representation of $\op{p}$ in the same basis, and $p_{mn}$ its elements, with $m, n = 1 \dots d$.
The left side of \eqref{eq:commconstant} is then represented as
\begin{multline}
  \displaybreak[2]  %% https://tex.stackexchange.com/a/24905
  QP - PQ =
  \mqty(
    q_1   &{}     &{} \\
    {}    &\ddots &{} \\
    {}    &{}     &q_d
  )
  \mqty(
    p_{11}  &\ldots &p_{1d} \\
    \vdots  &\ddots &\vdots \\
    p_{d1}  &\ldots &p_{dd}
  )
  -
  \mqty(
    p_{11}  &\ldots &p_{1d} \\
    \vdots  &\ddots &\vdots \\
    p_{d1}  &\ldots &p_{dd}
  )
  \mqty(
    q_1   &{}     &{} \\
    {}    &\ddots &{} \\
    {}    &{}     &q_d
  )
  \\
  =
  \mqty(
    q_{1}p_{11} &\ldots &q_{1}p_{1d}  \\
    \vdots      &\ddots &\vdots       \\
    q_{d}p_{d1} &\ldots &q_{d}p_{dd}
  )
  -
  \mqty(
    q_{1}p_{11} &\ldots &q_{d}p_{1d}  \\
    \vdots      &\ddots &\vdots       \\
    q_{1}p_{d1} &\ldots &q_{d}p_{dd}
  )
  =
  \qty\bigg{\qty(q_{m}-q_{n})p_{mn}}_{m, n = 1 \, \dots \, d}
  \;\text{.}
\end{multline}
All diagonal elements are thus zero,
hence such matrix cannot represent a constant operator in any basis.

In other words, it is not possible to identify a pair of operators $\op{q}$ and $\op{p}$
satisfying Eq.~\eqref{eq:commconstant} in a finite-dimensional Hilbert space.
%
The value of $\frac{1}{2} \qty|\ev{\qty[\op{q}, \op{p}]}| = \frac{1}{2} \qty|\ev{\qty[\op{q}, \op{p}]}{\psi}|$
would then  depend on the particular state vector $\ket{\psi} \in \hilb{H}_d$,
and one should compute, explicitly, the value of
$\displaystyle \min_{\ket{\psi} \in \hilb{H}_d} \frac{1}{2} \qty|\ev{\qty[\op{q}, \op{p}]}{\psi}|$
to obtain a general lower bound.

Nonetheless, if the discrete system is meant as an approximation of a continuous
one, %
% (and canonical operators are obtained from each other by means of discrete Fourier transformation rather than differentiation),
it can still be of interest to compare the spread product $\Delta{q}\Delta{p}$
(of two discrete operators)
with the lower limit predicted by the continuous theory.


% TODO? Excercise: ppendix \ref{sec:jpynb:finite-comm}.
% BUT it is not even necessary that one is DFT of the other...


%% REMOVED:
% Particularly, the entropic uncertainty relation holds
% (\cite[\s 2.4]{FiniteHilb}; \cite{Deutsch:Uncertainty}):
% \begin{equation}
%   S_q + S_p \geq \ln d
% \end{equation}
% where the quantities $S_q$ and $S_p$ are the \term{R\'enyi}-\term{Shannon} entropies
% \parencite[\s {\it I}.A]{EntroUncertaintyApp}; in this case:
% \begin{align}
%   S_q &= -\sum_n \qty|\lambda_n |^2  \ln\qty|\lambda_n|^2 \\
%   S_p &= -\sum_n \qty|\mu_n     |^2  \ln\qty|\mu_n    |^2
%   \,\text{,}
% \end{align}
% with $\lambda_n$ and $\mu_n$ being the discrete ``wave functions'' in the
% (generalized) position and momentum basis.

% TODO: See also \cite[\s III.A.1--2]{EntroUncertaintyApp}. Perhaps more importantly \cite{Deutsch:Uncertainty} (cited therein).
