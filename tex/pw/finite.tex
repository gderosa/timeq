\section{Finite-dimensional systems}\label{sec:finite-quantum}\label{sec:pw:theory_last}
\epigraph{\textelp{} discreteness in the world is simply the Fourier transform of compactness.}{%
  \emph{Physics and the Integers} \parencite{Tong_Integers}%
}

\noindent
An experimental illustration of the Page--Wootters model will be discussed in Sec. \ref{sec:pw:qubit}.
It is based on a clock
(defined through the corresponding frequency operator)
which has two discrete levels only.

Finite-dimensional systems are of interest, phenomenologically,
and in general easier to implement,
both experimentally and numerically.
However,
finite-dimension Hilbert spaces
present diffilcuties in identifying canonically conjugate
pairs, and one cannot use the same explicit formalism
that would be used for observables with a continuous spectrum.
Moreover, operators satisfying a canonical
commutation relation such as
\begin{equation}\label{eq:canonical_commutation_in_time}
  [\op{T}, \op{\Omega}] = i
\end{equation}
cannot be both bounded, therefore cannot exist in a finite-dimensional space.

This problem was studied in detail in
\cite{FiniteHilb}. It was shown therein
that statisfying the canonical commutation relation 
is not essential to build operators representing physical observables
with the same role of position and momentum.\footnote{
  Or $T$ and $\hbar\Omega$
  in $\hilb{H}_T$ for a finite-dimensional Page and Wootters model.
}

Discrete, bounded, position-like and momentum-like operators can be obtained from
each other via
the \term{finite} or \term{dicrete} \term{Fourier transform} (DFT).
In our case, we are particularly interested in relating the
time operator $\op{T}$ and the ``energy'' operator $-\hbar\op{\Omega}$
in $\hilb{H}_T$ ---via the angular frequency operator $\op{\Omega}$ which, in the continuous limit, would satisfy
eq.~\eqref{eq:canonical_commutation_in_time} exactly.

A benefit of finite-dimensional systems is the potential implementation on a finite array of
qubits in a quantum computer. The use of Discrete Fourier Transform extends the overlap
with technology and engineering to the domain of signal processing \parencite{FiniteHilb}.
In \emph{ordinary} quantum mechanics, the Fourier transform (discrete or continuous)
is generally used
to associate wavefunctions in position and momendum space
(whereas time and frequency are \emph{not} operators),
while in communication engineering it is used to convert signals
from the time to the frequency domain and vice versa.
Thanks to the introduction of the Hilbert space $\hilb{H}_T$,
the interpretation in terms of time and frequency
(or time and energy, up to a factor $\hbar$)
is applicable to quantum theory as well, not only formally
i.e. not in the sense of a mere operation among (``classical'') parameters;
but in the sense of conversion between representations of the
same quantum state vector with respect to different eigenbasis,
in full analogy with position and momentum in $\hilb{H}_S$.

\subsection{Discrete Fourier transform}

We will add some more details to the analysis of \citereset\cite{FiniteHilb} with
additional emphasis on sign and scaling conventions which will affect
our computation. We will then interpret the results in the context of
the Page--Wootters model, in particular with regards to the temporal subspace
$\hilb{H}_T$.

Let us start with a continuous system where the Fourier transform $\phi \eqbydef \mathcal{F}\psi$ of
an integrable function $\psi: \mathbb{R} \to \mathbb{C}$ is defined as
\begin{equation}\label{eq:fourier_transform:def}
  \phi(\omega) = \left(\mathcal{F}\psi\right) (\omega) =
    \frac{1}{\sqrt{2\pi}} \int \dd{t} \psi(t) \E^{- \iu t \omega} \text{,} 
\end{equation}
and the well known \term{inversion} property
\begin{equation}\label{eq:inverse_fourier_transform:def}
  \psi(t) = \left(\mathcal{F}^{-1} \phi \right) (t) =
    \frac{1}{\sqrt{2\pi}} \int \dd{\omega} \phi(\omega) \E^{\iu t \omega} 
\end{equation}
can be can be proven (see e.g. \cite{Folland:Fourier}).

Eq.~\eqref{eq:inverse_fourier_transform:def} can be interpreted
as a function of time
expressed as a linear superposition
of ``pure frequencies'' $\E^{\iu \omega t}$,
with coefficients given by the Fourier transform $\phi(\omega)$.

In a discrete system, a finite interval $(0, \Delta{T})$ is considered,
equally divided in $N$ sub-intervals. If we set $\delta{T} = \frac{\Delta{T}}{N}$,
$N$ discrete points in time can be considered
\begin{equation}\label{eq:DFT:t_spectrum}
  t_n \in \setof{0,\,\hdots,\,(N-1)\delta{T}} \text{.}
\end{equation} 
%
The corresponding frequencies are multiples of the fundamental frequency $\nu_1 = \frac{1}{\Delta{T}}$.
The fundamental \emph{angular} frequency is therefore $\omega_1 = \frac{2\pi}{\Delta{T}}$ and the
$N$ sample values are:\footnote{
  See
  \cite[Ch. ``The Discrete Fourier Transform'']{Oppenheim:Int1,Oppenheim:Int3}.
}
\begin{equation}\label{eq:DFT:omega_spectrum}
  \omega_n \in \setof{0, \frac{2\pi}{\Delta{T}}, \dots, \frac{2\pi(N-1)}{\Delta{T}}} \text{.}
\end{equation} 


% %\citereset
% It holds\footnote{
%   Contrary to what indicated in eq. (8) of \cite{FiniteHilb},
%   it can be easily verified that,
%   if $ \op{p} = F x F^{\dagger} $,
%   the correct inverse relation is
%   $ x = F^{\dagger} p F$ and not $ -x = F p F^{\dagger} $.
% } \parencite{FiniteHilb}:
% \begin{gather}\label{eq:FourierCanonicalRelations}
%   \op{\Omega}_{\mathrm{nat.}} = F \op{T}_{\mathrm{nat.}} F^{\dagger}\text{;} \quad
%   \op{T}_{\mathrm{nat.}} = F^{\dagger} \op{\Omega}_{\mathrm{nat.}} F
% \end{gather}
% where, in the ``position'' (or \emph{time}) finite eigenbasis,
% \begin{equation}
%   F = \frac{1}{\sqrt{N}} \sum_{m,n=0}^{N-1} \exp[i\frac{2\pi mn}{N}] \ketbra{m}{n} \, \text{,}
% \end{equation}
% while in the frequency eigenbasis
% \begin{equation}
%   F^{\dagger} = \frac{1}{\sqrt{N}} \sum_{\omega,\eta=0}^{N-1} \exp[-i\frac{2\pi \omega\eta}{N}] \ketbra{\omega}{\eta} \, \text{,}
% \end{equation}
% with $N$ being the finite dimension of the Hilbert space.

It holds
\begin{gather}\label{eq:DFT:deltas}
  \delta\Omega \delta T = \frac{2\pi}{N} \, \text{;} \quad
  \Delta\Omega \Delta T = 2\pi N \, \text{.}
\end{gather}

The discrete reformulation\footnote{
  See e.g.
  \cite[Ch. ``The Discrete Fourier Transform'']{Oppenheim:Int1,Oppenheim:Int3};
  \cite[Ch. 5]{ProakisManolakis}.
}
of \eqref{eq:fourier_transform:def} and \eqref{eq:inverse_fourier_transform:def}
is
\begin{equation}\label{eq:DFT:def}
  \phi_n \eqbydef \phi(\omega_n) = \frac{1}{\sqrt{N}} \sum_{m=0}^{N-1} \psi_m \E^{-\iu n m 2 \pi / N}
\end{equation}
and, respectively,
\begin{equation}\label{eq:IDFT:def}
  \psi_m \eqbydef \psi(t_m) = \frac{1}{\sqrt{N}} \sum_{n=0}^{N-1} \phi_n \E^{\iu n m 2 \pi / N} \text{.}
\end{equation}
Eq.~\eqref{eq:DFT:def} is the definition of the \term{Discrete Fourier Transform} (DFT).
Eq.~\eqref{eq:IDFT:def} defines the \term{Inverse Discrete Fourier Transform} (IDFT).
The factor $\frac{1}{\sqrt{N}}$ is chosen to guarantee \emph{unitarity} of the transformation:
$\sum_{n=0}^{N-1} \abs{\psi_n}^2 = \sum_{n=0}^{N-1} \abs{\phi_n}^2$.

From eqs.~\eqref{eq:DFT:def} and~\eqref{eq:IDFT:def} one can identify the
\term{Discrete Fourier Matrix}, with elements $F_{mn}$
(and its inverse,
with elements $F_{mn}^{\dagger}$):
\begin{align}
  F_{mn}            &= \frac{1}{\sqrt{N}} \E^{-\iu n m 2 \pi / N} \;\text{;} &
  F_{mn}^{\dagger}  &= \frac{1}{\sqrt{N}} \E^{ \iu n m 2 \pi / N}
  \text{.}
\end{align}

The (discrete) Fourier transform can be used to transform the time representation
of a vector $\ket{\psi}_T \in \hilb{H}_T$ into its angular frequency representation.
Namely, the sequence\footnote{
  We will omit the subscript ``${}_{T}$'' from the following equations
  when it is assumed ---and obvious, by the context and the other symbols used---
  that we are operating in the subspace $\hilb{H}_T$
  of the Page--Wootters model.
}
$\left\{\braket{t_n}{\psi}\right\}_{n=0, \dots, N-1}$ into
$\left\{\braket{\omega_m}{\psi}\right\}_{m=0, \dots, N-1}$:
\begin{equation}\label{eq:DFT:chrepr}  % TODO: inverse too?
  \braket{\omega_{m}}{\psi} = \sum_n F_{mn} \braket{t_n}{\psi} \text{.}
\end{equation}
It is convenient to write down the ``complex conjugate'' of eq.~\eqref{eq:DFT:chrepr} as well:
\begin{equation}\label{eq:DFT:chrepr:cconj}  % TODO: inverse too?
  \braket{\psi}{\omega_{m}} = \sum_n F_{mn}^{\dagger} \braket{\psi}{t_n}
\end{equation}
(recalling that $F$ is symmetric hence
$F_{mn}^{*} = F_{mn}^{\dagger}$,
$\forall m,n \in \setof{0, \dots, N-1}$).

As $\ket{\psi}$ (respectively: $\bra{\psi}$) is a generic ket (or bra) of the Hilbert space,
from eqs.~\eqref{eq:DFT:chrepr:cconj} and~\eqref{eq:DFT:chrepr:cconj} it follows:
\begin{align}
  \label{eq:DFT:bra}  \bra{\omega_{m}} &= \sum_n F_{mn}           \bra{t_n} \,\text{,}  \\
  \label{eq:DFT:ket}  \ket{\omega_{m}} &= \sum_n F_{mn}^{\dagger} \ket{t_n} \,\text{.}
\end{align}
This shows that the Fourier matrix can be used, not only to transform the components
of a vector from one eigenbasis to the canonically conjugate eigenbasis,
but also to obtain the eigenvectors themselves (of the canonically conjugate operator).

If we introduce the \term{Fourier operator}
$\displaystyle \op{F} \eqbydef \sum_{m,n=0}^{N-1} F_{mn} \ketbra{t_m}{t_n}$,
eq.~\eqref{eq:DFT:bra} can be reformulated as
\begin{equation}
  \label{eq:DFO:bra}  \bra{\omega_{m}} = \bra{t_m} \op{F} \text{.}
\end{equation}
Similarly, eq.~\eqref{eq:DFT:ket} can be expressed as
\begin{equation}
  \label{eq:DFO:ket}  \ket{\omega_{m}} = \op{F}^{\dagger} \ket{t_m} \text{.}
\end{equation}
%
Finally, we use the spectral decomposition of $\op{\Omega}$,
the equations \eqref{eq:DFO:bra} and~\eqref{eq:DFO:ket} above,
and the fact that $\omega_{m} = \frac{2\pi}{N(\delta{T})^{2}} t_{m}$
---from eqs.~\eqref{eq:DFT:t_spectrum}, \eqref{eq:DFT:omega_spectrum} and~\eqref{eq:DFT:deltas}---
to obtain:
\begin{multline}\label{eq:DFT:OmegaFTF}
  \op{\Omega} = \sum_{m} \omega_{m} \ketbra{\omega_{m}} =
  \sum_{m} \omega_{m} \op{F}^{\dagger} \ketbra{t_{m}} \op{F} =
  \\
  \frac{2\pi}{N(\delta{T})^{2}} \op{F}^{\dagger} \left(\sum_{m}t_{m}\ketbra{t_{m}}\right) \op{F} =
  \\
  \frac{2\pi}{N(\delta{T})^{2}} \op{F}^{\dagger} \op{T} \op{F} =
  \frac{2\pi N}{\qty(\Delta T)^2} \op{F}^{\dagger} \op{T} \op{F}
  \text{.}
\end{multline}

In conclusion:
\begin{gather}
  \label{eq:SI_Fourier:Omega}
    \op{\Omega} =
      \frac{2\pi}{N(\delta T)^2}          \op{F}^{\dagger} \op{T} \op{F} =
      \frac{2\pi N}{\qty(\Delta T)^2}     \op{F}^{\dagger} \op{T} \op{F}
      \, \text{;}
      \\
  \label{eq:SI_Fourier:T}
    \op{T} =
      \frac{2\pi}{N(\delta\Omega)^2}      \op{F} \op{\Omega} \op{F}^{\dagger} = 
      \frac{2\pi N}{\qty(\Delta\Omega)^2} \op{F} \op{\Omega} \op{F}^{\dagger}
      \, \text{;}
\end{gather}
where eq.~\eqref{eq:SI_Fourier:T} is obtained by
taking the ``conjugate'' of all the computation
from eq.~\eqref{eq:DFT:chrepr} to \eqref{eq:DFT:OmegaFTF}.

Please note eqs.~\eqref{eq:SI_Fourier:Omega} and~\eqref{eq:SI_Fourier:Omega}
differs from e.g. the results of \cite{FiniteHilb}.
This is due to the use of
``natural'' units,
in the sense that
both time and angular frequency
have (and only have) integer eigenvalues
and they are adimensional
i.e. $t_n = \omega_n = 0, 1, \dots, N-1$
(the paper, in actual fact, defines ``position'' $\op{x}$ and ``momentum'' $\op{p}$, but in a similar fashion).
Moreover, the roles of the Fourier transform and its inverse are swapped therein.
Other authors may                                                                            adopt different conventions and units.
The conventions chosen in this Section
are compatible with the computational software libraries\footnote{
  See \cite{comp:scipy} and, in particular,
  \url{https://docs.scipy.org/doc/scipy/reference/generated/scipy.linalg.dft.html}.
}
used later in this work.

\subsection{Uncertainty in finite-dimensional systems}\label{sec:finite_uncertainty}
\citereset
For canonical pairs of operators with a continuous, unbounded spectrum i.e.
$\op{q}$ and $\op{p} \eqbydef -i\hbar\op{\partial}_{q}$,
it is in general straightforward to prove that
$\qty[\op{q}, \op{p}] = i\hbar$ and therefore
$\Delta q \Delta p \geq {\frac{1}{2} \qty|\ev{\qty[\op{q}, \op{p}]}| = \frac{\hbar}{2}}$.

In finite $d$-dimensional Hilbert spaces, the above commutation relation doesn't hold
in general, and is less essential.
Canonically conjugate operators are related
through the (discrete) Fourier transform ($\op{p} = F\op{q}F^{\dagger}$)
rather then differentiation,
and uncertainty relations are based on
the properties of Fourier transformation
rather than commutation relations.

Particularly, the entropic uncertainty relation holds
(\cite[\s 2.4]{FiniteHilb}; \cite{Deutsch:Uncertainty}):
\begin{equation}
  S_q + S_p \geq \ln d
\end{equation}
where the quantities $S_q$ and $S_p$ are the \term{R\'enyi}-\term{Shannon} entropies
\parencite[\s {\it I}.A]{Wehner:Uncertainty}; in this case:
\begin{align}
  S_q &= -\sum_n \qty|\lambda_n |^2  \ln\qty|\lambda_n|^2 \\
  S_p &= -\sum_n \qty|\mu_n     |^2  \ln\qty|\mu_n    |^2
  \,\text{,}
\end{align}
with $\lambda_n$ and $\mu_n$ being the discrete ``wave functions'' in the
(generalized) position and momentum basis.
