\section{Finite-dimensional systems}\label{sec:finite-quantum}\label{sec:pw:theory_last}
\epigraph{\textelp{} discreteness in the world is simply the Fourier transform of compactness.}{%
  \emph{Physics and the Integers} \parencite{Tong_Integers}%
}

\noindent
An experimental illustration of the Page--Wootters model will be discussed in Sec. \ref{sec:pw:qubit}.
It is based on a clock
(defined through the corresponding frequency operator)
which has two discrete levels only.

Finite-dimensional systems are of interest, phenomenologically,
and in general easier to implement,
both experimentally and numerically.
However,
finite-dimension Hilbert spaces
present diffilcuties in identifying canonically conjugate
pairs, and one cannot use the same explicit formalism
that would be used for observables with a continuous spectrum.
Moreover, operators satisfying a canonical
commutation relation such as
\begin{equation}\label{eq:canonical_commutation_in_time}
  [\op{T}, \op{\Omega}] = i
\end{equation}
cannot be both bounded, therefore cannot exist in a finite-dimensional space.

This problem was studied in detail in
\cite{FiniteHilb}. It was shown therein
that statisfying the canonical commutation relation 
is not essential to build operators representing physical observables
with the same role of position and momentum.\footnote{
  Or $T$ and $\hbar\Omega$
  in $\hilb{H}_T$ for a finite-dimensional Page and Wootters model.
}

Discrete, bounded, position-like and momentum-like operators can be obtained from
each other via
the \term{finite} or \term{dicrete} \term{Fourier transform} (DFT).
In our case, we are particularly interested in relating the
time operator $\op{T}$ and the ``energy'' operator $-\hbar\op{\Omega}$
in $\hilb{H}_T$ ---via the angular frequency operator $\op{\Omega}$ which, in the continuous limit, would satisfy
eq.~\eqref{eq:canonical_commutation_in_time} exactly.

\subsection{Discrete Fourier transform}

We will add some more details to the analysis of \citereset\cite{FiniteHilb} with
additional emphasis on sign and scaling conventions which will affect
our computation.

Let us start with a continuous system where the Fourier transform $\phi \eqbydef \mathcal{F}\psi$ of
an integrable function $\psi: \mathbb{R} \to \mathbb{C}$ is defined as
\begin{equation}\label{eq:fourier_transform:def}
  \phi(\omega) = \left(\mathcal{F}\psi\right) (\omega) =
    \frac{1}{\sqrt{2\pi}} \int \dd{t} \psi(t) \E^{- \iu t \omega} \text{,} 
\end{equation}
and the well known \term{inversion} property
\begin{equation}\label{eq:fourier_antitransform:def}
  \psi(t) = \left(\mathcal{F}^{-1} \phi \right) (t) =
    \frac{1}{\sqrt{2\pi}} \int \dd{\omega} \phi(\omega) \E^{\iu t \omega} 
\end{equation}
can be can be proven (see e.g. \cite{Folland:Fourier}).

Eq.~\eqref{eq:fourier_antitransform:def} can be interpreted
as a function of time
expressed as a linear superposition
of ``pure frequencies'' $\E^{\iu \omega t}$,
with coefficients given by the Fourier transform $\phi(\omega)$.

In a discrete system, a finite interval $(0, \Delta{T})$ is considered,
equally divided in $N$ sub-intervals. If we set $\delta{T} = \frac{\Delta{T}}{N}$,
$N$ discrete points in time can be considered
\begin{equation}
  t_n \in \setof{0,\,\hdots,\,(N-1)\delta{T}} \text{.}
\end{equation} 


% %\citereset
% It holds\footnote{
%   Contrary to what indicated in eq. (8) of \cite{FiniteHilb},
%   it can be easily verified that,
%   if $ \op{p} = F x F^{\dagger} $,
%   the correct inverse relation is
%   $ x = F^{\dagger} p F$ and not $ -x = F p F^{\dagger} $.
% } \parencite{FiniteHilb}:
% \begin{gather}\label{eq:FourierCanonicalRelations}
%   \op{\Omega}_{\mathrm{nat.}} = F \op{T}_{\mathrm{nat.}} F^{\dagger}\text{;} \quad
%   \op{T}_{\mathrm{nat.}} = F^{\dagger} \op{\Omega}_{\mathrm{nat.}} F
% \end{gather}
% where, in the ``position'' (or \emph{time}) finite eigenbasis,
% \begin{equation}
%   F = \frac{1}{\sqrt{N}} \sum_{m,n=0}^{N-1} \exp[i\frac{2\pi mn}{N}] \ketbra{m}{n} \, \text{,}
% \end{equation}
% while in the frequency eigenbasis
% \begin{equation}
%   F^{\dagger} = \frac{1}{\sqrt{N}} \sum_{\omega,\eta=0}^{N-1} \exp[-i\frac{2\pi \omega\eta}{N}] \ketbra{\omega}{\eta} \, \text{,}
% \end{equation}
% with $N$ being the finite dimension of the Hilbert space.

\noindent{\color{red} \rule{\linewidth}{1pt}}  % 'TODO'

Please note the \eqref{eq:FourierCanonicalRelations} is valid in ``natural'' units,
in the sense that
both ``time'' and ``frequency'' 
always have (and only have) integer eigenvalues $\setof{0, 1, \dots, N-1}$.
The operator $\op{T}_{\mathrm{nat.}}$ and $\op{\Omega}_{\mathrm{nat.}}$ are defined indeed as
$\op{T}_{\mathrm{nat.}} = \sum_{n=0}^{N} n \ketbra{n}$ and
$\op{\Omega}_{\mathrm{nat.}} = \sum_{\omega=0}^{N} \omega \ketbra{\omega}$
(\citereset\cite{FiniteHilb} defines ``position'' $\op{x}$ and ``momentum'' $\op{p}$ in a similar fashion).

% Please note the \eqref{eq:FourierCanonicalRelations} is valid in normalized (``natural'') units
% where ``time'' and ``frequency'' are in fact respectively
% \term{samples} and \emph{cycles/samples rate},
% in a similar sense as in digital signal processing theory
% \parencite[469, 490]{Carlson}.

More generally, one is interested in studying a time interval
between $t=0$ and an arbitrary real value $\Delta{T}$,
the time interval being split in $N$ sub-intervals of length $\delta{T}$
so that $N\delta{T} = \Delta{T}$.
In other words, $\delta T$
is the size of a ``temporal sample'', or the size of a discrete
step in the clock;
and $\Delta T = N\delta T$ is the range of the clock.
The corresponding time operator $\op{T}$
is expected therefore to have spectrum $\setof{0, \delta{T}, \dots, (N-1)\delta{T}}$,
while the eigenvalues of the angular frequency operator $\op{\Omega}$
are expected to be multiples of
$\frac{2\pi}{\Delta{T}}$, which is
the fundamental
angular frequency for the discrete Fourier analysis.\footnote{
  See
  \cite[chapters
    ``The Discrete Fourier Transform'' and
    ``Fourier Analysis of Signals Using the Discrete Fourier Transform'']{Oppenheim:Int1,Oppenheim:Int3}.
}
The spectrum of $\op{\Omega}$ would then be $\setof{0, \frac{2\pi}{\Delta{T}}, \dots, \frac{2\pi(N-1)}{\Delta{T}}}$.
By defining $\delta{\Omega} = \frac{2\pi}{T}$,
one can easily verify that $\frac{\op{T}}{\delta{T}}$ and $\frac{\op{\Omega}}{\delta{\Omega}}$
have the same property of $\op{T}_{\mathrm{nat.}}$ and $\op{\Omega}_{\mathrm{nat.}}$.  
By making such substitution directly in eq.~\eqref{eq:FourierCanonicalRelations},
with some simple algebra, it follows:
\begin{gather}
  \label{eq:SI_Fourier:Omega}
    \op{\Omega} = \frac{2\pi}{N(\delta T)^2} F \op{T} F^{\dagger} = \frac{2\pi N}{\qty(\Delta T)^2} F \op{T} F^{\dagger} \\
  \label{eq:SI_Fourier:T}
    \op{T} = \frac{2\pi}{N(\delta\Omega)^2} F^{\dagger} \op{\Omega} F = \frac{2\pi N}{\qty(\Delta\Omega)^2} F^{\dagger} \op{\Omega} F
  \, \text{.}
\end{gather}
% which is essentially the rescaling property of the Fourier transform
% \parencite[Ch. ``Fast Fourier Transform'']{NRC}.
The above ``normalization'' is the same that will be used
for numerical computation from Sec.~\ref{sec:pw:apps_first} onwards.

It holds
\begin{gather}
  \delta\Omega \delta T = \frac{2\pi}{N} \, \text{;} \quad
  \Delta\Omega \Delta T = 2\pi N \, \text{.}
\end{gather}

A benefit of finite-dimensional systems is the potential implementation on a finite array of
qubits in a quantum computer. The use of Discrete Fourier Transform extends the overlap
with technology and engineering to the domain of signal processing \parencite{FiniteHilb}.
In \emph{ordinary} quantum mechanics, the Fourier transform (discrete or continuous)
is generally used
to associate wavefunctions in position and momendum space
(whereas time and frequency are \emph{not} operators),
while in communication engineering it is used to convert signals
from the time to the frequency domain and vice versa.
Thanks to the introduction of the Hilbert space $\hilb{H}_T$,
the interpretation in terms of time and frequency
(or time and energy, up to a factor $\hbar$)
is applicable to quantum theory as well, not only formally
i.e. not in the sense of a mere operation among (``classical'') parameters;
but in the sense of conversion between representations of the
same quantum state vector with respect to different eigenbasis,
in full analogy with position and momentum in $\hilb{H}_S$.

\subsection{Uncertainty in finite-dimensional systems}\label{sec:finite_uncertainty}
\citereset
For canonical pairs of operators with a continuous, unbounded spectrum i.e.
$\op{q}$ and $\op{p} \eqbydef -i\hbar\op{\partial}_{q}$,
it is in general straightforward to prove that
$\qty[\op{q}, \op{p}] = i\hbar$ and therefore
$\Delta q \Delta p \geq {\frac{1}{2} \qty|\ev{\qty[\op{q}, \op{p}]}| = \frac{\hbar}{2}}$.

In finite $d$-dimensional Hilbert spaces, the above commutation relation doesn't hold
in general, and is less essential.
Canonically conjugate operators are related
through the (discrete) Fourier transform ($\op{p} = F\op{q}F^{\dagger}$)
rather then differentiation,
and uncertainty relations are based on
the properties of Fourier transformation
rather than commutation relations.

Particularly, the entropic uncertainty relation holds
(\cite[\s 2.4]{FiniteHilb}; \cite{Deutsch:Uncertainty}):
\begin{equation}
  S_q + S_p \geq \ln d
\end{equation}
where the quantities $S_q$ and $S_p$ are the \term{R\'enyi}-\term{Shannon} entropies
\parencite[\s {\it I}.A]{Wehner:Uncertainty}; in this case:
\begin{align}
  S_q &= -\sum_n \qty|\lambda_n |^2  \ln\qty|\lambda_n|^2 \\
  S_p &= -\sum_n \qty|\mu_n     |^2  \ln\qty|\mu_n    |^2
  \,\text{,}
\end{align}
with $\lambda_n$ and $\mu_n$ being the discrete ``wave functions'' in the
(generalized) position and momentum basis.
