\section{Detector model}

\cite{TQM2} (Kijowski and Detector) is cited and summarized well in
\cite{Halliwell_Detector}.

\iftodo

\section{Can a POVM on a system, if then we see it as part of a bipartite one,
equivalent to a PVM on the other, entangled, system?}

TODO: backflow effect in both models.

This will show an equivalence of models based on POVM with the Page and Wootters...?

Well, yes.

From \cite{PreskillNotes}, Ch.3 
\begin{quotation}
We have seen that
a pure state of the bipartite system AB may behave like a mixed state
when we observe subsystem A alone, and that an orthogonal measurement
of the bipartite system can realize a (nonorthogonal) POVM on A alone.
\end{quotation}

and

\begin{quotation}
A POVM in $H_A$ can be realized as a unitary transformation on the tensor
product $H_A \otimes H_B$, followed by an orthogonal measurement in $H_B$.
\end{quotation}

The same chapter talks about quantum operations, quantum channels and Kraus opertators.

We might want to look at exponential decay from \url{https://arxiv.org/abs/1704.07236},
then compare with exponential decay with P and W using Lloyd Giovannetti and Maccone (ref).

\subsection{Purification}

See https://arxiv.org/pdf/quant-ph/0512125.pdf, P-W time as a purifying ancilla
of the (Kijowski?) time.

\subsection{4-partite universe?}
\begin{itemize}
  \item{The system being measured/detected}
  \item{The Ruschhaupt detector --- which does not measure time, but whose detection happens at a certain time}
  \item{The Page and Wootters clock, entangled with the system and/or the detector}
  \item{The rest of the Universe, aka the Environment, aka the Termal Bath or Reservoir}
\end{itemize}

Can any of the above be identified? If the lab is isolated enough,
the detector is the only macro object and can act as a Universe/bath/environment/reservoir\dots?

\fi