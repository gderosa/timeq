\section{Uncertainty relations in $\pwspace$}\label{sec:pw:uncertainty}

\subsection{For unitary evolutions -- general considerations}

In a continuous-time Page--Wootters universe,
one may observe that,
in $\hilb{H}_T$ and in the ``time representation'',
it is $E \eqbydef -\hbar\op{\Omega} \repr i\hbar\partial_{t}$,
in analogy with $\op{p} \repr -i\hbar\partial_{q}$ in $\hilb{H}_S$,
therefore a time-energy uncertainty relation can be derived
with a proof that is formally identical to the well known
position-momentum uncertainty relation.

However, unitary evolution means that a particle exists with
certainty at all times, i.e. the packet
is infinitely spread in time.
Intuitively,
one expects the standard deviation
of the probability distribution of time observable
to diverge i.e.
$\sigma_T = \infty$.

An extreme case is when $\dket{\Psi}$ encodes what in ordinary quantum mechanics would be the evolution
of an energy eigenstate $\ket{\epsilon_0}_S$, as seen in Sec. \ref{sec:pw:eeigenstate}.
In such case, the system is infinitely spread in time but also infinitely
``concentrated'' in one particular point of the ``frequency domain'',
where the point is $\omega_0 = \epsilon_{0}/\hbar$.
At least intuitively, we do expect the standard deviations to be $\sigma_T = \infty$
and $\sigma_{\Omega} = 0$
(in analogy to an eigenstate of linear momentum in ordinary quantum mechanics).

A physical state in $\pwspace$ is not generally
\emph{separable},
thus
the problem of time and energy relation cannot be immediately reduced to that of
a pure state in $\hilb{H}_T$.

The additional difficulty of dealing with infinites and improper state vectors
motivates the treatment of \emph{proper} states\footnote{
  Introduced in Sec. \ref{sec:properpw}.
}
of $\pwspace$
first, as it will be discussed in Sec.~\ref{sec:for-normalized-elements}.

\subsection{For normalized elements of $\hilb{H}_T \ox \hilb{H}_T$}\label{sec:for-normalized-elements}

A normalized element of $\hilb{H}_T \ox \hilb{H}_T$ allows in principle for a finite value
of the uncertainties $\sigma_{T}$ and $\sigma_{\Omega}$.

Still, it is generally an entangled state in the product space $\pwspace$.
Thus the spatial degrees of freedom should be \emph{traced out},
before deriving a time--energy relation in $\hilb{H}_T$.

With a slight change of notation we express:
\begin{equation}\label{eq:pw:uncertain:schmidt}
  \dket{\Phi} =
    \int \dd{t} \ket{t}_T \ox \ket{\phi(t)}_S =
    \int \dd\omega \ket{\omega}_T \ox \ket{\tilde{\phi}(\omega)}_S \, \text{,}
\end{equation}
with
$\setof{\ket{t}_T}$ and $\setof{\ket{\omega}_T}$
being
orthonormal bases of $\hilb{H}_T$,
and
$\ket{\tilde{\phi}(\omega)}_S$ being the Fourier transform\footnote{
  Fourier transform of a
  vector-valued function
  (vectors in $\hilb{H}_S$),
  similar to what found in \cite{Maccone:Pauli}.
}
of $\ket{\phi(t)}_S$,
neither kets being necessarily normalized, which is in fact a requirement
to satisfy the normalization in $\pwspace$
\[
  \int \dd{t} \braket{\phi(t)}_S =
    \int \dd{\omega} \braket{\tilde{\phi}(\omega)}_S =
    \dbradket{\Phi}{\Phi}_{T \ox S} =
    1
    \text{.}
\]

The vector $\dket{\Phi}$
cannot,
in general, be simply expressed as a tensor product of two pure states
in $\hilb{H}_T$ and $\hilb{H}_S$.
However,
a ``temporal part'' of $\dket{\Phi}$
can be identified as a \emph{mixed} state described
by a (reduced) density operator $\rho_T$ in $\hilb{H}_T$.

To obtain an explicit expression of $\rho_T$,
we use the definitions and results from Chapter~\ref{ch:decohere},
in particular eqs. \eqref{eq:bipartite_expansion} and \eqref{eq:density_A_expand},
replacing the discrete sums there with integrals,
and the index $j$ with the integration variable $t$ (or $\omega$, respectively);
also we identify the indices $i = \mu$ in eq.~\eqref{eq:bipartite_expansion}.

The reduced density operator can so be computed
as the partial trace:
\begin{align}
  \label{eq:ptrace_density_matrix_t}
  \op{\rho}_T = \Tr_S\qty(\dketdbra{\Phi}{\Phi}) &= \int \dd t \norm{\phi(t)}^2_S \ketbra{t}{t}
    \, \text{,}
  \\
  \label{eq:ptrace_density_matrix_omega}
  \op{\rho}_T = \Tr_S\qty(\dketdbra{\Phi}{\Phi}) &= \int \dd \omega \norm{\tilde{\phi}(\omega)}^2_S \ketbra{\omega}{\omega}
    \,\text{.} 
\end{align}
Here the probabilty distributions $\norm{\phi(t)}^2$ and $\norm{\tilde{\phi(\omega)}}^2$
are to be intended in the sense of a mixed state
i.e. probability that a system is in a certain state $\ket{t}_T$
(or $\ket{\omega}_T$,~respectively),
not in the sense of the probability of a measurement outcome on a \emph{known} pure state
(which is what, in its original formulation, the Heisenberg uncertainty principle 
refers to, in relation to position and linear momentum).

Nevertheless, with this notion of probability (which we may indicate as ``classical''\footnote{
  See, for example, \cite[eq.~2.20 and following discussion]{Schlosshauer_Decoherence_book}.
})
the relation $\sigma_T\sigma_{\hbar\Omega} = \hbar \sigma_{\phi} \sigma_{\tilde{\phi}} \geq \frac{\hbar}{2}$
can then be derived from the properties of the Fourier transform.

\subsection{Pure-state approach (in the product space)}\label{sec:pure-state-approach}

Another approach considers $\dket{\Phi}$ as a pure state of $\pwspace$
and the operators $\op{T} \ox \idop_{S}$ and $\hbar\op{\Omega} \ox \idop_{S}$
defined therein.

The same considerations above still hold, in terms of ``infinite history'' versus
time-localized, proper elements of $\pwspace$. 

We use the general Robertson--Schr\"{o}dinger uncertainty relation which yields:
\begin{multline}
  \sigma_T \sigma_E \eqbydef
  \sigma_{T \ox \idop_S} \sigma_{\hbar\Omega \ox \idop_S} \geq
  \frac{1}{2} \abs{\expval{\comm
    {\op{T}\ox\idop_S} {\hbar\op{\Omega}\ox\idop_S}
  }} =
  \\
  {\scriptstyle
    \frac{1}{2} \abs{\expval{
      \qty(\op{T}\ox\idop_S) \qty(\hbar\op{\Omega}\ox\idop_S) -
      \qty(\hbar\op{\Omega}\ox\idop_S) \qty(\op{T}\ox\idop_S)
    }}
  } =
  \frac{\hbar}{2} \abs{\expval{
    \op{T}\op{\Omega}\ox\idop_S - \op{\Omega}\op{T}\ox\idop_S
  }} = \\
  \frac{\hbar}{2} \abs{\expval{
    \comm{\op{T}}{\op{\Omega}}_T \ox \idop_S
  }} =
  \frac{\hbar}{2} \abs{\expval{\comm
    {\op{T}}{\op{\Omega}}
  }_T} =
  \frac{\hbar}{2}
  \,\text{.}
\end{multline}

\subsection{Concluding remarks}

We have shown that a time--``energy'' uncertainty relation,
mirroring the well known position--momentum uncertainty relation,
still holds, if we consider the ``clock energy'' $\op{E} \eqbydef \hbar\op{\Omega}$
in the Page--Wootters model.

In those states of $\pwspace$ where a time evolution actually emerges
(the ``physical states'' as defined in \cite{Lloyd:Time}),
the clock is in a maximally mixed state
(it is maximally entangled with the rest of the universe).

Therefore, it cannot be studied as an independent system
(and the position--momentum uncertainty relation is proved
for pure states in ordinary quantum mechanics); but a
time--energy relation is proved in the sense of probability
\emph{among states} of a density operator.

One might object to the interpretation in terms of (``classical'')
probability distribution, given
that the density matrix representation is
not unique.
However, the density matrices
in \eqref{eq:ptrace_density_matrix_t} and~\eqref{eq:ptrace_density_matrix_omega}
are diagonal with respect to
the orthonormal eigensystems of $\op{T}$ and $\hbar \op{\Omega}$ observables,
so the distributions $\abs{\phi(t)}^2$ and $\abs{\tilde{\phi}(\omega)}^2$
have a particular physical significance.

Classical probability emerges out of entanglement. More generally,
the property of time as a classical external parameter emerges as
a consequence of entanglement of the (quantum) clock
with the rest of the universe, where there's a Schmidt decomposition
that sums over eigenstates of time.

In fact, as we have seen in the last subsection,
an uncertainty relation for the \emph{quantum} probability distributions does hold,
but for the whole
$\textit{clock} + \textit{rest}$ system, which is indeed in a pure state.
It relates $\op{T}\ox\idop_S$ and $\hbar\op{\Omega}\ox\idop_S$
in $\pwspace$.