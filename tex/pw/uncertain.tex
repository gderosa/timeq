\section{Uncertainty}

\subsection*{For unitary evolutions}

In a continuous-time Page--Wootters universe,
one may observe that,
in $\hilb{H}_T$ and in the time representation,
it is $E \eqbydef -\hbar\hat{\Omega} \repr i\hbar\partial_{t}$,
in analogy with $\hat{p} \repr -i\hbar\partial_{q}$ in $\hilb{H}_S$,
therefore a time-energy uncertainty relation can be derived
with a proof that is formally identical to the well known
position-momentum uncertainty relation.

However, unitary evolution means that a particle exists with
certainty at all times, i.e. the packet
is infinitely spread in time.
%
As a function of $t$, $\ket{\psi(t)}_S$ will have a Fourier transform
$\ket{\tilde{\psi}(\omega)}$
so that it will also be
\begin{equation}\label{eq:pwfourier}
  \dket{\Psi} = \int \dd \omega \ket{\omega}_T \ox \ket{\tilde{\psi}(\omega)}_S \, \text{,}
\end{equation}
where $\ket{\tilde{\psi}(\omega)}_S$ is not necessarily normalized for each $\omega$
(and actually it's zero if $\hbar\omega$ does not belong to the energy spectrum of the system).
An extreme case is when $\dket{\Psi}$ encodes what in ordinary quantum mechanics would be the evolution
of an energy eigenstate $\ket{\epsilon}_S$: in such case
$\ket{\tilde{\psi}(\omega)}_S = \delta(\omega - \frac{\epsilon}{\hbar})\ket{\epsilon}_S$
and it is $\sigma_{T} = \infty$ and $\sigma_{\Omega} = 0$.

All other cases of unitary evolution correspond to $\sigma_{T}$
sill being infinite, but $\sigma_{\Omega}$ being finite, therefore
$\sigma_{T}\sigma_{\Omega} = \infty$, which is not particularly interesting:
any possible uncertainty relation would be valid, and $\dket{\Psi}$
is certainly \emph{not} a minimum uncertainty state\dots 

\subsection*{For normalized elements of $\hilb{H}_T \ox \hilb{H}_T$}\label{sec:for-normalized-elements}

A normalized element of $\hilb{H}_T \ox \hilb{H}_T$ allows in principle for a finite value
of the uncertainty $\sigma_{T}\sigma_{\Omega}$.

Still, it is not generally
a separable state,
and thus
the problem of time and energy relation cannot be reduced to that of
a pure state in $\hilb{H}_T$.
It is in general an entangled state in the product space.
Thus the spatial degrees of freedom should be \emph{traced out},
before deriving a time-energy relation.

With a slight change of notation we express:
\begin{equation}
  \dket{\Phi} =
    \int \dd{t} \ket{t}_T \ox \ket{\phi(t)}_S =
    \int \dd\omega \ket{\omega}_T \ox \ket{\tilde{\phi}(\omega)}_S \, \text{,}
\end{equation}
with $\ket{\tilde{\phi}(\omega)}_S$ being the Fourier transform
of $\ket{\phi(t)}_S$,
neither kets being necessarily normalized, which is by the way necessary
to satisfy the normalization in time (frequency)
\[
  \int \dd{t} \braket{\phi(t)}_S =
    \int \dd{\omega} \braket{\tilde{\phi}(\omega)}_S =
    \dbradket{\Phi}{\Phi}_{T \ox S} = 
    1
\]
while of course being $\setof{\ket{t}_T}$ and $\setof{\ket{\omega}_T}$
orthonormal.

Next we use \eqref{eq:density_A_expand}: replace the discrete sum with an integral,
set $i = \mu$ as this is a Schmidt decomposition, and then $j = t$
(or $j = \omega$ respectively).
The reduced density operator can so be computed
as partial trace:
\begin{align}
  \label{eq:ptrace_density_matrix_t}
  \hat{\rho}^T = \Tr_S\qty(\dketdbra{\Phi}{\Phi}) &= \int \dd t \norm{\phi(t)}^2_S \ketbra{t}{t}
    \, \text{,}
  \\
  \label{eq:ptrace_density_matrix_omega}
  \hat{\rho}^T = \Tr_S\qty(\dketdbra{\Phi}{\Phi}) &= \int \dd \omega \norm{\tilde{\phi}(\omega)}^2_S \ketbra{\omega}{\omega}
    \,\text{.} 
\end{align}
Here the probabilty distributions $\norm{\phi}^2$ and $\norm{\tilde{\phi}}^2$
are ``classical'' in the sense of a mixed state.

The relation $\sigma_T\sigma_{\hbar\Omega} = \hbar \sigma_{\phi} \sigma_{\tilde{\phi}} \geq \frac{\hbar}{2}$
can then be derived from the properties of the Fourier transform.\footnote{
  Fourier transform of a
  vector-valued function
  (vectors in $\hilb{H}_S$),
  similar to what found in \cite{Maccone:Pauli}.
}

A packet in $\pwspace$ that is localized in space \emph{and in time} can be interpreted as an \emph{event}.

As opposed to a localized event, an energy eigenstate is infinitely spread in time,
and infinitely concentrated in energy (or frequency).

\subsection*{Pure-state approach (in the product space)}\label{sec:pure-state-approach}

Another approach considers $\dket{\Phi}$ as a pure state of $\pwspace$
and the operators $\hat{T} \ox \idop_{S}$ and $\hbar\hat{\Omega} \ox \idop_{S}$
defined therein.

The same considerations above still hold, in terms of ``infinite history'' versus
time-localized, proper elements of $\pwspace$. 

We use the general Robertson--Schr\"{o}dinger uncertainty relation which yields:
\begin{multline}
  \sigma_T \sigma_E \eqbydef
  \sigma_{T \ox \idop_S} \sigma_{\hbar\Omega \ox \idop_S} \geq
  \frac{1}{2} \abs{\expval{\comm
    {\hat{T}\ox\idop_S} {\hbar\hat{\Omega}\ox\idop_S}
  }} =
  \\
  {\scriptstyle
    \frac{1}{2} \abs{\expval{
      \qty(\hat{T}\ox\idop_S) \qty(\hbar\hat{\Omega}\ox\idop_S) -
      \qty(\hbar\hat{\Omega}\ox\idop_S) \qty(\hat{T}\ox\idop_S)
    }}
  } =
  \frac{\hbar}{2} \abs{\expval{
    \hat{T}\hat{\Omega}\ox\idop_S - \hat{\Omega}\hat{T}\ox\idop_S
  }} = \\
  \frac{\hbar}{2} \abs{\expval{
    \comm{\hat{T}}{\hat{\Omega}}_T \ox \idop_S
  }} =
  \frac{\hbar}{2} \abs{\expval{\comm
    {\hat{T}}{\hat{\Omega}}
  }_T} =
  \frac{\hbar}{2}
  \,\text{.}
\end{multline}

\subsection*{Concluding remarks}

We have shown that a time--``energy'' uncertainty relation,
mirroring the well known position--momentum uncertainty relation,
still holds, if we consider the ``clock energy'' $\hat{E} \eqbydef \hbar\hat{\Omega}$
in the Page--Wootters model.

In those states of $\pwspace$ where a time evolution actually emerges
(the ``physical states'' as defined in \cite{Lloyd:Time}),
the clock is in a maximally mixed state
(it is maximally entangled with the rest of the universe).

Therefore, it cannot be studied as an independent system
(and the position--momentum uncertainty relation is proved
for pure states in ordinary quantum mechanics); but a
time--energy relation is proved in the sense of probability
\emph{among states} of a density operator.

One might object to the interpretation in terms of (``classical'')
probability distribution, given
that the density matrix representation is
not unique.
However, the density matrices
in \eqref{eq:ptrace_density_matrix_t} and~\eqref{eq:ptrace_density_matrix_omega}
are diagonal with respect to
the orthonormal eigensystems of $\hat{T}$ and $\hbar \hat{\Omega}$ observables,
so the distributions $\abs{\phi(t)}^2$ and $\abs{\tilde{\phi}(\omega)}^2$
have a particular physical significance.

Classical probability emerges out of entanglement. More generally,
the property of time as a classical external parameter emerges as
a consequence of entanglement of the (quantum) clock
with the rest of the universe, where there's a Schmidt decomposition
that sums over eigenstates of time.

In fact, as we have seen in the last subsection,
an uncertainty relation for the \emph{quantum} probability distributions does hold,
but for the whole
$\textit{clock} + \textit{rest}$ system, which is indeed in a pure state.
It relates $\hat{T}\ox\idop_S$ and $\hbar\hat{\Omega}\ox\idop_S$
in $\pwspace$.