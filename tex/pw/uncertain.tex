\section{Uncertainty relations in $\pwspace$}\label{sec:pw:uncertainty}

\subsection{For unitary evolutions -- general considerations}
\label{sec:pw:unitary-general}

In a continuous time Page--Wootters universe
---using the same symbols and definitions as in Sec.~\ref{sec:pw:theory_first}---
one may observe that,
in $\hilb{H}_T$ and in the ``time representation'',
it is $\hbar\op{\Omega} \repr -i\hbar\partial_{t}$.

Therefore, in analogy with $\op{p} \repr -i\hbar\partial_{q}$ in $\hilb{H}_S$,
a time--energy uncertainty relation can be derived
with a proof that is formally identical to the well known
position--momentum uncertainty relation.

Here by ``energy'' we mean the Hamiltonian in standard quantum mechanics which,
in Page--Wootters notation,
can be expressed as $\idop_T \ox \op{H}_S$, which is also equal to
$-\hbar\op{\Omega} \ox \idop_S$, in order to statisfy Eqs. \eqref{eq:Wheeler-DeWitt} and \eqref{eq:pwHamiltonian}.
The different sign does not change considerations related to the spread of statistical distributions.
A formal proof will follow in Sec.~\ref{sec:for-normalized-elements}.

However, unitary evolution means that a particle exists with
certainty at all times, i.e. the packet
is infinitely spread in time.
Intuitively,
one expects the standard deviation
of the probability distribution of time observable
to diverge i.e.
$\sigma_T = \infty$.

The additional difficulty of dealing with infinites and improper state vectors
motivates the study of \emph{proper} states\footnote{
  Introduced in Sec. \ref{sec:properpw}.
}
of $\pwspace$
instead, whose uncertainty relations for $T$ and $\hbar\Omega$ will be discussed in Sec.~\ref{sec:for-normalized-elements}.

\subsection{For normalized elements of $\hilb{H}_T \ox \hilb{H}_T$}\label{sec:for-normalized-elements}

A normalized element of $\hilb{H}_T \ox \hilb{H}_T$ allows in principle for a finite value
of the uncertainties $\sigma_{T}$ and $\sigma_{\Omega}$.

A physical state in $\pwspace$ is not generally
\emph{separable},
thus
the problem of time and energy relation cannot be immediately reduced to that of
a pure state in $\hilb{H}_T$.

Still, it is generally an entangled state in the product space $\pwspace$.
We can consider the
operators $\op{T} \ox \idop_{S}$ and $\hbar\op{\Omega} \ox \idop_{S}$
defined therein.

We use the general Robertson--Schr\"{o}dinger uncertainty relation which yields:
\begin{multline}
  \sigma_T \sigma_E \eqbydef
  \sigma_{T \ox \idop_S} \sigma_{\hbar\Omega \ox \idop_S} \geq
  \frac{1}{2} \abs{\expval{\comm
    {\op{T}\ox\idop_S} {\hbar\op{\Omega}\ox\idop_S}
  }} =
  \\
  {\scriptstyle
    \frac{1}{2} \abs{\expval{
      \qty(\op{T}\ox\idop_S) \qty(\hbar\op{\Omega}\ox\idop_S) -
      \qty(\hbar\op{\Omega}\ox\idop_S) \qty(\op{T}\ox\idop_S)
    }}
  } =
  \frac{\hbar}{2} \abs{\expval{
    \op{T}\op{\Omega}\ox\idop_S - \op{\Omega}\op{T}\ox\idop_S
  }} = \\
  \frac{\hbar}{2} \abs{\expval{
    \comm{\op{T}}{\op{\Omega}}_T \ox \idop_S
  }} =
  \frac{\hbar}{2} \abs{\expval{\comm
    {\op{T}}{\op{\Omega}}
  }_T} =
  \frac{\hbar}{2}
  \,\text{.}
\end{multline}

% \subsection{Concluding remarks}

% We have shown that a time--``energy'' uncertainty relation,
% mirroring the well known position--momentum uncertainty relation,
% still holds, for normalized states,
% if we consider the ``clock energy'' $\op{H}_T \eqbydef \hbar\op{\Omega}$
% in the Page--Wootters model.

% In those states of $\pwspace$ where a time evolution actually emerges
% (the ``physical states'' as defined in \cite{Lloyd:Time}),
% the clock is in a maximally mixed state
% (it is maximally entangled with the rest of the universe).

% Therefore, it cannot be studied as an independent system
% (and the position--momentum uncertainty relation is proved
% for pure states in ordinary quantum mechanics); but a
% time--energy relation is proved in the sense of probability
% \emph{among states} of a density operator.

% One might object to the interpretation in terms of (``classical'')
% probability distribution, given
% that the density matrix representation is
% not unique.
% However, the density matrices
% in \eqref{eq:ptrace_density_matrix_t} and~\eqref{eq:ptrace_density_matrix_omega}
% are diagonal with respect to
% the orthonormal eigensystems of $\op{T}$ and $\hbar \op{\Omega}$ observables,
% so the distributions $\abs{\phi(t)}^2$ and $\abs{\tilde{\phi}(\omega)}^2$
% have a particular physical significance.

% In fact, as we have seen in the Sec.~\ref{sec:pure-state-approach},
% an uncertainty relation
% \emph{related to the probability of a measurement outcome}
% does hold,
% but for the whole
% $\textit{clock} + \textit{rest}$ system, which is indeed in a pure state.
% Namely, it relates $\op{T}\ox\idop_S$ and $\hbar\op{\Omega}\ox\idop_S$
% in the overall space $\pwspace$.
