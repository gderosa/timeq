\section{Development of the Formalism}

About three decades after the first publication by
Don Page and William Wootters \parencite{PageWootters},
the model was further developed,
in particular by Lloyd, Giovannetti and Maccone,
with emphasis on addressing technical issues which have been raised meanwhile
\parencite{Lloyd:Time}.

In this formulation,
in addition to the ordinary Hilbert space of the system,
an extra Hilbert space $\mathcal{H}_T$ is considered,
where time is an observable
represented by a self-adjoint operator.
The ordinary Hilbert space can be labeled $\mathcal{H}_S$;
and we consider the tensor product space $\mathcal{H}_T \otimes \mathcal{H}_S$ as
the space in which both time and position are observables, and they act as
$T \otimes \idop_S$ and $\idop_T \otimes X$
respectively.

In principle, an appropriate system (a ``clock'') can be identified in such a way
to be described in $\mathcal{H}_T$, while $\mathcal{H}_S$ describes
the quantum states of \emph{the rest of the universe} \parencite{Marletto:Evolution}.

As explained in \cite{Lloyd:Time, Maccone:Pauli}, the overall Hamiltonian,
encompassing both position and time as observables, is given by
\begin{equation}\label{eq:pwHamiltonian}
  \op{\mathbb{J}} = \hbar\op{\Omega}\ox\idop_S + \idop_T\ox\op{H}_S \,\text{,}
\end{equation}
where $\hbar\op{\Omega}$ is the canonically conjugate to the operator $\op{T}$ in $\hilb{H}_T$.
Also, the \term{Wheeler-DeWitt equation} holds:
\begin{equation}\label{eq:Wheeler-DeWitt}
  \op{\mathbb{J}}\dket{\Psi} = 0 \,\text{,}
\end{equation}
describing a \emph{static} universe, where evolution is only
in terms of relations between parts of a multipartite system
(a ``clock'' and ``the rest'').

% Please note the special notation $\dket{\Psi}$ (double angle bracket),
% to indicate a state vector in the ``larger'' product space $\pwspace$,
% as opposed to vectors belonging to $\hilb{H}_S$ (or $\hilb{H}_T$) only.

The ``conventional'' state $\ket{\psi(t)}_S$ in $\hilb{H}_S$
can be obtained from $\dket{\Psi}$ via partial inner product\footnote{
  For the partial inner product,
  see Definition \ref{def:pBra} and \ref{def:pKet},
  and \cite[\s 1.3.3]{QMT_Jacobs}.
}
with a time eigenstate $\prescript{}{T}{\bra{t}}$:
\begin{equation*}
  \ket{\psi(t)}_S = \prescript{}{T}{\bradket{t}{\Psi}} \, \text{.}
\end{equation*}
The function $ t \rightarrow \ket{\psi(t)}_S \; $ is the
``wavefunction in time representation'', in analogy
with the wavefunction in position representation of standard quantum mechanics.
The state vector $\dket{\Psi}$ is univocally identified by the ``coefficients'' $\ket{\psi(t)}_S$
with respect to the basis $\setof{\ket{t}_T}$:
\begin{equation*}
  \dket{\Psi} = \int \dd{t} \ket{t}_T \ox \ket{\psi(t)}_S \, \text{,}
\end{equation*}
somewhat similar to the well known expression $\ket{\psi}_S = \int \dd{x} \psi_{S}(x) \ket{x}_S$
which defines the position representation.

Under this time representation (or $T$ representation), the operator $\hbar\Omega$,
canonically conjugate to the time operator T, is expressed as $-i\hbar\pdv{t}$,
and the commutation relation $\comm{t, -i\hbar\pdv{t}}{} = i\hbar$ can be proven
immediately, similarly to the well known position-momentum relation.

%% \hbar\Omega is -E ... and \Omega \repr -i t-derivative ... and [T, \Omega] = i .

Using the $T$ representation in $\hilb{H}_T$,
and comparing \eqref{eq:pwHamiltonian} and \eqref{eq:Wheeler-DeWitt}:
\begin{equation}\label{eq:schrod_from_pw}
  0 = \qty(\hbar\op{\Omega}\ox\idop_S + \idop_{T}\ox\op{H}_S)\dket{\Psi}
    \repr -i\hbar\pdv{t}\ket{\psi(t)}_{S} + \op{H}_S\ket{\psi(t)}_{S}
    \,\text{,}
\end{equation}
we recover the usual form of the Schr\"{o}dinger equation in $H_S$
(a detailed proof can be found in \cite[709--710]{Wootters:Loyola}).

Here $\op{\Omega}$ can be seen as a ``frequency operator''
represented as $-i\pdv{t}$ and having as eigenfunctions
those functions evolving in time with a phase factor $e^{i \omega t}$ only.

Canonical commutation relation should hold and can be easily verified
between $\op{t}$ and $\op{\Omega}$
i.e. $[\op{t}, \op{\Omega}] = i$,
therefore $\hbar\op{\Omega}$ can be seen as the ``linear momentum''
in the Hilbert space of time.

From another point of view, $\hbar\op{\Omega}$ is the ``Hamiltonian'' of $\hilb{H}_T$,
in that it plays a similar role of $H_S$ in the construction of
$\op{\mathbb{J}}$ in \eqref{eq:pwHamiltonian}. Some authors (e.g. in \citereset\cite{Wootters:Loyola})
use the symbol $H_C$ (where the supscript ``$C$'' stands for ``clock'') in place of $\hbar\Omega$, to emphasize this.

Such analogies among operators in $\hilb{H}_T$ and $\hilb{H}_S$ are summarized in Table \ref{tbl:op_comparison_pw}.

{
  %% https://tex.stackexchange.com/a/232874
  %% https://tex.stackexchange.com/a/2836
  \begin{table}
    \centering
    \begin{tabular}{l|l|l}
      & \multicolumn{1}{c|}{$\hilb{H}_T$}   & \multicolumn{1}{|c}{$\hilb{H}_S$}   \\
      \hline
      \multirow{2}{11em}{Canonical commutation relation} 
      & $\op{t}$                            & $\op{x}$                            \\
      %
      & $\hbar\op{\Omega}$                  & $\op{p}$                            \\
      \hline
      ``Energy operator''
      & $\hbar\op{\Omega}$ (or ``$H_C$'')   & $\op{H_S}$
    \end{tabular}
    {\caption{
      Analogies between observables in the two Hilbert spaces.
    }\label{tbl:op_comparison_pw}}
  \end{table}
}

% It would be interesting to a study relativistic extension of the
% Page and Wootters model that allows, for example, the derivation of the Klein-Gordon
% equation, thus eliminating the asimmetry between
% $\hilb{H}_T$ and $\hilb{H}_S$ (and momentum and energy as well).

In the bipartite universe, physical kets $\dket{\Psi}$ can have a Schmidt decomposition
made up of
eigenstates of $\op{\Omega}$ in $\hilb{H}_T$
entangled, respectively, with
eigenstates of $\op{H}_S$ in $\hilb{H}_S$;
or eigenstates of $\op{t}$ in $\hilb{H}_T$
entangled with time-evolved spatial states $\ket{\psi(t)}_S$ in $\hilb{H}_S$
(according to the evolution that is embedded in $\op{\mathbb{J}}$):
\begin{equation}
  \dket{\Psi} = \int d\omega \, \mu(\omega) \ket{\omega}_{T}\ox\ket{\phi(\omega)}_{S} = \int dt \, \ket{t}_{T} \ox \ket{\psi(t)}_{S}\,\text{.}
\end{equation}

In principle, different and less obvious decompositions are possible for other purposes.






