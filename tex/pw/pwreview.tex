\section{Page--Wootters basics and recent developements}

\subsection{The model}

Circa three decades after the first publication by
Don Page and William Wootters \parencite{PageWootters},
the model was further developed
in particular by Lloyd, Giovannetti and Maccone,
with emphasis on addressing technical issues which have been raised meanwhile
\parencite{Lloyd:Time}.
In this formulation,
in addition to the ordinary Hilbert space of the system,
an extra Hilbert space $\mathcal{H}_T$ is considered,
where time is an observable
represented by a self-adjoint operator
whose properties are similar to the ones of position
in ordinary quantum mechanics.

In this language, the ordinary Hilbert space can be labeled $\mathcal{H}_S$;
and we consider the tensor product space $\mathcal{H}_T \otimes \mathcal{H}_S$ as
the space in which both time and position are observables, and they act as
$T \otimes \idop_S$ and $\idop_T \otimes X$
respectively.

In principle, an appropriate system (a ``clock'') can be identified in such a way
to be described in $\mathcal{H}_T$, while $\mathcal{H}_S$ describes
the quantum states of \emph{the rest of the universe} \parencite{Marletto:Evolution}.

As explained in \cite{Lloyd:Time, Maccone:Pauli}, the overall Hamiltonian,
encompassing both position and time as observables, is given by
\begin{equation}\label{eq:pwHamiltonian}
  \op{\mathbb{J}} = \hbar\op{\Omega}\ox\idop_S + \idop_T\ox\op{H}_S \,\text{,}
\end{equation}
where $\hbar\op{\Omega}$ is the canonically conjugate of $\op{T}$ in $\hilb{H}_T$.
Also, the \term{Wheeler-DeWitt equation} holds:
\begin{equation}\label{eq:Wheeler-DeWitt}
  \op{\mathbb{J}}\dket{\Psi} = 0 \,\text{,}
\end{equation}
describing a \emph{static} universe, where evolution is only
in terms of relations between parts of a multipartite system
(a ``clock'' and ``the rest'').

Please note the special notation $\dket{\Psi}$ (double angle bracket),
to indicate a state vector in the ``larger'' product space $\pwspace$,
as opposed to vectors belonging to $\hilb{H}_S$ (or $\hilb{H}_T$) only.

TODO FROM HERE

Using the $T$ representation in $\hilb{H}_T$,
and comparing \eqref{eq:pwHamiltonian} and \eqref{eq:Wheeler-DeWitt}:
\begin{equation}\label{eq:schrod_from_pw}
  0 = \qty(\hbar\op{\Omega}\ox\idop_S + \idop_{T}\ox\op{H}_S)\dket{\Psi}
    \repr -i\hbar\pdv{t}\ket{\psi(t)}_{S} + \op{H}_S\ket{\psi(t)}_{S}
    \,\text{,}
\end{equation}
we recover the usual form of the Schr\"{o}dinger eqaution in $H_S$.
Details of this derivation are in the reference aforementioned.

Here $\op{\Omega}$ can be seen as a ``frequency operator''
represented as $-i\pdv{t}$ and having as eigenfunctions
those functions evolving in time with a phase factor $e^{i \omega t}$ only.

Canonical commutation relation holds and can be easily verified
between $\op{t}$ and $\op{\Omega}$
i.e. $[\op{t}, \op{\Omega}] = i$,
therefore $\hbar\op{\Omega}$ can be seen as the ``linear momentum''
in the Hilbert space of time.

From another point of view, $\hbar\op{\Omega}$ is the ``hamiltonian'' of $\hilb{H}_T$,
in that it plays a similar role of $H_S$ in the construction of
$\op{\mathbb{J}}$ in \eqref{eq:pwHamiltonian}. This dual role doesn't hold
for $H_S$. 

This ambiguity is related to the a\-sym\-me\-try of space and time in
non\-re\-la\-ti\-vi\-stic
mechanics and can be expressively synthesized in the below:
{
  %% https://tex.stackexchange.com/a/232874
  %% https://tex.stackexchange.com/a/2836
  \begin{table}[h!]
    \parbox{.45\linewidth}{
      \centering
      \begin{tabular}{c|c}
        $\hilb{H}_T$        & $\hilb{H}_S$  \\
        \hline
        \hline
        $\op{t}$           & $\op{x}$     \\
        \hline
        $\hbar\op{\Omega}$ & $\op{p}$     \\
        \hline
        $?$                 & $\op{H}$
      \end{tabular}
      {\caption{
        Operators in the two Hilbert spaces,
        with emphasis on the algebraic relation
        to other operators in the same space.
      }\label{op_comparison_alg}}
    }
    \hfill
    \parbox{.45\linewidth}{
      \centering
      \begin{tabular}{c|c}
        $\hilb{H}_T$        & $\hilb{H}_S$  \\
        \hline
        \hline
        $\op{t}$           & $\op{x}$     \\
        \hline
        $\hbar\op{\Omega}$ & $\op{H}$     \\
        \hline
        $?$                 & $\op{p}$
      \end{tabular}
      {\caption{
        Operators in the two Hilbert spaces,
        with emphasis on the role in the
        Page--Wootters ``hamiltonian'' (eq. \ref{eq:pwHamiltonian}).
      }\label{op_comparison_J}}
    }
  \end{table}
}

TODO: REMOVE SECOND TABLE AND TREAT IN THE RELATIVISTIC CHAPTER/SECTION.
THE FOLLOWING PARAGRAPH IS REDUNDANT.
It would be interesting to a study relativistic extension of the
Page and Wootters model that allows, for example, the derivation of the Klein-Gordon
equation, thus eliminating the asimmetry between
$\hilb{H}_T$ and $\hilb{H}_S$ (and momentum and energy as well).

In the bipartite universe, physical kets $\dket{\Psi}$ have a Schmidt decomposition
made up of
eigenstates of $\op{\Omega}$ in $\hilb{H}_T$
entangled, respectively, with
eigenstates of $\op{H}_S$ in $\hilb{H}_S$;
or eigenstates of $\op{t}$ in $\hilb{H}_T$
entangled with time-evolved spatial states $\ket{\psi(t)}_S$ in $\hilb{H}_S$
(according to the evolution that is embedded in $\op{\mathbb{J}}$):
\begin{equation}
  \dket{\Psi} = \int d\mu(\omega) \ket{\omega}_{T}\ox\ket{\psi(\omega)}_{S} = \int dt \ket{t}_{T} \ox \ket{\psi(t)}_{S}\,\text{.} 
\end{equation}
In principle, different and less obvious decompositions are possible for other purposes.


\subsection{One qubit universe: experimental realization and theoretical developments}
\sectionmark{One qubit universe: experimental and theoretical developments}
\label{sec:pw:qubit}

The Page and Wootters model has been illustrated experimentally in recent years,
with a very simple toy universe consisting of just one qubit acting as ``the system'' (or
``the rest of the universe'' if we will), and the clock being implemented by another qubit ---
physically, the polarization states of two photons \parencite{Moreva:synthetic,Moreva:illustration}).

In another experiment by the same authors \parencite{Moreva_position}, $\hilb{H}_S$ is still implemented with 
polarizations $\ket{H}$ or $\ket{V}$ of a photon, while the clock states in $\hilb{H}_T$
are given by the \emph{position} of the same photon along the conventional $x$ axis.

The latter is interesting because it implements a continuous time,
which, among other things, allows identifying a canonically conjugate observable
$\op{\Omega}$. Or, conversely, a time operator $\op{T}$, once $\op{\Omega}$ is given.
More precisely, it is impossible to satisfy
\begin{equation}\label{eq:canonical_commutation_in_time}
  [\op{T}, \op{\Omega}] = i
\end{equation}
in a finite-dimension Hilbert space, because the operators
cannot be both bounded \parencite{Weyl1927}.

On the other hand, one problematic aspect of the experiment with continuous time
is that
it relies on \term{photon position} which is another
still controversial topic (see, for example, \cite{HawtonPhotonPosition}),
similar, in that regards, to the quest for a quantum time operator that the experiment is trying to solve.

Just like time in quantum mechanics, position coordinates in quantum optics and other field theories
are (classical) external parameters and not quantum observables. 

However, the experiment in \cite{Moreva_position} verifies the violation of
\term{Legget-Garg inequalities}, as previously suggested in \cite{LeggettGarg+PageWootters},
for ``time'' measurement results
(In our notation, Leggett-Garg inequalities are to $\hilb{H}_T$ what the well known Bell inequalities
  are to $\hilb{H}_S$).
This proves the ``quantumness'' of this realization of Page and Wootters time,
regardless of the explicit expression of the corresponding operator (which, unsurprisingly,
is not given). It's tempting to infer that the experiment
rather tests Bell/Leggett-Garg inequalities for photon position.

The first experiment, on the other hand \parencite{Moreva:illustration,Moreva:synthetic},
uses (uncontroversial)
two-level quantum systems for both the clock and the rest of the universe.
While we can't derive a $\op{T}$ such that $[T, \Omega] = i$
because of the finite dimension of the space, both $\Omega$
and $H_S$ are given an explicit expression:
\begin{align}\label{eq:MorevaOmegaT}
  \Omega            &= i\omega(\ketbra{H}{V}- \ketbra{V}{H})_T \\
  H_S/{\hbar}       &= i\omega(\ketbra{H}{V}- \ketbra{V}{H})_S
  \,\text{,}
\end{align} 
as well as a zero-eigenstate of $\mathbb{J}$ (as in eq. \ref{eq:pwHamiltonian}):
\begin{equation}
  \dket{\Psi} = \frac{1}{\sqrt{2}}\qty(\ket{H}_T\ket{V}_S-\ket{V}_T\ket{H}_S)
  \,\text{.}
\end{equation}

It can be easily verified that the Wheeler-DeWitt condition
\eqref{eq:Wheeler-DeWitt} is satisfied.

In general, given a clock ($\Omega$), the problem of finding a
``rest of the universe'' ($H_S$) such that
$\hbar\op{\Omega}\ox\idop_S + \idop_T\ox\op{H}_S = 0$
(and vice versa)
is not trivial
(and it's particularly cumbersome in non-relativisitc
quantum mechanics where we can't avail of negative energies etc.).
Most literature focuses their examples on clocks only
\parencite{Prvanovic,RealisticClocks,HarmonicClocks},
implicitly relying on the scale of a realistic universe
in order to have the \eqref{eq:Wheeler-DeWitt} satisfied
(which was originally derived using General Relativity arguments)
but missing the opportunity to illustrate the entanglement mechanism in detail,
which is aimed at, instead, in the present work, to some extent.

\subsection{Overcoming limitations of finite-dimension spaces}\label{sec:finite-quantum}
\epigraph{\textelp{} discreteness in the world is simply the Fourier transform of compactness.}{%
  \emph{Physics and the Integers} \parencite{Tong_Integers}%
}

The experiment mentioned in Sec. \ref{sec:pw:qubit}
is based on a two-level clock
The ``frequency'' observable is defined and implemented experimentally
but the \emph{time} operator is not defined or calculated.
In finite-dimension Hilbert spaces
there are diffilcuties in identifying canonically conjugate
pairs and one cannot use the same explicit formalism
that would be used for observables with a continuous spectrum.
Moreover, operators satisfying a canonical
commutation relation like eq.~\eqref{eq:canonical_commutation_in_time}
cannot be both bounded, which excludes finit-dimensional systems.

This problem was studied in detail in
\cite{FiniteHilb}. It was shown therein
that statisfying the canonical commutation relation 
is not essential to build operators representing physical observables
with the same role of position and momentum.\footnote{
  Or $T$ and $\hbar\Omega$
  in $\hilb{H}_T$ for a finite-dimensional Page and Wootters model.
}

Discrete, bounded position-like and momentum-like operators can be obtained from
each other via
the \term{finite Fourier transform}.
In our case, we are particularly interested in relating the
time operator $\op{T}$ and the ``energy'' operator $\hbar\op{\Omega}$
in $\hilb{H}_T$ ---which in the continuous limit would satisfy the
\eqref{eq:canonical_commutation_in_time} exactly.

%\citereset
It holds\footnote{
  Contrary to what indicated in eq. (8) of \cite{FiniteHilb},
  it can be easily verified that,
  if $ \op{p} = F x F^{\dagger} $,
  the correct inverse relation is
  $ x = F^{\dagger} p F$ and not $ -x = F p F^{\dagger} $.
} \parencite{FiniteHilb}:
\begin{gather}\label{eq:FourierCanonicalRelations}
  \op{\Omega} = F \op{T} F^{\dagger}\text{;} \quad
  \op{T} = F^{\dagger} \op{\Omega} F
\end{gather}
where, in the ``position'' (or \emph{time}) finite eigenbasis,
\begin{equation}
  F = \frac{1}{\sqrt{N}} \sum_{m,n=0}^{N-1} \exp[i\frac{2\pi mn}{N}] \ketbra{m}{n} \, \text{,}
\end{equation}
while in the frequency eigenbasis
\begin{equation}
  F^{\dagger} = \frac{1}{\sqrt{N}} \sum_{\mu,\nu=0}^{N-1} \exp[-i\frac{2\pi \mu\nu}{N}] \ketbra{\mu}{\nu} \, \text{,}
\end{equation}
with $N$ being the finite dimension of the Hilbert space.

Please note the \eqref{eq:FourierCanonicalRelations} is valid in normalized (``natural'') units
where ``time'' and ``frequency'' are in fact respectively
\term{samples} and \emph{cycles/samples rate},
in a similar sense as in digital signal processing theory
\parencite[pp. 469, 490]{Signal}.

In SI units, the \eqref{eq:FourierCanonicalRelations} is replaced by
\begin{gather}
  \label{eq:SI_Fourier:Omega}
    \op{\Omega} = \frac{2\pi}{N(\delta T)^2} F \op{T} F^{\dagger} = \frac{2\pi N}{\qty(\Delta T)^2} F \op{T} F^{\dagger} \\
  \label{eq:SI_Fourier:T}
    \op{T} = \frac{2\pi}{N(\delta\Omega)^2} F^{\dagger} \op{\Omega} F = \frac{2\pi N}{\qty(\Delta\Omega)^2} F^{\dagger} \op{\Omega} F
  \, \text{,}
\end{gather}
where $\delta T$ (and analogously $\delta\Omega$)
is the size of a ``temporal sample'', or the size of a discrete
time step in the clock, and $\Delta T = N\delta T$ the range of the clock.
For example,
$\delta T = \text{1 hour}$ and $\Delta T=12\;\text{hours}$
for a common clock (hours hand) in our everyday life.

It holds
\begin{gather}
  \delta\Omega \delta T = \frac{2\pi}{N} \, \text{;} \quad
  \Delta\Omega \Delta T = 2\pi N \, \text{.}
\end{gather}

A benefit of finite-dimensional systems is the potential implementation on a finite array of
qubits in a quantum computer. The use of Discrete Fourier Transform extends the overlap
with technology and engineering to the domain of signal processing \parencite{FiniteHilb}.
In \emph{ordinary} quantum mechanics, the Fourier transform (discrete or continuous)
is generally used
to associate wavefunctions in position and momendum space
(whereas time and frequency are \emph{not} operators),
while in communication engineering it is used to convert signals
from the time to the frequency domain and vice versa.
Thanks to the introduction of the Hilbert space $\hilb{H}_T$,
the interpretation in terms of time and frequency
(or time and energy, up to a factor $\hbar$)
is applicable to quantum theory as well, not only formally
i.e. not in the sense of a mere operation among (``classical'') parameters;
but in the sense of conversion between representations of the
same quantum state vector with respect to different eigenbasis,
in full analogy with position and momentum in $\hilb{H}_S$.

\subsubsection{Uncertainty in finite-dimensional systems}\label{sec:finite_uncertainty}
\citereset
For canonical pairs of operators with a continuous, unbounded spectrum i.e.
$\op{q}$ and $\op{p} \eqbydef -i\hbar\op{\partial}_{q}$,
it is in general straightforward to prove that
$\qty[\op{q}, \op{p}] = i\hbar$ and therefore
$\Delta q \Delta p \geq {\frac{1}{2} \qty|\ev{\qty[\op{q}, \op{p}]}| = \frac{\hbar}{2}}$.

In finite $d$-dimensional Hilbert spaces, the above commutation relation doesn't hold
in general, and is less essential.
Canonically conjugate operators are related
through the (discrete) Fourier transform ($\op{p} = F\op{q}F^{\dagger}$)
rather then differentiation,
and uncertainty relations are based on
the properties of Fourier transformation
rather than commutation relations.

Particularly, the entropic uncertainty relation holds
(\cite[sec.2.4]{FiniteHilb}; \cite{Deutsch:Uncertainty}):
\begin{equation}
  S_q + S_p \geq \ln d
\end{equation}
where the quantities $S_q$ and $S_p$ are the \term{R\'enyi}-\term{Shannon} entropies
\parencite[sec.{\it I}.A]{Wehner:Uncertainty}; in this case:
\begin{align}
  S_q &= -\sum_n \qty|\lambda_n |^2  \ln\qty|\lambda_n|^2 \\
  S_p &= -\sum_n \qty|\mu_n     |^2  \ln\qty|\mu_n    |^2
  \,\text{,}
\end{align}
with $\lambda_n$ and $\mu_n$ being the discrete ``wave functions'' in the
(generalized) position and momentum basis.



\subsection{Normalizable vectors of $\pwspace$}
\label{sec:properpw}

A vector $\dket{\Psi}$ in $\hilb{H}_T \ox \hilb{H}_S$,
satisfying \eqref{eq:pwHamiltonian} and \eqref{eq:Wheeler-DeWitt},
encodes the whole (unitary) time evolution of a system.
\begin{equation}\label{eq:pwexpansion}
  \dket{\Psi} =
    \int \dd{t} \ket{t}_T \ox \ket{\psi(t)}_S =
    \int \dd{t}\dd[3]{\vec{r}} \ \Psi(t; \vec{r}) \; \ket{t}_T \ox \ket{\vec{r}}_S
    \,  \text{.}
\end{equation}
We know $\setof{\ket{t}_T \ox \ket{\vec{r}}_S}$ is an othonormal basis of $\hilb{H}_T \ox \hilb{H}_S$, therefore
\begin{equation}
  \norm{\dket{\Psi}}^2 =
    \int \dd{t}\dd[3]{\vec{r}} \ \abs{\Psi(t; \vec{r})}^2 =
    \int \dd{t} \int \dd[3]{\vec{r}} \ \abs{\Psi(t; \vec{r})}^2 =
    \int \dd{t} 1 \rightarrow +\infty
    \,  \text{,}
\end{equation}
which means that such $\dket{\Psi}$ is an \term{improper} vector of $\hilb{H}_T \ox \hilb{H}_S$.

Proper (i.e. normalizable) states are described in \cite{Lloyd:Time} as well, by replacing (or generalizing)
the \eqref{eq:pwexpansion} with
\begin{equation}\label{eq:pwphi}
  \dket{\Phi} =
    \int \dd{t} \phi(t) \ket{t}_T \ox \ket{\psi(t)}_S \, \text{.}
\end{equation}
If the function $\phi \in \mathscr{L}^2(\mathbb{R})$,
then $\dket{\Psi}$ is a proper element of the product space,
and $\norm{\dket{\Psi}}^2 = \int \dd{t} \abs{\phi(t)}^2$.

The case of non-normalizable $\dket{\Psi}$ in \eqref{eq:pwphi},
with normalized $\ket{\psi(t)}_S$ $\forall t \in \mathbb{R}$,
describes the unitary evolution, as seen throughout Chapter \ref{ch:pw}.
As observed in \cite{Maccone:QGR},
``%
  Quantum mechanics is formulated in terms of \emph{systems},
  typically limited in space but infinitely extended in time%
''.
If the state vector is \emph{conditioned} at a particular time $t$,
it holds $\norm{_{T}\bradket{t}{\Psi}}_S = \norm{\ket{\psi(t)}}_S = 1$,
meaning that, at each $t$,
\emph{the particle must certainly be in some (one) point in space}.

A normalized $\dket{\Psi}$ in the whole $\hilb{H}_T \ox \hilb{H}_S$,
instead,
can be interpreted as a total probability of~$1$ in both space and time combined.
It can be interpreted as describing an \term{event},
in that it
must certainly be in some point in space
\emph{and} time (in terms of outcome of an idealized measurement\footnote{
  Measurement in quantum mechanics requires the concepts
  of state of the system
  (and measurement apparatus)
  \emph{before} and \emph{after} the measurement and, therefore, the existence
  of an external time (external with respect of the Hilbert space of states).
  This logically contradicts the foundation of the Page--Wootters model if
  time itself is measured as a quantum observable. $\dket{\Psi} \in \pwspace$
  embeds the whole history of a system and therefore cannot have a
  ``before'' neither an ``after'' the measurement, ``when'' it collapses
  into an eigenstate of time. The apparent contradicion is resolved
  stressing that probability amplitues are intended in the sense of
  \emph{conditional} probabilities e.g. \emph{provided that the particle
  is in position} $x \in X$ (or the detector clicks)
  what is the probability density amplitude of time being (the clock showing) $t$?
  Consistently, the Bayes rule
  (see, for example, \cite{Stat:Conditional})
  is invoked in the following sections
  and references.
}).
A ``typical'' example of localized ``event wave packet'' would be
therefore represented by
a \emph{4\nobreakdash-dimensional} Gaussian wave function, % https://tex.stackexchange.com/a/330437
in analogy to well known examples of purely spatial Gaussian states
in quantum mechanics and quantum optics.

