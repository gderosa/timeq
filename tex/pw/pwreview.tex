\section{Development of the Formalism}

About three decades after the first publication by
Don Page and William Wootters \parencite{PageWootters},
the model was further developed,
in particular by Lloyd, Giovannetti and Maccone,
with emphasis on addressing technical issues which have been raised meanwhile
\parencite{Lloyd:Time}.
In this formulation,
in addition to the ordinary Hilbert space of the system,
an extra Hilbert space $\mathcal{H}_T$ is considered,
where time is an observable
represented by a self-adjoint operator.

In this language, the ordinary Hilbert space can be labeled $\mathcal{H}_S$;
and we consider the tensor product space $\mathcal{H}_T \otimes \mathcal{H}_S$ as
the space in which both time and position are observables, and they act as
$T \otimes \idop_S$ and $\idop_T \otimes X$
respectively.

In principle, an appropriate system (a ``clock'') can be identified in such a way
to be described in $\mathcal{H}_T$, while $\mathcal{H}_S$ describes
the quantum states of \emph{the rest of the universe} \parencite{Marletto:Evolution}.

As explained in \cite{Lloyd:Time, Maccone:Pauli}, the overall Hamiltonian,
encompassing both position and time as observables, is given by
\begin{equation}\label{eq:pwHamiltonian}
  \op{\mathbb{J}} = \hbar\op{\Omega}\ox\idop_S + \idop_T\ox\op{H}_S \,\text{,}
\end{equation}
where $\hbar\op{\Omega}$ is the canonically conjugate of $\op{T}$ in $\hilb{H}_T$.
Also, the \term{Wheeler-DeWitt equation} holds:
\begin{equation}\label{eq:Wheeler-DeWitt}
  \op{\mathbb{J}}\dket{\Psi} = 0 \,\text{,}
\end{equation}
describing a \emph{static} universe, where evolution is only
in terms of relations between parts of a multipartite system
(a ``clock'' and ``the rest'').

% Please note the special notation $\dket{\Psi}$ (double angle bracket),
% to indicate a state vector in the ``larger'' product space $\pwspace$,
% as opposed to vectors belonging to $\hilb{H}_S$ (or $\hilb{H}_T$) only.

The ``conventional'' state $\ket{\psi(t)}_S$ in $\hilb{H}_S$
can be obtained from $\dket{\Psi}$ TODO 

Using the $T$ representation in $\hilb{H}_T$,
and comparing \eqref{eq:pwHamiltonian} and \eqref{eq:Wheeler-DeWitt}:
\begin{equation}\label{eq:schrod_from_pw}
  0 = \qty(\hbar\op{\Omega}\ox\idop_S + \idop_{T}\ox\op{H}_S)\dket{\Psi}
    \repr -i\hbar\pdv{t}\ket{\psi(t)}_{S} + \op{H}_S\ket{\psi(t)}_{S}
    \,\text{,}
\end{equation}
we recover the usual form of the Schr\"{o}dinger eqaution in $H_S$.
Details of this derivation are in the reference aforementioned.

Here $\op{\Omega}$ can be seen as a ``frequency operator''
represented as $-i\pdv{t}$ and having as eigenfunctions
those functions evolving in time with a phase factor $e^{i \omega t}$ only.

Canonical commutation relation holds and can be easily verified
between $\op{t}$ and $\op{\Omega}$
i.e. $[\op{t}, \op{\Omega}] = i$,
therefore $\hbar\op{\Omega}$ can be seen as the ``linear momentum''
in the Hilbert space of time.

From another point of view, $\hbar\op{\Omega}$ is the ``hamiltonian'' of $\hilb{H}_T$,
in that it plays a similar role of $H_S$ in the construction of
$\op{\mathbb{J}}$ in \eqref{eq:pwHamiltonian}. This dual role doesn't hold
for $H_S$. 

This ambiguity is related to the a\-sym\-me\-try of space and time in
non\-re\-la\-ti\-vi\-stic
mechanics and can be expressively synthesized in the below:
{
  %% https://tex.stackexchange.com/a/232874
  %% https://tex.stackexchange.com/a/2836
  \begin{table}[h!]
    \parbox{.45\linewidth}{
      \centering
      \begin{tabular}{c|c}
        $\hilb{H}_T$        & $\hilb{H}_S$  \\
        \hline
        \hline
        $\op{t}$           & $\op{x}$     \\
        \hline
        $\hbar\op{\Omega}$ & $\op{p}$     \\
        \hline
        $?$                 & $\op{H}$
      \end{tabular}
      {\caption{
        Operators in the two Hilbert spaces,
        with emphasis on the algebraic relation
        to other operators in the same space.
      }\label{op_comparison_alg}}
    }
    \hfill
    \parbox{.45\linewidth}{
      \centering
      \begin{tabular}{c|c}
        $\hilb{H}_T$        & $\hilb{H}_S$  \\
        \hline
        \hline
        $\op{t}$           & $\op{x}$     \\
        \hline
        $\hbar\op{\Omega}$ & $\op{H}$     \\
        \hline
        $?$                 & $\op{p}$
      \end{tabular}
      {\caption{
        Operators in the two Hilbert spaces,
        with emphasis on the role in the
        Page--Wootters ``hamiltonian'' (eq. \ref{eq:pwHamiltonian}).
      }\label{op_comparison_J}}
    }
  \end{table}
}

TODO: REMOVE SECOND TABLE AND TREAT IN THE RELATIVISTIC CHAPTER/SECTION.
THE FOLLOWING PARAGRAPH IS REDUNDANT.
It would be interesting to a study relativistic extension of the
Page and Wootters model that allows, for example, the derivation of the Klein-Gordon
equation, thus eliminating the asimmetry between
$\hilb{H}_T$ and $\hilb{H}_S$ (and momentum and energy as well).

In the bipartite universe, physical kets $\dket{\Psi}$ have a Schmidt decomposition
made up of
eigenstates of $\op{\Omega}$ in $\hilb{H}_T$
entangled, respectively, with
eigenstates of $\op{H}_S$ in $\hilb{H}_S$;
or eigenstates of $\op{t}$ in $\hilb{H}_T$
entangled with time-evolved spatial states $\ket{\psi(t)}_S$ in $\hilb{H}_S$
(according to the evolution that is embedded in $\op{\mathbb{J}}$):
\begin{equation}
  \dket{\Psi} = \int d\mu(\omega) \ket{\omega}_{T}\ox\ket{\psi(\omega)}_{S} = \int dt \ket{t}_{T} \ox \ket{\psi(t)}_{S}\,\text{.} 
\end{equation}
In principle, different and less obvious decompositions are possible for other purposes.






