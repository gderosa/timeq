\section{Introduction}

The Page and Wootters model is based on
an additional Hilbert space $\mathcal{H}_T$,
where time is an observable
represented by a self-adjoint operator
whose properties are similar to the ones of position
in ordinary quantum mechanics
\parencite{Lloyd:Time, Maccone:Pauli}.

In this language, the ordinary Hilbert space can be labeled $\mathcal{H}_S$;
and we consider the product space $\mathcal{H}_T \otimes \mathcal{H}_S$ as
the space in which both time and position are observables, and they act as
$\hat{t} \otimes \idop_S$ and $\idop_T \otimes \hat{x}$
respectively.

\begin{remark}
  In a more ``relativistic friendly'' notation, we may set
  $\hat{x_0} = c\hat{t}$ and consider
  $\hat{x_0} \otimes \idop_S$ and $\idop_T \otimes \hat{x_1}$,
  and so on. An interesting development may be expressing
  Lorentz transformations as unitary transformations in
  $\mathcal{H}_T \otimes \mathcal{H}_S$.
\end{remark}

The Page and Wootters mechanism is rooted in the ``problem of time''
in quantum cosmology.
In principle, an approriate system (a ``clock'') can be identified in such a way
to be described in $\mathcal{H}_T$, while $\mathcal{H}_S$ describes
the quantum states of \emph{the rest of the universe} \parencite{Marletto:Evolution}.

As explained in \cite{Lloyd:Time, Maccone:Pauli}, the overall Hamiltonian,
encompassing both position and time as observables, is given by
\begin{equation}\label{eq:pwHamiltonian}
  \hat{\mathbb{J}} = \hbar\hat{\Omega}\ox\idop_S + \idop_T\ox\hat{H}_S \,\text{,}
\end{equation}
while the \term{Wheeler-DeWitt equation} holds:
\begin{equation}\label{eq:Wheeler-DeWitt}
  \hat{\mathbb{J}}\dket{\Psi} = 0 \,\text{,}
\end{equation}
describing a \emph{static} universe, where evolution is only
in terms of relations between parts of a multipartite system
(a ``clock'' and ``the rest'').

Using the $t$ representation in $\hilb{H}_T$,
and comparing \eqref{eq:pwHamiltonian} and \eqref{eq:Wheeler-DeWitt}:
\begin{equation}
  0 = \qty(\hbar\hat{\Omega}\ox\idop_S + \idop_{T}\ox\hat{H}_S)\dket{\Psi}
    = -i\hbar\pdv{t}\ket{\psi(t)}_{S} + \hat{H}_S\ket{\psi(t)}_{S}
    \,\text{,}
\end{equation}
we recover the usual form of the Schr\"{o}dinger eqaution in $H_S$.
Details of this derivation are in the reference aforementioned.

Here $\hat{\Omega}$ can be seen as a ``frequency operator''
represented as $-i\pdv{t}$ and having as eigenfunctions
those functions evolving in time with a phase factor $e^{i \omega t}$ only.

Canonical commutation relation holds and can be easily verified
between $\hat{t}$ and $\hat{\Omega}$
i.e. $[\hat{t}, \hat{\Omega}] = i$,
therefore $\hbar\hat{\Omega}$ can be seen as the ``linear momentum''
in the Hilbert space of time.

From another point of view, $\hbar\hat{\Omega}$ is the ``hamiltonian'' of $\hilb{H}_T$,
in that it plays a similar role of $H_S$ in the construction of
$\hat{\mathbb{J}}$ in \eqref{eq:pwHamiltonian}. This dual role doesn't hold
for $H_S$. It would be interesting to a study relativistic extension of the
Page and Wootters model that allows, for example, the derivation of the Klein-Gordon
equation, thus eliminating the asimmetry between
$\hilb{H}_T$ and $\hilb{H}_S$ (and momentum and energy as well).

In the bipartite universe, physical kets $\dket{\Psi}$ have a Schmidt decomposition
made up of
eigenstates of $\hat{\Omega}$ in $\hilb{H}_T$
entangled, respectively, with
eigenstates of $\hat{H}_S$ in $\hilb{H}_S$;
or eigenstates of $\hat{t}$ in $\hilb{H}_T$
entangled with time-evolved spatial states $\ket{\psi(t)}_S$ in $\hilb{H}_S$
(according to the evolution that is embedded in $\hat{\mathbb{J}}$):
\begin{equation}
  \dket{\Psi} = \int d\mu(\omega) \ket{\omega}_{T}\ox\ket{\psi(\omega)}_{S} = \int dt \ket{t}_{T} \ox \ket{\psi(t)}_{S}\,\text{.} 
\end{equation}
In principle, different and less obvious decompositions are possible for other purposes.

\section{One qubit universe: experimental realization and theoretical developments}

The Page and Wootters model has been illustrated experimentally in recent years,
with a very simple toy universe consisting of just one qubit acting as ``the system'' (or
``the rest of the universe'' if we will), and the clock being immplemented by another qubit ---
physically, the polarization states of two photons \parencite{Moreva:synthetic,Moreva:illustration}).

In another experiment by the same authors \parencite{Moreva_position}, $\hilb{H}_S$ is still implemented with 
polarizations $\ket{H}$ or $\ket{V}$ of a photon, while the clock states in $\hilb{H}_T$
are given by the \emph{position} of the same photon along the conventional $x$ axis.

The latter is interesting because it implements a continuous time,
which, among other things, allows identifying a canonically conjugate observable
$\hat{\Omega}$. Or, conversely, a time operator $\hat{T}$, once $\hat{\Omega}$ is given.
More precisely, it is impossible to satisfy
\begin{equation}\label{eq:canonical_commutation_in_time}
  [\hat{t}, \hat{\Omega}] = i
\end{equation}
in a finite-dimension Hilbert space, because the operators
cannot be both bounded \parencite{Weyl1927}.

On the other hand, one problematic aspect of the experiment with continuous time
is that
it relies on \term{photon position} which is another
still controversial topic (see, for example, \cite{HawtonPhotonPosition}),
similar, in that regards, to the quest for a quantum time operator that the experiment is trying to solve.

Just like time in quantum mechanics, position coordinates in quantum optics and other field theories
are (classical) external parameters and not quantum observables. 

However, the experiment in \cite{Moreva_position} verifies the violation of
\term{Legget-Garg inequalities}, as previously suggested in \cite{LeggettGarg+PageWootters},
for ``time'' measurement results
(In our notation, Leggett-Garg inequalities are to $\hilb{H}_T$ what the well known Bell inequalities
  are to $\hilb{H}_S$).
This proves the ``quantumness'' of this realization of Page and Wootters time,
regardless of the explicit expression of the corresponding operator (which, unsurprisingly,
is not given). It's tempting to infer that the experiment
rather tests Bell/Leggett-Garg inequalities for photon position.

The first experiment, on the other hand \parencite{Moreva:illustration,Moreva:synthetic},
uses (uncontroversial)
two-level quantum systems for both the clock and the rest of the universe.
While we can't derive a $\hat{t}$ such that $[t, \Omega] = i$
because of the finite dimension of the space, both $\Omega$
and $H_S$ are given an explicit expression:
\begin{align}
  \Omega            &= i\omega(\ketbra{H}{V}- \ketbra{V}{H})_T \\
  H_S/{\hbar}       &= i\omega(\ketbra{H}{V}- \ketbra{V}{H})_S
  \,\text{,}
\end{align} 
as well as a zero-eigenstate of $\mathbb{J}$ (as in eq. \ref{eq:pwHamiltonian}):
\begin{equation}
  \dket{\Psi} = \frac{1}{\sqrt{2}}\qty(\ket{H}_T\ket{V}_S-\ket{V}_T\ket{H}_S)
  \,\text{.}
\end{equation}

It can be easily verified that the Wheeler-DeWitt condition
\eqref{eq:Wheeler-DeWitt} is satisfied.

In general, given a clock ($\Omega$), the problem of finding a
``rest of the universe'' ($H_S$) such that
$\hbar\hat{\Omega}\ox\idop_S + \idop_T\ox\hat{H}_S = 0$
(and vice versa)
is not trivial
(and it's particularly cumbersome in non-relativisitc
quantum mechanics where we can't avail of negative energies etc.).
Most literature focuses their examples on clocks only \parencite{Prvanovic,RealisticClocks,HarmonicClocks},
implicitly relying on the scale of a realistic universe
in order to have the \eqref{eq:Wheeler-DeWitt} satisfied
(which was originally derived using General Relativity arguments)
but missing the opportunity to illustrate the entanglement mechanism in detail,
which is aimed at, instead, in the present work, to some extent.

\subsection{Overcoming limitations of finite-dimension spaces}

\epigraph{\textelp{} discreteness in the world is simply the Fourier transform of compactness.}{
  \emph{Physics and the Integers} \parencite{Tong_Integers}
}

\noindent{}This has been tackled, for example in
\cite{FiniteHilb},
where it has been shown that the lack of operators satisfying the canonical
commutation relation \eqref{eq:canonical_commutation_in_time}
is not essential to build operators representing physical observables
with the same role of position and momentum (or $t$ and $\hbar\Omega$
in $\hilb{H}_T$ for a finite-dimensional Page and Wootters model).

Discrete, bounded position and momentum operators can be derived from
each other via
the \term{finite Fourier transform}.
In our case, we are particularly interested in relating the
time operator $\hat{t}$ and the ``energy'' operator $\hbar\hat{\Omega}$
in $\hilb{H}_T$ ---which in the continuous limit would satisfy the
\eqref{eq:canonical_commutation_in_time} exactly.

It holds\footnote{
  Contrary to what indicated in eq. (8) of \cite{FiniteHilb},
  where $ p = F x F^{\dagger} $,
  it can be easily verified that the correct inverse relation is
  $ x = F^{\dagger} p F$ and not $ -x = F p F^{\dagger} $.
} \parencite{FiniteHilb}:
\begin{gather}
  \Omega = F t F^{\dagger}\text{;} \quad
  t = F^{\dagger} \Omega F
\end{gather}
where, in the ``position'' (or \emph{time}) finite eigenbasis,
\begin{equation}
  F = \frac{1}{\sqrt{d}} \sum_{m,n=0}^{d-1} \exp[i\frac{2\pi mn}{d}] \ketbra{m}{n}
\end{equation}
with $d$ being the finite dimension of the Hilbert space.

\iftodo


\section{Prvanovic and P\&W}
In \cite{Prvanovic}, essentially the clock observable is the Hamiltonian.
The two example clocks are an harmonic oscillator and a free particle.
The harmonic oscillator features discrete time. Generally a time which is
{bounded from below}
is consistent with the Big Bang...

Prvanovic uses ``relativisitc'' constants...

Ciao.

\section{Analyze latest Moreva experiment with previous ones}

Let's consider a photon, as a two level system with its two linear polarization
sates $\ket{H}, \ket{V}$;
and the Hamiltonian
\begin{equation}
  \mathcal{H}=i\hbar\omega\qty(\ket{H}\ket{V}-\ket{V}\ket{H})\,\text{.}
\end{equation}
There has:
\[
  \mathcal{H}^n \ket{H, V} = \begin{cases}
      \hbar^n\omega^n\ket{H, V} &\text{for n even} \\
    -i\hbar^n\omega^n\ket{V, H} &\text{for n odd}
  \end{cases}
\]

\url{http://physweb.bgu.ac.il/COURSES/PHYSICS3_physics/CLASS_ymeir/polarization.pdf}

\section{Entanglement and decoherence (Arrow of time)}
See also \cite{EntanglementVsDecoherence}.

Decoherence is an irreversible process, it also happens in measurement.

According to Marletto and Vedral, arrow of time is increase in Entanglement
between the clock and the rest.

So, there seems to be a contradiction: is entanglement ``decreasing''
(i.e. destroyed by decoherence) with time
or increasing?

We can avoid the contradiction saying that
entanglement between two finite systems is
destroyed while the entanglement of each of them with the universe
is increasing.




\section{TODOs}

TODO: \cite{Lloyd:Time} does not only deal with improper eigenstates in $\hilb{H}_T$
(full evolutions?)
but also normalized ones (in $\hilb{H}_T$! \emph{``Events''}?)

TODO: \cite{RealisticClocks}.

\cite{HarmonicClocks} concludes ``Classical clock can be described by an Hamiltonian linear in momentum''\dots
like in relativity?

Extra: \cite{TimeAnyons}.

TODO: use the harmonic oscillator in \cite{HarmonicClocks}
as a PaW clock for the same packet that is measured in
Ruschhaupt's detector model.

Therein, fading wave function: is minus derivative an event?
L4 normalized?

TODO: qubit PW vs Ruschhaupt detector on 2-level system  i.e. \url{https://arxiv.org/pdf/1109.5087.pdf},
particularly ``EMISSION FROM A TWO-LEVEL SYSTEM''.

Other systems of interest: decays. Prvanovic new.

TQM Book: time of residence: applicable to qubits, therefore Moreva experiment.

\url{https://arxiv.org/abs/1708.04302} majorization, thermo.



\section{and paths}

Both \cite{YearsleyHalliwell_Clocks} and \cite{Gambini_PW}
reason in terms of paths and actions, maybe Feynmann stuff
in following chapter... and maybe conistent historiesapproach can help
towards linking PaW and ToA?

Also \url{http://quantum.phys.cmu.edu/CHS/CHS_transp.pdf}.

\section{and dwell time}

\cite[\S 5]{TQM2}, \cite{YearsleyHalliwell_Clocks}.

\section{In relation with the \term{time of residence}}

Here we compare \cite{Moreva:synthetic, Moreva:illustration}
(a Page and Wootters problem)
with the ``standard'' \term{time of residence}
treated in \cite[\S 5.5.2]{TQM2},
the only usable concept since there is no notion of position
for a qubit.

\section{Comparison with (Kijowski/Aharonov-Bohm) time of arrival}

Time of arrival is derived for non-relativistic \parencite{Delgado_TOA, Delgado_TOA2}
and relativisitic \parencite{Leon_TOA_R}
particle.

Kijowski: \cite{Kijowski_Time, Kijowski_Comment}.

A Page and Wootters time of arrival is mentioned in \cite{Gambini_PW}.

Time of arrival and clocks: again, \cite{YearsleyHalliwell_Clocks}.
Which maybe suggests we should not wory too much of $H\ket{\Psi} = 0$. 

BUT please note \cite{YearsleyHalliwell_Clocks} uses a clock that is
\emph{coupled} with the system, while in PaW they are ``only'' entangled.
So their calculation may be unnecessarily complicated.
Maybe the weakjly coupling case can be used?

\section{In relation to Leggett-Garg inequalities}
(Also mentioned in \cite{Moreva_position}).

Ref \cite{LeggettGarg+PageWootters}.

But also Lloyd: \url{https://arxiv.org/abs/1608.05672},
\emph{Decoherent histories approach to the cosmological measure problem}.
Lindbladt, Markov, Open Systems.

Halliwell, \url{https://arxiv.org/abs/1604.01659}. Ancilla, decay (spontaneous emission?).

A phylosophical object to decoherent histories / Everett (Everett mentioned in Marletto/ref)
is at
\url{https://arxiv.org/abs/1603.04845}.
\section{Detector model}

\cite{TQM2} (Kijowski and Detector) is cited and summarized well in
\cite{Halliwell_Detector}.



\section{Can a POVM on a system, if then we see it as part of a bipartite one,
equivalent to a PVM on the other, entangled, system?}

TODO: backflow effect in both models.

This will show an equivalence of models based on POVM with the Page and Wootters...?

Well, yes.

From \cite{PreskillNotes}, Ch.3 
\begin{quotation}
We have seen that
a pure state of the bipartite system AB may behave like a mixed state
when we observe subsystem A alone, and that an orthogonal measurement
of the bipartite system can realize a (nonorthogonal) POVM on A alone.
\end{quotation}

and

\begin{quotation}
A POVM in $H_A$ can be realized as a unitary transformation on the tensor
product $H_A \otimes H_B$, followed by an orthogonal measurement in $H_B$.
\end{quotation}

The same chapter talks about quantum operations, quantum channels and Kraus opertators.

We might want to look at exponential decay from \url{https://arxiv.org/abs/1704.07236},
then compare with exponential decay with P and W using Lloyd Giovannetti and Maccone (ref).

\subsection{Purification}

See https://arxiv.org/pdf/quant-ph/0512125.pdf, P-W time as a purifying ancilla
of the (Kijowski?) time.

\subsection{4-partite universe?}
\begin{itemize}
  \item{The system being measured/detected}
  \item{The Ruschhaupt detector --- which does not measure time, but whose detection happens at a certain time}
  \item{The Page and Wootters clock, entangled with the system and/or the detector}
  \item{The rest of the Universe, aka the Environment, aka the Termal Bath or Reservoir}
\end{itemize}

Can any of the above be identified? If the lab is isolated enough,
the detector is the only macro object and can act as a Universe/bath/environment/reservoir\dots?

\fi