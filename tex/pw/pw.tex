\section{Introduction}

The Page and Wootters model is based on
an additional Hilbert space $\mathcal{H}_T$,
where time is an observable
represented by a self-adjoint operator
whose properties are similar to the ones of position
in ordinary quantum mechanics
\parencite{Lloyd:Time, Maccone:Pauli}.

In this language, the ordinary Hilbert space can be labeled $\mathcal{H}_S$;
and we consider the product space $\mathcal{H}_T \otimes \mathcal{H}_S$ as
the space in which both time and position are observables, and they act as
$\hat{t} \otimes \idop_S$ and $\idop_T \otimes \hat{x}$
respectively.

\begin{remark}
  In a more ``relativistic friendly'' notation, we may set
  $\hat{x_0} = c\hat{t}$ and consider
  $\hat{x_0} \otimes \idop_S$ and $\idop_T \otimes \hat{x_1}$,
  and so on. An interesting development may be expressing
  Lorentz transformations as unitary transformations in
  $\mathcal{H}_T \otimes \mathcal{H}_S$.
\end{remark}

The Page and Wootters mechanism is rooted in the ``problem of time''
in quantum cosmology.
In principle, an approriate system (a ``clock'') can be identified in such a way
to be described in $\mathcal{H}_T$, while $\mathcal{H}_S$ describes
the quantum states of \emph{the rest of the universe} \parencite{Marletto:Evolution}.

As explained in \cite{Lloyd:Time, Maccone:Pauli}, the overall Hamiltonian,
encompassing both position and time as observables, is given by
\begin{equation}\label{eq:pwHamiltonian}
  \hat{\mathbb{J}} = \hbar\hat{\Omega}\ox\idop_S + \idop_T\ox\hat{H}_S \,\text{,}
\end{equation}
while the \term{Wheeler-DeWitt equation} holds:
\begin{equation}\label{eq:Wheeler-DeWitt}
  \hat{\mathbb{J}}\dket{\Psi} = 0 \,\text{,}
\end{equation}
describing a \emph{static} universe, where evolution is only
in terms of relations between parts of a multipartite system
(a ``clock'' and ``the rest'').

Using the $T$ representation in $\hilb{H}_T$,
and comparing \eqref{eq:pwHamiltonian} and \eqref{eq:Wheeler-DeWitt}:
\begin{equation}
  0 = \qty(\hbar\hat{\Omega}\ox\idop_S + \idop_{T}\ox\hat{H}_S)\dket{\Psi}
    = -i\hbar\pdv{t}\ket{\psi(t)}_{S} + \hat{H}_S\ket{\psi(t)}_{S}
    \,\text{,}
\end{equation}
we recover the usual form of the Schr\"{o}dinger eqaution in $H_S$.
Details of this derivation are in the reference aforementioned.

Here $\hat{\Omega}$ can be seen as a ``frequency operator''
represented as $-i\pdv{t}$ and having as eigenfunctions
those functions evolving in time with a phase factor $e^{i \omega t}$ only.

Canonical commutation relation holds and can be easily verified
between $\hat{t}$ and $\hat{\Omega}$
i.e. $[\hat{t}, \hat{\Omega}] = i$,
therefore $\hbar\hat{\Omega}$ can be seen as the ``linear momentum''
in the Hilbert space of time.

From another point of view, $\hbar\hat{\Omega}$ is the ``hamiltonian'' of $\hilb{H}_T$,
in that it plays a similar role of $H_S$ in the construction of
$\hat{\mathbb{J}}$ in \eqref{eq:pwHamiltonian}. This dual role doesn't hold
for $H_S$. 

This ambiguity is related to the asymmetry of space and time in non-relativistic
mechanics and can be expressively synthetised in the below:
{
  %% https://tex.stackexchange.com/a/232874
  %% https://tex.stackexchange.com/a/2836
  \begin{table}[h!]
    \parbox{.45\linewidth}{
      \centering
      \begin{tabular}{c|c}
        $\hilb{H}_T$        & $\hilb{H}_S$  \\
        \hline
        \hline
        $\hat{t}$           & $\hat{x}$     \\
        \hline
        $\hbar\hat{\Omega}$ & $\hat{p}$     \\
        \hline
        $?$                 & $\hat{H}$
      \end{tabular}
      {\caption{
        Operators in the two Hilbert spaces,
        with emphasis on the algebraic relation
        to other operators in the same space.
      }\label{op_comparison_alg}}
    }
    \hfill
    \parbox{.45\linewidth}{
      \centering
      \begin{tabular}{c|c}
        $\hilb{H}_T$        & $\hilb{H}_S$  \\
        \hline
        \hline
        $\hat{t}$           & $\hat{x}$     \\
        \hline
        $\hbar\hat{\Omega}$ & $\hat{H}$     \\
        \hline
        $?$                 & $\hat{p}$
      \end{tabular}
      {\caption{
        Operators in the two Hilbert spaces,
        with emphasis on the role in the
        Page-Wootters ``hamiltonian'' (eq. \ref{eq:pwHamiltonian}).
      }\label{op_comparison_J}}
    }
  \end{table}
}

It would be interesting to a study relativistic extension of the
Page and Wootters model that allows, for example, the derivation of the Klein-Gordon
equation, thus eliminating the asimmetry between
$\hilb{H}_T$ and $\hilb{H}_S$ (and momentum and energy as well).

In the bipartite universe, physical kets $\dket{\Psi}$ have a Schmidt decomposition
made up of
eigenstates of $\hat{\Omega}$ in $\hilb{H}_T$
entangled, respectively, with
eigenstates of $\hat{H}_S$ in $\hilb{H}_S$;
or eigenstates of $\hat{t}$ in $\hilb{H}_T$
entangled with time-evolved spatial states $\ket{\psi(t)}_S$ in $\hilb{H}_S$
(according to the evolution that is embedded in $\hat{\mathbb{J}}$):
\begin{equation}
  \dket{\Psi} = \int d\mu(\omega) \ket{\omega}_{T}\ox\ket{\psi(\omega)}_{S} = \int dt \ket{t}_{T} \ox \ket{\psi(t)}_{S}\,\text{.} 
\end{equation}
In principle, different and less obvious decompositions are possible for other purposes.

\section[
  One qubit universe \dots
]{One qubit universe: experimental realization and theoretical developments}

The Page and Wootters model has been illustrated experimentally in recent years,
with a very simple toy universe consisting of just one qubit acting as ``the system'' (or
``the rest of the universe'' if we will), and the clock being immplemented by another qubit ---
physically, the polarization states of two photons \parencite{Moreva:synthetic,Moreva:illustration}).

In another experiment by the same authors \parencite{Moreva_position}, $\hilb{H}_S$ is still implemented with 
polarizations $\ket{H}$ or $\ket{V}$ of a photon, while the clock states in $\hilb{H}_T$
are given by the \emph{position} of the same photon along the conventional $x$ axis.

The latter is interesting because it implements a continuous time,
which, among other things, allows identifying a canonically conjugate observable
$\hat{\Omega}$. Or, conversely, a time operator $\hat{T}$, once $\hat{\Omega}$ is given.
More precisely, it is impossible to satisfy
\begin{equation}\label{eq:canonical_commutation_in_time}
  [\hat{t}, \hat{\Omega}] = i
\end{equation}
in a finite-dimension Hilbert space, because the operators
cannot be both bounded \parencite{Weyl1927}.

On the other hand, one problematic aspect of the experiment with continuous time
is that
it relies on \term{photon position} which is another
still controversial topic (see, for example, \cite{HawtonPhotonPosition}),
similar, in that regards, to the quest for a quantum time operator that the experiment is trying to solve.

Just like time in quantum mechanics, position coordinates in quantum optics and other field theories
are (classical) external parameters and not quantum observables. 

However, the experiment in \cite{Moreva_position} verifies the violation of
\term{Legget-Garg inequalities}, as previously suggested in \cite{LeggettGarg+PageWootters},
for ``time'' measurement results
(In our notation, Leggett-Garg inequalities are to $\hilb{H}_T$ what the well known Bell inequalities
  are to $\hilb{H}_S$).
This proves the ``quantumness'' of this realization of Page and Wootters time,
regardless of the explicit expression of the corresponding operator (which, unsurprisingly,
is not given). It's tempting to infer that the experiment
rather tests Bell/Leggett-Garg inequalities for photon position.

The first experiment, on the other hand \parencite{Moreva:illustration,Moreva:synthetic},
uses (uncontroversial)
two-level quantum systems for both the clock and the rest of the universe.
While we can't derive a $\hat{t}$ such that $[t, \Omega] = i$
because of the finite dimension of the space, both $\Omega$
and $H_S$ are given an explicit expression:
\begin{align}\label{eq:MorevaOmegaT}
  \Omega            &= i\omega(\ketbra{H}{V}- \ketbra{V}{H})_T \\
  H_S/{\hbar}       &= i\omega(\ketbra{H}{V}- \ketbra{V}{H})_S
  \,\text{,}
\end{align} 
as well as a zero-eigenstate of $\mathbb{J}$ (as in eq. \ref{eq:pwHamiltonian}):
\begin{equation}
  \dket{\Psi} = \frac{1}{\sqrt{2}}\qty(\ket{H}_T\ket{V}_S-\ket{V}_T\ket{H}_S)
  \,\text{.}
\end{equation}

It can be easily verified that the Wheeler-DeWitt condition
\eqref{eq:Wheeler-DeWitt} is satisfied.

In general, given a clock ($\Omega$), the problem of finding a
``rest of the universe'' ($H_S$) such that
$\hbar\hat{\Omega}\ox\idop_S + \idop_T\ox\hat{H}_S = 0$
(and vice versa)
is not trivial
(and it's particularly cumbersome in non-relativisitc
quantum mechanics where we can't avail of negative energies etc.).
Most literature focuses their examples on clocks only
\parencite{Prvanovic,RealisticClocks,HarmonicClocks},
implicitly relying on the scale of a realistic universe
in order to have the \eqref{eq:Wheeler-DeWitt} satisfied
(which was originally derived using General Relativity arguments)
but missing the opportunity to illustrate the entanglement mechanism in detail,
which is aimed at, instead, in the present work, to some extent.

\subsection{Overcoming limitations of finite-dimension spaces}

\epigraph{\textelp{} discreteness in the world is simply the Fourier transform of compactness.}{
  \emph{Physics and the Integers} \parencite{Tong_Integers}
}

\noindent{}This has been tackled, for example in
\cite{FiniteHilb},
where it has been shown that the lack of operators satisfying the canonical
commutation relation \eqref{eq:canonical_commutation_in_time}
is not essential to build operators representing physical observables
with the same role of position and momentum (or $T$ and $\hbar\Omega$
in $\hilb{H}_T$ for a finite-dimensional Page and Wootters model).

Discrete, bounded position and momentum operators can be derived from
each other via
the \term{finite Fourier transform}.
In our case, we are particularly interested in relating the
time operator $\hat{T}$ and the ``energy'' operator $\hbar\hat{\Omega}$
in $\hilb{H}_T$ ---which in the continuous limit would satisfy the
\eqref{eq:canonical_commutation_in_time} exactly.

It holds\footnote{
  Contrary to what indicated in eq. (8) of \cite{FiniteHilb},
  it can be easily verified that,
  if $ \hat{p} = F x F^{\dagger} $,
  the correct inverse relation is
  $ x = F^{\dagger} p F$ and not $ -x = F p F^{\dagger} $.
} \parencite{FiniteHilb}:
\begin{gather}\label{eq:FourierCanonicalRelations}
  \hat{\Omega} = F \hat{T} F^{\dagger}\text{;} \quad
  \hat{T} = F^{\dagger} \hat{\Omega} F
\end{gather}
where, in the ``position'' (or \emph{time}) finite eigenbasis,
\begin{equation}
  F = \frac{1}{\sqrt{N}} \sum_{m,n=0}^{N-1} \exp[i\frac{2\pi mn}{N}] \ketbra{m}{n} \, \text{,}
\end{equation}
while in the frequency eigenbasis
\begin{equation}
  F^{\dagger} = \frac{1}{\sqrt{N}} \sum_{\mu,\nu=0}^{N-1} \exp[-i\frac{2\pi \mu\nu}{N}] \ketbra{\mu}{\nu} \, \text{,}
\end{equation}
with $N$ being the finite dimension of the Hilbert space.

Please note the \eqref{eq:FourierCanonicalRelations} is valid in normalized (``natural'') units
where ``time'' and ``frequency'' are in fact respectively
\term{samples} and \emph{cycles/samples rate},
in a similar sense as in digital signal processing theory
\parencite[pp. 469, 490]{Signal}.

In SI units, the \eqref{eq:FourierCanonicalRelations} is replaced by
\begin{gather}
  \label{eq:SI_Fourier:Omega}
    \hat{\Omega} = \frac{2\pi}{N(\delta T)^2} F \hat{T} F^{\dagger} = \frac{2\pi N}{\qty(\Delta T)^2} F \hat{T} F^{\dagger} \\
  \label{eq:SI_Fourier:T}
    \hat{T} = \frac{2\pi}{N(\delta\Omega)^2} F^{\dagger} \hat{\Omega} F = \frac{2\pi N}{\qty(\Delta\Omega)^2} F^{\dagger} \hat{\Omega} F
  \, \text{,}
\end{gather}
where $\delta T$ (and analogously $\delta\Omega$)
is the size of a ``temporal sample'', or the size of a discrete
time step in the clock, and $\Delta T = N\delta T$ the range of the clock.
For example,
$\delta T = \text{1 hour}$ and $\Delta T=12\;\text{hours}$
for a common clock (hours hand) in our everyday life.

It holds
\begin{gather}
  \delta\Omega \delta T = \frac{2\pi}{N} \, \text{;} \quad
  \Delta\Omega \Delta T = 2\pi N \, \text{.}
\end{gather}

A benefit of finite-dimensional systems is the potential implementation on a finite array of
qubits in a quantum computer. The use of Discrete Fourier Transform extends the overlap
with technology and engineering to the domain of signal processing \parencite{FiniteHilb}.
In \emph{ordinary} quantum mechanics, the Fourier transform (discrete or continuous)
is generally used
to associate wavefunctions in position and momendum space
(whereas time and frequency are \emph{not} operators),
while in communication engineering it is used to convert signals
from the time to the frequency domain and vice versa.
Thanks to the introduction of the Hilbert space $\hilb{H}_T$,
the interpretation in terms of time and frequency
(or time and energy, up to a factor $\hbar$)
is applicable to quantum theory as well, not only formally
i.e. not in the sense of a mere operation among (``classical'') parameters;
but in the sense of conversion between representations of the
same quantum state vector with respect to different eigenbasis,
in full analogy with position and momentum in $\hilb{H}_S$.

\subsection{The $1 + 1$ qubit experiment \parencite{Moreva:synthetic, Moreva:illustration}}

In \cite{Moreva:illustration}, the frequency operator $\hat{\Omega} = H_T / \hbar$
is given by the \eqref{eq:MorevaOmegaT}. With respect to the polarization basis
$\qty{\ket{H}, \ket{V}}$ it is represented in the the matrix form
\begin{equation}
  \hat{\Omega} \repr {
    i\omega
    \begin{pmatrix}
      0 & 1 \\
     -1 & 0
    \end{pmatrix}
  } \, \text{.}
\end{equation}

The spectrum of $\hat{\Omega}$ is $\qty{-\omega, \omega}$.
Therefore, it's $N=2$ and $\delta\Omega = 2\omega$ in the sense of
\eqref{eq:SI_Fourier:T}.
We can thus derive the time operator matrix:
\begin{equation}
  \hat{T}
  \repr
  \frac{\pi}{4\omega^2} F^{\dagger} \Omega F
  =
  \frac{i\pi}{8\omega}
  \begin{pmatrix}
    1 & 1 \\
    1 & -1
  \end{pmatrix}
  \begin{pmatrix}
    0 & 1 \\
   -1 & 0
  \end{pmatrix}
  \begin{pmatrix}
    1 & 1 \\
    1 & -1
  \end{pmatrix}
  =
  \frac{\pi}{4\omega}
  \begin{pmatrix}
    0 & -i \\
    i &  0
  \end{pmatrix}
  \,\text{.}
\end{equation}
We notice that time is not diagonal in the polarization basis.
It can be diagonalized with:
\begin{equation}
  R^{\dagger} T R
  =
\frac{\pi}{4\omega}
\begin{pmatrix}
  -1  & 0 \\
  0   & 1
\end{pmatrix}
\,\text{,}
\end{equation}
and $R$ being the matrix of eigenvectors of T as columns
\begin{equation}
  R
  =
  \frac{1}{\sqrt{2}}
  \begin{pmatrix}
    i & -i \\
    1 & 1
  \end{pmatrix}
  \,\text{.}
\end{equation}

Therefore the clock can measure only two times: $-\frac{\pi}{4\omega}$ and $\frac{\pi}{4\omega}$.