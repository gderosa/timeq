\section{Introduction}

In Section \ref{sec:trel} it was observed that time is the only observable quantity
(even if it's not a quantum observable \emph{strictu sensu})
of which all other observables and the state of the system itself are a function.
This justifies the very concept of \term{time evolution}.
Taking two other observables \emph{in general} there is no functional dependency
between one another.
However, this can be true in some particular cases.
An extreme example is a bipartite system,
the two subsystems of which happen to be
perfectly entangled i.e. $ \ket{\Psi} = \sum_{j} \ket{j}_{A} \ox \ket{\psi_{j}}_{B} $,
where $\qty{\ket{j}}$ is an eigenbasis of some observable in system A.
If we regard this observable as ``the position of the hand of a clock'',
the state of system B is functionally dependent on the ``parameter'' $j$,
which is not \emph{just a parameter} but also an index labeling
eigenvalues and eigenvectors of an actual quantum observable
with its own Hermitian operator.
Thus the relation between quantum systems A and B effectively
reproduces a ``time evolution'' of system B,
while the whole bipartite system
is stationary, in state
$\ket{\Psi}$.
There is no logical requirement that $\ket{\Psi}$
depends, other than trivially,
upon any further external parameter.
The stationary state $\ket{\Psi}$ embeds the whole ``history''
of subsystem B.

The Page and Wootters model of ``evolution without evolution'',
first proposed in 1983,
is essentially based on the above ideas \parencite{PageWootters}.
Circa three decades after the first publication by
Don Page and William Wootters, the model was further developed
in particular by Lloyd, Giovannetti and Maccone,
with emphasis on addressing technical issues which have been raised meanwhile
\parencite{Lloyd:Time}.
In this formulation,
in addition to the ordinary Hilbert space of the system,
an extra Hilbert space $\mathcal{H}_T$ is considered,
where time is an observable
represented by a self-adjoint operator
whose properties are similar to the ones of position
in ordinary quantum mechanics.

In this language, the ordinary Hilbert space can be labeled $\mathcal{H}_S$;
and we consider the product space $\mathcal{H}_T \otimes \mathcal{H}_S$ as
the space in which both time and position are observables, and they act as
$\hat{t} \otimes \idop_S$ and $\idop_T \otimes \hat{x}$
respectively.

%% REMOVED: a relativistic chapter was created since. The intro chap. also deals w/ relativity.
% \begin{remark}
%   In a more ``relativistic friendly'' notation, we may set
%   $\hat{x_0} = c\hat{t}$ and consider
%   $\hat{x_0} \otimes \idop_S$ and $\idop_T \otimes \hat{x_1}$,
%   and so on. An interesting development may be expressing
%   Lorentz transformations as unitary transformations in
%   $\mathcal{H}_T \otimes \mathcal{H}_S$.
% \end{remark}

The Page and Wootters mechanism is rooted in the ``problem of time''
in quantum cosmology.
In principle, an appropriate system (a ``clock'') can be identified in such a way
to be described in $\mathcal{H}_T$, while $\mathcal{H}_S$ describes
the quantum states of \emph{the rest of the universe} \parencite{Marletto:Evolution}.

As explained in \cite{Lloyd:Time, Maccone:Pauli}, the overall Hamiltonian,
encompassing both position and time as observables, is given by
\begin{equation}\label{eq:pwHamiltonian}
  \hat{\mathbb{J}} = \hbar\hat{\Omega}\ox\idop_S + \idop_T\ox\hat{H}_S \,\text{,}
\end{equation}
while the \term{Wheeler-DeWitt equation} holds:
\begin{equation}\label{eq:Wheeler-DeWitt}
  \hat{\mathbb{J}}\dket{\Psi} = 0 \,\text{,}
\end{equation}
describing a \emph{static} universe, where evolution is only
in terms of relations between parts of a multipartite system
(a ``clock'' and ``the rest'').

Please note the special notation $\dket{\Psi}$ (double angle bracket),
to indicate a state vector in the ``larger'' product space $\pwspace$,
as opposed to vectors belonging to $\hilb{H}_S$ (or $\hilb{H}_T$) only.

Using the $T$ representation in $\hilb{H}_T$,
and comparing \eqref{eq:pwHamiltonian} and \eqref{eq:Wheeler-DeWitt}:
\begin{equation}\label{eq:schrod_from_pw}
  0 = \qty(\hbar\hat{\Omega}\ox\idop_S + \idop_{T}\ox\hat{H}_S)\dket{\Psi}
    \repr -i\hbar\pdv{t}\ket{\psi(t)}_{S} + \hat{H}_S\ket{\psi(t)}_{S}
    \,\text{,}
\end{equation}
we recover the usual form of the Schr\"{o}dinger eqaution in $H_S$.
Details of this derivation are in the reference aforementioned.

Here $\hat{\Omega}$ can be seen as a ``frequency operator''
represented as $-i\pdv{t}$ and having as eigenfunctions
those functions evolving in time with a phase factor $e^{i \omega t}$ only.

Canonical commutation relation holds and can be easily verified
between $\hat{t}$ and $\hat{\Omega}$
i.e. $[\hat{t}, \hat{\Omega}] = i$,
therefore $\hbar\hat{\Omega}$ can be seen as the ``linear momentum''
in the Hilbert space of time.

From another point of view, $\hbar\hat{\Omega}$ is the ``hamiltonian'' of $\hilb{H}_T$,
in that it plays a similar role of $H_S$ in the construction of
$\hat{\mathbb{J}}$ in \eqref{eq:pwHamiltonian}. This dual role doesn't hold
for $H_S$. 

This ambiguity is related to the asymmetry of space and time in non-relativistic
mechanics and can be expressively synthetised in the below:
{
  %% https://tex.stackexchange.com/a/232874
  %% https://tex.stackexchange.com/a/2836
  \begin{table}[h!]
    \parbox{.45\linewidth}{
      \centering
      \begin{tabular}{c|c}
        $\hilb{H}_T$        & $\hilb{H}_S$  \\
        \hline
        \hline
        $\hat{t}$           & $\hat{x}$     \\
        \hline
        $\hbar\hat{\Omega}$ & $\hat{p}$     \\
        \hline
        $?$                 & $\hat{H}$
      \end{tabular}
      {\caption{
        Operators in the two Hilbert spaces,
        with emphasis on the algebraic relation
        to other operators in the same space.
      }\label{op_comparison_alg}}
    }
    \hfill
    \parbox{.45\linewidth}{
      \centering
      \begin{tabular}{c|c}
        $\hilb{H}_T$        & $\hilb{H}_S$  \\
        \hline
        \hline
        $\hat{t}$           & $\hat{x}$     \\
        \hline
        $\hbar\hat{\Omega}$ & $\hat{H}$     \\
        \hline
        $?$                 & $\hat{p}$
      \end{tabular}
      {\caption{
        Operators in the two Hilbert spaces,
        with emphasis on the role in the
        Page--Wootters ``hamiltonian'' (eq. \ref{eq:pwHamiltonian}).
      }\label{op_comparison_J}}
    }
  \end{table}
}

It would be interesting to a study relativistic extension of the
Page and Wootters model that allows, for example, the derivation of the Klein-Gordon
equation, thus eliminating the asimmetry between
$\hilb{H}_T$ and $\hilb{H}_S$ (and momentum and energy as well).

In the bipartite universe, physical kets $\dket{\Psi}$ have a Schmidt decomposition
made up of
eigenstates of $\hat{\Omega}$ in $\hilb{H}_T$
entangled, respectively, with
eigenstates of $\hat{H}_S$ in $\hilb{H}_S$;
or eigenstates of $\hat{t}$ in $\hilb{H}_T$
entangled with time-evolved spatial states $\ket{\psi(t)}_S$ in $\hilb{H}_S$
(according to the evolution that is embedded in $\hat{\mathbb{J}}$):
\begin{equation}
  \dket{\Psi} = \int d\mu(\omega) \ket{\omega}_{T}\ox\ket{\psi(\omega)}_{S} = \int dt \ket{t}_{T} \ox \ket{\psi(t)}_{S}\,\text{.} 
\end{equation}
In principle, different and less obvious decompositions are possible for other purposes.
