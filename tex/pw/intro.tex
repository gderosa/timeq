\section{Introduction}

To introduce the Page--Wootters model, let us first consider
a bipartite system $\hilb{H}_A \ox \hilb{H}_B$,
the two subsystems of which have finite dimension\footnote{
  As it will be clearer later in the Chapter,
  this example is simplified for the sake of the argument: in principle
  the system may be infinite-dimensional;
  the sum may be replaced by an integral (continuous spectrum),
  and the vector in the tensor-product space may be not normalizable (an improper vector).
}$N$,
and are completely
entangled, i.e.
\begin{equation}\label{eq:pw:finite-entanglement}
  \hilb{H}_A \ox \hilb{H}_B \ni \dket{\Psi}
  =
  \frac{1}{\sqrt{N}} \sum_{n=0}^{N-1} \ket{\tau_{n}}_{A} \ox \ket{\psi_{n}}_{B} \text{.}
\end{equation}
Here
a ``double angle braket'' notation $\dket{\Psi}$ has been adopted
for vectors in the tensor product space $\hilb{H}_A \ox \hilb{H}_B$.
We assume $\ket{\psi_{n}}_{B}$ are all \emph{normalized} in $\hilb{H}_B$;
and
$\qty{\ket{{\tau}_n}}$ is an orthonormal eigenbasis of some observable (say, $T$) in system A.
Let also $\dket{\Psi}$ be \emph{stationary}, meaning it is an eigenstate
of the overall Hamiltonian $\mathbb{J}$ in $\hilb{H}_A \ox \hilb{H}_B$:
$$
  \mathbb{J} \dket{\Psi} = \epsilon \dket{\Psi} \text{.}
$$
Finally, we assume that the two subsystems $A$ and $B$ are \emph{non interacting},
meaning that the global Hamiltonian $\mathbb{J}$ can be expressed as a sum of two terms
which only act on the respective subspaces $\hilb{H}_A$ and $\hilb{H}_B$
(and there is no explicit ``interaction term''). In other words,
$\mathbb{J}$ can be expressed as
$$
  \mathbb{J} = H_A \ox \idop_B + \idop_A \ox H_B \, \text{.} 
$$
Therefore, with the ordinary notion of ``time as a paramerter'',
the evolution of system B will only depend on ``its own'' Hamiltonian $H_B$, namely
\begin{equation}\label{eq:pw:ordinary_B}
  \ket{\psi(t)}_B = e^{- i \hbar t / H_B } \ket{\psi(0)}_B \text{.}  
\end{equation}
 
In the Page--Wootters model, time evolution is based on an internal entanglement
relation among subsystems of an isolated system,
including the case where such isolated system is the whole universe \parencite{PageWootters}.
The overall system
---as the one described by $\hilb{H}_A \ox \hilb{H}_B$ in \eqref{eq:pw:finite-entanglement}---
is stationary and there is no notion
of time external to it.
This is also known as the
``timeless'' approach \parencite{Marletto:Evolution}.
\citereset
Within this approach,
a suitable subsystem is chosen within the ``universe'' so that it acts as
\emph{clock} for the \emph{rest} of it, in the sense that
there is an observable $T$ (as in the above example, in system $A$)
that can be used to
describe time evolution ``without evolution''\footnote{
  The expression ``evolution without evolution''
  is directly quoted from the title of \cite{PageWootters},
  and recalled in \cite{Marletto:Evolution}.
}
of system $B$
and is canonically conjugate to the Hamiltonian $H_A$.

Please note, while the system of interest (for its time evolution) is system $B$,
the time operator $T$ is defined in system $A$ ---and canonically conjugate to $H_A$, not $H_B$.
In other words, the time operator is defined in \emph{another} Hilber space, therefore
the Pauli objection no longer applies. Entanglement between the two systems is necessary
so the ``time'' measured by a clock in one system is relevant to the evolution of the other.

Qantitatively, within the example of eq.~\eqref{eq:pw:finite-entanglement},
the observable $T$ in $\hilb{H}_A$
has eigenvalues $\tau_n$ which can be interpreted as possible
instants of time in the evolution of $\ket{\psi}_B$, in the sense that
\begin{equation}\label{eq:pw:discrete_approx}
  \Big[ \ket{\psi(t)}_B \Big]_{t=\tau_n} \simeq \,\,\,\, \ket{\psi_n}_B \, \text{,}
\end{equation}
where $\ket{\psi(t)}_B$ is the same of~\eqref{eq:pw:ordinary_B} i.e.
the evolution one would obtain in ``ordinary'' way
by simply resolving the Schr\"{o}dinger equation in system B, with Hamiltonian $H_B$.
Please note in eq.~\eqref{eq:pw:discrete_approx} we do not use strict equality:
in its original formulation the Page--Wooters model is still based on a continuous
notion of time, and the same applies to more recent developments \parencite{Lloyd:Time}:
whether a discrete formulation
brings a good approximation
will be verifyied numerically.

An intuitive description may be as follows.
From the perspective of system B, time is a parameter $t$, however it can have values
$\tau_n$ (or \emph{all} real values, in a continuous model)
which are also eigenvalues of an observable $T$ defined in \emph{another system} B,
whith which it is entangled. Such entanglement relation estabishes a correspondence
between ``instants of time'' (eigenvalues and eigenstates of $T$ in system A)
and (``evolved'') states in system B. Thus, time can be regarded as a quantum observable,
although defined in a different Hibert space than the one of the Hamiltonian of interest.
As $H_B$ and $T$ acts on two different Hilbert spaces, the Pauli objection is no longer valid
(how canonical conjugation and uncertainty relations hold will be clarer in Sec.~\ref{sec:pw:uncertainty}).

A slightly different notation than $\hilb{H}_A$ and $\hilb{H}_B$ (and their tensor product)
is useful, once
the physical interpretation of the two spaces is set out by this introduction to the model.
Some authors use TODO

TODO position of hands not strictly time.

\subsection*{Notation}

In this and following chapters, when treating quantum time models,
$\hilb{H}_S$ will indicate the Hilbert space of ordinary quantum mechanics.
$\hilb{H}_T$, an extra space where a time operator $\hat{T}$ is introduced. The tensor
product space $\pwspace$
will often be referred to as well.
Bras and kets in this tensor product space will be indicated with a special double-angle-bracket
notation, as in $\dket{\Psi}$.

${\mathbb{J}}$ (blackboard bold) indicates an  operator defined in
a larger space than the one of standard quantum mechanics,
typically the product $\pwspace$.

The symbol `$\repr$', as in $\hat{A} \repr \mqty(a&b\\c&d)$, $\ket{\psi} \repr \mqty[\alpha \\ \beta]$
means: representation with respect to a particular basis (as opposed to intrinsic equality `$=$').

The symbol `$\eqbydef$'
means: equal by definition, equal by settings i.e. postulated and not derived logically.
