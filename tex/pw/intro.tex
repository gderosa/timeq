% \subsection*{\it Structure of the Chapter}

This Chapter
will be devoted to outlining the Page and Wootters theory
of time,
also known as the model of ``evolution without evolution''.
Existing literature will be reviewed,
with some original considerations,
in particular with regards to
time--energy uncertainty relations,
finite-dimensional systems and discrete clocks
(Sections \ref{sec:pw:theory_first}--\ref{sec:pw:theory_last}).

A second part
(Sections \ref{sec:pw:apps_first}--\ref{sec:pw:apps_last})
will be dedicated to
applications of the Page--Wootters model, starting with a discussion of existing
experiments (or ``experimental illustrations''),
then implementing numerical simulations, with a larger number of clock levels.
Finally, the Page and Wootters model will be compared to the results of
detection and time-of-arrival models
based on non-unitary evolution.

\section{The Model}\label{sec:pw:theory_first}

To introduce the Page--Wootters model, let us first consider
a bipartite system $\hilb{H}_T \ox \hilb{H}_S$,
and a state vector $\dket{\Psi} \in \pwspace$.
%% Move to finite-dim section? is this just a redundant repetition?
% the two subsystems of which have finite dimension\footnote{
%   In general, within the model,
%   the system may as well be infinite-dimensional;
%   the sum may be replaced by an integral (continuous spectrum);
%   and the vector in the tensor-product space may be not normalizable (an improper vector).
%   Nonetheless, a finite-dimensional Hilbert space is sufficient to construct several applications of interest.
% } $N$ each.
% Let us also consider the following entangled state:
% \begin{equation}\label{eq:pw:finite-entanglement}
%   \hilb{H}_T \ox \hilb{H}_S \ni \dket{\Psi}
%   =
%   \frac{1}{\sqrt{N}} \sum_{n=0}^{N-1} \ket{\tau_{n}}_{T} \ox \ket{\psi_{n}}_{S} \text{.}
% \end{equation}
A ``double angle bracket'' notation $\dket{\Psi}$ is used here
for vectors in the tensor product space.
% We assume that $\ket{\psi_{n}}_{S}$ are all \emph{normalized} in $\hilb{H}_S$;
% and
% $\qty{\ket{{\tau}_n}}$ is an orthonormal eigenbasis of some observable (say, $\op{T}$)
% in $\hilb{H}_T$.
% Furthermore,

Let $\dket{\Psi}$ be \emph{stationary}, meaning it is an eigenstate
of an overall Hamiltonian $\op{\mathbb{J}}$ in $\hilb{H}_T \ox \hilb{H}_S$:
\begin{equation}\label{eq:Wheeler-DeWitt}  %%  eq:Wheeler-DeWitt + eq:pwHamiltonian
  \op{\mathbb{J}} \dket{\Psi} = 0 \text{.}
\end{equation}
Finally, we assume that the two subsystems
are \emph{non interacting},
meaning that the global Hamiltonian $\mathbb{J}$ can be expressed as a sum of two terms
which only act on the respective subspaces $\hilb{H}_T$ and $\hilb{H}_S$
(and there is no explicit ``interaction term''). In other words,
$\op{\mathbb{J}}$ can be expressed as
\begin{equation}\label{eq:pwHamiltonian}
  \op{\mathbb{J}} = \op{H}_T \ox \idop_S + \idop_T \ox \op{H}_S \, \text{.}
\end{equation}
Therefore,
the evolution of the subsystem described by $\hilb{H}_S$
will only depend on ``its own''
Hamiltonian $H_S$ (assumed as time-independent), namely
\begin{equation}\label{eq:pw:ordinary_S}
  \ket{\psi(t)}_S = e^{- i \op{H}_S t / \hbar} \ket{\psi(0)}_S \text{.}
\end{equation}
Note the familiar single-bracket notation $\ket{\psi(t)}_S$ for a ket in
one of the two parts (subspaces) of $\pwspace$ ($\hilb{H}_S$ in this case).

The overall bipartite system is \emph{isolated}. A notable example of isolated system is, of course, the universe.
This simple observation is relevant because,
in its original formulation \parencite{PageWootters}, $\pwspace$ is regarded
as a partition of the whole universe, which is considered, overall, in a stationary state.
% There cannot be notion of time external to it.

This is known as the
``timeless'' approach \parencite{Marletto:Evolution}.
\citereset
A suitable subsystem ---described by $\hilb{H}_T$ within this framework---
is chosen in the ``universe'' so that it acts as
\emph{clock} for the \emph{rest} of it, in the sense that
there is an observable $\op{T}$
which can be used to
describe the time evolution ``without evolution''
of the other subsystem\footnote{
  A slightly different notation
  uses indeed $\hilb{H}_C$ and $\hilb{H}_R$ (instead of $\hilb{H}_T$ and $\hilb{H}_S$)
  to indicate the Hilbert spaces of the clock~(C) and the rest~(R) of the ``universe''
  (or isolated system):
  see, for example, \cite{Marletto:Evolution}.
}
$\hilb{H}_S$.

The operator $\op{T}$ can be regarded as
the \emph{time operator}. It is defined in $\hilb{H}_T$ (not $\hilb{H}_S$),
and canonically conjugate to the ``Hamiltonian'' $\op{H}_T$ (not $\op{H}_S$).
In other words, the time operator is defined in a different Hilbert space than the ``system of interest'' $\hilb{H}_S$,
therefore
the Pauli objection (Sec.~\ref{proof}) no longer applies.

As we will see below, in the Page--Wootters model, time evolution is based on an internal \emph{entanglement}
relation among elements of subspaces $\hilb{H}_S$ and $\hilb{H}_T$ respectively.

% Quantitatively,
% the observable $T$ in $\hilb{H}_T$
% has eigenvalues $\tau_n$
% (assuming a discrete spectrum to simplify notation)
% which can be interpreted as possible
% instants of time in the evolution of $\ket{\psi}_S$, in the sense that
% \begin{equation}\label{eq:pw:discrete_Tpprox}
%   \Big[ \ket{\psi(t)}_S \Big]_{t=\tau_n} = \,\,\,\, \ket{\psi_n}_S \, \text{,}
% \end{equation}
% where $\ket{\psi(t)}_S$ is given in~\eqref{eq:pw:ordinary_S} i.e.
% the evolution one would obtain in ``ordinary'' way
% by resolving the Schr\"{o}dinger equation in $\hilb{H}_S$.
% Please note:
% in its original formulation, the Page--Wootters model is based on a continuous
% notion of time, and the same applies to more recent developments \parencite{Lloyd:Time}.
% A discrete formulation, as in Eq.~\eqref{eq:pw:discrete_Tpprox},
% will be verified numerically in Sections~\ref{sec:beyondMoreva}--\ref{sec:multiLevelClock}.

An intuitive description may be as follows.
From the perspective of the system described by $\hilb{H}_S$,
time is a parameter $t$;
however $t$ can have values
which are also eigenvalues of an observable $\op{T}$ defined in \emph{another} space $\hilb{H}_T$,
with which it is entangled. Such entanglement relation establishes a correspondence
between ``instants of time'' (eigenvalues and eigenstates of observable $\op{T}$)
and (``evolved'') states in $\hilb{H}_S$.

%%
%% \section{Development of the Formalism}\label{sec:pw:formalism}\label{sec:pw:theory_first}
%%

About three decades after the original work
by Page and Wootters \parencite{PageWootters},
the formalism was developed further
by Giovannetti, Lloyd and Maccone \citereset\parencite{Lloyd:Time}.
Most technical details introduced below are based
on that paper.

First of all, the ``conventional'' state $\ket{\psi(t)}_S$ in $\hilb{H}_S$
can be obtained from $\dket{\Psi}$ via partial inner product\footnote{
  For the partial inner product,
  see Definition \ref{def:pBra} and \ref{def:pKet},
  and \cite[\s 1.3.3]{QMT_Jacobs}.
}
with a time eigenstate $\prescript{}{T}{\bra{t}}$:
\begin{equation}\label{eq:pw:standard_evolution}
  \ket{\psi(t)}_S = \prescript{}{T}{\bradket{t}{\Psi}} \, \text{.}
\end{equation}
The function $ t \rightarrow \ket{\psi(t)}_S \; $ is the
``wavefunction in time representation'', in analogy
with the wavefunction in position representation of standard quantum mechanics.
The state vector $\dket{\Psi}$ is univocally identified by the ``coefficients'' $\ket{\psi(t)}_S$
with respect to the basis $\setof{\ket{t}_T}$:
\begin{equation}\label{eq:pw:}
  \dket{\Psi} = \int \dd{t} \ket{t}_T \ox \ket{\psi(t)}_S \, \text{,}
\end{equation}
somewhat similar to the well known expression $\ket{\psi}_S = \int \dd{x} \psi_{S}(x) \ket{x}_S$
which defines the position representation.

Under this time representation%\footnote{  % no, we're not...
%   Please note we are using use the following sign convention:
%   \begin{align*}
%     \op{\Omega} &\repr[T] i\pdv{t} \text{,} \\
%     \op{H}_T = -\hbar\op{\Omega} &\repr[T] -i\hbar\pdv{t} \text{;}
%   \end{align*}
%   as opposed to the sign convention used in \cite{Lloyd:Time}, and by other authors, where
%   \begin{align*}
%     \op{\Omega} &\repr[T] -i\pdv{t} \text{,} \\
%     \op{H}_T = \hbar\op{\Omega} &\repr[T] -i\hbar\pdv{t} \text{.}
%   \end{align*}
% }
(or $T$ representation), the operator $\op{H}_T = -\hbar\op{\Omega}$,
canonically conjugate to the time operator $\op{T}$, is expressed as $-i\hbar\pdv{t}$,
and the commutation relation
$[\op{T}, \op{H}_T] \repr[T] \comm{t, -i\hbar\pdv{t}}{} = i\hbar$
can be proven
immediately, similarly to the well known position-momentum relation
$\comm{\op{x}}{\op{p}} \repr[x] \comm{x}{-i\hbar\pdv{x}} = i\hbar$.
Thus, $\hbar\op{\Omega}$ can be seen as similar to a ``linear momentum''
in the Hilbert space of time.

%% \hbar\Omega is -E ... and \Omega \repr -i t-derivative ... and [T, \Omega] = i .

Using the $T$ representation in $\hilb{H}_T$,
and comparing \eqref{eq:pwHamiltonian} and \eqref{eq:Wheeler-DeWitt}:
\begin{equation}\label{eq:schrod_from_pw}
  0 = \qty(\hbar\op{\Omega}\ox\idop_S + \idop_{T}\ox\op{H}_S)\dket{\Psi}
    \repr -i\hbar\pdv{t}\ket{\psi(t)}_{S} + \op{H}_S\ket{\psi(t)}_{S}
    \,\text{,}
\end{equation}
we recover the usual form of the Schr\"{o}dinger equation in $H_S$
(see also \cite[709--710]{Wootters:Loyola}).

Here $\op{\Omega}$ can be seen as a ``frequency operator''
represented as $-i\pdv{t}$ and having as eigenfunctions
those functions evolving in time with a phase factor $e^{i \omega t}$ only.

From another point of view, $\op{H}_{T} = \hbar\op{\Omega}$ is the ``Hamiltonian'' of $\hilb{H}_T$,
in that it plays a similar role of $H_S$ in the construction of
$\op{\mathbb{J}}$ in \eqref{eq:pwHamiltonian}.

Such analogies among operators in $\hilb{H}_T$ and $\hilb{H}_S$ are summarized in Table \ref{tbl:op_comparison_pw}.

{
  %% https://tex.stackexchange.com/a/232874
  %% https://tex.stackexchange.com/a/2836
  \begin{table}
    \centering
    \begin{tabular}{l|l|l}
      & \multicolumn{1}{c|}{$\hilb{H}_T$}                 & \multicolumn{1}{|c}{$\hilb{H}_S$}       \\
      \hline
      \multirow{2}{11em}{Canonical commutation relation}
      & $\op{T}$                                          & $\op{x}$                            \\
      %
      & $\op{H}_T \repr[T] -i\hbar\pdv{t} $  & $\op{p} \repr[x] -i\hbar\pdv{x}$       \\
      \hline
      Hamiltonian
      & $\op{H}_T$                                        & $\op{H_S}$
    \end{tabular}
    {\caption{
      Analogies between observables in the two Hilbert spaces.
    }\label{tbl:op_comparison_pw}}
  \end{table}
}

% It would be interesting to a study relativistic extension of the
% Page and Wootters model that allows, for example, the derivation of the Klein-Gordon
% equation, thus eliminating the asimmetry between
% $\hilb{H}_T$ and $\hilb{H}_S$ (and momentum and energy as well).

In the bipartite universe, physical kets $\dket{\Psi}$ can have a Schmidt decomposition
made up of
eigenstates $\ket{\omega}_{T}$ of $\op{H}_T$ in $\hilb{H}_T$
entangled with
eigenstates $\ket{\tilde{\psi}(\omega)}_{S}$ of $\op{H}_S$ in $\hilb{H}_S$
(the eigenvalue is $\hbar\omega$ for both $\op{H}_T$ and $\op{H}_S$ respectively);
or eigenstates $\ket{t}_{T}$ of $\op{T}$ in $\hilb{H}_T$
entangled with time-evolved spatial states~$\ket{\psi(t)}_S$ in $\hilb{H}_S$:
\begin{equation}\label{eq:pw:time_freq_nomu}
  \dket{\Psi} =
    \int \dd{\omega} \, \ket{\omega}_{T}\ox\ket{\tilde{\psi}(\omega)}_{S} =
    \int \dd{t} \, \ket{t}_{T} \ox \ket{\psi(t)}_{S} \text{.}
\end{equation}
%
% \begin{tcolorbox}
%   Eq.~\eqref{eq:pw:time_freq_mu} here is based on \cite[eq. 10]{Lloyd:Time}.
%   The paper, in the first integral,
%   would use a ``measure'' $\dd{\mu(\omega)}$, instead of simply $\dd{\omega}$.
%   This is to take into account that not all values of $\omega$ necessarily contribute
%   to the integral: in particular if $\hbar\omega$ is not in the spectrum of $\op{H}_S$,
%   the corresponding integrand would be zero.
%
%   So,
%   I can certainly remove $\dd{\mu(\omega)}$ as requested\footnote{
%     I actually even wrote $\dd{\omega}\mu(\omega)$ in my previous draft,
%     somewhat incorrectly:\linebreak
%     technically I should have written $\dd{\mu(\omega)}$,
%     or $\dd{\omega}\mu'(\omega)$ (with a ``prime'')
%     so that $\dd{\mu(\omega)} = \frac{\dd{\mu}}{\dd{\omega}}\dd{\omega} = \mu'(\omega) \dd{\omega}$
%     to match the paper,
%     but that's probably not the most important point here\dots
%   },
%   and simply write $\dd{\omega}$, but at least I need to specify what follows:
%   %
%   that, while it is $\braket{\psi(t)}_S = 1 \ \forall t \in \mathbb{R}$
%   (unitary evolution, as in standard~QM),
%   one should not necessarily expect that
%   $\ket{\phi(\omega)}_S$ has norm $1$ for every real value of $\omega$.
%   Actually, given that the spectrum of $\op{H}_S$ is bounded from below,
%   certainly there are values of $\omega$ for which $\ket{\phi(\omega)}_S = \nullvector$.
%
%   In other words, we can certainly remove $\mu(\omega)$, but it must be ``absorbed''
%   in the properties of $\ket{\phi(\omega)}_S$.
% \end{tcolorbox}

Using the relations
$
\, \prescript{}{T}{\braket{t}{t'}}_T = \delta(t-t') \,
$
and
$
  \,
  \ket{\omega}_T =
  \frac{1}{\sqrt{2\pi}} \int\dd{t}\E^{\iu\omega t}\ket{t}_T
  \,
$
(see also \cite[2]{Lloyd:Time}),
it follows that\footnote{%
  In analogy with the well-known
  inner product of position and momentum eigenvectors in standard quantum mechanics:
  $\braket{\vb{x}}{\vb{p}} = (2\pi\hbar)^{-3/2}\E^{\iu \vb{p} \vdot \vb{x}  / \hbar}$
  (or $\braket{x}{p} = \frac{\E^{\iu p x / \hbar}}{\sqrt{2\pi\hbar}}$ for one-dimensional systems).
  See, for example, \cite[126--127]{Ballentine}.
}
\begin{equation}\label{eq:pw:tscalaromega}
  \prescript{}{T}{\braket{t}{\omega}}_T =
  \frac{1}{\sqrt{2\pi}}\prescript{}{T}{\bra{t}} \int\dd{t'}\E^{-\iu\omega t'}\ket{t'}_T  =
  \frac{1}{\sqrt{2\pi}}\int\dd{t'}\E^{\iu\omega t'}\delta(t-t') =
  \frac{1}{\sqrt{2\pi}} \E^{\iu\omega t} \; \text{.}
\end{equation}

Note that,
while it is $\braket{\psi(t)}_S = 1 \ \forall t \in \mathbb{R}$ (unitary evolution,
as in standard quantum mechanics),
one should not necessarily expect that
$\ket{\tilde{\psi}(\omega)}_S$ has norm~$1$ for each real value of $\omega$.
In particular, if $\hbar\omega$ is not in the energy spectrum of the system,
then $\ket{\tilde{\psi}(\omega)}_S = \nullvector$, i.e. the vector does not contribute to the integral.
Given that the energy spectrum is bounded,
certainly there are values of $\omega$ for which this is the case.
With these considerations, Eq.~\eqref{eq:pw:time_freq_nomu}, with regards to the first integral, may also be reformulated as
\begin{equation}\label{eq:pw:time_freq_mu}
  \dket{\Psi} =
    \int \dd{\omega} \mu(\omega) \ket{\omega}_{T} \ox \ket{\phi(\omega)}_{S} \text{,}
\end{equation}
with $\ket{\phi(\omega)}_{S}$ normalized for any real value of $\omega$,
and $\mu(\omega)\ket{\phi(\omega)}_{S} = \ket{\tilde{\psi}(\omega)}_S$
--- see also \cite[eq.~(10)]{Lloyd:Time}.
This notation will be useful in the example of Sec.~\ref{sec:pw:ex-hamiltonian-eigenstate}.

\subsubsection{Non-zero eigenvalues}
Generalizing \eqref{eq:pwHamiltonian} and \eqref{eq:Wheeler-DeWitt}, let us consider the case when
$\dket{\Psi}$ is an eigenvector of $\op{\mathbb{J}}$ related to a general eigenvalue $\epsilon$
instead of zero. This brings to
\begin{equation}\label{eq:pw:non0e:first}
  \epsilon \dket{\Psi} = \qty(\hbar\op{\Omega} \ox \idop_S + \idop_T \ox \op{H}_S) \dket{\Psi} \text{.}
\end{equation}
Partial scalar product by $\prescript{}{T}{\bra{t}}$ on the left yields
\begin{equation}\label{eq:pw:nonzero-schrod-1}
  \epsilon\ket{\psi_{PW}(t)}_{S} = -i\hbar\pdv{t}\ket{\psi_{PW}(t)}_{S} + \op{H}_S\ket{\psi_{PW}(t)}_{S}
  \text{,}
\end{equation}
where we have defined $\psi_{PW}(t) \eqbydef \prescript{}{T}{\bradket{t}{\Psi}}$.
% This is a generalization of \eqref{eq:schrod_from_pw}.
Rearranging Eq.~\eqref{eq:pw:nonzero-schrod-1}:
\begin{equation}\label{eq:pw:nonzero-schrod-2}
   \qty(\op{H}_S - \epsilon \idop_{S}) \ket{\psi_{PW}(t)}_{S} = i\hbar\pdv{t}\ket{\psi_{PW}(t)}_{S}
   \text{,}
\end{equation}
which is, formally, the Schr\"{o}dinger equation
for the state $\ket{\psi_{PW}(t)}_{S}$ with
the Hamiltonian $\qty(\op{H}_S - \epsilon \idop_{S})$.
With the initial condition $\ket{\psi_{PW}(0)} \eqbydef \ket{\psi_0}$, we have therefore:
\begin{equation}\label{eq:pw:non0e:last}
  \ket{\psi_{PW}(t)}_S =
  \E^{-\frac{\iu \qty(\op{H}_{S} - \epsilon) t}{\hbar}} \ket{\psi_0}_S =
  \E^{\frac{\iu \epsilon t}{\hbar}} \E^{-\frac{\iu \op{H}_{S} t}{\hbar}} \ket{\psi_0}_S
  \text{,}
\end{equation}
hence:
\begin{equation}
  \E^{-\frac{\iu \op{H}_{S} t}{\hbar}} \ket{\psi_0}_S = \E^{-\frac{\iu \epsilon t}{\hbar}} \ket{\psi_{PW}(t)}_S
  \,\text{,}
\end{equation}
where we recognize $\E^{-\frac{\iu \op{H}_{S} t}{\hbar}} \ket{\psi_0}_S$ as the time evolution
from an initial state $\ket{\psi_0}_S$ at time $t = 0$,
under the Hamiltonian $\op{H}_S$,
in standard quantum mechanics.
If we define this evolved state as, simply, $\ket{\psi(t)}_S$, and recall the definition of $\ket{\psi_{PW}(t)}_S$,
we have
\begin{equation}
  \ket{\psi(t)}_S = \E^{-\frac{\iu \epsilon t}{\hbar}} \prescript{}{T}{\bradket{t}{\Psi}} \, \text{,}
\end{equation}
which, compared to Eq.~\eqref{eq:pw:standard_evolution},
shows that the projection of the eigenstate $\dket{\Psi}$ of $\op{\mathbb{J}}$
over an eigenbasis of time within the Page--Wootters model,
namely $\prescript{}{T}{\bradket{t}{\Psi}}$,
is compatible with the prediction $\ket{\psi(t)}_S$ of standard quantum mechanics
% $\E^{-\frac{\iu \op{H}_{S} t}{\hbar}} \ket{\psi_0}_S$
only \emph{up to a corrective phase} $\E^{-\frac{\iu \epsilon t}{\hbar}}$.
In other terms, with a non-zero eigenvalue $\epsilon$ of $\op{\mathbb{J}}$,
the model can still return the ordinary quantum mechanics evolution, but
with an ``energy shift'' of value $\epsilon$.
See also \cite[\it ``The Zero-eigenvalue'']{Lloyd:Time}.

\subsection{Example: Hamiltonian eigenstate}\label{sec:pw:ex-hamiltonian-eigenstate}

An extreme case for the distribution % $\ket{\phi(\omega)}_{S}$ in Eq.~\eqref{eq:pw:time_freq_nomu} ---or
$\mu(\omega)$ in \eqref{eq:pw:time_freq_mu} % ---
is, as an example, what in standard quantum mechanics would be the time evolution of an energy eigenstate
(or an eigenstate of $\op{H}_S$ in $\hilb{H}_S$, within the Page--Wootters formalism).
Let the eigenvalue be $E_0 \eqbydef \hbar\omega_{0}$;
and let the initial state $\ket{\psi_{0}(0)}_S = \ket{\psi_0^i}_S$ a related eigenstate,
such that $\op{H}_{S}\ket{\psi_0^i}_S = E_{0}\ket{\psi_0^i}_S$.
Then the time evolution is
\begin{equation}\label{eq:pw:eigenevol}
  \ket{\psi_{0}(t)}_S = \E^{-\frac{i}{\hbar}E_{0}t}\ket{\psi_0^i}_S = \E^{-i \omega_{0} t}\ket{\psi_0^i}_S \text{.}
\end{equation}

In Page--Wootters terms, the corresponding ``history vector'' is
\begin{equation}\label{eq:pw:Psi_0_t}
  \dket{\Psi_0} = \int \dd{t} \ket{t}_T \ox \ket{\psi_0(t)}_S \text{,}
\end{equation}
where the integrand is the tensor product of an eigenvector of $\op{T}$ in $\hilb{H}_T$
and the time-evolved state vector in $\hilb{H}_S$;
but $\dket{\Psi_0}$ can also be expressed as
\begin{equation}\label{eq:pw:Psi_0_omega__int}
  \dket{\Psi_0} = \int \dd{\omega} \, \ket{\omega}_{T}\ox\ket{\tilde{\psi}_{0}(\omega)}_{S} \text{,}
\end{equation}
in terms of eigenvectors of the ``Hamiltonians''
$\op{H}_T$ and $\op{H}_S$ in the two spaces respectively % $\hilb{H}_T$ and $\hilb{H}_S$ respectively
---see also Eq. \eqref{eq:pw:time_freq_nomu}.

As $\ket{\tilde{\psi}_{0}(\omega)}_{S}$ is the Fourier transform of $\ket{\psi_0(t)}_S$, and using Eq. \eqref{eq:pw:eigenevol}, it holds:
\[
  \ket{\tilde{\psi}_0(\omega)}_S = \frac{1}{\sqrt{2\pi}} \int \dd{t} \E^{iwt} \E^{-i\omega_{0}t} \ket{\psi_0^i}_S
  = \delta(\omega-\omega_0)\ket{\psi_0^i}_S \text{;}
\]
then, replacing into Eq.~\eqref{eq:pw:Psi_0_omega__int}:
\begin{equation}\label{eq:pw:Psi_0_omega}
  \dket{\Psi_0} = \ket{\omega_0}_{T}\ox\ket{\tilde{\psi}_{0}(\omega_0)}_{S} \text{,}
\end{equation}
%
showing that the frequency distribution in $\hilb{H}_T$ is in fact concentrated at $\omega = \omega_0$.

% $\dket{\Psi_0}$ is also an egienstate of $\idop_T \ox \op{H}_S$, related to the same eigenvalue $E_0$:
% \[
%   \qty(\idop_T \ox \op{H}_S) \dket{\Psi_0} = \int \dd{t} \ket{t}_T \ox E_0 \ket{\psi_{0}(t)}_S = E_0 \dket{\Psi_0} \text{.}
% \]

% In order to statisfy Eq.~\eqref{eq:Wheeler-DeWitt} and \eqref{eq:pwHamiltonian},
% $\dket{\Psi_0}$ must also be an eigenvector of $\hbar\op{\Omega} \ox \idop$,
% but related to the \emph{opposite} eigenvalue $-E_0 = -\hbar\omega_0$.

% %% REMOVE FROM HERE
% From Eq.~\eqref{eq:pw:time_freq_mu}, we can write, in particular:
% \begin{equation}
%   \dket{\Psi_0} =
%     \int \dd{\omega} \mu_0(\omega) \ket{\omega}_{T} \ox \ket{\phi_0(\omega)}_{S} \text{,}
% \end{equation}
% which expands $\dket{\Psi_0}$ over a continuous orthonormal eigenbasis
% $\setof{\ket{\omega}_{T} \ox \ket{\phi_0(\omega)}_{S}}$
% of $\op{\Omega} \ox \idop_S$.
% But $\dket{\Psi_0}$ is an eigenvector itself,
% for the eigenvalue $-\omega_0$,
% therefore the coefficients $\mu_0(\omega)$ must be
% \begin{equation}\label{eq:pw:mudelta}
%   \mu_0(\omega) = \delta\qty(\omega - \qty(-\omega_0) ) = \delta\qty(\omega + \omega_0) \text{.}
% \end{equation}

% Now, from \eqref{eq:pw:mudelta} and \eqref{eq:pw:time_freq_mu} it holds:
% \[
%   \dket{\Psi_0} = \int \dd{\omega} \delta(\omega + \omega_0) \ket{\omega}_T \ox \ket{\phi(\omega)}_S =
%     \ket{-\omega_0}_T \ox \ket{\phi(-\omega_0)}_S \text{,}
% \]
% %% TO HERE

% Using Eq. \eqref{eq:pw:tscalaromega} and \eqref{eq:pw:Psi_0_omega}, the \emph{evolution} (in $\hilb{H}_S$) is
% \begin{equation}
%   \ket{\psi(t)}_S = \prescript{}{T}{\bradket{t}{\Psi_0}} = \prescript{}{T}{\braket{t}{\omega_0}}_{T} \ket{\tilde{\psi}(\omega_0)}_S =
%     \frac{1}{\sqrt{2\pi}} \E^{-\iu\omega_0 t} \ket{\tilde{\psi}(\omega_0)}_S \text{.}
% \label{eq:pw:stat_eigenst_of_H_S}
% \end{equation}
% By evaluating and comparing Eq.~\eqref{eq:pw:stat_eigenst_of_H_S} at two arbitrary instants of time
% $t_0$~and~$t$,
% one easily gets
% $\ket{\psi(t)}_{S} = \E^{-\iu\omega_{0}(t-t_{0})} \ket{\psi(t_0)}_{S}$,
% which is the expected stationary evolution for an energy eigenstate in quantum mechanics.

\subsection{Normalizable vectors of $\pwspace$}
\label{sec:properpw}

A vector $\dket{\Psi}$ in $\hilb{H}_T \ox \hilb{H}_S$,
satisfying \eqref{eq:pwHamiltonian} and \eqref{eq:Wheeler-DeWitt},
encodes the whole (unitary) time evolution of a system.
\begin{equation}\label{eq:pwexpansion}
  \dket{\Psi} =
    \int \dd{t} \ket{t}_T \ox \ket{\psi(t)}_S =
    \int \dd{t}\dd[3]{\vec{r}} \ \Psi(t; \vec{r}) \; \ket{t}_T \ox \ket{\vec{r}}_S
    \,  \text{,}
\end{equation}
with $\ket{\vec{r}}_S$ eigenstate of the position observable in $\hilb{H}_S$,
so that $\Psi(t; \vec{r})$ can be regarded as the wavefunction in position representation of the system at time $t$,
in ordinary quantum mechanics terms.
We know $\setof{\ket{t}_T \ox \ket{\vec{r}}_S}$ is an orthonormal basis of $\hilb{H}_T \ox \hilb{H}_S$, therefore
\begin{equation}
  \norm{\dket{\Psi}}^2 =
    \int \dd{t}\dd[3]{\vec{r}} \ \abs{\Psi(t; \vec{r})}^2 =
    \int \dd{t} \int \dd[3]{\vec{r}} \ \abs{\Psi(t; \vec{r})}^2 =
    \int \dd{t} 1 \rightarrow +\infty
    \,  \text{,}
\end{equation}
which means that such $\dket{\Psi}$ is an \term{improper} vector of $\hilb{H}_T \ox \hilb{H}_S$.

Proper (i.e. normalizable) states are described in \citereset\cite{Lloyd:Time} as well, by replacing (or generalizing)
the \eqref{eq:pwexpansion} with
% [{\color{red} TODO}: consistent notation? See \url{https://drive.google.com/file/d/1Doi9HI_1qjux6gYsjX5anySZR5YpBKGb/view?usp=sharing}]
\begin{equation}\label{eq:pwphi}
  \dket{\Phi} =
    \int \dd{t} \phi(t) \ket{t}_T \ox \ket{\psi(t)}_S \, \text{.}
\end{equation}
If the function $\phi \in \mathscr{L}^2(\mathbb{R})$,
then $\dket{\Psi}$ is a proper element of the product space,
and $\norm{\dket{\Psi}}^2 = \int \dd{t} \abs{\phi(t)}^2$.

The case of non-normalizable $\dket{\Psi}$ in \eqref{eq:pwphi},
with normalized $\ket{\psi(t)}_S$ $\forall t \in \mathbb{R}$,
describes the unitary evolution, as seen throughout Chapter \ref{ch:pw}.
As observed in \cite{Maccone:QGR},
``%
  Quantum mechanics is formulated in terms of \emph{systems},
  typically limited in space but infinitely extended in time%
''.
If the state vector is \emph{conditioned} at a particular time $t$,
it holds $\norm{_{T}\bradket{t}{\Psi}}_S = \norm{\ket{\psi(t)}}_S = 1$,
meaning that, at each $t$,
\emph{the particle must certainly be in some (one) point in space}.

A normalized $\dket{\Psi}$ in the whole $\hilb{H}_T \ox \hilb{H}_S$,
instead,
can be interpreted as a total probability of~$1$ in both space and time combined.
It can be interpreted as describing an \term{event},
in that it
must certainly be in some point in space
\emph{and} time (in terms of outcome of an idealized measurement%\footnote{ %% Andreas...
%   Measurement in quantum mechanics requires the concepts
%   of state of the system
%   (and measurement apparatus)
%   \emph{before} and \emph{after} the measurement and, therefore, the existence
%   of an external time (external with respect of the Hilbert space of states).
%   This logically contradicts the foundation of the Page--Wootters model if
%   time itself is measured as a quantum observable. $\dket{\Psi} \in \pwspace$
%   embeds the whole history of a system and therefore cannot have a
%   ``before'' neither an ``after'' the measurement, ``when'' it collapses
%   into an eigenstate of time. The apparent contradicion is resolved
%   stressing that probability amplitues are intended in the sense of
%   \emph{conditional} probabilities e.g. \emph{provided that the particle
%   is in position} $x \in X$ (or the detector clicks)
%   what is the probability density amplitude of time being (the clock showing) $t$?
%   Consistently, the Bayes rule
%   (see, for example, \cite{Stat:Conditional})
%   is invoked in the following sections
%   and references.
% }
).
Such states have no correspondence to the full ``history'' of a system
(and, therefore, to its unitary evolution according to standard quantum mechanics).

Such notion of \term{event} was formalized, recently, in \cite{MacconeGeomEvents}.
Apart from different notation, one can recognize, in Eq.~1 therein
(by setting $\mu = \nu = 0$),
the canonical commutation relations for the time operator $\op{T}$ and its conjugate $\op{H}_T$
(while $\mu = \nu = 1, 2, 3$ would bring to the well-known position--momentum relations).
Moreover, Eq.~6 from \cite{MacconeGeomEvents} can easily be recognized as a more compact form of Eq.~\eqref{eq:pwphi}
by rearranging its integration variables.

% Relativistic extension? connect also somewhat with our work, Ch. 5?

% \subsubsection*{The clock as an open system}

% Let us consider again Eq.~\eqref{eq:pwphi}. There has:
% \begin{multline}\label{eq:pw:uncertain:schmidt}
%   \dket{\Phi} =
%     \int \dd{t} \phi(t) \ket{t}_T \ox \ket{\psi(t)}_S
%     \\
%     = \int \dd{t} \ket{t}_T \ox \ket{\phi(t)}_S
%     = \int \dd\omega \ket{\omega}_T \ox \ket{\tilde{\phi}(\omega)}_S
%     \text{,}
% \end{multline}
% where we define $\ket{\phi(t)}_S \eqbydef \phi(t)\ket{\psi(t)}_S$;
% %
% % Thus the spatial degrees of freedom should be \emph{traced out},
% % before deriving a time--energy relation in $\hilb{H}_T$.
% %
% % \begin{equation}\label{eq:pw:uncertain:schmidt}
% %   \dket{\Phi} =
% %     \int \dd{t} \ket{t}_T \ox \ket{\phi(t)}_S =
% %     \int \dd\omega \ket{\omega}_T \ox \ket{\tilde{\phi}(\omega)}_S \, \text{,}
% % \end{equation}
% the sets $\setof{\ket{t}_T}$ and $\setof{\ket{\omega}_T}$
% are
% orthonormal bases of $\hilb{H}_T$;
% and
% $\ket{\tilde{\phi}(\omega)}_S$ is the Fourier transform%\footnote{
% %   Fourier transform of a
% %   vector-valued function
% %   (vectors in $\hilb{H}_S$),
% %   similar to what found in \cite{Maccone:Pauli}.
% % }
%  of $\ket{\phi(t)}_S$. Let us stress again that
% neither $\ket{\phi(t)}_S$ nor $\ket{\tilde{\phi}(\omega)}_S$
% are necessarily normalized (in $\hilb{H}_S$)
% for all values of $t, \omega \in \mathbb{R}$,
% which is in fact a requirement
% to satisfy the normalization in $\pwspace$:
% \[
%   \int \dd{t} \braket{\phi(t)}_S =
%     \int \dd{\omega} \braket{\tilde{\phi}(\omega)}_S =
%     \dbradket{\Phi}{\Phi}_{T \ox S} =
%     1
%     \text{.}
% \]

% %% Andreas: "unclear"
% % The vector $\dket{\Phi}$
% % cannot,
% % in general, be simply expressed as a tensor product of two pure states
% % in $\hilb{H}_T$ and $\hilb{H}_S$.
% % However,
% % a ``temporal part'' of $\dket{\Phi}$
% % can be identified as a \emph{mixed} state described
% % by a (reduced) density operator $\op{\rho}_T$ in $\hilb{H}_T$.
% % In other words, the ``spatial'' part (corresponding to the subspaces $\hilb{H}_S$)
% % is \term{traced out} as the \term{environment} for the clock.

% % To obtain an explicit expression of $\op{\rho}_T$,
% % we use the definitions and results from Chapter~\ref{ch:decohere},
% % in particular Eqs. \eqref{eq:bipartite_expansion} and \eqref{eq:density_A_expand},
% % replacing the discrete sums there with integrals,
% % and the index $j$ with the integration variable $t$ (or $\omega$, respectively);
% % also we identify the indices%\footnote{
% % %   A particular case of Eq.~\eqref{eq:bipartite_expansion} is when $\alpha_{i\mu} = \delta_{i\mu}$,
% % %   therefore it becomes
% % %   $$\ket{\psi_S} = \sum_{i} |\alpha_{i}|^{2} \ket{i}_A \ox \ket{i}_B \text{,}$$
% % %   which can be compared to Eq.~\eqref{eq:pw:uncertain:schmidt}.
% % % }
% % $i = \mu$ therein.

% The reduced density operator can so be computed
% as the partial trace
% \begin{align}
%   \label{eq:ptrace_density_matrix_t}
%   \op{\rho}_T = \Tr_S\qty(\dketdbra{\Phi}{\Phi}) %&= \int \dd t \norm{\phi(t)}^2_S \ketbra{t}{t}
%     \, \text{,}
%   \\
%   \label{eq:ptrace_density_matrix_omega}
%   \op{\rho}_S = \Tr_T\qty(\dketdbra{\Phi}{\Phi}) %&= \int \dd \omega \norm{\tilde{\phi}(\omega)}^2_S \ketbra{\omega}{\omega}
%     \,\text{.}
% \end{align}
% % Here the probabilty distributions $\norm{\phi(t)}^2$ and $\norm{\tilde{\phi}(\omega)}^2$
% % are to be intended in the sense of a mixed state
% % i.e. probability that a system is in a certain state $\ket{t}_T$
% % (or $\ket{\omega}_T$,~respectively),
% % not in the sense of the probability of a measurement outcome on a \emph{known} pure state
% % (which is what, in its original formulation, the Heisenberg uncertainty principle
% % refers to).
% % Although with these limitations, it is still of interest to observe that
% % the relation $\sigma_T\sigma_{\hbar\Omega} = \hbar \sigma_{\phi} \sigma_{\tilde{\phi}} \geq \frac{\hbar}{2}$,
% % formally resembling a time--energy uncertainty relation,
% % can be proven
% % due to the properties of the Fourier transform,
% % which relates $\phi(t)$ and $\tilde{\phi}(\omega)$.
