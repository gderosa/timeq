\section{Basic ideas of the Model}

To introduce the Page--Wootters model, let us first consider
a bipartite system $\hilb{H}_T \ox \hilb{H}_S$,
the two subsystems of which have finite dimension\footnote{
  In general, within the model,
  the system may as well be infinite-dimensional;
  the sum may be replaced by an integral (continuous spectrum);
  and the vector in the tensor-product space may be not normalizable (an improper vector).
  Nonetheless, a finite-dimensional Hilbert space is sufficient to construct several applications of interest.
} $N$ each.
Let us also consider the following entangled state:
\begin{equation}\label{eq:pw:finite-entanglement}
  \hilb{H}_T \ox \hilb{H}_S \ni \dket{\Psi}
  =
  \frac{1}{\sqrt{N}} \sum_{n=0}^{N-1} \ket{\tau_{n}}_{T} \ox \ket{\psi_{n}}_{S} \text{.}
\end{equation}
Here
a ``double angle bracket'' notation $\dket{\Psi}$ has been adopted
for vectors in the tensor product space $\hilb{H}_T \ox \hilb{H}_S$.
We assume that $\ket{\psi_{n}}_{S}$ are all \emph{normalized} in $\hilb{H}_S$;
and
$\qty{\ket{{\tau}_n}}$ is an orthonormal eigenbasis of some observable (say, $T$) in $\hilb{H}_T$.
Furthermore, let $\dket{\Psi}$ be \emph{stationary}, meaning it is an eigenstate
of an overall Hamiltonian $\mathbb{J}$ in $\hilb{H}_T \ox \hilb{H}_S$:
$$
  \mathbb{J} \dket{\Psi} = \epsilon \dket{\Psi}\text{,} \quad \text{where} \kuad \epsilon \in \mathbb{R} \text{.}
$$
Finally, we assume that the two subsystems
are \emph{non interacting},
meaning that the global Hamiltonian $\mathbb{J}$ can be expressed as a sum of two terms
which only act on the respective subspaces $\hilb{H}_T$ and $\hilb{H}_S$
(and there is no explicit ``interaction term''). In other words,
$\mathbb{J}$ can be expressed as
$$
  \mathbb{J} = H_T \ox \idop_S + \idop_T \ox H_S \, \text{.} 
$$
Therefore,
the evolution of the subsystem described by $\hilb{H}_S$
will only depend on ``its own''
Hamiltonian $H_S$, namely
\begin{equation}\label{eq:pw:ordinary_S}
  \ket{\psi(t)}_S = e^{- i \hbar t / H_S } \ket{\psi(0)}_S \text{.}  
\end{equation}
 
In the Page--Wootters model, time evolution is based on an internal entanglement
relation among subsystems of an isolated system
---including the case where such isolated system is the whole universe \parencite{PageWootters}.

The overall system
---as the one described by $\hilb{H}_T \ox \hilb{H}_S$ in \eqref{eq:pw:finite-entanglement}---
is stationary and there is no notion
of time external to it.
This is also known as the
``timeless'' approach \parencite{Marletto:Evolution}.
\citereset
A suitable subsystem is chosen within the ``universe'' so that it acts as
\emph{clock} for the \emph{rest} of it, in the sense that
there is an observable $T$
that can be used to
describe time evolution ``without evolution''\footnote{
  The expression ``evolution without evolution''
  is directly quoted from the title of \cite{PageWootters},
  and recalled in \cite{Marletto:Evolution}.
}
of the other subsystem.

Please note,
the \emph{time operator} $T$ is defined in $\hilb{H}_T$ ---and canonically conjugate to $H_T$, not $H_S$.
In other words, the time operator is defined in a different Hilbert space than the ``system of interest'' $\hilb{H}_S$,
therefore
the Pauli objection no longer applies. A suitable entanglement relation ensures that 
the ``time'' measured by a clock in one system $\hilb{H}_T$ is relevant to the evolution of the other
system $\hilb{H}_S$.

Quantitatively,
the observable $T$ in $\hilb{H}_T$
has eigenvalues $\tau_n$
(assuming a discrete spectrum to simplify notation)
which can be interpreted as possible
instants of time in the evolution of $\ket{\psi}_S$, in the sense that
\begin{equation}\label{eq:pw:discrete_Tpprox}
  \Big[ \ket{\psi(t)}_S \Big]_{t=\tau_n} = \,\,\,\, \ket{\psi_n}_S \, \text{,}
\end{equation}
where $\ket{\psi(t)}_S$ is given in~\eqref{eq:pw:ordinary_S} i.e.
the evolution one would obtain in ``ordinary'' way
by resolving the Schr\"{o}dinger equation in $\hilb{H}_S$.
Please note:
in its original formulation, the Page--Wootters model is based on a continuous
notion of time, and the same applies to more recent developments \parencite{Lloyd:Time}.
A discrete formulation, as in eq. \eqref{eq:pw:discrete_Tpprox},
will be verified numerically in Sections~\ref{sec:beyondMoreva}--\ref{sec:multiLevelClock}.

An intuitive description may be as follows.
From the perspective of the system described by $\hilb{H}_S$,
time is a parameter $t$;
however $t$ can have values
$\tau_n$ (or \emph{all} real values, in a continuous model)
which are also eigenvalues of an observable $T$ defined in \emph{another} space $\hilb{H}_T$,
with which it is entangled. Such entanglement relation establishes a correspondence
between ``instants of time'' (eigenvalues and eigenstates of observable $T$)
and (``evolved'') states in $\hilb{H}_S$.
% Thus, time can be regarded as a quantum observable,
% although defined in a different Hilbert space than the one of the Hamiltonian of interest.
% As $H_S$ and $T$ acts on two different Hilbert spaces, the Pauli objection is no longer valid
% (how canonical conjugation and uncertainty relations hold will be clearer in Sec.~\ref{sec:pw:uncertainty}).

The symbols $\hilb{H}_T$ and $\hilb{H}_S$
naturally indicate
the Hilbert spaces where a time operator $T$ is defined,
and, respectively,
the ordinary Hilbert space of the \emph{S}ystem.
This notation is based on \citereset\cite{Lloyd:Time}
(where also the blackboard bold typeface, e.g. ``$\mathbb{J}$'',
is used for operators defined in a larger space
than the one of standard quantum mechanics,
typically the product $\hilb{H}_T \ox \hilb{H}_S$).
It is tempting to relate $\hilb{H}_T$ and $\hilb{H}_S$
to temporal (T) and spatial (S)
degrees of freedom, thus forming a sort of
``space-time'' (and this will be even more apparent
with a relativistic extension that will be sketched in Sec.~\ref{sec:KG}).
%
A slightly different notation
uses $\hilb{H}_C$ and $\hilb{H}_R$
to indicate the Hilbert spaces of the clock (C) and the rest (R) of the ``universe''
(or isolated system):
see, for example, \cite{Marletto:Evolution}.

% \subsection*{Notation}

% In this and following chapters, when treating quantum time models,
% $\hilb{H}_S$ will indicate the Hilbert space of ordinary quantum mechanics.
% $\hilb{H}_T$, an extra space where a time operator ${T}$ is introduced. The tensor
% product space $\pwspace$
% will often be referred to as well.
% Bras and kets in this tensor product space will be indicated with a special double-angle-bracket
% notation, as in $\dket{\Psi}$.

%% TODO: find a place for this
% ${\mathbb{J}}$ (blackboard bold) indicates an  operator defined in
% a larger space than the one of standard quantum mechanics,
% typically the product $\pwspace$.

%% TODO: move this (amd the above?) to where these symbols are first used.
% The symbol `$\repr$', as in $\op{A} \repr \mqty(a&b\\c&d)$, $\ket{\psi} \repr \mqty[\alpha \\ \beta]$
% means: representation with respect to a particular basis (as opposed to intrinsic equality `$=$').

% The symbol `$\eqbydef$'
% means: equal by definition, equal by settings i.e. postulated and not derived logically.
