\section{Introduction}

To introduce the Page--Wootters model, let us first consider
a bipartite system $\hilb{H}_A \ox \hilb{H}_B$,
the two subsystems of which have, for example, finite dimension\footnote{
  As it will be clearer later in the Chapter,
  this example is simplified for the sake of argument: in principle the system
  may be infinite-dimensional, the sum may be replaced by an integral, and
  the vector in the tensor product space may be not normalizable
  (an improper vector).
}$N$,
and are completely
entangled, i.e.
\begin{equation}\label{eq:pw:finite-entanglement}
  \hilb{H}_A \ox \hilb{H}_B \ni \dket{\Psi}
  =
  \frac{1}{\sqrt{N}} \sum_{n=0}^{N-1} \ket{\tau_{n}}_{A} \ox \ket{\psi_{n}}_{B} \text{.}
\end{equation}
Here
a ``double angle braket'' notation $\dket{\Psi}$ has been adopted
for vectors in the tensor product space $\hilb{H}_A \ox \hilb{H}_B$.
We assume $\ket{\psi_{n}}_{B}$ are all \emph{normalized} in $\hilb{H}_B$;
and
$\qty{\ket{{\tau}_n}}$ is an eigenbasis of some observable (say, $T$) in system A.
Let also $\dket{\Psi}$ be \emph{stationary}, meaning it is an eigenstate
of the overall Hamiltonian $\mathbb{J}$ in $\hilb{H}_A \ox \hilb{H}_B$:
$$
  \mathbb{J} \dket{\Psi} = \epsilon \dket{\Psi} \text{.}
$$
Finally, we assume that the two subsystems $A$ and $B$ are \emph{non interacting},
meaning that the global Hamiltonian $\mathbb{J}$ can be expressed as a sum of two terms
which only act on the respective subspaces $\hilb{H}_A$ and $\hilb{H}_B$
(and there is no explicit ``interaction term''). In other words,
$\mathbb{J}$ can be expressed as
$$
  \mathbb{J} = H_A \ox \idop_B + \idop_A \ox H_B \, \text{.} 
$$
Therefore, with the     ordinary notion of ``time as a paramerter'' in quantum mechanics,
the evolution of system B will only depend on ``its own'' Hamiltonian $H_B$, namely
$\ket{\psi(t)}_B = e^{- i \hbar t / H_B } \ket{\psi(0)}_B$. 

In the Page--Wootters model, time evolution is based on an internal entanglement
relation among subsystems of an isolated system (including the case of the whole universe,
\cite{PageWootters}), while the whole system
---as the one described by $\hilb{H}_A \ox \hilb{H}_B$ in \eqref{eq:pw:finite-entanglement}---
is stationary and there is no notion
of time as a parameter external to it. This is also known as the
``timeless'' approach \parencite{Marletto:Evolution}.
\citereset
Within this approach,
a suitable subsystem is chosen within the ``universe'' so that it acts as
clock for the rest of it, in the sense that
there is an observable $T$ (as in the above example, in system $A$)
that can be used to
describe time evolution ``without evolution'' of system $B$
---where the expression ``evolution without evolution''
is taken directly from the original Page and Wootters paper just mentioned,
and used in \cite{Marletto:Evolution} as well.

If we regard this observable as ``the position of the hand of a clock'',
the state of system B can be regarded as a function of the ``parameter'' $n$,
which is also an index labeling
eigenvalues and eigenvectors of a quantum observable
with its own Hermitian operator.
Thus the relation between quantum systems A and B is thus regarded as
a ``time evolution'' of system B,
while the whole bipartite system
is in a stationary state
$\ket{\Psi}$.
There is no logical requirement that $\ket{\Psi}$
depends, other than trivially,
upon any further external parameter.
The stationary state $\ket{\Psi}$ embeds the whole ``history''
of subsystem B.

\subsection*{Notation}

In this and following chapters, when treating quantum time models,
$\hilb{H}_S$ will indicate the Hilbert space of ordinary quantum mechanics.
$\hilb{H}_T$, an extra space where a time operator $\hat{T}$ is introduced. The tensor
product space $\pwspace$
will often be referred to as well.
Bras and kets in this tensor product space will be indicated with a special double-angle-bracket
notation, as in $\dket{\Psi}$.

${\mathbb{J}}$ (blackboard bold) indicates an  operator defined in
a larger space than the one of standard quantum mechanics,
typically the product $\pwspace$.

The symbol `$\repr$', as in $\hat{A} \repr \mqty(a&b\\c&d)$, $\ket{\psi} \repr \mqty[\alpha \\ \beta]$
means: representation with respect to a particular basis (as opposed to intrinsic equality `$=$').

The symbol `$\eqbydef$'
means: equal by definition, equal by settings i.e. postulated and not derived logically.
