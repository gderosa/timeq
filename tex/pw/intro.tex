\section{The Model}

To introduce the Page--Wootters model, let us first consider
a bipartite system $\hilb{H}_T \ox \hilb{H}_S$.
%% Move to finite-dim section? is this just a redundant repetition?
% the two subsystems of which have finite dimension\footnote{
%   In general, within the model,
%   the system may as well be infinite-dimensional;
%   the sum may be replaced by an integral (continuous spectrum);
%   and the vector in the tensor-product space may be not normalizable (an improper vector).
%   Nonetheless, a finite-dimensional Hilbert space is sufficient to construct several applications of interest.
% } $N$ each.
% Let us also consider the following entangled state:
% \begin{equation}\label{eq:pw:finite-entanglement}
%   \hilb{H}_T \ox \hilb{H}_S \ni \dket{\Psi}
%   =
%   \frac{1}{\sqrt{N}} \sum_{n=0}^{N-1} \ket{\tau_{n}}_{T} \ox \ket{\psi_{n}}_{S} \text{.}
% \end{equation}
Here
a ``double angle bracket'' notation $\dket{\Psi}$ has been adopted
for vectors in the tensor product space $\hilb{H}_T \ox \hilb{H}_S$.
We assume that $\ket{\psi_{n}}_{S}$ are all \emph{normalized} in $\hilb{H}_S$;
and
$\qty{\ket{{\tau}_n}}$ is an orthonormal eigenbasis of some observable (say, $\op{T}$)
in $\hilb{H}_T$.
Furthermore, let $\dket{\Psi}$ be \emph{stationary}, meaning it is an eigenstate
of an overall Hamiltonian $\op{\mathbb{J}}$ in $\hilb{H}_T \ox \hilb{H}_S$:
$$
  \op{\mathbb{J}} \dket{\Psi} = \epsilon \dket{\Psi}\text{,} \quad \text{where} \kuad \epsilon \in \mathbb{R} \text{.}
$$
Finally, we assume that the two subsystems
are \emph{non interacting},
meaning that the global Hamiltonian $\mathbb{J}$ can be expressed as a sum of two terms
which only act on the respective subspaces $\hilb{H}_T$ and $\hilb{H}_S$
(and there is no explicit ``interaction term''). In other words,
$\op{\mathbb{J}}$ can be expressed as
$$
  \op{\mathbb{J}} = \op{H}_T \ox \idop_S + \idop_T \ox \op{H}_S \, \text{.} 
$$
Therefore,
the evolution of the subsystem described by $\hilb{H}_S$
will only depend on ``its own''
Hamiltonian $H_S$, namely
\begin{equation}\label{eq:pw:ordinary_S}
  \ket{\psi(t)}_S = e^{- i \op{H}_S t / \hbar} \ket{\psi(0)}_S \text{.}  
\end{equation}
 
In the Page--Wootters model, time evolution is based on an internal entanglement
relation among subsystems of an isolated system
---including the case where such isolated system is the whole universe \parencite{PageWootters}.

The overall system
---as the one described by $\hilb{H}_T \ox \hilb{H}_S$ in \eqref{eq:pw:finite-entanglement}---
is stationary and there is no notion
of time external to it.
This is also known as the
``timeless'' approach \parencite{Marletto:Evolution}.
\citereset
A suitable subsystem is chosen within the ``universe'' so that it acts as
\emph{clock} for the \emph{rest} of it, in the sense that
there is an observable $T$
that can be used to
describe time evolution ``without evolution''\footnote{
  The expression ``evolution without evolution''
  is directly quoted from the title of \cite{PageWootters},
  and recalled in \cite{Marletto:Evolution}.
}
of the other subsystem.

Please note,
the \emph{time operator} $\op{T}$ is defined in $\hilb{H}_T$ ---and canonically conjugate to $\op{H}_T$, not $\op{H}_S$.
In other words, the time operator is defined in a different Hilbert space than the ``system of interest'' $\hilb{H}_S$,
therefore
the Pauli objection no longer applies. A suitable entanglement relation ensures that 
the ``time'' measured by a clock in one system $\hilb{H}_T$ is relevant to the evolution of the other
system $\hilb{H}_S$.

Quantitatively,
the observable $T$ in $\hilb{H}_T$
has eigenvalues $\tau_n$
(assuming a discrete spectrum to simplify notation)
which can be interpreted as possible
instants of time in the evolution of $\ket{\psi}_S$, in the sense that
\begin{equation}\label{eq:pw:discrete_Tpprox}
  \Big[ \ket{\psi(t)}_S \Big]_{t=\tau_n} = \,\,\,\, \ket{\psi_n}_S \, \text{,}
\end{equation}
where $\ket{\psi(t)}_S$ is given in~\eqref{eq:pw:ordinary_S} i.e.
the evolution one would obtain in ``ordinary'' way
by resolving the Schr\"{o}dinger equation in $\hilb{H}_S$.
Please note:
in its original formulation, the Page--Wootters model is based on a continuous
notion of time, and the same applies to more recent developments \parencite{Lloyd:Time}.
A discrete formulation, as in eq. \eqref{eq:pw:discrete_Tpprox},
will be verified numerically in Sections~\ref{sec:beyondMoreva}--\ref{sec:multiLevelClock}.

An intuitive description may be as follows.
From the perspective of the system described by $\hilb{H}_S$,
time is a parameter $t$;
however $t$ can have values
$\tau_n$ (or \emph{all} real values, in a continuous model)
which are also eigenvalues of an observable $T$ defined in \emph{another} space $\hilb{H}_T$,
with which it is entangled. Such entanglement relation establishes a correspondence
between ``instants of time'' (eigenvalues and eigenstates of observable $T$)
and (``evolved'') states in $\hilb{H}_S$.
% Thus, time can be regarded as a quantum observable,
% although defined in a different Hilbert space than the one of the Hamiltonian of interest.
% As $H_S$ and $T$ acts on two different Hilbert spaces, the Pauli objection is no longer valid
% (how canonical conjugation and uncertainty relations hold will be clearer in Sec.~\ref{sec:pw:uncertainty}).

The symbols $\hilb{H}_T$ and $\hilb{H}_S$
naturally indicate
the Hilbert spaces where a time operator $\op{T}$ is defined,
and, respectively,
the ordinary Hilbert space of the \emph{S}ystem.
This notation is based on \citereset\cite{Lloyd:Time}
(where also the blackboard bold typeface, e.g. ``$\op{\mathbb{J}}$'',
is used for operators defined in a larger space
than the one of standard quantum mechanics,
typically the product $\hilb{H}_T \ox \hilb{H}_S$).
It is tempting to relate $\hilb{H}_T$ and $\hilb{H}_S$
to temporal (T) and spatial (S)
degrees of freedom, thus forming a sort of
``space-time'' (and this will be even more apparent
with a relativistic extension that will be sketched in Sec.~\ref{sec:KG}).
%
A slightly different notation
uses $\hilb{H}_C$ and $\hilb{H}_R$
to indicate the Hilbert spaces of the clock (C) and the rest (R) of the ``universe''
(or isolated system):
see, for example, \cite{Marletto:Evolution}.

\subsection*{\it Structure of the Chapter}

Sections \ref{sec:pw:theory_first}--\ref{sec:pw:theory_last}
will be devoted to outlining the theory, reviewing existing papers,
with some original considerations ---particularly with regards to
time--energy (or time--\emph{frequency}) uncertainty relations, and
finite-dimensional systems (discrete clocks).

Sections \ref{sec:pw:apps_first}--\ref{sec:pw:apps_last} will be dedicated to
applications of the Page--Wootters model, starting with a critical review of existing
experiments (or ``experimental illustrations''),
then implementing numerical simulations, with a larger number of clock levels.
Finally, the Page and Wootters model will be compared to the results of
detection and time-of-arrival models
based on non-unitary evolution.
\section{Development of the Formalism}\label{sec:pw:formalism}\label{sec:pw:theory_first}

About three decades after the first publication by
Don Page and William Wootters \parencite{PageWootters},
the model was further developed,
in particular by Lloyd, Giovannetti and Maccone,
with emphasis on addressing technical issues which have been raised meanwhile
\parencite{Lloyd:Time}.

In this formulation,
in addition to the ordinary Hilbert space of the system,
an extra Hilbert space $\mathcal{H}_T$ is considered,
where time is an observable
represented by a self-adjoint operator.
The ordinary Hilbert space can be labeled $\mathcal{H}_S$;
and we consider the tensor product space $\mathcal{H}_T \otimes \mathcal{H}_S$ as
the space in which both time and position are observables, and they act as
$T \otimes \idop_S$ and $\idop_T \otimes X$
respectively.

As explained in \cite{Lloyd:Time, Maccone:Pauli}, the overall Hamiltonian,
encompassing both position and time as observables, is given by
\begin{equation}\label{eq:pwHamiltonian}
  \op{\mathbb{J}} = \hbar\op{\Omega}\ox\idop_S + \idop_T\ox\op{H}_S \,\text{,}
\end{equation}
where $\hbar\op{\Omega}$ is the canonically conjugate to the operator $\op{T}$ in $\hilb{H}_T$.
Also, the \term{Wheeler-DeWitt equation} holds:
\begin{equation}\label{eq:Wheeler-DeWitt}
  \op{\mathbb{J}}\dket{\Psi} = 0 \,\text{,}
\end{equation}
describing a \emph{static} universe, where evolution is only
in terms of relations between parts of a multipartite system
(a ``clock'' and ``the rest'').

% Please note the special notation $\dket{\Psi}$ (double angle bracket),
% to indicate a state vector in the ``larger'' product space $\pwspace$,
% as opposed to vectors belonging to $\hilb{H}_S$ (or $\hilb{H}_T$) only.

The ``conventional'' state $\ket{\psi(t)}_S$ in $\hilb{H}_S$
can be obtained from $\dket{\Psi}$ via partial inner product\footnote{
  For the partial inner product,
  see Definition \ref{def:pBra} and \ref{def:pKet},
  and \cite[\s 1.3.3]{QMT_Jacobs}.
}
with a time eigenstate $\prescript{}{T}{\bra{t}}$:
\begin{equation*}
  \ket{\psi(t)}_S = \prescript{}{T}{\bradket{t}{\Psi}} \, \text{.}
\end{equation*}
The function $ t \rightarrow \ket{\psi(t)}_S \; $ is the
``wavefunction in time representation'', in analogy
with the wavefunction in position representation of standard quantum mechanics.
The state vector $\dket{\Psi}$ is univocally identified by the ``coefficients'' $\ket{\psi(t)}_S$
with respect to the basis $\setof{\ket{t}_T}$:
\begin{equation*}
  \dket{\Psi} = \int \dd{t} \ket{t}_T \ox \ket{\psi(t)}_S \, \text{,}
\end{equation*}
somewhat similar to the well known expression $\ket{\psi}_S = \int \dd{x} \psi_{S}(x) \ket{x}_S$
which defines the position representation.

Under this time representation (or $T$ representation), the operator $\hbar\Omega$,
canonically conjugate to the time operator T, is expressed as $-i\hbar\pdv{t}$,
and the commutation relation $\comm{t, -i\hbar\pdv{t}}{} = i\hbar$ can be proven
immediately, similarly to the well known position-momentum relation.

%% \hbar\Omega is -E ... and \Omega \repr -i t-derivative ... and [T, \Omega] = i .

Using the $T$ representation in $\hilb{H}_T$,
and comparing \eqref{eq:pwHamiltonian} and \eqref{eq:Wheeler-DeWitt}:
\begin{equation}\label{eq:schrod_from_pw}
  0 = \qty(\hbar\op{\Omega}\ox\idop_S + \idop_{T}\ox\op{H}_S)\dket{\Psi}
    \repr -i\hbar\pdv{t}\ket{\psi(t)}_{S} + \op{H}_S\ket{\psi(t)}_{S}
    \,\text{,}
\end{equation}
we recover the usual form of the Schr\"{o}dinger equation in $H_S$
(a detailed proof can be found in \cite[709--710]{Wootters:Loyola}).

Here $\op{\Omega}$ can be seen as a ``frequency operator''
represented as $-i\pdv{t}$ and having as eigenfunctions
those functions evolving in time with a phase factor $e^{i \omega t}$ only.

Canonical commutation relation should hold and can be easily verified
between $\op{T}$ and $\op{\Omega}$
i.e. $[\op{T}, \op{\Omega}] = i$,
therefore $\hbar\op{\Omega}$ can be seen as the ``linear momentum''
in the Hilbert space of time.

From another point of view, $\hbar\op{\Omega}$ is the ``Hamiltonian'' of $\hilb{H}_T$,
in that it plays a similar role of $H_S$ in the construction of
$\op{\mathbb{J}}$ in \eqref{eq:pwHamiltonian}. Some authors (e.g. in \citereset\cite{Wootters:Loyola})
use the symbol $H_C$ (where the subscript ``$C$'' stands for ``clock'') in place of
$\hbar\op{\Omega}$, to emphasize this.

Such analogies among operators in $\hilb{H}_T$ and $\hilb{H}_S$ are summarized in Table \ref{tbl:op_comparison_pw}.

{
  %% https://tex.stackexchange.com/a/232874
  %% https://tex.stackexchange.com/a/2836
  \begin{table}
    \centering
    \begin{tabular}{l|l|c}
      & \multicolumn{1}{c|}{$\hilb{H}_T$}   & \multicolumn{1}{|c}{$\hilb{H}_S$}   \\
      \hline
      \multirow{2}{11em}{Canonical commutation relation} 
      & $\op{t}$                                & $\op{x}$                            \\
      %
      & $\hbar\op{\Omega}$                      & $\op{p}$                            \\
      \hline
      ``Energy operator''
      & $\hbar\op{\Omega}$ (or ``$\op{H}_C$'')  & $\op{H_S}$
    \end{tabular}
    {\caption{
      Analogies between observables in the two Hilbert spaces.
    }\label{tbl:op_comparison_pw}}
  \end{table}
}

% It would be interesting to a study relativistic extension of the
% Page and Wootters model that allows, for example, the derivation of the Klein-Gordon
% equation, thus eliminating the asimmetry between
% $\hilb{H}_T$ and $\hilb{H}_S$ (and momentum and energy as well).

In the bipartite universe, physical kets $\dket{\Psi}$ can have a Schmidt decomposition
made up of
eigenstates of $\op{\Omega}$ in $\hilb{H}_T$
entangled, respectively, with
eigenstates of $\op{H}_S$ in $\hilb{H}_S$;
or eigenstates of $\op{T}$ in $\hilb{H}_T$
entangled with time-evolved spatial states~$\ket{\psi(t)}_S$ in $\hilb{H}_S$
(according to the evolution that is embedded in $\op{\mathbb{J}}$):
\begin{equation}\label{eq:pw:time_freq_mu}
  \dket{\Psi} =
    \int \dd{\omega} \, \mu(\omega) \ket{\omega}_{T}\ox\ket{\phi(\omega)}_{S} =
    \int \dd{t} \, \ket{t}_{T} \ox \ket{\psi(t)}_{S} \text{.}
\end{equation}
Here $\mu(\omega)$ is a measure on the real axis which selects
values of $\omega$ such that~$\hbar\omega$ is part of the system's energy spectrum
(in ordinary quantum mechanics terms). See \cite[eq. (10)]{Lloyd:Time}.

\subsection{Example: eigenstate of the Hamiltonian $\op{H}_S$ in $\hilb{H}_S$}\label{sec:pw:eeigenstate}

An extreme case for the measure $\mu$ in \eqref{eq:pw:time_freq_mu}
is the evolution of an energy eigenstate
(related, say, to eigenvalue\footnote{
  Some care must be taken with the sign convention used in
  \cite{Lloyd:Time} where
  energy and frequency have \emph{opposite} signs.
}
$E_0 \eqbydef -\hbar\omega_{0}$):
in this case the evolution is
$\ket{\psi(t)}_{S} = \E^{\iu\omega_{0}(t-t_{0})} \ket{\psi(t_0)}_{S}$
and
the measure is
$\mu(\omega) = \delta(\omega-\omega_0)$
(it  ``selects''only $\omega_0$ among all possible values
of $\omega$).

Indeed:
\[
  \dket{\Psi} = \int \dd{\omega} \delta(\omega-\omega_0)\ket{\omega}_T \ox \ket{\phi(\omega)}_S =
    \ket{\omega_0}_T \ox \ket{\phi(\omega_0)}_S \text{.}
\]
Also, using the relations
$
\, \prescript{}{T}{\braket{t}{t'}}_T = \delta(t-t') \,
$
and
$
  \,
  \ket{\omega}_T =
  \frac{1}{\sqrt{2\pi}} \int\dd{t}\E^{\iu\omega t}\ket{t}_T
  \,
$
\parencite{Lloyd:Time},
it follows that\footnote{%
  Please note the analogy with the well-known
  inner product of position and momentum eigenvectors in standard quantum mechanics:
  $\braket{\vb{x}}{\vb{p}} = (2\pi\hbar)^{-3/2}\E^{\iu \vb{p} \vdot \vb{x}  / \hbar}$
  (or $\braket{x}{p} = \frac{\E^{\iu p x / \hbar}}{\sqrt{2\pi\hbar}}$ for one-dimensional systems).
  See, for example, \cite[126--127]{Ballentine}.
}
\[
  \prescript{}{T}{\braket{t}{\omega}}_T =
  \frac{1}{\sqrt{2\pi}}\prescript{}{T}{\bra{t}} \int\dd{t'}\E^{\iu\omega t'}\ket{t'}_T  =
  \frac{1}{\sqrt{2\pi}}\int\dd{t'}\E^{\iu\omega t'}\delta(t-t') =
  \frac{1}{\sqrt{2\pi}} \E^{\iu\omega t} \; \text{;}
\]
therefore the \emph{evolution} (in $\hilb{H}_S$) is
\begin{equation}
  \ket{\psi(t)}_S = \prescript{}{T}{\bradket{t}{\Psi}} = \prescript{}{T}{\braket{t}{\omega_0}}_{T} \ket{\phi(\omega_0)}_S =
    \frac{1}{\sqrt{2\pi}} \E^{\iu\omega_0 t} \ket{\phi(\omega_0)}_S \text{.}
\label{eq:pw:stat_eigenst_of_H_S}
\end{equation}
By evaluating and comparing eq. \eqref{eq:pw:stat_eigenst_of_H_S} at two arbitrary instants of time
$t_0$~and~$t$,
one easily gets
$\ket{\psi(t)}_{S} = \E^{\iu\omega_{0}(t-t_{0})} \ket{\psi(t_0)}_{S}$,
which is the expected stationary evolution for an energy eigenstate in standard quantum mechanics. 






\subsection{Normalizable vectors of $\pwspace$}
\label{sec:properpw}

A vector $\dket{\Psi}$ in $\hilb{H}_T \ox \hilb{H}_S$,
satisfying \eqref{eq:pwHamiltonian} and \eqref{eq:Wheeler-DeWitt},
encodes the whole (unitary) time evolution of a system.
\begin{equation}\label{eq:pwexpansion}
  \dket{\Psi} =
    \int \dd{t} \ket{t}_T \ox \ket{\psi(t)}_S =
    \int \dd{t}\dd[3]{\vec{r}} \ \Psi(t; \vec{r}) \; \ket{t}_T \ox \ket{\vec{r}}_S
    \,  \text{.}
\end{equation}
We know $\setof{\ket{t}_T \ox \ket{\vec{r}}_S}$ is an othonormal basis of $\hilb{H}_T \ox \hilb{H}_S$, therefore
\begin{equation}
  \norm{\dket{\Psi}}^2 =
    \int \dd{t}\dd[3]{\vec{r}} \ \abs{\Psi(t; \vec{r})}^2 =
    \int \dd{t} \int \dd[3]{\vec{r}} \ \abs{\Psi(t; \vec{r})}^2 =
    \int \dd{t} 1 \rightarrow +\infty
    \,  \text{,}
\end{equation}
which means that such $\dket{\Psi}$ is an \term{improper} vector of $\hilb{H}_T \ox \hilb{H}_S$.

Proper (i.e. normalizable) states are described in \citereset\cite{Lloyd:Time} as well, by replacing (or generalizing)
the \eqref{eq:pwexpansion} with
\begin{equation}\label{eq:pwphi}
  \dket{\Phi} =
    \int \dd{t} \phi(t) \ket{t}_T \ox \ket{\psi(t)}_S \, \text{.}
\end{equation}
If the function $\phi \in \mathscr{L}^2(\mathbb{R})$,
then $\dket{\Psi}$ is a proper element of the product space,
and $\norm{\dket{\Psi}}^2 = \int \dd{t} \abs{\phi(t)}^2$.

The case of non-normalizable $\dket{\Psi}$ in \eqref{eq:pwphi},
with normalized $\ket{\psi(t)}_S$ $\forall t \in \mathbb{R}$,
describes the unitary evolution, as seen throughout Chapter \ref{ch:pw}.
As observed in \cite{Maccone:QGR},
``%
  Quantum mechanics is formulated in terms of \emph{systems},
  typically limited in space but infinitely extended in time%
''.
If the state vector is \emph{conditioned} at a particular time $t$,
it holds $\norm{_{T}\bradket{t}{\Psi}}_S = \norm{\ket{\psi(t)}}_S = 1$,
meaning that, at each $t$,
\emph{the particle must certainly be in some (one) point in space}.

A normalized $\dket{\Psi}$ in the whole $\hilb{H}_T \ox \hilb{H}_S$,
instead,
can be interpreted as a total probability of~$1$ in both space and time combined.
It can be interpreted as describing an \term{event},
in that it
must certainly be in some point in space
\emph{and} time (in terms of outcome of an idealized measurement\footnote{
  Measurement in quantum mechanics requires the concepts
  of state of the system
  (and measurement apparatus)
  \emph{before} and \emph{after} the measurement and, therefore, the existence
  of an external time (external with respect of the Hilbert space of states).
  This logically contradicts the foundation of the Page--Wootters model if
  time itself is measured as a quantum observable. $\dket{\Psi} \in \pwspace$
  embeds the whole history of a system and therefore cannot have a
  ``before'' neither an ``after'' the measurement, ``when'' it collapses
  into an eigenstate of time. The apparent contradicion is resolved
  stressing that probability amplitues are intended in the sense of
  \emph{conditional} probabilities e.g. \emph{provided that the particle
  is in position} $x \in X$ (or the detector clicks)
  what is the probability density amplitude of time being (the clock showing) $t$?
  Consistently, the Bayes rule
  (see, for example, \cite{Stat:Conditional})
  is invoked in the following sections
  and references.
}).
Such states have no correspondence with standard quantum mechanics states and their unitary evolution.