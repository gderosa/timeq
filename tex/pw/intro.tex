\section{Introduction}

In Section \ref{sec:trel} it was observed that time is the only observable quantity
(even if it's not a quantum observable \emph{strictu sensu})
of which all other observables and the state of the system itself are a function.
This justifies the very concept of \term{time evolution}.
Taking two other observables \emph{in general} there is no functional dependency
between one another.
However, this can be true in some particular cases.
An extreme example is a bipartite system,
the two subsystems of which happen to be
perfectly entangled i.e. $ \ket{\Psi} = \sum_{j} \ket{j}_{A} \ox \ket{\psi_{j}}_{B} $,
where $\qty{\ket{j}}$ is an eigenbasis of some observable in system A.
If we regard this observable as ``the position of the hand of a clock'',
the state of system B is functionally dependent on the ``parameter'' $j$,
which is not \emph{just a parameter} but also an index labeling
eigenvalues and eigenvectors of an actual quantum observable
with its own Hermitian operator.
Thus the relation between quantum systems A and B effectively
reproduces a ``time evolution'' of system B,
while the whole bipartite system
is stationary, in state
$\ket{\Psi}$.
There is no logical requirement that $\ket{\Psi}$
depends, other than trivially,
upon any further external parameter.
The stationary state $\ket{\Psi}$ embeds the whole ``history''
of subsystem B.

\subsection*{Notation}

In this and following chapters, when treating quantum time models,
$\hilb{H}_S$ will indicate the Hilbert space of ordinary quantum mechanics.
$\hilb{H}_T$, an extra space where a time operator $\hat{T}$ is introduced. The tensor
product space $\pwspace$
will often be referred to as well.
Bras and kets in this tensor product space will be indicated with a special double-angle-bracket
notation, as in $\dket{\Psi}$.

${\mathbb{J}}$ (blackboard bold) indicates a self-adjoint operator defined in
a larger space than the one of standard quantum mechanics,
typically the product $\pwspace$.

The symbol `$\repr$', as in $\hat{A} \repr \mqty(a&b\\c&d)$, $\ket{\psi} \repr \mqty[\alpha \\ \beta]$
means: representation with respect to a particular basis (as opposed to intrinsic equality `$=$').

The symbol `$\eqbydef$'
means: equal by definition, equal by settings i.e. postulated and not derived logically.
