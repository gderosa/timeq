\section[
  One qubit universe \dots
]{One qubit universe: experimental realization and theoretical developments}

The Page and Wootters model has been illustrated experimentally in recent years,
with a very simple toy universe consisting of just one qubit acting as ``the system'' (or
``the rest of the universe'' if we will), and the clock being immplemented by another qubit ---
physically, the polarization states of two photons \parencite{Moreva:synthetic,Moreva:illustration}).

In another experiment by the same authors \parencite{Moreva_position}, $\hilb{H}_S$ is still implemented with 
polarizations $\ket{H}$ or $\ket{V}$ of a photon, while the clock states in $\hilb{H}_T$
are given by the \emph{position} of the same photon along the conventional $x$ axis.

The latter is interesting because it implements a continuous time,
which, among other things, allows identifying a canonically conjugate observable
$\hat{\Omega}$. Or, conversely, a time operator $\hat{T}$, once $\hat{\Omega}$ is given.
More precisely, it is impossible to satisfy
\begin{equation}\label{eq:canonical_commutation_in_time}
  [\hat{T}, \hat{\Omega}] = i
\end{equation}
in a finite-dimension Hilbert space, because the operators
cannot be both bounded \parencite{Weyl1927}.

On the other hand, one problematic aspect of the experiment with continuous time
is that
it relies on \term{photon position} which is another
still controversial topic (see, for example, \cite{HawtonPhotonPosition}),
similar, in that regards, to the quest for a quantum time operator that the experiment is trying to solve.

Just like time in quantum mechanics, position coordinates in quantum optics and other field theories
are (classical) external parameters and not quantum observables. 

However, the experiment in \cite{Moreva_position} verifies the violation of
\term{Legget-Garg inequalities}, as previously suggested in \cite{LeggettGarg+PageWootters},
for ``time'' measurement results
(In our notation, Leggett-Garg inequalities are to $\hilb{H}_T$ what the well known Bell inequalities
  are to $\hilb{H}_S$).
This proves the ``quantumness'' of this realization of Page and Wootters time,
regardless of the explicit expression of the corresponding operator (which, unsurprisingly,
is not given). It's tempting to infer that the experiment
rather tests Bell/Leggett-Garg inequalities for photon position.

The first experiment, on the other hand \parencite{Moreva:illustration,Moreva:synthetic},
uses (uncontroversial)
two-level quantum systems for both the clock and the rest of the universe.
While we can't derive a $\hat{T}$ such that $[T, \Omega] = i$
because of the finite dimension of the space, both $\Omega$
and $H_S$ are given an explicit expression:
\begin{align}\label{eq:MorevaOmegaT}
  \Omega            &= i\omega(\ketbra{H}{V}- \ketbra{V}{H})_T \\
  H_S/{\hbar}       &= i\omega(\ketbra{H}{V}- \ketbra{V}{H})_S
  \,\text{,}
\end{align} 
as well as a zero-eigenstate of $\mathbb{J}$ (as in eq. \ref{eq:pwHamiltonian}):
\begin{equation}
  \dket{\Psi} = \frac{1}{\sqrt{2}}\qty(\ket{H}_T\ket{V}_S-\ket{V}_T\ket{H}_S)
  \,\text{.}
\end{equation}

It can be easily verified that the Wheeler-DeWitt condition
\eqref{eq:Wheeler-DeWitt} is satisfied.

In general, given a clock ($\Omega$), the problem of finding a
``rest of the universe'' ($H_S$) such that
$\hbar\hat{\Omega}\ox\idop_S + \idop_T\ox\hat{H}_S = 0$
(and vice versa)
is not trivial
(and it's particularly cumbersome in non-relativisitc
quantum mechanics where we can't avail of negative energies etc.).
Most literature focuses their examples on clocks only
\parencite{Prvanovic,RealisticClocks,HarmonicClocks},
implicitly relying on the scale of a realistic universe
in order to have the \eqref{eq:Wheeler-DeWitt} satisfied
(which was originally derived using General Relativity arguments)
but missing the opportunity to illustrate the entanglement mechanism in detail,
which is aimed at, instead, in the present work, to some extent.

\subsection{Overcoming limitations of finite-dimension spaces}

\epigraph{\textelp{} discreteness in the world is simply the Fourier transform of compactness.}{
  \emph{Physics and the Integers} \parencite{Tong_Integers}
}

\noindent{}This has been tackled, for example in
\cite{FiniteHilb},
where it has been shown that the lack of operators satisfying the canonical
commutation relation \eqref{eq:canonical_commutation_in_time}
is not essential to build operators representing physical observables
with the same role of position and momentum (or $T$ and $\hbar\Omega$
in $\hilb{H}_T$ for a finite-dimensional Page and Wootters model).

Discrete, bounded position and momentum operators can be derived from
each other via
the \term{finite Fourier transform}.
In our case, we are particularly interested in relating the
time operator $\hat{T}$ and the ``energy'' operator $\hbar\hat{\Omega}$
in $\hilb{H}_T$ ---which in the continuous limit would satisfy the
\eqref{eq:canonical_commutation_in_time} exactly.

It holds\footnote{
  Contrary to what indicated in eq. (8) of \cite{FiniteHilb},
  it can be easily verified that,
  if $ \hat{p} = F x F^{\dagger} $,
  the correct inverse relation is
  $ x = F^{\dagger} p F$ and not $ -x = F p F^{\dagger} $.
} \parencite{FiniteHilb}:
\begin{gather}\label{eq:FourierCanonicalRelations}
  \hat{\Omega} = F \hat{T} F^{\dagger}\text{;} \quad
  \hat{T} = F^{\dagger} \hat{\Omega} F
\end{gather}
where, in the ``position'' (or \emph{time}) finite eigenbasis,
\begin{equation}
  F = \frac{1}{\sqrt{N}} \sum_{m,n=0}^{N-1} \exp[i\frac{2\pi mn}{N}] \ketbra{m}{n} \, \text{,}
\end{equation}
while in the frequency eigenbasis
\begin{equation}
  F^{\dagger} = \frac{1}{\sqrt{N}} \sum_{\mu,\nu=0}^{N-1} \exp[-i\frac{2\pi \mu\nu}{N}] \ketbra{\mu}{\nu} \, \text{,}
\end{equation}
with $N$ being the finite dimension of the Hilbert space.

Please note the \eqref{eq:FourierCanonicalRelations} is valid in normalized (``natural'') units
where ``time'' and ``frequency'' are in fact respectively
\term{samples} and \emph{cycles/samples rate},
in a similar sense as in digital signal processing theory
\parencite[pp. 469, 490]{Signal}.

In SI units, the \eqref{eq:FourierCanonicalRelations} is replaced by
\begin{gather}
  \label{eq:SI_Fourier:Omega}
    \hat{\Omega} = \frac{2\pi}{N(\delta T)^2} F \hat{T} F^{\dagger} = \frac{2\pi N}{\qty(\Delta T)^2} F \hat{T} F^{\dagger} \\
  \label{eq:SI_Fourier:T}
    \hat{T} = \frac{2\pi}{N(\delta\Omega)^2} F^{\dagger} \hat{\Omega} F = \frac{2\pi N}{\qty(\Delta\Omega)^2} F^{\dagger} \hat{\Omega} F
  \, \text{,}
\end{gather}
where $\delta T$ (and analogously $\delta\Omega$)
is the size of a ``temporal sample'', or the size of a discrete
time step in the clock, and $\Delta T = N\delta T$ the range of the clock.
For example,
$\delta T = \text{1 hour}$ and $\Delta T=12\;\text{hours}$
for a common clock (hours hand) in our everyday life.

It holds
\begin{gather}
  \delta\Omega \delta T = \frac{2\pi}{N} \, \text{;} \quad
  \Delta\Omega \Delta T = 2\pi N \, \text{.}
\end{gather}

A benefit of finite-dimensional systems is the potential implementation on a finite array of
qubits in a quantum computer. The use of Discrete Fourier Transform extends the overlap
with technology and engineering to the domain of signal processing \parencite{FiniteHilb}.
In \emph{ordinary} quantum mechanics, the Fourier transform (discrete or continuous)
is generally used
to associate wavefunctions in position and momendum space
(whereas time and frequency are \emph{not} operators),
while in communication engineering it is used to convert signals
from the time to the frequency domain and vice versa.
Thanks to the introduction of the Hilbert space $\hilb{H}_T$,
the interpretation in terms of time and frequency
(or time and energy, up to a factor $\hbar$)
is applicable to quantum theory as well, not only formally
i.e. not in the sense of a mere operation among (``classical'') parameters;
but in the sense of conversion between representations of the
same quantum state vector with respect to different eigenbasis,
in full analogy with position and momentum in $\hilb{H}_S$.

\subsection{The $1 + 1$ qubit experiment}

This section aims at a theoretical analysys of the experiment
described in \cite{Moreva:synthetic, Moreva:illustration},
plus drawing some general considerations about the model. 

In \cite{Moreva:illustration}, the frequency operator $\hat{\Omega} = H_T / \hbar$
is given by the \eqref{eq:MorevaOmegaT}. With respect to the polarization basis
$\qty{\ket{H}, \ket{V}}$ it is represented in the the matrix form
\begin{equation}
  \hat{\Omega} \repr {
    i\omega
    \begin{pmatrix}
      0 & 1 \\
     -1 & 0
    \end{pmatrix}
  } \, \text{.}
\end{equation}
The Hamiltonian in $\hilb{H}_S$, in that particular experiment, is represented as
\begin{equation}
  \hat{H}_S \repr {
    i\hbar\omega
    \begin{pmatrix}
      0 & 1 \\
     -1 & 0
    \end{pmatrix}
  } \, \text{,}
\end{equation}
with respect to horizontal/vertical polarization \emph{of the second photon},
acting as ``the rest of the universe''.
There is no particular physical reason for being $H_S = H_T$
(in principle they could even act on spaces of different dimensionality)
other than simplicity of experimental realization.

The spectrum of $\hat{\Omega}$ is $\qty{-\omega, \omega}$.
Therefore, it's $N=2$ and $\delta\Omega = 2\omega$ in the sense of
\eqref{eq:SI_Fourier:T}.
We can thus derive the time operator matrix:
\begin{equation}
  \hat{T}
  \repr
  \frac{\pi}{4\omega^2} F^{\dagger} \Omega F
  =
  \frac{i\pi}{8\omega}
  \begin{pmatrix}
    1 & 1 \\
    1 & -1
  \end{pmatrix}
  \begin{pmatrix}
    0 & 1 \\
   -1 & 0
  \end{pmatrix}
  \begin{pmatrix}
    1 & 1 \\
    1 & -1
  \end{pmatrix}
  =
  \frac{\pi}{4\omega}
  \begin{pmatrix}
    0 & -i \\
    i &  0
  \end{pmatrix}
  \,\text{.}
\end{equation}
We notice that time is not diagonal in the polarization basis.
It can be diagonalized with:
\begin{equation}\label{eq:moreva_diag_T}
  \mathcal{E}_T^{\dagger} T \mathcal{E}_T
  =
\frac{\pi}{4\omega}
\begin{pmatrix}
  -1  & 0 \\
  0   & 1
\end{pmatrix}
\,\text{,}
\end{equation}
and $\mathcal{E}_T$ being the matrix of eigenvectors of T as columns
\begin{equation}
  \mathcal{E}_T
  =
  \frac{1}{\sqrt{2}}
  \begin{pmatrix}
    i & -i \\
    1 & 1
  \end{pmatrix}
  \,\text{.}
\end{equation}
Thus the clock can measure (only) the two times: $-\frac{\pi}{4\omega}$ and $\frac{\pi}{4\omega}$
(or superpositions of them):
\begin{align}
  \ket{-\frac{\pi}{4\omega}} &= \frac{1}{\sqrt{2}} \qty(\ket{V}+i\ket{H}) \eqdef \ket{L} \\
  \ket{ \frac{\pi}{4\omega}} &= \frac{1}{\sqrt{2}} \qty(\ket{V}-i\ket{H}) \eqdef \ket{R} \, \text{.}
\end{align}
In terms of the physics of the experiment,
time is determined with quantum certainty when the clock photon is
in one of the two circular polarization states.

It's worth observing that
only if $\hat{T}$ is diagonal in a certain basis $\qty{t}$,
the components of a vector in $\hilb{H}_T \ox \hilb{H}_S$
over that basis
can be interpreted as a ``time evolution'':
\begin{equation}
  \dket{\Psi}
  \repr_{\qty{t} \ox \qty{H,V}}
  \qty{\psi_H(t_0), \psi_V(t_0), \psi_H(t_1), \psi_V(t_1)}
\end{equation}
or, in general:
\begin{equation}
  \dket{\Psi}
  \repr
  \qty{
    \psi_0(t_0),
    \dotsc,
    \psi_{N_{S} - 1}(t_0),
    \dots,
    \psi_0(t_{N_{T}-1}),
    \dotsc,
    \psi_{N_{S} - 1}(t_{N_{T}-1})
  } \,\text{,}
\end{equation}
with $N_{S}$ and $N_{T}-1$ being the dimensions of $\hilb{H}_T$ and $\hilb{H}_S$ respectively.

The basis of interest is therefore
\begin{equation}
  \qty{\ket{-\frac{\pi}{4\omega}}, \ket{\frac{\pi}{4\omega}}}_T \ox \qty{\ket{H}, \ket{V}}_S
  \, \text{.}
\end{equation}
How does the matrix of $\hat{\Omega}$ transform? In general, it's
\begin{equation}
  \Omega \rightarrow F \mathcal{E}_T^{\dagger} F^{\dagger} \Omega F \mathcal{E}_T F^{\dagger}
  \, \text{,}
\end{equation}
but we already know $F^{\dagger} \Omega F = T$,
and $\mathcal{E}_T^{\dagger} T \mathcal{E}_T$ is the diagonal matrix
(let's call it $T_d$) of the \eqref{eq:moreva_diag_T}.
In conclusion, with some algebra, it's simply
\begin{equation}
  \Omega_{T_d} \eqdef F^{\dagger} T_d F = \left(\begin{matrix}0 & - \omega\\- \omega & 0\end{matrix}\right)
\end{equation}
and we are interested in the eigensystem of
\begin{multline}
  \mathbb{J}_{T_d} \eqdef \hbar \Omega_{T_d} \ox \idop_{S} + \idop_{T_d} \ox H_S =
    \hbar
    \begin{pmatrix}0 & - \omega\\- \omega & 0\end{pmatrix}
    \ox
    \begin{pmatrix} 1 & 0 \\  0 & 1 \end{pmatrix}
    + \\
    \hbar\omega
    \begin{pmatrix} 1 & 0 \\  0 & 1 \end{pmatrix}
    \ox
    \begin{pmatrix} 0 & 1 \\ -1 & 0 \end{pmatrix}
    =
    \hbar\omega
    \begin{pmatrix}
      0   &i  &-1 &0  \\
      -i  &0  &0  &-1 \\
      -1  &0  &0  &i  \\
      0   &-1 &-i &0  
    \end{pmatrix}
  \text{,}
\end{multline}
which is
\begin{itemize}
  \item Eigenvalue: $0$; degeneracy: $2$; eigenvectors: $(0, -i, 1, 0)$, $(i, 0, 0 ,1)$
  \item Eigenvalue: $-2\hbar\omega$;  non-degenerate; eigenvector: $(-i, 1, -i, 1)$
  \item Eigenvalue: $2\hbar\omega$;   non-degenerate; eigenvector: $(-i, -1, i, 1)$ .
\end{itemize}

See \ref{nb:jupyter:moreva} for more details.