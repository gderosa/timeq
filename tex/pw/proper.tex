\section{Proper vectors of $\pwspace$}\label{sec:properpw}

A vector $\dket{\Psi}$ in $\hilb{H}_T \ox \hilb{H}_S$,
satisfying \eqref{eq:pwHamiltonian} and \eqref{eq:Wheeler-DeWitt},
encodes the whole (unitary) time evolution of a system.
\begin{equation}\label{eq:pwexpansion}
  \dket{\Psi} =
    \int \dd{t} \ket{t}_T \ox \ket{\psi(t)}_S =
    \int \dd{t}\dd[3]{\vec{r}} \ \Psi(t; \vec{r}) \; \ket{t}_T \ox \ket{\vec{r}}_S
    \,  \text{.}
\end{equation}
We know $\setof{\ket{t}_T \ox \ket{\vec{r}}_S}$ is an othonormal basis of $\hilb{H}_T \ox \hilb{H}_S$, therefore
\begin{equation}
  \norm{\dket{\Psi}}^2 =
    \int \dd{t}\dd[3]{\vec{r}} \ \abs{\Psi(t; \vec{r})}^2 =
    \int \dd{t} \int \dd[3]{\vec{r}} \ \abs{\Psi(t; \vec{r})}^2 =
    \int \dd{t} 1 \rightarrow +\infty
    \,  \text{,}
\end{equation}
which means that such $\dket{\Psi}$ is an \term{improper} vector of $\hilb{H}_T \ox \hilb{H}_S$.

Proper (i.e. normalizable) states are described in \cite{Lloyd:Time} as well, by replacing (or generalizing)
the \eqref{eq:pwexpansion} with
\begin{equation}\label{eq:pwphi}
  \dket{\Phi} =
    \int \dd{t} \phi(t) \ket{t}_T \ox \ket{\psi(t)}_S \, \text{.}
\end{equation}
If the function $\phi \in \mathscr{L}^2(\mathbb{R})$,
then $\dket{\Psi}$ is a proper element of the product space,
and $\norm{\dket{\Psi}}^2 = \int \dd{t} \abs{\phi(t)}^2$.

The case of non-normalizable $\dket{\Psi}$ in \eqref{eq:pwphi},
with normalized $\ket{\psi(t)}_S$ $\forall t \in \mathbb{R}$,
describes the unitary evolution, as seen throughout Chapter \ref{ch:pw}.
As observed in \cite{Maccone:QGR},
``%
  Quantum mechanics is formulated in terms of \emph{systems},
  typically limited in space but infinitely extended in time%
''.
If the state vector is \emph{conditioned} at a particular time $t$,
it holds $\norm{_{T}\bradket{t}{\Psi}}_S = \norm{\ket{\psi(t)}}_S = 1$,
meaning that, at each $t$,
\emph{the particle must certainly be in some (one) point in space}.

A normalized $\dket{\Psi}$ in the whole $\hilb{H}_T \ox \hilb{H}_S$,
instead,
can be interpreted as a total probability of~$1$ in both space and time combined.
It can be interpreted as describing an \term{event},
in that it
must certainly be in some point in space
\emph{and} time (in terms of outcome of an idealized measurement\footnote{
  Measurement in quantum mechanics requires the concepts
  of state of the system
  (and measurement apparatus)
  \emph{before} and \emph{after} the measurement and, therefore, the existence
  of an external time (external with respect of the Hilbert space of states).
  This logically contradicts the foundation of the Page--Wootters model if
  time itself is measured as a quantum observable. $\dket{\Psi} \in \pwspace$
  embeds the whole history of a system and therefore cannot have a
  ``before'' neither an ``after'' the measurement, ``when'' it collapses
  into an eigenstate of time. The apparent contradicion is resolved
  stressing that probability amplitues are intended in the sense of
  \emph{conditional} probabilities e.g. \emph{provided that the particle
  is in position} $x \in X$ (or the detector clicks)
  what is the probability density amplitude of time being (the clock showing) $t$?
  Consistently, the Bayes rule
  (see, for example, \cite{Stat:Conditional})
  is invoked in the following sections
  and references.
}).
A ``typical'' example of localized ``event wave packet'' would be
therefore represented by
a \emph{4\nobreakdash-dimensional} Gaussian wave function, % https://tex.stackexchange.com/a/330437
in analogy to well known examples of purely spatial Gaussian states
in quantum mechanics and quantum optics.

