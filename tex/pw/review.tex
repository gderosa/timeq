\section{Introduction}

Papers: \cite{Lloyd:Time, Marletto:Evolution, Prvanovic, Maccone:Pauli, RealisticClocks}.

% Extra: \cite{TimeAnyons}.

And an experiment: \cite{Moreva:synthetic,Moreva:illustration}.
Some theoretical implications of the experiment are reviewed in
\cite{LeggettGarg+PageWootters}.

And a NEW experiment! \parencite{Moreva_position}.

The basic idea is an additional Hilbert space $\mathcal{H}_T$ where time is an observable
corresponding to
a self-adjoint operator whose mathematical properties are the same of position  n the
ordinary Hilbert space of a quantum particle in one dimension.

In this language, the ordinary Hilbert space can be labeled $\mathcal{H}_S$;
and we consider the product space $\mathcal{H}_T \otimes \mathcal{H}_S$ as
the space in which both time and position are observables, and they act as
$\hat{t} \otimes \idop_S$ and $\idop_T \otimes \hat{x}$
respectively.

\begin{remark}
  In a more ``relativistic friendly'' notation, we may set
  $\hat{x_0} = c\hat{t}$ and consider
  $\hat{x_0} \otimes \idop_S$ and $\idop_T \otimes \hat{x_1}$,
  and so on. An interesting development may be expressing
  Lorentz transformations as unitary transformations in
  $\mathcal{H}_T \otimes \mathcal{H}_S$.
\end{remark}

\iftodo

\section{Prvanovic and P\&W}
In \cite{Prvanovic}, essentially the clock observable is the Hamiltonian.
The two example clocks are an harmonic oscillator and a free particle.
The harmonic oscillator features discrete time. Generally a time which is
{bounded from below}
is consistent with the Big Bang...

Prvanovic uses ``relativisitc'' constants...

Ciao.

\section{Analyze latest Moreva experiment with previous ones}

Let's consider a photon, as a two level system with its two linear polarization
sates $\ket{H}, \ket{V}$;
and the Hamiltonian
\begin{equation}
  \mathcal{H}=i\hbar\omega\qty(\ket{H}\ket{V}-\ket{V}\ket{H})\,\text{.}
\end{equation}
There has:
\[
  \mathcal{H}^n \ket{H, V} = \begin{cases}
      \hbar^n\omega^n\ket{H, V} &\text{for n even} \\
    -i\hbar^n\omega^n\ket{V, H} &\text{for n odd}
  \end{cases}
\]

\url{http://physweb.bgu.ac.il/COURSES/PHYSICS3_physics/CLASS_ymeir/polarization.pdf}

\section{Entanglement and decoherence (Arrow of time)}
See also \cite{EntanglementVsDecoherence}.

Decoherence is an irreversible process, it also happens in measurement.

According to Marletto and Vedral, arrow of time is increase in Entanglement
between the clock and the rest.

So, there seems to be a contradiction: is entanglement ``decreasing''
(i.e. destroyed by decoherence) with time
or increasing?

We can avoid the contradiction saying that
entanglement between two finite systems is
destroyed while the entanglement of each of them with the universe
is increasing.

\fi

\iftodo
\section{Further TODOs}

\cite{HarmonicClocks} concludes ``Classical clock can be described by an Hamiltonian linear in momentum''\dots
like in relativity?

TODO: use the harmonic oscillator in \cite{HarmonicClocks}
as a PaW clock for the same packet that is measured in
Ruschhaupt's detector model.
Therein, fading wave function: is minus derivative an event?
L4 normalized?

Other systems of interest: decays. Prvanovic new.

TQM Book: time of residence: applicable to qubits, therefore Moreva experiment.

https://arxiv.org/abs/1708.04302 majorization, thermo.

\fi

