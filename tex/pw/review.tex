\section{Blah blah}

Three papers: \cite{Lloyd:Time}, \cite{Marletto:Evolution}, \cite{Prvanovic}.

And an experiment: \cite{Moreva:synthetic,Moreva:illustration}.

The basic idea is an additional Hilbert space $\mathcal{H}_T$ where time is an observable
corresponding to
a self-adjoint operator whose mathematical properties are the same of position in the
ordinary Hilbert space of a quantum particle in one dimension.

In this language, the ordinary Hilbert space can be labeled $\mathcal{H}_S$;
and we consider the product space $\mathcal{H}_T \otimes \mathcal{H}_S$ as
the space in which both time and position are observables, and they act as
$\hat{t} \otimes \idop_S$ and $\idop_T \otimes \hat{x}$
respectively.

\section{Questions and observations --- some of which may be resolved already}

With regards to \cite{Lloyd:Time}, section B, \textit{Measurement}.

\begin{enumerate}
  \item What is a memory system in quantum mechanics? 
  \begin{itemize}
    \item Do they mean a non-Markovian system? Look at \cite{MeasurementMarkovian}\dots
    \item Memory in the sense of the Maxwell daemon?
    \item Any general theory of quantum memory systems?
  \end{itemize}
  \item What is a fiducial state? Any particular ``quantum'' meaning of the term?
  \item ``\emph{the case where a measurement is performed at time t1}'' suggests they are back to time as an external parameter\dots
\end{enumerate}
