\section{Discussion}

The present work has illustrated,
mainly through numerical examples
at various degrees of complexity,
and developing from existing theoretical work,
how time can be trated as a quantum observable
(described by a self-adjoint operator).

The Pauli objection (presented in Sec. \ref{proof})
is overcome because the time operator $\hat{T}$
is defined in a different Hilbert space (that we name $\hilb{H}_T$)
than the space, say $\hilb{H}_S$, where the Hamiltonian is defined,
according to the Page and Wootters model
\parencite{PageWootters, Lloyd:Time, Marletto:Evolution, Maccone:QMOT, Maccone:Pauli}.

One can introduce an extra Hilbert space, to that of ordinary quantum mechanics,
in which the time operator is defined.