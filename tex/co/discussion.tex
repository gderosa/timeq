\section{Discussion}

The initial draft title for this work was: 
\textit{It's $\it\ket{1}$ o'clock -- Relational Time and Applications}.
Can time have eigenvectors and eigenvalues?

We have illustrated,
mainly through numerical examples
at various degrees of complexity,
and developing from existing theoretical work,
how time can be trated as a quantum observable
(described by a self-adjoint operator with its eigenvalues ---and eigenvectors).

The Pauli objection (presented in Section~\ref{proof})
can be overcome if the time operator $\hat{T}$
is defined in a different Hilbert space (that we name $\hilb{H}_T$),
distinct from
the space, say $\hilb{H}_S$, where the Hamiltonian is defined,
as set out by the Page and Wootters model
\parencite{PageWootters, Lloyd:Time, Marletto:Evolution, Maccone:QMOT, Maccone:Pauli}.

One can introduce an extra Hilbert space, to that of ordinary quantum mechanics,
in which the time operator is defined.
Or, more ``realistically'', identify, in a closed system, a subsystem
acting as a clock for the rest of it. Requirements are non-interaction and
full entanglement.

