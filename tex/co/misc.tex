\section{Other topics}\label{sec:outlook-misc}

\subsection{Computation methods and decomposition}

In Sec.~\ref{sec:finite-quantum}, it was noted that
``a benefit of finite-dimensional systems is the potential implementation on a finite array of
qubits in a quantum computer''.
Indeed,
experimentally,
besides looking for a suitable $N$-level system to act as a clock,
we may want to ``encode'' $2^{n}$ levels into the combined state of $n$
qubits and
decompose $\mathbb{J}$ (or, actually, $e^{-i\mathbb{J}\tau/\hbar}$)
into simpler Hamiltonians (respectively: evolutions)
acting on individual qubits (\term{gates}),
therefore allowing the experiment to be run on a quantum processor.
A decomposition method that can be considered for the purpose
is the Trotter-Suzuki scheme
\parencite{Trotter-Suzuki:exp, Trotter-Suzuki:GPU}.

\subsection{Additional themes}

Further studies
would be required to explore the potential to model
---in terms of relational time---
other quantum systems and processes such as:
%
\term{time crystals} \parencite{crystal2,crystal3,crystal2012};
%
\term{atomic clocks} \parencite{TQM2};
%
\term{irreversibility} and ``arrow of time'' \parencite{Josset_Thermo};
%
\term{spontaneous emission} \parencite{Souza_Spontaneous};
%
as well as
\term{reference frames} and transformations \parencite{Adlam_ReferenceFrames}.
