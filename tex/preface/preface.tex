This work relates to the problem of time in quantum physics%
\footnote{
  See, for a review, \cite{TQM1, TQM2}.
},
from Wolfgang Pauli's argument
on the impossibility of a time observable \parencite{PauliFootnote},
to \emph{relational} models developed in the last decades,
based
on the entanglement of the ordinary Hilbert space of quantum mechanics
with an extra space where a time operator is introduced.

Pauli's objection is analyzed in possibly more detail, as well as the models
that have been proposed
to overcome it. 

One of the aims of the present work
is bridging a gap between the theoretical
and experimental literature on the topic.
To this end, examples and applications are developed on top of existing theoretical results;
and, conversely,
a conceptual critique and theoretical analysis
of recent experiments is also carried out.
