This work relates to the problem of time in quantum physics%
\footnote{
  See, for a review, \cite{TQM1, TQM2}.
},
from Wolfgang Pauli's argument
on the impossibility of defining a time observable \parencite{PauliFootnote},
to \emph{relational} models developed during the last decades:
according to such models,
time and unitary evolution only emerge in
terms of \emph{entanglement} between noninteracting subsystems
---one of which acts as a ``clock''---
of an otherwise stationary universe \parencite{PageWootters, Marletto:Evolution}.

Pauli's objection is reviewed and commented,
as well as some of the alternative models
that have been proposed
to overcome it.

One of the aims of the present work
is contributing to bridging a gap between the theoretical
and experimental literature on the topic.
To this end, examples and applications are developed on top of existing theoretical results.
Conversely,
a conceptual critique and theoretical analysis
of recent experiments is carried out as well.
Numerical computation is also extensively employed in order to compare
predictions from different models, relational or not.