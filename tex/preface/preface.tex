This work relates to the problem of time in quantum physics%
\footnote{
  See, for a review, \cite{TQM1, TQM2}.
},
from Wolfgang Pauli's argument
on the impossibility of a time observable \parencite{PauliFootnote},
to \emph{relational} models developed in the last decades,
based
on the entanglement of the ordinary Hilbert space of quantum mechanics
with an extra space where a time operator is introduced.

Pauli's objection is analyzed in possibly more details, as well as the models
aimed at overcoming it, in an effort of generalization on one side,
and at bridging a gap between the theoretical
and experimental literature on the other
---adding examples and applications to existing theoretical results,
and a degree of conceptual critique and theoretical analysis
to recent experiments.
