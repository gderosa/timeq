\section*{Quick note on symbols}

Generally, $\hilb{H}_S$ indicates the Hilbert space of ordinary quantum mechanics.
$\hilb{H}_T$, an extra space where a time operator $\hat{T}$ is defined. The product space $\pwspace$
is often referred to in the present work.

$\hat{A}$ denotes a self-adjoint operator.

$U$, $F$, etc. denote unitary operators, generally not self-adjoint.

$\mathcal{K}$, a non self-adjoint, non-unitary operator e.g. ``Hamiltonians'' embedding a complex potential,
and such that $\exp(-\iu \mathcal{K} t / \hbar)$ is not unitary either.

$\hat{\mathbb{J}}$ indicates a self-adjoint operator defined in a product space with respect to the original Hilbert space
of quantum mechanics, tipically $\pwspace$. No circumflex accent on the symbol if the operator is not self-adjoint.

The symbol `$\repr$', as in $\hat{A} \repr \mqty(a&b\\c&d)$, $\ket{\psi} \repr \mqty[\alpha \\ \beta]$
means: representation with respect to a particular basis (as opposed to intrinsic equality `$=$').

The symbol `$\eqbydef$'
means: equal by definition, equal by settings i.e. postulated and not derived logically.
